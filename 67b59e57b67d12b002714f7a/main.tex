\documentclass[a4paper,12pt]{article}

\usepackage{amsmath, amssymb, amsthm}
\usepackage{geometry}
\geometry{a4paper, margin=1in}
\usepackage{hyperref}

\title{Die Freese-Formel: Eine neue Skalenordnung für die Riemann-Zeta-Nullstellen}
\author{Tim Freese}
\date{\today}

\begin{document}

\maketitle

\begin{abstract}
Diese Arbeit präsentiert eine neue Skalierungsformel für die Abstände der nichttrivialen Nullstellen der Riemannschen Zetafunktion.  
Erste numerische Tests zeigen, dass diese Skalenordnung mit einer fundamentalen mathematischen Konstante \( f \) zusammenhängt:

\[
f = \frac{\pi - \varphi}{\pi} \approx 0.4884.
\]

Dieser Zusammenhang deutet auf eine Fibonacci-Logarithmische Struktur in den Nullstellenabständen hin.  
Die exakte mathematische Herleitung wird in zukünftigen Arbeiten untersucht.
\end{abstract}

\section{Einleitung}
Die Verteilung der Nullstellen der Riemannschen Zetafunktion \( \zeta(s) \) hat eine tiefe Verbindung zur Primzahlenverteilung.  
In dieser Arbeit wird eine neue Skalierungsformel vorgeschlagen:

\[
L(N) = \alpha \cdot N^f
\]

wobei \( f \) als fundamentale Naturkonstante auftritt.

\section{Numerische Evidenz}
Erste Simulationen zeigen eine Übereinstimmung der Zeta-Nullstellenabstände mit der Freese-Formel.  
Die vollständige mathematische Analyse dieser Struktur wird in zukünftiger Arbeit folgen.

\section{Fazit}
Die vorgestellte Skalierungsformel zeigt eine neue mathematische Ordnung in der Verteilung der Nullstellen.  
Weitere Forschung wird sich auf die theoretische Fundierung und Verbindung zur Spektraltheorie konzentrieren.

\end{document}
