import numpy as np
import matplotlib.pyplot as plt
from mpl_toolkits.mplot3d import Axes3D

# Daten für die Nullstellen simulieren
n_values = np.arange(1, 5000)
real_parts = 0.5 * np.ones_like(n_values)
imaginary_parts = np.linspace(0, 5000, len(n_values))

# 3D-Plot der Nullstellen in der komplexen Ebene
fig = plt.figure(figsize=(8, 6))
ax = fig.add_subplot(111, projection='3d')
ax.plot(real_parts, imaginary_parts, zs=0, zdir='z', label='Nullstellen')

ax.set_xlabel('Re(s)')
ax.set_ylabel('Im(s)')
ax.set_zlabel('Index')
ax.set_title("3D-Darstellung der Riemann-Nullstellen")
plt.show()

# Mandelbrotmenge mit Nullstellen als Overlay
def mandelbrot(c, max_iter):
    z = 0
    for n in range(max_iter):
        if abs(z) > 2:
            return n
        z = z**2 + c
    return max_iter

# Bildgröße
width, height = 600, 600
xmin, xmax = -2, 1
ymin, ymax = -1.5, 1.5
max_iter = 500

# Mandelbrotmenge berechnen
image = np.zeros((height, width))
for x in range(width):
    for y in range(height):
        cx = xmin + (x / width) * (xmax - xmin)
        cy = ymin + (y / height) * (ymax - ymin)
        color = mandelbrot(complex(cx, cy), max_iter)
        image[y, x] = color

# Nullstellen auf die Mandelbrotmenge plotten
plt.figure(figsize=(8, 8))
plt.imshow(image, cmap='hot', extent=[xmin, xmax, ymin, ymax])
plt.colorbar(label="Iterationen bis Divergenz")

# Nullstellen hinzufügen
plt.scatter(real_parts - 1, imaginary_parts / 5000, color='cyan', s=5, label="Riemann-Nullstellen")
plt.legend()
plt.title("Mandelbrotmenge & Riemannsche Nullstellen")
plt.xlabel("Re(c)")
plt.ylabel("Im(c)")
plt.show()