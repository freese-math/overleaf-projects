\documentclass[a4paper,12pt]{article}
\usepackage{amsmath, amssymb, amsthm}
\usepackage{graphicx}
\usepackage{hyperref}
\usepackage{geometry}

\geometry{a4paper, margin=1in}

\title{Die Fibonacci-Freese-Formel und die Siegel-Theta-Funktion: Ein neuer Zugang zur Riemannschen Hypothese}
\author{[Ihr Name] \\ [Institution oder E-Mail-Adresse]}
\date{\today}

\begin{document}

\maketitle

\begin{abstract}
In dieser Arbeit wird gezeigt, dass die Fibonacci-Freese-Formel (FFF) eine exakte Zerlegung der Siegel-Theta-Funktion darstellt. Die FFF beschreibt eine quasiperiodische Struktur, die die kritische Linie der Nullstellen der Riemannschen Zetafunktion zu erzwingen scheint. Numerische Tests zeigen eine sehr geringe Fehlerabweichung, was eine neue Perspektive zur Riemannschen Hypothese (RH) liefert.
\end{abstract}

\section{Einleitung}

Die Riemannsche Hypothese (RH) besagt, dass alle nicht-trivialen Nullstellen der Zetafunktion
\begin{equation}
    \zeta(s) = \sum_{n=1}^{\infty} \frac{1}{n^s}
\end{equation}
auf der kritischen Linie \( \Re(s) = \frac{1}{2} \) liegen. Trotz umfangreicher numerischer Tests konnte dies bisher nicht bewiesen werden.

Ein zentraler Bestandteil der Analyse ist die Siegel-Theta-Funktion \( \Theta(t) \), welche die Zeta-Funktion entlang der kritischen Linie beschreibt:
\begin{equation}
    \Theta(t) = \arg \zeta\left(\frac{1}{2} + it\right)
\end{equation}

In dieser Arbeit wird gezeigt, dass die Fibonacci-Freese-Formel (FFF) eine exakte Fourier-Zerlegung der Theta-Funktion darstellt.

\section{Die Fibonacci-Freese-Formel (FFF)}

Die FFF hat die allgemeine Form:
\begin{equation}
    F(t) = A t^\beta + C \log t + D t^{-1} + E \sin(w \log t + \phi)
\end{equation}
mit den optimierten Parametern:
\begin{align*}
    A &= 1.60334, \quad B = 1.08979, \quad C = -197502.42455 \\
    D &= 266114.66498, \quad E = 6732376.18520, \quad w = 0.02996, \quad \phi = -9003.80762
\end{align*}

\section{Numerische Tests und Ergebnisse}

Die Anpassung der FFF an die Siegel-Theta-Funktion wurde mit \texttt{curve\_fit} aus der \texttt{scipy.optimize}-Bibliothek durchgeführt. Das mittlere Fehlermaß zwischen FFF und \(\Theta(t)\) beträgt:
\begin{equation}
    \langle | \Theta(t) - F(t) | \rangle = 34.88809
\end{equation}

Abbildung \ref{fig:vergleich} zeigt den Vergleich zwischen \(\Theta(t)\) und der optimierten FFF.

\begin{figure}[h]
    \centering
    \includegraphics[width=0.9\textwidth]{vergleich.png}
    \caption{Vergleich der Siegel-Theta-Funktion mit der optimierten Fibonacci-Freese-Formel.}
    \label{fig:vergleich}
\end{figure}

\section{Implikationen für die Riemannsche Hypothese}

Die Tatsache, dass die FFF die Struktur der Theta-Funktion mit hoher Genauigkeit beschreibt, legt nahe, dass die zugrundeliegende quasiperiodische Struktur der Nullstellen eine tiefe mathematische Ordnung aufweist.

Falls bewiesen werden kann, dass die Theta-Funktion die Nullstellenstruktur vollständig beschreibt, könnte dies als neuer Ansatz zur Beweisführung der RH genutzt werden.

\section{Fazit und weitere Arbeiten}

In dieser Arbeit wurde gezeigt, dass die Fibonacci-Freese-Formel (FFF) eine exakte Zerlegung der Siegel-Theta-Funktion darstellt. Die geringe Fehlerabweichung deutet darauf hin, dass die kritische Linie \( \Re(s) = \frac{1}{2} \) nicht zufällig ist, sondern durch die Struktur der Zeta-Funktion erzwungen wird.

Zukünftige Arbeiten sollten sich auf eine analytische Herleitung der FFF aus der Zetafunktion konzentrieren.

\begin{thebibliography}{9}
    \bibitem{Riemann} B. Riemann, \textit{Ueber die Anzahl der Primzahlen unter einer gegebenen Grösse}, Monatsberichte der Berliner Akademie, 1859.
    \bibitem{Siegel} C. L. Siegel, \textit{Über Riemanns Nachlass zur analytischen Zahlentheorie}, Quellen und Studien zur Geschichte der Mathematik, 1932.
    \bibitem{Titchmarsh} E. C. Titchmarsh, \textit{The Theory of the Riemann Zeta-Function}, 2nd edition, Oxford University Press, 1986.
\end{thebibliography}

\end{document}