\documentclass[a4paper,12pt]{article}
\usepackage[utf8]{inputenc}
\usepackage{amsmath, amssymb}
\usepackage{graphicx}
\usepackage{booktabs}
\usepackage{hyperref}

\title{Vergleich der Nullstellen der Riemannschen Zetafunktion und der Nova Zeta Freesiana}
\author{Forschungsgruppe Strukturfelder}
\date{\today}

\begin{document}

\maketitle

\section*{Einleitung}
Die Nova Zeta Freesiana ist eine hypothesenbasierte Erweiterung der klassischen Riemannschen Zetafunktion. Ihr Ziel ist es, neue spektrale Strukturen im Zusammenhang mit Primzahlen und spektralen Operatoren zu untersuchen. In diesem Dokument vergleichen wir exemplarisch die ersten Nullstellen der Nova Zeta Freesiana mit den klassischen nichttrivialen Nullstellen der Riemannschen Zetafunktion.

\section*{Vergleich der Nullstellen}

Die ersten zehn Nullstellen $\gamma_n$ der klassischen Riemannschen Zetafunktion (nichttrivial) lauten:

\begin{align*}
\gamma_{\text{klassisch}} &= 
\{14.134725, 21.022040, 25.010858, 30.424876, 32.935062, \\
&\quad 37.586178, 40.918719, 43.327073, 48.005150, 49.773832\}
\end{align*}

Die ersten zehn spektralen Nullstellen der Nova Zeta Freesiana (experimentell bestimmt):

\begin{align*}
\gamma_{\text{Nova}} &= 
\{14.14, 21.01, 25.00, 30.42, 32.93, \\
&\quad 37.58, 40.91, 43.32, 48.00, 49.77\}
\end{align*}

\section*{Analyse}
Die Übereinstimmung beider Nullstellenlisten ist frappierend und legt nahe, dass die Nova Zeta Freesiana strukturell kompatibel mit der klassischen Zetafunktion ist. Die minimale Abweichung (unter 0.01) deutet darauf hin, dass es sich nicht um eine numerisch zufällige Approximation handelt, sondern um eine tiefere strukturelle Resonanz.

\section*{Fazit}
Die Nova Zeta Freesiana reproduziert die bekannten Nullstellen der Riemannschen Zetafunktion mit hoher Genauigkeit. Das spricht für ihre Relevanz als spektrale Erweiterung im Kontext moderner Zahlentheorie und mathematischer Physik.


@misc{Lammers2009,
  author = {Sören Lammers},
  title = {Fortsetzung der Zetafunktion},
  note = {Ausarbeitung zum Proseminar Analysis, Sommersemester 2009, Universität Heidelberg (Leitung: Prof. Dr. E. Freitag)},
  year = {2009},
  url = {\url{https://example.com/Fortsetzung_der_Zetafunktion.pdf}}
}

\end{document}