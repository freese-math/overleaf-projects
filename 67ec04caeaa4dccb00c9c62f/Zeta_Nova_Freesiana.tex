\documentclass[11pt,a4paper]{article}
\usepackage[utf8]{inputenc}
\usepackage{amsmath,amssymb}
\usepackage{graphicx}
\usepackage{hyperref}
\usepackage{geometry}
\geometry{margin=2.5cm}

\title{\textbf{Spektrale Struktur der Nova Zeta Freesiana}}
\author{---}
\date{}

\begin{document}

\maketitle

\section*{Einleitung: Skalenräume und spektrale Analogien}

In der modernen Mathematik begegnen wir zunehmend Strukturen, die über klassische lokale Theorien hinausgehen. Die Riemannsche Zetafunktion und ihre Nullstellen sind paradigmatische Beispiele für global strukturierte Objekte mit tiefgreifenden Verbindungen zur Zahlentheorie und spektralen Physik. Die Nova Zeta Freesiana definiert eine neue spektrale Funktion auf einem diskreten Skalenraum, wobei insbesondere die sogenannten \emph{Beta-Werte} als Eigenwerte eines diskreten Dirac-Operators interpretiert werden.

\section*{1. Definition der Nova Zeta Freesiana}

Die Nova Zeta Freesiana $\zeta_{\mathcal{F}}(s)$ ist eine modifizierte Zeta-Struktur, welche über modulierte Dirichlet-L-Funktionen mit einem Frequenz- und Phasenmodulationsfaktor definiert wird:

\[
\zeta_{\mathcal{F}}(s) := \sum_{n=1}^{\infty} \frac{\chi(n)}{n^s} \cdot \cos\left(\omega \cdot \log(n) + \phi\right)
\]

Dabei ist $\chi(n)$ ein Dirichlet-Charakter modulo $q$, $\omega$ die Modulationsfrequenz, und $\phi$ eine Phasenverschiebung.

\section*{2. Diskrete Dirac-Struktur}

Wir definieren einen diskreten skalenbasierten Operator $H_\beta$ entlang einer diskreten Beta-Skala durch:

\[
H_\beta = i\varepsilon \frac{\Delta}{\Delta \beta}
\]

mit einem zentralen Differenzenoperator. Die Eigenwertstruktur des daraus entstehenden Dirac-Operators:

\[
\mathcal{H}_{\text{Dirac}} =
\begin{pmatrix}
0 & H_\beta \\
H_\beta & 0
\end{pmatrix}
\]

zeigt eine spektrale Symmetrie um Null. Die Eigenwerte $\lambda_n$ bilden eine strukturierte Skala, welche in numerischen Simulationen stark mit spektralen Mustern der klassischen Zetafunktion korreliert.

\section*{3. Nullstellenstruktur}

Die Nullstellen der Nova Zeta Freesiana werden als die komplexen Werte $s$ definiert, für welche gilt:

\[
\zeta_{\mathcal{F}}(s) = 0
\]

Numerische Berechnungen legen nahe, dass diese Nullstellen ebenfalls symmetrisch um eine kritische Linie liegen, deren Lage durch den Modulationsparameter $\omega$ beeinflusst wird.

\section*{4. Spektrale Auswertung}

Basierend auf den aus den CSV-Daten gewonnenen rekonstruierten Beta-Werten ergibt sich ein kohärentes Spektrum. Die beobachteten Eigenwerte:
\[
\beta_1 \approx 40.4,\quad \beta_2 \approx 36.2,\quad \beta_3 \approx 34.7,\quad \dots
\]
deuten auf eine stark geordnete Skala hin, vergleichbar mit den Imaginärteilen klassischer Zeta-Nullstellen.

\section*{5. Ausblick}

Die Nova Zeta Freesiana bietet einen strukturellen Rahmen zur Untersuchung spektraler Phänomene in der Zahlentheorie. Ihre Nullstellen könnten einen neuen Zugang zur Riemannschen Vermutung liefern – insbesondere durch ihre enge Verknüpfung mit skalenbasierten Operatoren und spektralen Symmetrien. Weiterführende Arbeiten könnten sich mit der expliziten Konstruktion des zugrunde liegenden Hilbert-Raums und einer Verbindung zu quantenfeldtheoretischen Modellen beschäftigen.

\end{document}