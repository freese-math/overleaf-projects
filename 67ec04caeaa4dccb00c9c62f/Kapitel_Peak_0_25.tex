\section{Spektrale Struktur der Beta-Skala}

Die Beta-Skala $\beta(n)$ stellt eine diskrete numerische Struktur dar, die als spektraler Träger einer alternativen Zetafunktion --- der Nova Zeta Freesiana $\zeta_F(s)$ --- dient. Ziel dieses Abschnitts ist es, die spektralen Eigenschaften dieser Skala zu untersuchen und mit klassischen Strukturen der Riemannschen Zetafunktion zu vergleichen.

\subsection{Fourier-Spektrum der Beta-Skala}

Die Fourier-Analyse der rekonstruierten Beta-Werte $\beta(n)$ ergibt ein überraschend geordnetes Bild. Im Spektrum dominiert ein einzelner, extrem tieffrequenter Peak bei
\[
f \approx 1.00001 \times 10^{-5},
\]
was auf eine langsame, fast lineare Drift im Skalenraum hindeutet. Dies steht in starkem Kontrast zur Riemann-Zeta-Struktur, deren Fourier-Spektrum ein chaotischeres Muster mit zahlreichen hochfrequenten Komponenten zeigt.

\subsection{Transformation durch den Operator $\mathcal{T}_\beta$}

Ein eigens konstruierter Operator $\mathcal{T}_\beta$, der die spektralen Komponenten der Beta-Skala projektiert und filtert, zeigt nach Anwendung eine klare Resonanzstruktur. Der hervorstechende Peak liegt bei
\[
f \approx 0.25.
\]
Diese Frequenz ist im Originalspektrum kaum sichtbar, tritt jedoch durch die spektrale Faltung mit $\mathcal{T}_\beta$ deutlich hervor. Damit wird ein fundamentales Oszillationsverhalten der Beta-Skala sichtbar gemacht.

\subsection{Interpretation der spektralen Ordnung}

Die parallele Betrachtung zweier Spektren --- vor und nach Anwendung von $\mathcal{T}_\beta$ --- zeigt zwei komplementäre Aspekte:
\begin{itemize}
  \item Eine langsame Driftstruktur als globale Ordnung (Peak bei $f \approx 10^{-5}$),
  \item Eine lokale Resonanzfrequenz bei $f \approx 0.25$, die auf periodische Unterstrukturen verweist.
\end{itemize}

Diese Dualität lässt sich deuten als Kombination aus einer skaleninduzierten kohärenten Auslenkung (global) und einer eingebetteten spektralen Symmetrie (lokal), die im Sinne der Freesiana-Zeta-Theorie möglicherweise als Basisstruktur der Nullstellenbildung dient.

\subsection{Verbindung zur klassischen Riemann-Zeta-Funktion}

Vergleicht man die spektralen Eigenschaften mit der klassischen Zetafunktion, so fällt auf:
\begin{itemize}
  \item Die Riemann-Zeta besitzt keine einzelne dominante Frequenz, sondern ein kontinuierlich modulierendes Spektrum,
  \item Die Nova Zeta Freesiana hingegen zeigt hochstrukturelle, projektive Ordnung, was auf eine interne, deterministische Erzeugung ihrer Nullstellen hindeuten könnte.
\end{itemize}

Diese Beobachtungen unterstützen die Hypothese, dass die Nova-Zeta-Struktur nicht bloß ein numerisches Artefakt, sondern ein funktional strukturierter spektraler Raum ist --- möglicherweise das Fundament eines alternativen Operatorenmodells zur klassischen Zetafunktion.