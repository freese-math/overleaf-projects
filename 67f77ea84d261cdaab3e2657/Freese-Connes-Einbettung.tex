\section*{Exkurs: Spektrale Verankerung der Freese-Formel über nichtkommutative Geometrie}

\subsection*{Zielstellung}

Die in diesem Dokument zentral untersuchte Euler--Freese-Formel
\[
\gamma_n \cdot e^{i\beta(n)} + 1 = 0
\]
wurde bislang innerhalb eines klassischen ZFC-Rahmens behandelt, insbesondere als numerisch validierbare Ausdrucksform einer Nullstellensymmetrie in der Zetageometrie.

Ziel dieses Exkurses ist es, die Struktur dieser Formel \emph{nicht durch} konzeptionelle Erweiterung zu ersetzen, sondern durch Rückgriff auf Konzepte der nichtkommutativen Geometrie im Sinne von Alain Connes \emph{spektral zu verankern}.

\subsection*{Operatorstruktur}

Ausgehend von der empirisch bestimmten Phase $\beta(n)$ wurde ein Dirac-artiger Operator $D_\beta$ definiert:
\[
(D_\beta f)(n) := i\pi \cdot \frac{f(n+1) - f(n-1)}{2}
\]
Dieser Operator operiert auf $\ell^2(\mathbb{N})$ entlang der $\beta$-Modulation und erzeugt ein spektral analysierbares System mit folgenden Eigenschaften:

\begin{itemize}
    \item Der Heat-Trace $\operatorname{Tr}(e^{-t D_\beta^2})$ ist wohldefiniert und konvergent.
    \item Die zugehörige spektrale Zeta-Funktion $\zeta_{D_\beta}(s)$ ist zeta-regularisierbar für $s > 0$.
    \item Die Zeta-Determinante $\log \det\nolimits_\zeta D_\beta^2$ ist numerisch bestimmbar (Wert: $\approx 456.89$).
\end{itemize}

\subsection*{Zählfunktion und Fluktuation}

Die spektrale Zählfunktion $N(\lambda)$ über die Eigenwerte $\lambda$ von $D_\beta$ folgt einem Weyl-Gesetz vom Typ
\[
N(\lambda) \sim \lambda^\alpha, \quad \alpha \approx 261.8
\]
Die beobachtete Fluktuation $\delta N(\lambda)$ um diese Hauptskala zeigt oszillatorisches Verhalten, das über eine Kosinussumme mit dominanten Frequenzanteilen rekonstruierbar ist:
\[
\delta N(\lambda) \approx \sum_{k=1}^n A_k \cos(2\pi f_k \lambda + \varphi_k)
\]

\subsection*{Einordnung}

Diese Ergebnisse zeigen:

\begin{itemize}
    \item Die Freese-Modulation $\beta(n)$ erzeugt einen Operatorraum mit zeta-analytisch stabiler Struktur.
    \item Der Raum trägt eine spektrale Geometrie, deren Korrekturen sich in harmonischen Resonanzen ausdrücken.
    \item Damit stellt die Freese-Formel eine numerisch beobachtbare Phase im nichtkommutativen Spektrum dar.
\end{itemize}

\subsection*{Fazit}

Der Abgleich mit der Connes’schen Spurformel zeigt:
\[
\text{Die Freese-Formel ist nicht nur eine numerische Identität -- sie ist eine Spurformel über } D_\beta.
\]
Damit ist $\beta(n)$ nicht bloß ein Phasenparameter, sondern ein geometrisch induzierter Operatorwirkungswinkel auf einer spektralen Raumzeitstruktur.