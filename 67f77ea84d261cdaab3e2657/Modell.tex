\documentclass{article}
\usepackage{amsmath, amssymb, graphicx}

\title{Modell zur Analyse der Beta-Skala, Riemann-Nullstellen und Primzahldifferenzen}
\author{}
\date{}

\begin{document}
\maketitle

\section*{1. Hintergrund}

Ausgehend von den ersten $2\,001\,052$ nicht-trivialen Nullstellen der Riemannschen Zetafunktion (nach Odlyzko) sowie einer empirisch rekonstruierten Beta-Skala $\{\beta_n\}$ mit gleicher Länge, wird ein analytischer und spektraler Vergleich durchgeführt. Ergänzt wird dies durch die Folge der ersten $2\,001\,052$ Primzahlen $\{p_n\}$.

\section*{2. Definitionen}

\subsection*{2.1 Riemann-Nullstellen (Imaginärteile)}

Die nicht-trivialen Nullstellen der Zetafunktion liegen in der kritischen Linie $\Re(s) = \tfrac{1}{2}$, sodass sie als
\[
\rho_n = \tfrac{1}{2} + i \gamma_n, \quad \text{mit } \gamma_n \in \mathbb{R}, \quad n \in \mathbb{N}
\]
notiert werden können.

\subsection*{2.2 Primzahlen und Differenzen}

\[
p_1 = 2,\quad p_2 = 3, \quad p_3 = 5,\ldots
\]
\[
\Delta p_n = p_{n+1} - p_n
\]

\subsection*{2.3 Beta-Skala}

Die rekonstruierten Beta-Werte werden als Folge $\{\beta_n\}$ dargestellt. Sie repräsentieren eine nichtlineare, spektral kodierte Skala, empirisch abgeleitet über Operatoren in kosmologischen Metriken und Theta/Euler-Siegel-Kalibrierung.

Die Differenzen lauten:
\[
\Delta \beta_n = \beta_{n+1} - \beta_n
\]

\section*{3. Normalisierung}

Zur Vergleichbarkeit werden die Differenzen zentriert und skaliert:
\[
\tilde{\Delta x}_n = \frac{\Delta x_n - \mu_{\Delta x}}{\sigma_{\Delta x}}, \quad x \in \{\gamma, p, \beta\}
\]

\section*{4. Fourier-Analyse}

\[
\mathcal{F}[\tilde{\Delta x}_n](f) = \sum_{n=1}^{N} \tilde{\Delta x}_n \, e^{-2\pi i f n}, \quad f \in [0, 0.5]
\]

\section*{5. Korrelationsanalyse}

Die Pearson-Korrelation zwischen normierten Differenzfolgen $x, y$ ergibt sich als:
\[
r = \frac{\sum_{n} (\tilde{\Delta x}_n \cdot \tilde{\Delta y}_n)}{N - 1}
\]

\section*{6. Ergebnis}

\begin{itemize}
  \item Die normierten Differenzen $\tilde{\Delta \beta_n}$ und $\tilde{\Delta \gamma_n}$ sind spektral nahezu identisch:
  \[
  r_{\text{Pearson}} \approx 1.0000
  \]
  \item Die Fourier-Spektren zeigen dominante Peaks bei $f \approx 0$, was auf hochgradige Ordnung und Kohärenz hinweist.
\end{itemize}

\section*{7. Ausblick}

Das Modell legt nahe, dass die Beta-Skala nicht nur ein numerisches Artefakt, sondern möglicherweise eine codierte Darstellung der Nullstellenstruktur ist. Eine axiomatische Herleitung unter Verwendung der Freese-Formel, der Euler-Identität sowie gravitativer Operatoren bleibt Gegenstand zukünftiger Arbeiten.

\end{document}