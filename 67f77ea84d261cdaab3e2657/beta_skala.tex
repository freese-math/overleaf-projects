\documentclass{article}
\usepackage{amsmath}
\usepackage{amssymb}
\usepackage{geometry}
\geometry{a4paper, margin=2.5cm}

\begin{document}

\section*{Analytische Form der Beta-Skala}

Die analytische Form der spektralen Beta-Skala lautet:

\paragraph{Für $\delta \neq -1$:}

\begin{equation}
\beta(n) = \beta_0 + \frac{C \cdot \omega}{\omega^2 + 1}
+ C \cdot \left( \frac{n \cdot \omega \cdot \cos(\omega \log n)}{\omega^2 + 1}
- \frac{n \cdot \sin(\omega \log n)}{\omega^2 + 1} \right)
+ \frac{\alpha}{\delta + 1} \left( n^{\delta + 1} - 1 \right)
\end{equation}

\paragraph{Für $\delta = -1$:}

\begin{equation}
\beta(n) = \beta_0 + \frac{C \cdot \omega}{\omega^2 + 1}
+ C \cdot \left( \frac{n \cdot \omega \cdot \cos(\omega \log n)}{\omega^2 + 1}
- \frac{n \cdot \sin(\omega \log n)}{\omega^2 + 1} \right)
+ \alpha \cdot \log(n)
\end{equation}

\bigskip

\noindent
Dabei sind:
\begin{itemize}
    \item $\beta_0$ ein konstanter Startwert (Initialwert),
    \item $C$ die Amplitude der harmonischen Modulation,
    \item $\omega$ die logarithmische Frequenzkomponente,
    \item $\alpha$ der Driftkoeffizient,
    \item $\delta$ der exponentielle Skalierungsparameter.
\end{itemize}

\end{document}