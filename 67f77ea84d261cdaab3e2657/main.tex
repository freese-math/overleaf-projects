\documentclass[a4paper,12pt]{article}
\usepackage{amsmath, amssymb}
\usepackage{geometry}
\usepackage{graphicx}
\geometry{margin=2.5cm}

\title{Modellierung und Herleitung der Beta-Skala \\[0.5em] \large Struktur, Frequenz und Drift}
\author{Freese Research}
\date{\today}

\begin{document}
\maketitle

\section*{Einleitung}

Die \(\beta(n)\)-Skala stellt eine rekonstruktive spektrale Skala dar, die aus den Nullstellen der Riemann-Zeta-Funktion extrahiert und durch spektrale Operatormethoden verfeinert wurde. 
Zur analytischen Beschreibung dieser Struktur wird ein funktionaler Ansatz verwendet, der folgende Form besitzt:
\[
\beta(n) = A + \frac{a}{\log n} + b \cdot \cos(2\pi f n + \phi) + c \cdot n.
\]
Im Folgenden werden die einzelnen Terme dieses Modells motiviert und hergeleitet.
\bibliographystyle{plain}
\bibliography{references} % Ensure the file 'references.bib' exists in the project directory or update the filename here.

\section{Herleitung des Log-Terms \(\frac{a}{\log n}\)}

\subsection*{1. Verbindung zur Primzahldichte}

Die klassische Primzahldichte ist gegeben durch:
\[
\pi(x) \sim \frac{x}{\log x},
\]
woraus sich durch Umkehrung und Ableitung eine typische Frequenz- oder Skalenstruktur ergibt, die proportional zu \(\frac{1}{\log n}\) skaliert.

\subsection*{2. Mellin-Transformation}

In der analytischen Zahlentheorie tritt der Ausdruck \(\frac{1}{\log n}\) in Verbindung mit Mellin-Transformationen logarithmischer Ausdrücke auf:
\[
\mathcal{M}\{(\log x)^{-s}\}(s) \sim \frac{1}{(s - 1) \log^s x},
\]
was die natürliche Emergenz solcher Terme in asymptotischen Entwicklungen unterstützt.

\section{Herleitung des Cosinus-Terms \(b \cdot \cos(2\pi f n + \phi)\)}

\subsection*{1. Fourier-Synthese der Beta-Skala}

Die Fourier-Analyse der \(\beta(n)\)-Reihe zeigt dominant periodische Anteile bei sehr niedrigen Frequenzen. Dies motiviert einen Cosinus-Term mit Frequenz \(f \approx 10^{-5}\), wie er empirisch aus der Spektralanalyse gewonnen wurde.

\subsection*{2. Physikalische Analogie: Resonante Modulation}

In physikalischen Systemen mit diskreten Eigenzuständen treten Modulationen durch überlagerte Wellen auf. Der Cosinus-Term modelliert somit eine Oszillation in der Energie- oder Frequenzskala, wie sie auch bei gestörten harmonischen Oszillatoren auftritt:
\[
E_n = E_n^{(0)} + b \cos(2\pi f n + \phi).
\]

\section{Herleitung des Linearterms \(c \cdot n\)}

\subsection*{1. Lineare Drift aus Systeminstabilität}

Ein analoges Konzept zu nicht-konservativen Systemen mit konstanter Energiezunahme oder Verlust ergibt einen linearen Term:
\[
E_n \rightarrow E_n + \gamma \cdot n.
\]

\subsection*{2. Asymptotische Reststruktur}

Bei Subtraktion dominanter Terme bleibt eine lineare Komponente als Restglied bestehen:
\[
\beta(n) - \frac{a}{\log n} \approx c \cdot n.
\]

\subsection*{3. Fourier-Kompensation unstetiger Skalen}

In diskretisierten Fourier-Transformationen können lineare Terme als Korrekturfaktor zur Zentrierung der Spektralkomponenten auftreten.

\section*{Fazit}

Alle drei Terme lassen sich sowohl numerisch bestätigen als auch analytisch motivieren. Damit entsteht eine robuste Modellstruktur für die Beta-Skala, die spektrale, analytische und physikalische Aspekte vereint.

\section{Herleitung des Cosinus-Terms aus der Siegel-Theta-Funktion}

Die periodische Struktur des empirischen Termes $b \cos(2\pi f n + \phi)$ in der spektralen Skala $\beta(n)$ lässt sich im Kontext der Theorie modularer Formen und insbesondere der Siegel-Theta-Funktion formal motivieren. Die klassische Jacobi-Theta-Funktion besitzt die Darstellung:
\[
\vartheta(z,\tau) = \sum_{n=-\infty}^{\infty} \exp\left( \pi i n^2 \tau + 2\pi i n z \right),
\]
wobei $\tau$ in der oberen Halbebene liegt und $z \in \mathbb{C}$ ist. Für $\tau = i t$, $t > 0$ und $z = \frac{n}{N}$ ergibt sich:
\[
\vartheta\left(\frac{n}{N}, it\right) = \sum_{k \in \mathbb{Z}} e^{-\pi k^2 t} \cdot e^{2\pi i k \frac{n}{N}}.
\]

\subsection*{Poisson-Transformation}

Wendet man auf diese Darstellung die Poisson-Summationsformel an, ergibt sich ein spektrales Bild mit realen Oszillationen:
\[
\vartheta\left(\frac{n}{N}, it\right) = \sum_{m \in \mathbb{Z}} \left( \frac{1}{\sqrt{t}} e^{-\pi \frac{(m - n/N)^2}{t}} \right),
\]
welches durch Fouriertransformation eine dominante harmonische Modulation nahe
\[
\cos\left(2\pi f n + \phi\right)
\]
erzeugt, wobei $f = \frac{k}{N}$ für dominanten $k$.

\subsection*{Oszillatorische Beiträge im Spektrum}

Da $\beta(n)$ numerisch eine strukturierte spektrale Dichte mit einem klaren dominanten Frequenzanteil bei $f \approx 10^{-5}$ zeigt, liegt der Schluss nahe, dass dieser durch die modulare Struktur der zugehörigen Theta-Transformation erzeugt wird. Insbesondere ist der Cosinus-Term als spektraler Anteil im Frequenzraum der ursprünglichen Gitterstruktur der Zeta-Nullstellen interpretierbar.

\subsection*{Fazit}

Somit lässt sich der Oszillationsterm $b \cos(2\pi f n + \phi)$ als ein durch die modulare Transformation der Siegel-Theta-Funktion induzierter dominanter Modus interpretieren. Er stellt kein Artefakt dar, sondern folgt aus der harmonischen Analyse des zetaassoziierten Theta-Kerns im Spektralraum.



\end{document}