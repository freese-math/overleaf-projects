\documentclass{article}
\usepackage{amsmath}
\usepackage{physics}
\usepackage{fourier}
\usepackage{graphicx}

\title{Herleitung der Beta-Skala aus der Theta-Transformation der Zeta-Funktion}
\author{Freese, GPT-Kollaboration}
\date{}

\begin{document}
\maketitle

\section*{1. Ausgangspunkt: Theta-Integral der Zeta-Funktion}

Die Riemannsche Zeta-Funktion lässt sich über die Theta-Funktion schreiben als:
\[
\zeta(s) = \frac{1}{\Gamma\left(\frac{s}{2}\right)} \int_0^\infty \left( \theta(it) - 1 \right) t^{\frac{s}{2}} \frac{dt}{t}
\]
mit
\[
\theta(it) = \sum_{n=-\infty}^{\infty} e^{-\pi n^2 t}
\]

\section*{2. Transformation: Mellin und Frequenzraum}
Wende die Mellin-Transformation an:
\[
\zeta(s) \propto \sum_{n=1}^{\infty} \int_0^\infty e^{-\pi n^2 t} t^{\frac{s}{2} - 1} dt
= \sum_{n=1}^{\infty} \left( \frac{1}{(\pi n^2)^{s/2}} \Gamma\left( \frac{s}{2} \right) \right)
\]

\section*{3. Extraktion der harmonischen Struktur}

Durch die Substitution $t = e^u$ wird:
\[
\theta(e^u) = \sum_{n=-\infty}^{\infty} e^{-\pi n^2 e^u}
\]
Eine Fourier-Zerlegung in $u = \log n$ liefert harmonische Moden:
\[
\beta(n) \sim \alpha n^\delta + \sum_k C_k \cos(\omega_k \log n + \varphi_k)
\]

\section*{4. Zielstruktur}

Die numerisch gefittete Struktur lautet:
\[
\beta(n) = C \cdot \cos(\omega \log n + \varphi) + \alpha n^\delta + \beta_0
\]

\end{document}