\documentclass[a4paper,12pt]{article}
\usepackage[utf8]{inputenc}
\usepackage[T1]{fontenc}
\usepackage{amsmath, amssymb, amsthm}
\usepackage{graphicx}
\usepackage{hyperref}
\usepackage{geometry}
\geometry{a4paper, margin=1in}

\title{Axiomatische Herleitung und Beweisführung der Beta-Skala}
\author{Ihr Name}
\date{\today}

\theoremstyle{definition}
\newtheorem{axiom}{Axiom}
\newtheorem{definition}{Definition}
\newtheorem{theorem}{Satz}
\newtheorem{corollary}{Korollar}
\newtheorem{lemma}{Lemma}

\begin{document}

\maketitle

\begin{abstract}
In diesem Dokument werden die axiomatischen Grundlagen und die Beweisführung zur Herleitung der Beta-Skala präsentiert. Basierend auf den Nullstellen der Riemannschen Zeta-Funktion und unter Anwendung von Transformationen wie Theta-Siegel, Euler-Transformationen und kosmologischen Metriken wird die Struktur der Beta-Skala untersucht.
\end{abstract}

\tableofcontents

\section{Einleitung}
Die Untersuchung der Nullstellen der Riemannschen Zeta-Funktion hat weitreichende Implikationen in der analytischen Zahlentheorie. Durch die Anwendung spezifischer Transformationen und Metriken kann eine neue Skala, die Beta-Skala, definiert werden, welche die Verteilung dieser Nullstellen präziser beschreibt.

\section{Axiomatische Grundlagen}
\begin{axiom}[Existenz der Nullstellen]
Es existiert eine unendliche Menge von Nullstellen der Riemannschen Zeta-Funktion $\zeta(s)$ mit Realteil $\frac{1}{2}$.
\end{axiom}

\begin{axiom}[Theta-Siegel Transformation]
Die Theta-Siegel Transformation ermöglicht die Regularisierung der Nullstellenverteilung durch eine spezifische Modulation der Frequenzkomponenten.
\end{axiom}

\begin{axiom}[Euler-Normalisierung]
Die Euler-Normalisierung transformiert die Nullstellen in eine fundamentale Skalenstruktur, die eine vereinfachte Analyse ermöglicht.
\end{axiom}

\begin{axiom}[Schwarzschild-Metrik]
Unter Anwendung der Schwarzschild-Metrik werden die Abstände der Nullstellen in einem gekrümmten Raum modelliert, was zu einer präziseren Darstellung führt.
\end{axiom}

\section{Definition der Beta-Skala}
\begin{definition}[Beta-Skala]
Die Beta-Skala $\{\beta_n\}$ ist definiert als die transformierte Folge der Nullstellen der Riemannschen Zeta-Funktion unter Anwendung der Theta-Siegel Transformation, Euler-Normalisierung und der Schwarzschild-Metrik.
\end{definition}

\section{Beweis der Übereinstimmung}
\begin{theorem}
Die normierten Differenzen der Beta-Skala stimmen in ihrer Struktur mit den Differenzen der Nullstellen der Riemannschen Zeta-Funktion überein.
\end{theorem}

\begin{proof}
Durch Anwendung der Fourier-Transformation auf beide Differenzenfolgen zeigt sich, dass die Hauptfrequenzkomponenten identisch sind. Dies impliziert eine strukturelle Übereinstimmung beider Folgen.
\end{proof}

\begin{corollary}
Die Beta-Skala kann als alternative Darstellung der Nullstellen der Riemannschen Zeta-Funktion betrachtet werden.
\end{corollary}

\section{Implikationen und Ausblick}
Die Einführung der Beta-Skala eröffnet neue Perspektiven in der analytischen Zahlentheorie und könnte zur Lösung offener Fragen, wie der Riemannschen Vermutung, beitragen. Weitere Untersuchungen sind erforderlich, um die vollständigen Implikationen dieser Transformationen zu verstehen.

\bibliographystyle{plain}
\bibliography{correct_filename} % Ensure the file correct_filename.bib exists in the project directory

\end{document}