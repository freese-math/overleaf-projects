\documentclass{article}
\usepackage{amsmath, amssymb, amsfonts}
\usepackage{mathtools}
\usepackage{tcolorbox}
\usepackage{physics}

\begin{document}

\section*{Theorem: Die harmonische Beta-Skala als universeller Spektralträger}

\subsection*{Definition}
\[
\boxed{\beta(n) := A \cdot n^\delta} \quad \text{mit} \quad A = 0.003652, \quad \delta = -1.336085
\]

\begin{enumerate}
    \item \textbf{Operatorstruktur:}
    \[
    H_\beta \psi(n) := \beta(n) \cdot \psi(n)
    \quad \text{mit } \psi(n) \text{ als Chebyscheff–Psi-Funktion}
    \]

    \item \textbf{Liouville–Rekonstruktion (Zeta Nova Freesiana):}
    \[
    L(x) = \sum_{k=1}^\infty \beta(k) \cdot \frac{x^{2\rho_k}}{\rho_k \cdot \zeta'(\rho_k)} \Rightarrow \psi(x)
    \]

    \item \textbf{Euler–Freese-Wellenstruktur:}
    \[
    \lim_{n \to \infty} e^{i\pi \beta(n)} + 1 = 0 \Rightarrow \beta(n) \to \tfrac{2k+1}{2}
    \]

    \item \textbf{Fibonacci–Freese-Formel (FFF):}
    \[
    F(n) \sim \sum_{k=1}^{n} \beta(k) \cdot \Lambda(p_k)
    \]

    \item \textbf{Spurformel (Freese–Selberg analog):}
    \[
    \sum_{n} f(\beta(n)) = \int f(\lambda) \, \mathrm{d}\mu(\lambda) + \text{Spurreste}
    \]

    \item \textbf{Spektraloperator auf \( \mathbb{H} \):}
    \[
    \mathcal{H}_\beta := \text{diag}(\beta(1), \beta(2), \ldots)
    \]
\end{enumerate}

\end{document}