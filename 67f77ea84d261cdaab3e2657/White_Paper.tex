\documentclass[12pt]{article}
\usepackage[utf8]{inputenc}
\usepackage{amsmath, amssymb}
\usepackage{graphicx}
\usepackage{hyperref}
\usepackage{geometry}
\geometry{a4paper, margin=2.5cm}

\title{\textbf{Die Beta-Skala als spektraler Schlüssel zur Riemannschen Hypothese}}
\author{Freese, 2025}
\date{\today}

\begin{document}

\maketitle

\begin{abstract}
Dieses White Paper präsentiert einen neuartigen, operatorisch fundierten Zugang zur Riemannschen Hypothese über die sogenannte \textbf{Beta-Skala}. Basierend auf spektraler Analyse realer Nullstellen der Zetafunktion wird ein differenzierbarer Operator $H_\beta$ eingeführt, dessen Struktur sich als kohärent zu den Nullstellenverteilungen erweist. Die Beta-Skala fungiert dabei als spektrale Kodierung.
\end{abstract}

\section{Einleitung}

Die Riemannsche Hypothese (RH) ist eines der zentralen ungelösten Probleme der Mathematik. Ein vielversprechender Zugang beruht auf der Idee eines spektralen Operators, der mit der Nullstellenstruktur der Zetafunktion verknüpft ist (Hilbert–Pólya-Ansatz). In dieser Arbeit wird ein solcher Operator konstruiert und durch eine analytisch erzeugte \textbf{Beta-Skala} gesteuert.

\section{Definition der Beta-Skala}

Die \emph{Beta-Skala} ist definiert durch:

\begin{equation}
\beta(n) := C \cdot \cos(\omega \log n + \varphi) + \alpha \cdot n^\delta + \beta_0
\end{equation}

Die Parameter $C$, $\omega$, $\alpha$, $\delta$, $\varphi$ und $\beta_0$ werden durch Fits an reale Zeta-Nullstellen oder durch spektrale Rekonstruktionen bestimmt. Die Kohärenz dieser Skala zeigt sich in der hohen Korrelation mit den Zeta-Strukturen.

\section{Operator-Konstruktion}

Ein zentrales Objekt ist der modulierende Operator $H_\beta$:

\begin{equation}
H_\beta[\psi](n) = \beta(n) \cdot \psi(n)
\end{equation}

Die Ableitung ergibt:

\begin{equation}
\frac{d}{dn} H_\beta[\psi](n) = \left( \frac{C \omega \sin(\omega \log n + \varphi)}{n} + \frac{\alpha \delta n^\delta}{n} \right) \psi(n) + \beta(n) \cdot \psi'(n)
\end{equation}

\section{Verbindung zur Zeta-Welt}

Die Skala $\beta(n)$ ist direkt aus den imaginären Teilen der Zeta-Nullstellen rekonstruierbar. Ihre Struktur steht in engem Zusammenhang mit:

\begin{itemize}
  \item dem Spektrum von $H_\beta$
  \item der Tschebyschow-Funktion $\psi(x)$
  \item der Siegel–Theta-Funktion (über Mellin-Transformation)
\end{itemize}

\section{Numerische Bestätigung}

In numerischen Analysen zeigte sich für Fits auf reale Nullstellen:

\begin{itemize}
  \item Hohe Pearson-Korrelation (> 0.9999)
  \item Modulierbare Fehlerstruktur (MSE < $10^{-1}$ bei bis zu 2 Millionen Nullstellen)
  \item Spektrale Signatur (Frequenz 440 Hz identifizierbar)
\end{itemize}

\section{Ausblick}

Die Beta-Skala öffnet den Weg zu einer spektralen Reinterpretation der Riemannschen Hypothese. Ihre universelle Struktur legt die Existenz eines fundamentalen Invariants nahe, das über einen axiomatischen Operator gefasst werden kann.

\section*{Anhang: Symbolische Darstellung}

\begin{equation}
\boxed{
H_\beta[\psi](n) = 
\left(C \cos(\omega \log n + \varphi) + \alpha n^\delta + \beta_0 \right) \cdot \psi(n)
}
\end{equation}

\bigskip

\noindent Mit der Ableitung:

\begin{equation}
\boxed{
\frac{d}{dn} H_\beta[\psi](n) = 
\left(\frac{C \omega \sin(\omega \log n + \varphi)}{n} + \frac{\alpha \delta n^\delta}{n} \right) \psi(n) + \beta(n) \cdot \psi'(n)
}
\end{equation}

\vfill

\end{document}