\documentclass{article}
\usepackage{amsmath, amssymb}
\usepackage{amsfonts}
\usepackage{mathtools}
\usepackage{tcolorbox}
\usepackage{physics}

\begin{document}

\section*{Theorem: Kanonische Beta-Skala als Träger harmonischer Spektralstruktur}

Sei  
\[
\boxed{\beta(n) := A \cdot n^{\beta}} \quad \text{mit} \quad A = 0.003652, \quad \beta = -1.336085,
\]  
die kanonisch-normalisierte Beta-Skala, spektral extrahiert aus den Zeta-Nullstellen \( \rho_k \). Dann gilt:

\begin{enumerate}
    \item \textbf{Liouville-Rekonstruktion (Zeta Nova Freesiana, ZNF):}
    \[
    L(x) = \sum_{k=1}^{\infty} \beta(k) \cdot \frac{x^{2\rho_k}}{\rho_k \cdot \zeta'(\rho_k)} \quad \text{mit} \quad \rho_k = \tfrac{1}{2} + i\gamma_k
    \]

    \item \textbf{Operatorstruktur:}
    \[
    H_\beta \psi(n) := \beta(n) \cdot \psi(n) = A \cdot n^{\beta} \cdot \psi(n)
    \]

    \item \textbf{Kohärenzlängenstruktur:}
    \[
    \Delta \beta(n) = \beta(n+1) - \beta(n) \approx -\alpha \cdot n^{\beta - 1}
    \quad \Rightarrow \quad \Delta \beta(n) \in \mathcal{O}(n^{-2.336})
    \]

    \item \textbf{Euler--Freese-Spektralidentität:}
    \[
    \boxed{
    \lim_{n \to \infty} e^{i\pi \beta(n)} + 1 = 0
    }
    \quad \Rightarrow \quad \beta(n) \to \tfrac{2k+1}{2}
    \]

    \item \textbf{ZNF-Kohärenzbedingung:}
    \[
    \boxed{
    \psi(x) \sim \sum_{k=1}^n \beta(k) \cdot \frac{x^{2\rho_k}}{\rho_k \cdot \zeta'(\rho_k)} \to \psi(x)
    }
    \quad \text{für } n \to \infty
    \]
\end{enumerate}

\end{document}