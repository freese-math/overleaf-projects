\documentclass{article}
\usepackage{amsmath}
\usepackage{amssymb}
\usepackage{geometry}
\geometry{a4paper, margin=2.5cm}
\usepackage{lmodern}
\usepackage[T1]{fontenc}
\usepackage[utf8]{inputenc}

\title{Analytische Darstellung der Beta-Skala}
\author{Freese, 2025}
\date{}

\begin{document}

\maketitle

\section*{Analytische Form der Beta-Skala}

Die analytische Darstellung der spektralen Beta-Skala basiert auf einem harmonisch driftenden Modell mit logarithmischer Modulation. Die folgende Gleichung beschreibt $\beta(n)$ als Funktion der natürlichen Zahl $n$:

\subsection*{Fall $\delta \neq -1$}

\begin{equation}
\beta(n) = \beta_0 + \frac{C \cdot \omega}{\omega^2 + 1}
+ C \cdot \left( \frac{n \cdot \omega \cdot \cos(\omega \log n)}{\omega^2 + 1}
- \frac{n \cdot \sin(\omega \log n)}{\omega^2 + 1} \right)
+ \frac{\alpha}{\delta + 1} \left( n^{\delta + 1} - 1 \right)
\end{equation}

\subsection*{Fall $\delta = -1$}

\begin{equation}
\beta(n) = \beta_0 + \frac{C \cdot \omega}{\omega^2 + 1}
+ C \cdot \left( \frac{n \cdot \omega \cdot \cos(\omega \log n)}{\omega^2 + 1}
- \frac{n \cdot \sin(\omega \log n)}{\omega^2 + 1} \right)
+ \alpha \cdot \log(n)
\end{equation}

\bigskip

\noindent
\textbf{Legende:}
\begin{itemize}
  \item $\beta_0$: Initialwert
  \item $C$: Amplitude der Modulation
  \item $\omega$: Frequenz auf der logarithmischen Skala
  \item $\alpha$: Driftkoeffizient
  \item $\delta$: Exponent der Drift
\end{itemize}

\end{document}