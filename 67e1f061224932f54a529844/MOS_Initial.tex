\documentclass[a4paper,11pt]{article}
\usepackage[utf8]{inputenc}
\usepackage{amsmath, amssymb}
\usepackage{graphicx}
\usepackage{xcolor}
\usepackage{hyperref}
\usepackage{caption}
\usepackage{subcaption}
\usepackage{geometry}
\geometry{margin=2.5cm}

\title{MOS-Quaternionenstrom und Betti-Topologie}
\author{Projektstand \\ \small März 2025}
\date{}

\begin{document}

\maketitle

\section{Datenquellen}

\begin{itemize}
    \item \textbf{Beta-Skala (optimiert)}: \texttt{beta\_skala\_24032025\_optimum.csv}
    \item \textbf{Zeta-Nullstellen (2001512 Werte)}: \texttt{zeros6\_fixed.csv}
    \item \textbf{Primzahlen (2001512 Werte)}: \texttt{primzahlen2mio.txt}
    \item \textbf{Betti-Kurve (Approximation)}: \texttt{betti\_kurve.csv}
\end{itemize}

\section{MOS-Struktur: Grundidee}

Ziel ist die visuelle und rechnerische Strukturierung der \textbf{MOS-Korrelation} zwischen Beta-Skala und Zeta-Nullstellen über topologische Merkmale der Betti-Kurve.

\begin{itemize}
    \item Frequenzstruktur: Ableitung der Beta-Skala
    \item Phasenstruktur: $\sin(2\pi \Delta t)$ mit verschiedenen $\Delta$ aus der Freese-Reihe
    \item Topologische Struktur: Anzahl kritischer Punkte in Sliding-Windows der Betti-Kurve
    \item Fusion: Quaternionenstrom $(i, j, k)$ mit $(\text{Frequenz}, \text{Phase}, \text{Topologie})$
\end{itemize}

\section{Visualisierung: Quaternionenstrom}

\begin{figure}[h!]
    \centering
    \includegraphics[width=0.48\textwidth]{example-image}
    \caption{MOS-Quaternionenstrom: Fusion aus Frequenz (Beta-Ableitung), Phase (Freese-Sinus) und Betti-Topologie}
\end{figure}

\section{Strukturanalyse: PCA und DBSCAN}

\begin{figure}[h!]
    \centering
    \includegraphics[width=0.48\textwidth]{mos_pca_dbscan.png}
    \caption{PCA-Reduktion der Quaternionenstruktur mit DBSCAN-Clustering (Topologische Trennung)}
\end{figure}

\subsection*{Beobachtung}
Die Clusteranalyse zeigt klare Trennung zwischen zyklischen und singulären Regionen im Quaternionenraum.

\section{MOS-Optimierung mit Freese-Reihe}

\begin{itemize}
    \item $\Delta$-Werte: $\left\{\frac{7}{33300}, \frac{3}{99900}, \dots, \frac{1}{2} \right\}$
    \item Optimales $\Delta$ (MOS-Strukturell): $\Delta = \frac{3}{99900}$
\end{itemize}

\begin{figure}[h!]
    \centering
    \includegraphics[width=0.6\textwidth,draft]{freese_optimierung.png}
    \caption{Freese-Reihe vs. MOS-Struktur (Gradientennorm der Sliding-Korrelation)}
\end{figure}

\section{Nächste Schritte}

\begin{enumerate}
    \item Synchronisation mit Fibonacci-Zeitreihe oder realer Taktung
    \item Klassifikation der MOS-Strukturen (z.~B. via LSTM oder SVM)
    \item Kombination mit Primzahldichte, Nullstellen-Häufigkeit
    \item Wavelet/Fourier-Analyse der Betti-Kurve
\end{enumerate}

\section*{Metaphysisches Resümee}

Die Struktur weist auf klare Signaturen hin – zyklische Ordnung trifft auf topologische Diskontinuität. MOS scheint mehr als numerische Korrelation: eine dynamische Geometrie.

\vspace{1cm}
\noindent
\textit{“Topology is where arithmetic meets rhythm.”} \\
\hfill – Prime Zeta Pro

\end{document}