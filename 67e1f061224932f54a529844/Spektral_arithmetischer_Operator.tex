\documentclass[12pt]{article}
\usepackage{amsmath, amssymb, amsfonts, physics, geometry, hyperref, graphicx}
\geometry{a4paper, margin=2.5cm}
\usepackage{lmodern}
\usepackage{color}
\usepackage{bm}
\usepackage{caption}
\usepackage{mathtools}
\usepackage{listings}
\usepackage{tikz}
\usepackage{fancyhdr}
\pagestyle{fancy}
\fancyhead[L]{\textit{T. H. Freese}}
\fancyhead[R]{\textit{März 2025}}

\title{\textbf{Spektral-arithmetischer Operatoransatz zur Riemannschen Hypothese}}
\author{Tim Hendrik Freese}
\date{}

\begin{document}
\maketitle

\section*{1. Ziel}
Wir untersuchen, ob sich die nicht-trivialen Nullstellen der Riemannschen Zetafunktion \(\zeta(s)\) als Eigenwerte eines geeigneten, selbstadjungierten Operators \(\mathcal{D}_\mu\) beschreiben lassen. Dies folgt dem spektralen Zugang zur Riemannschen Hypothese.

\section*{2. Der Operator \(\mathcal{D}_\mu\)}

\subsection*{Definition}
Sei \(\mathcal{D}_\mu : \ell^2(\mathbb{N}) \rightarrow \ell^2(\mathbb{N})\) ein Operator der Form:
\[
(\mathcal{D}_\mu f)(n) = \sum_{k=1}^{K} w_k \cdot \exp\left( -\frac{(n - \gamma_k)^2}{2\sigma^2} \right) f(n) + \lambda \sum_{d=1}^{n-1} \mu(d) \left\lfloor \frac{n}{d} \right\rfloor f(n)
\]
mit
\begin{itemize}
  \item \(\rho_k = \frac{1}{2} + i\gamma_k\) den nicht-trivialen Nullstellen von \(\zeta(s)\),
  \item \(w_k = \frac{1}{|\zeta'(\rho_k)|}\) als spektrales Gewicht,
  \item \(\mu(d)\) der Möbius-Funktion,
  \item \(\lambda\), \(\sigma\) Regularisierungsparameter.
\end{itemize}

\subsection*{Hypothese}
\begin{quote}
Die Riemannsche Hypothese gilt genau dann, wenn \(\mathcal{D}_\mu\) selbstadjungiert ist.
\end{quote}

\section*{3. Vergleich mit weiteren Operatoren}

\subsection*{(a) Hamilton-Operator \( H = -\frac{d^2}{dx^2} + V(x) \)}
Mit geeignetem Potential \(V(x)\), das auf \(\log n\), \(\beta(n)\) oder Siegel-Theta basiert, lässt sich das Spektrum von \(H\) mit den \(\gamma_k\) korrelieren:
\[
V(q) = \alpha \cdot \log q + \beta(q)
\]
Die numerische Korrelation mit den echten Nullstellen beträgt bis zu \(r \approx 0.999\).

\subsection*{(b) Viererimpuls-Konstruktion}
Analog zur relativistischen Energie:
\[
m_n^2 = E_n^2 - p_n^2
\]
mit \(E_n \sim \zeta(n)\), \(p_n \sim \log n\), ergibt sich eine massenartige Skala, die die Zeta-Nullstellen reproduziert.

\subsection*{(c) Dirac-Operator \(\mathcal{D}_{\text{Dirac}}\)}
Ein komplexer, hermitescher Operator auf \(\mathbb{C}^{2N}\), aufgebaut aus normierten Impulsanteilen:
\[
\mathcal{D} = \begin{pmatrix}
0 & m_n - i p_n \\
m_n + i p_n & 0
\end{pmatrix}
\]
Sein Spektrum zeigt komplexe Konjugation und liefert Eigenwerte, die stark mit den Zeta-Nullstellen korrelieren.

\section*{4. Beta-Skala \(\beta(n)\) und Struktur}
\begin{itemize}
  \item \(\beta(n)\) wurde empirisch als strukturtragende Korrektur gefunden.
  \item Aus \(\Theta(t)\)-Funktion rekonstruierbar.
  \item Führt zu Potenzialen \(V(x)\), die die Spektralstruktur erklären.
\end{itemize}

\section*{5. Numerischer Test}
Vergleich verschiedener Operatoren mit realen Zeta-Nullstellen (z.~B. Odlyzko-Datensatz):
\begin{itemize}
  \item Korrelation (Dirac-Modell vs. Zeta): \(r \approx 0.9886\)
  \item Fehler in der Euler-Freese-Identität: \(\approx 0.085\)
  \item FFT-Spektrum: dominante Frequenz bei \(f = \pm 0.001\)
\end{itemize}

\section*{6. Ausblick}
\begin{itemize}
  \item Formaler Beweis der Selbstadjungiertheit von \(\mathcal{D}_\mu\)
  \item Verbindung zur Random-Matrix-Theorie (GOE-Statistik)
  \item Anwendung auf Primzahlstruktur (Hardy-Littlewood-Folge?)
  \item Symbolische Ableitung von \(\beta(n)\) aus \(\Theta(t)\)
\end{itemize}

\section*{7. Anhang: Python-Test (Numerik)}
\vspace{-0.7em}
\begin{verbatim}
# Eigenwerte & Selbstadjungiertheit prüfen
inner1 = np.dot(Dmu_f, g)
inner2 = np.dot(f, Dmu_g)
diff = abs(inner1 - inner2)
\end{verbatim}

\end{document}