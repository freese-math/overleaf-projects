\subsection*{MOS-Phasenraumorbit: Zustandsbahn im Beta-Raum}

Die folgende Darstellung zeigt die explizite Bahn im Operator-Phasenraum, definiert durch die Koordinate $q = \log n$ und die konjugierte Größe $p = \beta(n)$. Diese Bahn ergibt sich direkt aus der Beta-Korrekturfunktion der MOS-Operatorstruktur:

\[
\beta(n) = A \cdot n^B + C + \varepsilon(n)
\]

wobei $\varepsilon(n)$ eine modulierte Korrektur repräsentiert. Der entstehende Orbit im Phasenraum $(q, p)$ beschreibt eine zeta-induzierte Zustandsentwicklung mit stetiger Skalenmodulation.

\begin{center}
\includegraphics[width=0.9\textwidth]{mos_phasenbahn.png}
\end{center}

\paragraph{Beobachtung:} Die Phasenbahn verläuft glatt und konvex. Die Monotonie von $\beta(n)$ entlang $\log n$ zeigt, dass die Skalenstruktur stabil unter Zustandsentwicklung bleibt und keine Rückläufer (nichtinvertierbare Operatorabschnitte) auftreten. Resonanzen oder nichtlineare Knoten würden sich als Schleifen oder Verzweigungen zeigen — im gezeigten Bereich nicht vorhanden.e