\subsection*{Topologische Phasenraumdarstellung des MOS-Operators mittels Quaternionenstromanalyse}

Zur Beschreibung der internen Dynamik des MOS-Operators wurde eine Phasenraumanalyse auf Basis von Quaternionen durchgeführt. Dabei wird der Zustandsraum um eine topologische Dimension erweitert, sodass die Quaternionenstruktur $(q = a + bi + cj + dk)$ als Träger der Frequenz-, Phasen- und Topologiedaten fungiert.

\paragraph{Interpretation der Komponenten:}
\begin{itemize}
    \item $i$-Komponente: kodiert die modulierte Frequenzstruktur (z.\,B. Ableitung von $\log n$)
    \item $j$-Komponente: beschreibt die Phasenlage, insbesondere durch $\sin(\omega \log n + \varphi)$
    \item $k$-Komponente: enthält topologische Information, z.\,B. Betti-Zahlen, Windungszahlen oder Cluster-Indizes
\end{itemize}

\paragraph{Analyseverfahren:}
\begin{enumerate}
    \item Quaternionenströme werden aus den Beta-korrigierten Operatorbahnen generiert.
    \item Diese werden per PCA (Hauptkomponentenanalyse) in den \emph{dynamischen Zustandshauptraum} projiziert.
    \item Anschließend erfolgt eine Clusterdetektion über das DBSCAN-Verfahren, um Orbittypen zu extrahieren.
\end{enumerate}

\paragraph{Beobachtungen:}
\begin{itemize}
    \item Elliptische oder helikale Bahnen im $(i,j,k)$-Raum zeigen stabilisierte Eigenmoden des Operators.
    \item Lineare Pfade deuten auf gerichtete Spektralflüsse hin.
    \item Isolierte Punkte im PCA/DBSCAN-Bild entsprechen \emph{spektralen Ausschlusszonen} – z.\,B. bei Resonanzfrequenzen wie $\frac{1}{33}$.
\end{itemize}

\begin{figure}[h!]
\centering
\includegraphics[width=0.45\textwidth]{mos_quaternionenstrom_3d.png}
\includegraphics[width=0.45\textwidth]{mos_quaternionen_pca_clusters.png}
\caption{Links: Quaternionischer Strom im MOS-System (Fusion: Frequenz + Phase + Topologie). \\
Rechts: PCA-Analyse mit DBSCAN-Clustering der Operatorbahnen im Quaternionenraum.}
\end{figure}

\paragraph{Schlussfolgerung:}
Die Quaternionenstromanalyse eröffnet eine topologisch fundierte Perspektive auf die Operatorstruktur der Zeta-Nullstellen. Die $k$-Komponente fungiert dabei als \emph{topologische Ladung}, deren Stabilität mit der selbstadjungierten Symmetrie der MOS-Komponenten korreliert.