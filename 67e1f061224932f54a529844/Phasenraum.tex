\subsection*{Phasenraumstruktur des MOS-Operators}

Analog zur klassischen Mechanik lässt sich die Dynamik des MOS-Operators als Bewegung im Phasenraum beschreiben. Dabei entspricht der log-skalierte Zustandsindex $q = \log n$ einer generalisierten Koordinate, und die modulierte Beta-Korrekturfunktion $\beta(n)$ einer konjugierten Größe $p$.

\paragraph{Definition:} Der MOS-Phasenraum ist der Raum der Punkte
\[
(q_n, p_n) = (\log n, \beta(n)) \in \mathbb{R}^2
\]
wobei $\beta(n)$ durch eine harmonisch modulierte Freese-Funktion gegeben ist:
\[
\beta(n) = A n^B + C + \varepsilon \cdot \sin(\omega \log n + \varphi)
\]

\paragraph{Beobachtung:} Die resultierende Trajektorie im $(q, p)$-Raum beschreibt eine elliptische, spiralartige Bahn, ähnlich der Bewegung eines quantisierten harmonischen Oszillators. Dabei ist die Frequenzmodulation $\varepsilon \ll 1$ entscheidend für die fraktale Feinstruktur des Spektrums.

\begin{center}
\includegraphics[width=0.8\textwidth]{mos_phasenraum.png}
\end{center}

\paragraph{Interpretation:} Die Fermat-Spirale der Zeta-Nullstellen ist eine Projektion dieser Phasenraumkurve auf die komplexe Ebene. Resonante Frequenzen (z. B. $1/33$) erzeugen Abweichungen von der stabilen Ellipsenbahn und markieren spektrale Ausschlusszonen im dynamischen MOS-System.