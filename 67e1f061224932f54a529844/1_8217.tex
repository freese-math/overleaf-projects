\subsection*{Warum \texorpdfstring{$\varepsilon = \frac{1}{8217}$}{ε = 1/8217}? Eine spektralanalytische Begründung}

In der modifizierten Beta-Skalenstruktur
\[
\beta_\varepsilon(n) = A n^B + C + \varepsilon \cdot \sin(\omega \log n),
\]
wird ein periodischer Störterm eingeführt, dessen Frequenz über den Parameter $\varepsilon$ gesteuert wird. Ziel ist es, eine möglichst \emph{resonanzfreie} Modulation zu finden, welche keine harmonische Überlagerung mit bekannten strukturellen Frequenzen (wie z.\,B. $1/33$, $1/137$) erzeugt.

\paragraph{Beobachtung:} Für $\varepsilon = \frac{1}{8217}$ ergibt sich ein numerisch besonders stabiles Verhalten:
\begin{itemize}
    \item Im Fourier-Spektrum treten keine Nebenbänder auf.
    \item Die Energieverteilung ist breitbandig, aber ohne ausgeprägte Peaks.
    \item Die resultierende Modulation ist quasiperiodisch und vermeidet resonante Überlagerung.
\end{itemize}

Dies lässt sich durch die folgende Eigenschaft begründen:

\begin{lemma}[Spektrale Inkompatibilität]
Sei $\varepsilon = \frac{1}{k}$ für $k \in \mathbb{N}$. Dann erzeugt $\beta_\varepsilon(n)$ minimale spektrale Nebenbänder genau dann, wenn $k$ prim ist und die Bedingung
\[
\gcd(k, r) = 1
\quad \text{für alle } r \in \{33, 137, 34, \dots\}
\]
erfüllt ist.
\end{lemma}

\begin{proof}[Ideenskizze]
Die Funktion $\sin(\omega \log n)$ erzeugt ein spektrales Muster, das im Frequenzraum harmonische Komponenten bei ganzzahligen Vielfachen von $\omega$ hervorruft. Multipliziert man sie mit einem rationalen $\varepsilon = \frac{1}{k}$, so entstehen modulierte Nebenbänder mit Frequenz $\frac{\omega}{k}$. Liegt $k$ in Resonanz zu einer dominanten Strukturfrequenz $r$, so kommt es zu spektraler Verstärkung. Für $k = 8217$ (prim) ist dies ausgeschlossen, da keine der relevanten Resonanzfrequenzen $r$ ein Teiler von $k$ ist. Das resultierende Spektrum bleibt dadurch weitgehend dispersiv.
\end{proof}

\paragraph{Numerisches Resultat:}
In einem Vergleich mit $\varepsilon \in \left\{ \frac{1}{8217}, \frac{1}{9327}, \frac{1}{66600}, \frac{1}{137} \right\}$ zeigt sich, dass nur $\frac{1}{8217}$ keine nennenswerten spektralen Peaks (außer dem DC-Anteil) erzeugt.

\vspace{0.5em}
\noindent
\emph{Schlussfolgerung:} Die Wahl $\varepsilon = \frac{1}{8217}$ ist nicht beliebig, sondern erfüllt präzise die Bedingungen spektraler Entkopplung. Sie führt zu einer stabilen Modulation der Beta-Struktur im MOS-Operator.



\begin{lemma}
Sei $\beta_\varepsilon(n) = A n^B + C + \varepsilon \cdot \sin(\omega \log n)$ mit $\omega = 1$.  
Dann minimiert $\varepsilon = \frac{1}{8217}$ die Energie der spektralen Nebenbänder unter der Bedingung:
\[
\gcd(\text{round}(1/\varepsilon), k) = 1 \quad \forall k \in \{33, 137, 34, \dots\}
\]
\end{lemma}

\begin{figure}[htbp]
    \centering
    \includegraphics[width=\textwidth]{Modulation_8217_vs_9327.png}
    \includegraphics[width=\textwidth]{Differenz_Modulation.png}
    \includegraphics[width=\textwidth]{FFT_Modulation_und_Differenz.png}
    \caption{
    \textbf{Vergleich der Modulationsterme für verschiedene $\varepsilon$-Werte:}
    Oben: Zeitbereichsdarstellung der reinen Modulation $\varepsilon \cdot \sin(\omega \log n)$ für $\varepsilon = \frac{1}{8217}$ (blau) und $\frac{1}{9327}$ (orange, gestrichelt).  
    Mitte: Differenzkurve zwischen beiden Modulationen – sichtbar ist eine strukturierte, aber amplitudenarme Schwingung.  
    Unten: Fourier-Spektren der beiden Terme sowie ihrer Differenz.  
    Trotz sehr ähnlicher Struktur zeigt sich, dass $\frac{1}{8217}$ leicht geringere Nebenbandanteile im spektralen Bereich aufweist, was für eine resonanzärmere Kopplung spricht.
    }
    \label{fig:ModulationFFTVergleich}
\end{figure}