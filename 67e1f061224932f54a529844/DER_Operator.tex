\documentclass[12pt]{article}
\usepackage{amsmath, amssymb, amsfonts}
\usepackage{physics}
\usepackage{graphicx}
\usepackage{bm}
\usepackage{geometry}
\usepackage{tikz}
\usepackage{hyperref}
\geometry{margin=2.5cm}

\title{Der MOS-Master-Operator zur Analyse zeta-naher Strukturen}
\author{---}
\date{}

\begin{document}

\maketitle

\section*{1. Ausgangspunkt: Spektrale Interpretation}

Ausgehend von der Idee, dass die Nullstellen der Riemannschen Zetafunktion eine spektrale Struktur tragen, wird ein selbstadjungierter Operator konstruiert, dessen Eigenwerte in enger Beziehung zur kritischen Linie stehen.

\section*{2. Definition des Master Operators}

\subsection*{2.1 Grundstruktur}

Der Operator ist formal definiert als:

\[
\boxed{
\hat{H}_{\text{MOS}} = -\frac{d^2}{dx^2} + V_{\beta}(x)
}
\]

mit einem skalenmodulierten Potenzial:

\[
V_{\beta}(x) = \frac{A}{1 + e^{-B(x - C)}} + D \cdot \sin(\omega x) + \varepsilon \cdot \sin(\Omega \cdot \log(x))
\]

\subsection*{2.2 Skalenstruktur: Beta-Skala}

Die zugrunde liegende Frequenzstruktur ist in der sogenannten Beta-Skala kodiert:

\[
\beta(n) = A \cdot n^B + C + \sum_{k=2}^K \frac{a_k}{n^k}
\]

Dabei ist das Restglied

\[
\varepsilon(n) := 1 - \beta(n)
\]

interpretierbar als kohärenter Korrekturterm, z.\,B. \(\varepsilon \approx \frac{1}{8217}\).

\subsection*{2.3 Eigenwertproblem}

Das zugehörige Eigenwertproblem lautet:

\[
\hat{H}_{\text{MOS}} \psi_n(x) = \lambda_n \psi_n(x)
\]

wobei die \(\lambda_n\) im Idealfall mit den Ordinaten \(\gamma_n\) der nichttrivialen Nullstellen der Zetafunktion korrelieren:

\[
\zeta\left(\frac{1}{2} + i \gamma_n\right) = 0
\]

\section*{3. Interpretation}

\begin{itemize}
  \item \textbf{Beta-Skala} erzeugt fraktale Frequenzstruktur im Raum der natürlichen Zahlen.
  \item \textbf{Sinus- und Log-Term} wirken als spektrale Modulatoren auf verschiedenen Skalenebenen.
  \item \textbf{Operatorstruktur} erlaubt spektrale Analyse über Eigenwertvergleich zu Zeta-Nullstellen.
  \item \textbf{Kohärenz} entsteht durch asymptotische Normierung:
  \[
  \sum_{n=1}^{\infty} \beta(n) + \varepsilon = 1
  \]
\end{itemize}

\section*{4. Geometrisches Bild}

Die Struktur wirkt wie ein inverses Prisma:  
Nicht Zerstreuung, sondern Bündelung entlang der kritischen Linie.  
Die Beta-Korrektur wirkt dabei als harmonische Einpassung in einen diskreten Resonatorraum.

\section*{5. Nächste Schritte}

\begin{itemize}
  \item Vergleich der Spektren \(\lambda_n\) mit den \(\gamma_n\) der Zeta-Funktion.
  \item Operatorausweitung auf verallgemeinerte L-Funktionen.
  \item Analyse von Ausschlusszonen und Resonanzspitzen.
  \item Verbindung zur GUE-Statistik und zu universellen Skalenkonstanten.
\end{itemize}

\end{document}