\documentclass[12pt]{article}
\usepackage{amsmath, amssymb, geometry}
\geometry{a4paper, margin=2.5cm}
\title{Beta-Skala, Spektrale Operatoren und die Zeta Nova}
\author{[Dein Name]}
\date{}

\begin{document}
\maketitle

\section*{Abstract}

Ziel dieser Arbeit ist die Konstruktion einer spektralen Skala $\beta(n)$, deren Struktur sowohl arithmetisch als auch harmonisch fundiert ist. Die Skala basiert auf einem Driftterm kombiniert mit dominanten Frequenzkomponenten und bildet asymptotisch die Logarithmen der Primzahlen sowie die Ordinaten der Nullstellen der Riemannschen Zetafunktion ab. Ausgehend von dieser Skala wird eine zugehörige Dirichletreihe $L_\beta(s) = \sum \beta(n)/n^s$ konstruiert, die als spektral generierte Zeta-Analogie interpretiert werden kann.

Die Analyse umfasst:
\begin{itemize}
  \item die numerische Rekonstruktion der Beta-Skala,
  \item die spektrale Interpretation via Hamiltonoperatoren,
  \item die Fourier-Analyse der Frequenzstruktur,
  \item und die Konstruktion einer L-Funktion aus $\beta(n)$.
\end{itemize}

Im Fokus steht die Frage, ob $L_\beta(s)$ zetaähnliche Eigenschaften aufweist, insbesondere hinsichtlich Funktionalgleichungen und Nullstellenstruktur. Damit verbindet die Arbeit numerische Spektralanalyse mit modularen Konzepten aus der Theorie der Hecke-Operatoren, Theta-Funktionen und automorphen Formen.

\section*{Zielsetzung}

\begin{enumerate}
  \item Konstruktion einer spektral harmonischen Skala $\beta(n)$, die arithmetische Strukturen abbildet.
  \item Definition eines Operators $\hat{H}_\beta$ mit Eigenwerten $\beta(n)$ (Hamiltonstruktur).
  \item Aufbau der zugehörigen L-Funktion $L_\beta(s) = \sum \beta(n)/n^s$.
  \item Vergleich von $L_\beta(s)$ mit klassischen L-Funktionen (z.\,B. $\zeta(s)$, Hecke).
  \item Untersuchung der Frage, ob $L_\beta(s)$ eine neue Klasse spektraler L-Funktionen begründet.
\end{enumerate}

\end{document}