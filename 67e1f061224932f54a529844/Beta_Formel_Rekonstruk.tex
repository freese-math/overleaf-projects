\usepackage{amsmath, amssymb, tcolorbox}
\tcbuselibrary{skins, breakable}

\begin{tcolorbox}[
  title=Beta-Skala: Frequenzbasierte Struktur,
  colback=white!95!blue!2,
  colframe=blue!40,
  coltitle=black,
  fonttitle=\bfseries,
  enhanced,
  breakable
]

Die rekonstruierte Struktur basiert auf einem spektral harmonisierten Ausdruck:

\[
\boxed{
\beta(n) = \frac{A}{n^p} + \sum_{k=1}^{K} A_k \cdot \sin(2\pi f_k n)
}
\]

\vspace{0.5em}

Mit dominanten Frequenzen \(f_k\) und gewichteten Amplituden \(A_k\) stellt \(\beta(n)\) ein deterministisches, reproduzierbares Oszillationsmodell dar.

\vspace{1em}

\textbf{Kumulative Struktur:}

Die aufsummierte Skala nähert asymptotisch die Primzahllogarithmen an:

\[
\boxed{
\sum_{k=1}^{n} \beta(k) \approx \log(p_n) + \varepsilon
\quad \text{mit} \quad \varepsilon \approx 15{,}88
}
\]

\vspace{1em}

Dies zeigt: \(\beta(n)\) trägt die spektrale Signatur des Primzahlraums.

\end{tcolorbox}