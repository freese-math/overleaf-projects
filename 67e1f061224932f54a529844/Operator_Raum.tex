\documentclass[12pt]{article}
\usepackage{amsmath, amssymb, physics, geometry, hyperref, graphicx, listings}
\geometry{a4paper, margin=2.5cm}
\usepackage{lmodern}
\usepackage{xcolor}

\title{Spektral-arithmetischer Operatoransatz zur Riemannschen Hypothese}
\author{Tim Hendrik Freese}
\date{März 2025}

\begin{document}
\maketitle

\section*{1. Operatoransatz \boldmath$\mathcal{D}_\mu$}

Sei \(\mathcal{D}_\mu\) ein Operator mit Wirkung auf \(f : \mathbb{N} \to \mathbb{R}\), definiert durch:

\begin{equation}
(\mathcal{D}_\mu f)(n) = \sum_{k=1}^{K} w_k \cdot \delta_\sigma(n - \gamma_k) \cdot f(n) + \lambda \sum_{d=1}^{n-1} \mu(d) \left\lfloor \frac{n}{d} \right\rfloor f(n)
\end{equation}

\textbf{Notation:}
\begin{itemize}
  \item \(\rho_k = \sigma_k + i \gamma_k\): nicht-triviale Nullstellen der Riemannschen Zeta-Funktion
  \item \(w_k = \frac{1}{|\zeta'(\rho_k)|}\): spektrales Gewicht
  \item \(\delta_\sigma(n - \gamma_k) = \exp\left( -\frac{(n - \gamma_k)^2}{2 \sigma^2} \right)\): gaußförmige Approximation von \(\delta\)
  \item \(\mu(d)\): Möbius-Funktion
  \item \(\lambda\): Rückkopplungsfaktor
\end{itemize}

\vspace{1em}

\section*{2. Hypothese (informell)}

Die Riemannsche Hypothese (RH) ist äquivalent zur Selbstadjungiertheit von \(\mathcal{D}_\mu\) auf \(\ell^2(\mathbb{N})\), d.\,h.:

\begin{quote}
\textbf{Satz (RH-Selbstadjungiertheit):} \\
RH gilt genau dann, wenn \(\mathcal{D}_\mu = \mathcal{D}_\mu^\dagger\) auf einem geeigneten Hilbertraum.
\end{quote}

\vspace{1em}

\section*{3. Numerischer Test (Python)}

\noindent Test der Adjungiertheit von \(\mathcal{D}_\mu\) durch Vergleich der Skalarprodukte:

\begin{lstlisting}[language=Python, basicstyle=\ttfamily\footnotesize, breaklines=true, backgroundcolor=\color{gray!10}]
import numpy as np
import matplotlib.pyplot as plt
from mpmath import zetazero, diff, zeta, mp
from sympy.ntheory import mobius

mp.dps = 50
N = 1000
K = 50
epsilon = 0.0
sigma = 0.5
lambda_mu = 0.05

gammas = [float(zetazero(k).imag) for k in range(1, K + 1)]
sigmas = [0.5 + epsilon for _ in range(K)]
weights = [1.0 / abs(diff(zeta, complex(s, g))) for s, g in zip(sigmas, gammas)]
weights = [w / sum(weights) for w in weights]

n_vals = np.arange(N)
f = np.sin(0.01 * np.pi * n_vals)
g = np.cos(0.01 * np.pi * n_vals)

Dmu_f = np.zeros_like(f)
for w, gamma in zip(weights, gammas):
    for n in range(N):
        delta = np.exp(-((n - gamma)**2) / (2 * sigma**2))
        Dmu_f[n] += w * delta * f[n]

for n in range(2, N):
    mobius_sum = sum([mobius(d) * (n // d) for d in range(1, n)])
    Dmu_f[n] += lambda_mu * mobius_sum * f[n]

inner1 = np.dot(Dmu_f, g)
inner2 = np.dot(f, Dmu_f)
print(f"⟨Dµ f, g⟩ = {inner1}")
print(f"⟨f, Dµ g⟩ = {inner2}")
print(f"Differenz = {abs(inner1 - inner2)}")
\end{lstlisting}

\vspace{1em}

\section*{4. Ziel der Zusammenarbeit}

\begin{itemize}
  \item Exakte Definition von \(\mathcal{D}_\mu\) als Operator auf \(\ell^2\) oder geeignetem Raum
  \item Formale Untersuchung der Selbstadjungiertheit
  \item Herleitung eines RH-äquivalenten Satzes mit vollständigem Beweis
\end{itemize}

\section*{5. Kontakt}
Tim Hendrik Freese\\
\texttt{[E-Mail-Adresse einsetzen]}\\
\texttt{[ggf. GitHub / arXiv-Link]}

\end{document}