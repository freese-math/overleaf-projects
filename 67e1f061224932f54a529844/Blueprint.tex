\documentclass[12pt]{article}
\usepackage{amsmath, amssymb, geometry, graphicx}
\geometry{a4paper, margin=2.5cm}

\title{Rekonstruktion der Beta-Skala\\
\large Eine spektrale Approximation von Primzahlen und Zeta-Nullstellen}
\author{[Dein Name]}
\date{}

\begin{document}

\maketitle

\section*{1. Ziel}
Ziel ist die Konstruktion einer vollständig deterministischen, spektralen Skala $\beta(n)$, deren kumulative Struktur asymptotisch die Primzahllogarithmen $\log(p_n)$ abbildet – und damit indirekt auch die ordinaten $\gamma_n$ der nichttrivialen Nullstellen der Riemannschen Zetafunktion.

\section*{2. Struktur der Beta-Skala}

Die rekonstruierte Skala $\beta(n)$ besteht aus einem Drift-Term und überlagerten Frequenzmodulationen:

\[
\boxed{
\beta(n) = \text{Drift}(n) + \sum_{k=1}^{K} A_k \cdot \sin(2\pi f_k \cdot n + \phi_k)
}
\]

mit:
\begin{itemize}
    \item $\text{Drift}(n) \approx \dfrac{A}{n^p}$ oder logarithmisch
    \item $f_k$: bekannte Frequenzen
    \item $A_k$: Amplituden (rekonstruiert oder gefittet)
    \item $\phi_k$: Phasen (approximiert oder numerisch bestimmt)
\end{itemize}

\section*{3. Kumulative Näherung}

Die kumulative Summe der Skala nähert die Logarithmen der Primzahlen bis auf eine Konstante:

\[
\boxed{
\sum_{k=1}^{n} \beta(k) \approx \log(p_n) + C
}
\qquad \text{mit } C \approx 15.88
\]

Diese Konstante ergibt sich aus dem Mittelwert des Residuums:

\[
r(n) = \log(p_n) - \sum_{k=1}^n \beta(k)
\]

\section*{4. Verbindung zur Zeta-Funktion}

Da empirisch gilt:

\[
\gamma_n \approx \log(p_n) + \mu
\quad \Rightarrow \quad
\gamma_n \approx \sum_{k=1}^{n} \beta(k) + \mu'
\]

mit $\mu' = \mu - C$, liefert $\beta(n)$ eine spektrale Approximation der Zeta-Geometrie.

\section*{5. Ausblick}

Mögliche Weiterentwicklungen:
\begin{itemize}
    \item Operatorform: $H \psi_n = \beta(n) \psi_n$
    \item Verallgemeinerung auf L-Funktionen
    \item Ableitung eines analytischen Ausdrucks für $\phi_k$
\end{itemize}

\end{document}