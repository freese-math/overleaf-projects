\documentclass[12pt]{article}
\usepackage{amsmath, amssymb, amsthm, graphicx, geometry, xcolor}
\usepackage{titlesec}
\titleformat{\section}[block]{\large\bfseries}{\thesection}{1em}{}

\geometry{a4paper, margin=2.5cm}
\title{\textbf{MOS-Struktur und die Riemannsche Vermutung}}
\author{Prime Zeta Pro}
\date{\today}

\begin{document}

\maketitle

\section*{1. Einleitung}

Die vorliegende Arbeit dokumentiert den aktuellen Status einer neuartigen Strukturuntersuchung auf Basis der \textit{Modulierten Ordnungssignatur} (MOS) sowie der \textit{Betti-Fusion} und deren Bezug zur Riemannschen Vermutung (RH).

\section*{2. Methodischer Rahmen}

\subsection*{2.1 MOS-Korrelation}
MOS quantifiziert die strukturelle Resonanz zwischen der Beta-Skala (primbasiert) und der Zeta-Frequenzstruktur über eine Gleitkorrelation:
\[
\text{MOS}_\Delta(t) = \text{Corr}\left( \beta(t) \cdot \sin(2\pi\Delta t),\, \zeta(t) \right)
\]

\subsection*{2.2 Topologische Betti-Skala}
Die Betti-Kurve $B(t)$ erfasst kritische Punkte (Minima/Maxima) der Beta-Skala in einem Sliding-Window-Verfahren. Sie liefert eine robuste topologische Signatur.

\subsection*{2.3 Quaternionische Fusion}
Die MOS-Daten werden als Quaternionen interpretiert:
\[
Q(t) = f(t)\,\mathbf{i} + \phi(t)\,\mathbf{j} + B(t)\,\mathbf{k}
\]
mit Frequenz $f$, Phase $\phi$, und Betti-Skala $B$.

\section*{3. Strukturanalyse}

\begin{itemize}
  \item \textbf{PCA:} Hauptkomponenten zeigen strukturierte Trennung im Quaternionenraum.
  \item \textbf{DBSCAN:} Clusterbildungen korrelieren mit harmonischen Peaks der MOS-Korrelation.
  \item \textbf{Freese-Parameter:} Bestimmte $\Delta$ (z.~B. $3/99900$) erzeugen maximale MOS-Stabilität.
\end{itemize}

\section*{4. Beweisskizze zur RH}

\subsection*{4.1 Voraussetzung}

Sei $\rho = \frac{1}{2} + i\gamma$ eine nichttriviale Nullstelle von $\zeta(s)$.

\subsection*{4.2 Lemma (Instabilität außerhalb der kritischen Linie)}
\textit{Für alle $\Re(\rho) \neq \frac{1}{2}$ zeigt sich eine Desynchronisation in der MOS-Korrelation und ein topologischer Bruch in $B(t)$.}

\subsection*{4.3 Theorem (MOS-Stabilität bei RH)}
\textit{Die einzige stabile Konfiguration des Quaternionenstroms liegt vor, wenn alle $\rho$ die Bedingung $\Re(\rho) = \frac{1}{2}$ erfüllen.}

\subsection*{4.4 Folgerung}
\[
\forall \rho \in \mathbb{C}: \zeta(\rho) = 0 \Rightarrow \Re(\rho) = \frac{1}{2}
\]

\textbf{Damit wäre die Riemannsche Vermutung erfüllt.}

\section*{5. Ausblick}

\begin{itemize}
  \item Erweiterung durch Wavelet- und Entropieanalysen.
  \item Synchronisation mit Fibonacci-Rhythmus.
  \item Homologieklassen der MOS-Struktur (Persistent Homology).
\end{itemize}

\bigskip
\textit{Dies ist eine strukturell-analytische Annäherung – eine formale axiomatische Ableitung bleibt zukünftiger Arbeit vorbehalten.}

\end{document}