\documentclass[12pt]{article}
\usepackage[utf8]{inputenc}
\usepackage{amsmath, amssymb}
\usepackage{graphicx}
\usepackage{hyperref}
\usepackage{geometry}
\geometry{a4paper, margin=2.5cm}

\title{Die Beta-Skala als Fundamentalinvariant der Spektralstruktur der Zetawelt}
\author{[Dein Name] \\ \small{\textit{freese-math research project}}}
\date{\today}

\begin{document}
\maketitle

\begin{abstract}
Die vorliegende Arbeit entwickelt eine analytisch konstruierte, spektral motivierte \(\beta\)-Skala auf Basis der Riemannschen Nullstellen und Primzahlen. Es wird gezeigt, dass diese Skala als Fundamentalinvariant in einer Operatorformulierung erscheint, mit direkten Bezügen zur Quantisierungsstruktur à la Connes. Damit wird die \(\beta(n)\)-Funktion zu einem zentralen Bindeglied zwischen Spektraltheorie, Zahlentheorie und der Riemannschen Vermutung.
\end{abstract}

\tableofcontents

\section{Einleitung}
\begin{itemize}
    \item Motivation: Riemannsche Hypothese und spektrale Interpretation
    \item Ziel: Herleitung und Interpretation der \(\beta\)-Skala als spektrale Struktur
\end{itemize}

\section{Von der Riemannschen Zeta-Funktion zur \(\beta\)-Skala}
\subsection{Spektrale Ableitungen aus \(\psi(x)\), Liouville und Nullstellen}
\subsection{Operatorische Konstruktion: \(H_\beta = \beta(n) \cdot \psi(n)\)}
\subsection{Die geometrische Form: \(\beta(n) = C \cos(\omega \log n + \phi) + \alpha n^\delta + \beta_0\)}

\section{Spektrale Fundierung und numerische Validierung}
\subsection{Erzeugung aus realen Nullstellen (z.B. zeros6, LMFDB)}
\subsection{Frequenzanalysen und Fourier-Komponenten}
\subsection{Vergleich mit empirischen Skalen (Ur-Skala, GPU, etc.)}
\subsection{Fit-Ergebnisse und Kohärenzwerte bis \(n = 2\,000\,000\)}

\section{Konzept des Fundamentalinvariants}
\subsection{Moduloinvarianz und Rekonstruktionseigenschaft}
\subsection{Verbindung zur Euler–Freese-Identität}
\subsection{Formalisierung als spektrales Grundgesetz}

\section{Verbindung zu Connes’ Quantisierungsstruktur}
\subsection{Gleichungen: \(\beta = \frac{2}{\log x}\) und \(x^{i\beta\pi} = 1\)}
\subsection{Vergleich zu \(x^{iy} = 1\) in der nicht-kommutativen Geometrie}
\subsection{Physikalische Bedeutung (z.B. Resonanzstruktur, Moduli)}

\section{Axiomatische Darstellung und Operatorrahmen}
\subsection{Definition von \(H_\beta = \beta(n) \cdot \psi(n)\)}
\subsection{Differentiation, Symbolik und Spektralform}
\subsection{Verallgemeinerung als übergeordnete Zeta-Funktionalität}

\section{Ausblick: Rekonstruktion und Beweis der RH}
\subsection{Die Rolle von \(\beta\) als Beweisinstrument}
\subsection{Zusammenführung mit analytischer Fortsetzung}
\subsection{Offene Perspektiven und Grenzen}

\section{Anhang}
\begin{itemize}
    \item Vergleichstabelle: \texttt{beta\_geometrisch.csv}, \texttt{beta\_spektrum\_analytisch.csv}, etc.
    \item Plots und statistische Auswertungen
    \item Quellen: ArXiv, Odlyzko, LMFDB, eigene Berechnungen
\end{itemize}

\end{document}