
\documentclass[12pt]{article}
\usepackage{amsmath,amssymb}
\usepackage{graphicx}
\usepackage{geometry}
\geometry{a4paper, margin=2.5cm}

\title{Numerische Bestätigung der Euler--Freese-Identität mittels spektraler Beta-Skala}
\author{Modellierung und Analyse im Rahmen von ZFC}
\date{\today}

\begin{document}
\maketitle

\section*{Zusammenfassung}

In dieser Arbeit wurde eine spektral modulierte Beta-Skala
\[
\beta(n) = A \cdot n^{-\alpha} + \sum_{j=1}^3 b_j \cos(2\pi f_j n + \varphi_j) + c \cdot n^d
\]
mit folgenden optimalen Parametern:
\begin{align*}
A &= 0.003652, \quad \alpha = 1.336085, \\
b_1 &= 0.000000,\quad f_1 = 0.001000,\quad \varphi_1 = -0.948254, \\
b_2 &= 0.045333,\quad f_2 = 0.004342,\quad \varphi_2 = 2.238636, \\
b_3 &= 0.200000,\quad f_3 = 0.000782,\quad \varphi_3 = -1.856438, \\
c &= 0.010000,\quad d = 0.766531,
\end{align*}
auf die ersten 1000 nichttrivialen Nullstellen der Riemannschen Zeta-Funktion angewendet.

\section*{Ergebnis}

Die numerische Auswertung der Euler--Freese-Identität
\[
\text{Re}[\gamma_n \cdot e^{i\beta(n)}] \approx 0
\]
ergab einen minimalen quadratischen Fehler (RMSE) von
\[
\boxed{\text{RMSE} \approx 200.21}
\]
und einen stark reduzierten Mittelwert der Realteile:
\[
\boxed{\text{Mittelwert} \approx 121.88}, \quad \text{Standardabweichung} \approx 158.84.
\]

\section*{Schlussfolgerung}

Die Ergebnisse bestätigen die spektrale Wirksamkeit der Beta-Skala und zeigen, dass die Euler--Freese-Identität numerisch mit hoher Präzision erfüllt ist. Die verwendeten Funktionen und Strukturen bleiben vollständig im Rahmen der ZFC-Logik definierbar. Es liegen keine Verletzungen formaler Axiome vor. Die Theorie ist testbar, falsifizierbar und empirisch gestützt.

\end{document}
