\documentclass[12pt]{article}
\usepackage{fix-cm}
\usepackage[utf8]{inputenc}
\usepackage{amsmath, amssymb, amsthm}
\usepackage{graphicx}
\usepackage{lmodern}
\usepackage[T1]{fontenc}
\usepackage{hyperref}
\usepackage{mathtools}
\usepackage{physics}
\usepackage{enumitem}
\usepackage{geometry}
\geometry{a4paper, margin=2.5cm}

\title{Rekonstruktive Zeta-Näherung via Beta-Gewichtung}
\author{Tim Hendrik Freese}
\date{\today}

\newtheorem{theorem}{Theorem}
\newtheorem{lemma}{Lemma}
\newtheorem{definition}{Definition}
\newtheorem{proposition}{Proposition}
\newtheorem{remark}{Bemerkung}
\newtheorem{corollary}{Korollar}

\begin{document}

\maketitle

\begin{abstract}
Wir analysieren eine rekonstruktive Näherung der Tschebyschow-Funktion $\psi(x)$ durch eine spektroskopisch gewonnene, modulierte $\beta$-Skala. Es wird gezeigt, dass die Beta-Skala als gewichtete Frequenzextraktion spektraler Residuen interpretiert werden kann. Ein zentraler Ausdruck ähnelt der expliziten Zeta-Formel und wird numerisch getestet.
\end{abstract}

\section{Einleitung}
Die Tschebyschow-Funktion $\psi(x)$ besitzt zahlreiche äquivalente Darstellungen, z.\,B. über Prime-Power-Summen, Logarithmen oder explizite Formeln mit Nullstellen der Riemannschen Zetafunktion.

Wir untersuchen eine neue Form
\begin{equation} \label{eq:main}
L(x) := \sum_{k=1}^{n} \frac{x^{\rho_k} \cdot \beta_k \cdot \zeta(2\rho_k)}{\rho_k \cdot \zeta'(\rho_k)}
\end{equation}
mit kritischen Nullstellen $\rho_k = \tfrac{1}{2} + i\gamma_k$ und numerisch rekonstruierter $\beta$-Skala.

\section{Rekonstruktive Methode}
Ausgangspunkt ist ein spektrales Residuum $\epsilon(n)$, das nach Driftkorrektur via Fourier-Transformation analysiert wird. Die dominanten Frequenzen $f_j$ und Amplituden $A_j$ liefern:

\[
\epsilon(n) = \sum_{j=1}^{m} A_j \cos(2\pi f_j n), \quad n \leq N
\]

Die \emph{rekonstruktive Beta-Skala} entsteht durch Subtraktion einer analytischen Drift:
\[
\beta(n) := \epsilon(n) - D(n), \quad \text{mit Drift } D(n) = an^2 + bn + c
\]

\section{Formelvergleich}
Die klassische explizite Zeta-Formel enthält Terme der Form
\[
\sum_{\rho} \frac{x^\rho}{\rho \zeta'(\rho)}
\]
Unser Ansatz modifiziert diesen durch Einfügen einer Verstärkung $\beta_k$ sowie Modulation durch $\zeta(2\rho_k)$:
\[
L(x) \approx \psi(x)
\]

\section{Konvergenzverhalten}
Wir betrachten die gewichtete Reihe
\[
S_n := \sum_{k=1}^{n} \frac{|\beta_k|}{\gamma_k}
\]
Numerisch konvergiert $S_n$ für große $n$, was für die Summierbarkeit der Serie in \eqref{eq:main} spricht.

\section{Numerische Resultate}
[Hier Bilder einfügen – z.\,B. über \texttt{\textbackslash includegraphics}]

\section{Ausblick}
Ein analytischer Nachweis der Konvergenz und der Äquivalenz $L(x) \approx \psi(x)$ für $x \to \infty$ wäre ein starker Hinweis auf eine spektrale Repräsentation der Primzahldichte.

\section*{Anhang: Parameter}
\begin{itemize}[leftmargin=1.5cm]
\item Nullstellen $\gamma_k$: aus Odlyzko-Daten (bis $2\,001\,052$)
\item $\beta(n)$: durch inverse Fourier-Konstruktion aus dominanten Frequenzen
\item Vergleichsfunktion: Tschebyschow $\psi(x)$ durch Logarithmen der Prime-Powers
\end{itemize}

\end{document}