\documentclass{article}
\usepackage{amsmath, amssymb}
\usepackage{mathtools}
\usepackage{physics}
\usepackage{geometry}
\geometry{margin=2.5cm}

\title{Spektrale Liouville-Formel mit \(\beta_k\)-Gewichtung}
\author{}
\date{}

\begin{document}
\maketitle

\section*{1. Allgemeine Struktur}

Die klassische spektrale Approximation der Tschebyschew-Funktion \(L(x)\) lautet:

\begin{equation}
L(x) = \sum_{k=1}^{n} \frac{\beta_k \cdot x^{\rho_k}}{\rho_k \cdot \zeta'(\rho_k)}
\end{equation}

Dabei ist:
\begin{itemize}
  \item \(\rho_k\) die \(k\)-te nichttriviale Nullstelle der Riemannschen Zeta-Funktion.
  \item \(\zeta'(\rho_k)\) die Ableitung an dieser Stelle.
  \item \(\beta_k\) ein gewichtender Skalenfaktor (z.\,B. aus Operatorstruktur).
\end{itemize}

\section*{2. Konkretisierung für \(\rho_k = \tfrac{1}{2} + i \gamma_k\)}

\begin{equation}
L(x) = \sum_{k=1}^{n} \frac{\beta_k \cdot x^{\frac{1}{2} + i\gamma_k}}{\left( \frac{1}{2} + i\gamma_k \right) \cdot \zeta'\left( \frac{1}{2} + i\gamma_k \right)}
\end{equation}

Diese Formel ermöglicht die spektrale Approximation der Tschebyschew-Funktion mit modulierender \(\beta\)-Struktur.

\end{document}