\documentclass[12pt]{article}
\usepackage{amsmath,amssymb,amsfonts}
\usepackage{physics}
\usepackage{hyperref}
\usepackage{graphicx}
\usepackage{geometry}
\geometry{a4paper, margin=2.5cm}

\title{Spektrale Rekonstruktion der Tschebyschew-Funktion mittels \(\beta\)-gewichteter Zeta-Nullstellen}
\author{Dein Name}
\date{\today}

\begin{document}

\maketitle

\section*{1. Ziel}

Diese Arbeit untersucht die Möglichkeit, die Tschebyschew-Funktion \(\psi(x)\) durch eine spektrale Entwicklung über die nichttrivialen Nullstellen der Riemannschen Zeta-Funktion \(\zeta(s)\) zu rekonstruieren. Dabei wird eine gewichtete Form der klassischen Formel~(76) verwendet, wobei eine empirisch oder konstruktiv motivierte Folge \(\beta(n)\) als spektrales Gewicht dient.

\section*{2. Definitionen und Voraussetzungen}

\subsection*{2.1 Zeta-Nullstellen}

Seien \(\rho_k = \tfrac{1}{2} + i\gamma_k\) die nichttrivialen Nullstellen der Riemannschen Zeta-Funktion, geordnet nach wachsendem Imaginärteil.

\subsection*{2.2 Tschebyschew-Funktion}

Die Tschebyschew-Funktion ist definiert durch:
\[
\psi(x) = \sum_{p^m \leq x} \log p
\]

\subsection*{2.3 Klassische spektrale Approximation (Formel 76)}

Eine klassische Darstellung der Liouville-Funktion \(L(x)\) in Bezug auf Zeta-Nullstellen lautet:
\[
L(x) = 1 + \sum_{k=1}^{N} \Re \left( \frac{x^{\rho_k} \cdot \zeta(2\rho_k)}{\rho_k \cdot \zeta'(\rho_k)} \right)
\]

\subsection*{2.4 Beta-gewichtete Variante}

Wir erweitern diese Struktur zu einer \(\beta\)-gewichteten Spektralsumme:
\[
L_\beta(x) = 1 + \sum_{k=1}^{N} \beta_k \cdot \Re \left( \frac{x^{\rho_k} \cdot \zeta(2\rho_k)}{\rho_k \cdot \zeta'(\rho_k)} \right)
\]
Dabei ist \(\beta_k \in \mathbb{R}\) eine frei wählbare Gewichtungssequenz.

\section*{3. Definition: Zeta-kompatible Beta-Skala}

\textbf{Definition.} Eine Folge \(\beta = (\beta_k)_{k\in\mathbb{N}}\) heißt \textbf{Zeta-kompatible Beta-Skala}, wenn:

\begin{enumerate}
    \item \(\beta_k \in \mathbb{R}\) für alle \(k\)
    \item Die Reihe
    \[
    \sum_{k=1}^{\infty} \left| \beta_k \cdot \frac{x^{\rho_k} \cdot \zeta(2\rho_k)}{\rho_k \cdot \zeta'(\rho_k)} \right|
    \]
    konvergiert (punktweise oder gleichmäßig) für \(x > 1\)
    \item Die Funktion \(L_\beta(x)\) approximiert \(\psi(x)\) im Grenzfall \(N \to \infty\)
\end{enumerate}

\section*{4. Spektrale Rekonstruktionsvermutung}

\textbf{Vermutung.}
\textit{Es existiert mindestens eine Folge \(\beta_k\), sodass gilt:}
\[
\psi(x) = \lim_{N \to \infty} \left(1 + \sum_{k=1}^N \beta_k \cdot \Re \left( \frac{x^{\rho_k} \cdot \zeta(2\rho_k)}{\rho_k \cdot \zeta'(\rho_k)} \right) \right)
\]
\textit{für alle \(x > 1\), im Sinne punktweiser oder gleichmäßiger Konvergenz.}

\section*{5. Relevanz und Zielsetzung}

Diese Struktur liefert eine neuartige spektrale Interpretation der Primzahlinformation in \(\psi(x)\) über modulierte Beiträge der Zeta-Nullstellen. Ein Beweis dieser Vermutung würde die \(\beta\)-Skala als spektralen Vermittler zwischen Zeta-Nullstellen und Primzahldichte etablieren.

\section*{6. Weiteres Vorgehen}

\begin{itemize}
    \item Ableitung analytischer Bedingungen für \(\beta_k\) (z.B. aus rekonstruktiven Verfahren)
    \item Untersuchung der Konvergenzgeschwindigkeit von \(L_\beta(x)\)
    \item Vergleich zu expliziten klassischen Formeln (Guinand-Weil, Riesz, etc.)
    \item Untersuchung einer möglichen Fourier-Struktur der \(\beta(n)\)
\end{itemize}

\vfill
\noindent\textbf{Kontakt:} \texttt{deine.email@domain.tld}

\section*{7. Eigenschaften der rekonstruktiven \(\beta_k\)}

\subsection*{7.1 Ursprung der rekonstruktiven Skala}

Die Folge \(\beta_k^{\text{rek}}\) wurde empirisch über ein Inversionsverfahren rekonstruiert, das auf der spektralen Rückprojektion aus der bekannten \(\psi(x)\)-Kurve basiert. Dabei wurde der Zusammenhang:

\[
\psi(x) \approx \sum_{k=1}^{N} \beta_k \cdot \Re \left( \frac{x^{\rho_k} \cdot \zeta(2\rho_k)}{\rho_k \cdot \zeta'(\rho_k)} \right)
\]

umgestellt und auf numerischem Weg invertiert, um die optimale Folge \(\beta_k\) aus den bekannten Werten von \(\psi(x)\) und \(\rho_k\) zu ermitteln.

\subsection*{7.2 Ziel: Mathematisch explizite Eigenschaften von \(\beta_k\)}

Es soll nun untersucht werden, ob diese \(\beta_k^{\text{rek}}\) eine oder mehrere der folgenden Eigenschaften besitzt:

\begin{itemize}
    \item \textbf{(A)} Glattheit oder reguläres asymptotisches Verhalten
    \item \textbf{(B)} Zugehörigkeit zu bekannten Funktionalklassen (z.\,B. Fourier-Koeffizienten, orthogonale Systeme)
    \item \textbf{(C)} Strukturelle Entsprechung zu Residuen oder Dichten der Zeta-Funktion
    \item \textbf{(D)} Rekonstruktionsstabilität: Kleine Änderungen in \(\psi(x)\) führen zu kleinen Änderungen in \(\beta_k\)
\end{itemize}

\subsection*{7.3 Konvergenz und analytisches Verhalten}

Die Reihe

\[
L_\beta(x) = 1 + \sum_{k=1}^{\infty} \beta_k^{\text{rek}} \cdot \Re \left( \frac{x^{\rho_k} \cdot \zeta(2\rho_k)}{\rho_k \cdot \zeta'(\rho_k)} \right)
\]

ist im Kontext ihrer Konvergenzgeschwindigkeit und Regularität zu analysieren. Dabei interessieren insbesondere:

\begin{itemize}
    \item Punktweise oder gleichmäßige Konvergenz auf Kompakten
    \item Fehlerabschätzungen \(|\psi(x) - L_\beta(x)|\) in Abhängigkeit von \(N\)
    \item Dominierte Konvergenz, sofern \(\beta_k\) durch wohldefinierte Schranken kontrolliert ist
\end{itemize}

\subsection*{7.4 Ausblick: Vergleich mit klassischen Entwicklungen}

Die klassische Entwicklung etwa der Guinand-Weil-Formel liefert:

\[
\psi(x) = x - \sum_{\rho} \frac{x^{\rho}}{\rho} - \frac{\zeta'}{\zeta}(0) - \frac{1}{2} \log\left(1 - \frac{1}{x^2} \right)
\]

Ein Vergleich der rekonstruierenden \(\beta_k\)-Formel mit dieser klassischen Struktur kann Hinweise geben, ob und wie die \(\beta_k\)-Formel eine Variante oder verallgemeinerte Interpretation dieser klassischen Ansätze darstellt.

\documentclass{article}
\usepackage{amsmath}
\usepackage{amsfonts}
\usepackage{siunitx}
\usepackage{graphicx}
\usepackage{physics}
\usepackage{booktabs}
\usepackage{geometry}
\geometry{a4paper, margin=2.5cm}

\begin{document}

\section*{Rekonstruktive Formel für \boldmath$\beta_k$ aus dominanten Frequenzen}

Die rekonstruktive Beta-Skala basiert auf einer trigonometrischen Superposition dominanter Frequenzanteile des Spektrums der Fehlerfunktion \( \varepsilon(n) \). Die rekonstruktive Näherung lautet:

\[
\beta_k \approx \sum_{n=1}^{N} \frac{A_n}{k^\delta} \cdot \cos\left(2\pi f_n \cdot \log(k) + \varphi_n \right)
\]

\textbf{mit:}
\begin{itemize}
    \item \( k \in \mathbb{N} \): Index des Terms
    \item \( f_n \): extrahierte dominante Frequenzen aus dem Spektrum von \( \varepsilon(n) \)
    \item \( A_n \): zugehörige Amplituden (Skalierungsfaktoren)
    \item \( \varphi_n \approx 0 \): optionale Phasenverschiebung (hier ignoriert)
    \item \( \delta > 0 \): Dämpfungsfaktor zur Sicherung der Konvergenz
\end{itemize}

\subsection*{Dominante Frequenzen (extrahiert)}

\begin{table}[h!]
\centering
\begin{tabular}{@{}rr@{}}
\toprule
\textbf{Frequenz \( f_n \)} & \textbf{Amplitude \( A_n \)} \\
\midrule
9.994743e-06 & 1.82e+08 \\
9.495006e-06 & 1.91e+08 \\
8.995268e-06 & 2.00e+08 \\
8.495531e-06 & 2.11e+08 \\
7.995794e-06 & 2.23e+08 \\
7.496057e-06 & 2.36e+08 \\
6.996320e-06 & 2.51e+08 \\
6.496583e-06 & 2.68e+08 \\
5.996846e-06 & 2.87e+08 \\
5.497109e-06 & 3.09e+08 \\
4.997371e-06 & 3.35e+08 \\
4.497634e-06 & 3.66e+08 \\
3.997897e-06 & 4.03e+08 \\
3.498160e-06 & 4.47e+08 \\
2.998423e-06 & 5.02e+08 \\
2.498686e-06 & 5.70e+08 \\
1.998949e-06 & 6.54e+08 \\
9.994743e-07 & 7.44e+08 \\
1.499211e-06 & 7.45e+08 \\
4.997371e-07 & 1.30e+09 \\
\bottomrule
\end{tabular}
\caption{Extrahierte dominante Frequenzen und Amplituden zur rekonstruktiven Näherung der Beta-Skala.}
\end{table}

\end{document}
\end{document}