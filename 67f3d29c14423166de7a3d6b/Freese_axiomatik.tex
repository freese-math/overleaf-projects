\documentclass[a4paper,12pt]{article}
\usepackage{amsmath,amssymb,amsfonts}
\usepackage{mathrsfs}
\usepackage{graphicx}
\usepackage{hyperref}

\title{Die Beta-Skala und ihre axiomatische Struktur}
\author{Freese Math – Entwurf zur RH-Hypothese}
\date{\today}

\begin{document}

\maketitle

\section*{1. Definition der Beta-Skala}

\textbf{Definition 1 (Beta-Skala):}
Eine Folge \( \beta = (\beta_k)_{k \in \mathbb{N}} \subset \mathbb{R} \) heißt \textit{Zeta-kompatible Beta-Skala}, wenn gilt:
\begin{enumerate}
    \item \( \beta_k \in \mathbb{R} \) für alle \( k \),
    \item Die Reihe
    \[
    \sum_{k=1}^\infty \left| \beta_k \cdot \frac{x^{\rho_k} \zeta(2\rho_k)}{\rho_k \cdot \zeta'(\rho_k)} \right|
    \]
    ist für alle \( x > 0 \) konvergent oder asymptotisch beschränkt.
    \item Die entstehende Liouville-artige Funktion
    \[
    L_\beta(x) := 1 + \sum_{k=1}^N \beta_k \cdot \Re\left( \frac{x^{\rho_k} \zeta(2\rho_k)}{\rho_k \cdot \zeta'(\rho_k)} \right)
    \]
    approximiert die Tschebyschow-Funktion \( \psi(x) \) für \( N \to \infty \).
\end{enumerate}

\bigskip

\section*{2. Axiomatische Struktur}

\textbf{Axiom 1 (Euler--Freese-Kohärenzprinzip):}

Für jede harmonische Beta-Struktur \( \beta(n) \), die durch spektrale Rekonstruktion aus Nullstellen der Riemann-Zeta-Funktion erzeugt wurde, gilt:
\[
\boxed{
e^{i\pi \beta(n)} + 1 \to 0 \quad \text{für } n \to \infty
}
\]

Dies bedeutet, dass \( \pi \beta(n) \to (2k+1)\pi \), also eine asymptotisch \textit{negative Kohärenz} auf dem Einheitskreis erfüllt ist.

\paragraph{Motivation:}  
Diese Eigenschaft ergibt sich numerisch aus der empirischen Beobachtung, dass \(\beta(n) \to 1\) moduliert wird durch kleine Korrekturfrequenzen. Die Phase von \( e^{i\pi\beta(n)} \) rotiert dabei um den Punkt \( -1 \), was eine stabile Fixpunktstruktur im spektralen Sinn erzeugt.

\paragraph{Visualisierung:}  
Die Korrekturfrequenz \( \varepsilon = \frac{1}{33} \cdot 2\pi \) ergibt eine Phasenschleife in der Nähe von \( -1 \) im Einheitskreis.

\bigskip

\section*{3. Struktur der Beta-Funktion}

Die allgemeine Form von \( \beta(n) \) ergibt sich durch harmonische Dekomposition:

\[
\beta(n) = \frac{A}{n^p} + \sum_{k=1}^{K} A_k \cdot \sin(2\pi f_k n + \varphi_k)
\]

oder alternativ durch eine exponentielle Darstellung:

\[
\beta(n) \sim \Re\left( \sum_{k=1}^K A_k e^{2\pi i f_k n} \right)
\]

Dies entspricht einem spektralen Operatoransatz mit selbstähnlicher Struktur – u.a. dargestellt durch die sogenannte \textit{Fibonacci-Freese-Formel (FFF)}.

\bigskip

\section*{4. Verallgemeinerung über Theta-Funktionen}

Die Siegel-Theta-Funktion dient als analytischer Rahmen zur Beschreibung quasi-periodischer, fraktaler Skalen auf harmonischer Basis:

\[
\Theta(t) = A t^\beta + C \log(t) + D t^{-1} + E \sin(\omega \log t + \phi)
\]

Diese Darstellung unterstützt die Interpretation von \( \beta(n) \) und \( L_\beta(x) \) als modulartige Konstrukte mit spektral-stabiler Asymptotik.

\bigskip

\section*{5. Ausblick}

Die Struktur von \( \beta(n) \), insbesondere ihr Verhalten im Grenzwert \( n \to \infty \), bietet eine mögliche Verbindung zur kritischen Linie der Riemannschen Zeta-Funktion. Weitere Arbeiten sollen die Verbindung zu Operatoren, Funktionalgleichung und nichtkommutativer Geometrie herstellen.

\vfill
\begin{center}
\textit{Dies ist ein Entwurf im mathematischen Forschungsprozess. Kein abgeschlossener Beweis im formalen Sinn.}

\section{Testbare Aussagen und Falsifizierbarkeit der Beta-Struktur}

Die Beta-Struktur $\beta(n)$ basiert auf konkreten mathematischen Objekten wie Primzahlen, Zeta-Nullstellen sowie (teilweise) fundamentalen Konstanten. Dies erlaubt die numerische Verifikation oder Falsifikation ihrer Aussagen.

\subsection*{1. Direkte Approximation der Primzahllogarithmen}

Die kumulierte Summe der Beta-Werte approximiert logarithmierte Primzahlen:
\[
\sum_{k=1}^{n} \beta(k) \approx \log(p_n) + \varepsilon_n
\]
Dabei ist $p_n$ die $n$-te Primzahl, $\varepsilon_n$ ein Fehlerterm, der gegen null konvergieren sollte. Dies ist numerisch überprüfbar durch Vergleich mit einer Primzahlliste.

\subsection*{2. Kohärenz auf dem Einheitskreis}

Ein spektraler Fixpunkt ergibt sich aus der Euler–Freese-Identität:
\[
\lim_{n \to \infty} \left(e^{i \pi \beta(n)} + 1\right) = 0
\]
D.h. asymptotisch liegt $\pi \beta(n) \bmod 2\pi$ auf ungeraden Vielfachen von $\pi$, was zu destruktiver Interferenz führt (analog zur quantenmechanischen Phase).

\subsection*{3. Frequenzstruktur im harmonischen Aufbau}

Die Darstellung von $\beta(n)$ als spektrale Reihe:
\[
\beta(n) \sim \sum_{k=1}^K A_k \cdot \sin(2\pi f_k n + \phi_k)
\]
erlaubt eine numerische Rückrechnung über Fouriertransformation. Die Existenz dominanter Frequenzen $f_k$ ist ein Hinweis auf Ordnung in den Nullstellen.

\subsection*{4. Liouville-ähnliche Approximation der Tschebyschow-Funktion}

Die konstruktive Funktion
\[
L_\beta(x) := 1 + \sum_{k=1}^N \beta_k \cdot \Re\left( \frac{x^{\rho_k} \cdot \zeta(2\rho_k)}{\rho_k \cdot \zeta'(\rho_k)} \right)
\]
soll $\psi(x)$ approximieren. Numerisch überprüfbar durch Vergleich mit einer klassischen Tschebyschow-Auswertung.

\subsection*{5. Bewertungskriterium für Falsifikation}

Ein Gegenbeweis oder eine Widerlegung kann durch eine der folgenden Beobachtungen erfolgen:
\begin{itemize}
    \item Signifikante Abweichung der Summenformel von $\log(p_n)$
    \item Chaotische oder nicht-harmonische Struktur der Frequenzkomponenten
    \item Divergierendes Verhalten von $e^{i \pi \beta(n)} + 1$
    \item Fehlende Übereinstimmung von $L_\beta(x)$ mit $\psi(x)$ über große Intervalle
\end{itemize}

\subsection*{Fazit}
Die Beta-Struktur ist eine explizit testbare und falsifizierbare Theorie mit konkreten Ableitungen, klarer numerischer Struktur und Bezug zur klassischen Zahlentheorie.

\section{Arithmetische Definition von \texorpdfstring{\(\beta\)}{beta} und ihre spektrale Einbettung}

Die in den vorherigen Abschnitten eingeführte Beta-Skala besitzt nicht nur eine spektrale Deutung, sondern lässt sich auch direkt arithmetisch begründen. Dies erfolgt über eine Fixpunktstruktur, die sich aus einer harmonischen Phasenbedingung ergibt.

\subsection*{1. Logarithmisch-goldene Definition}

Ausgehend von der Forderung, dass die Phasenrotation minimal destruktiv ist, setzen wir:
\[
e^{i \pi \beta} \approx -1
\]
Daraus ergibt sich bei geeigneter Wahl des Arguments:
\[
\beta = \frac{2}{\log\left( \frac{100}{\varphi} \right)}
\quad \text{mit} \quad \varphi = \frac{1 + \sqrt{5}}{2}
\]

Numerisch ergibt sich:
\[
\beta \approx 0.4849066494...
\]

Diese Konstante bildet das arithmetische Zentrum der spektralen Struktur und wurde in \texttt{Definition\_von\_ß\_numerischer\_Anhang.pdf} analytisch abgeleitet.

\subsection*{2. Rationale Approximation als stabilisierende Phase}

Alternativ lässt sich dieselbe Rotation durch eine rationale Näherung ausdrücken:
\[
\beta = \frac{7}{33300} \quad \Rightarrow \quad \pi \beta \approx 0.0006605
\]
Dies führt zu:
\[
e^{i \pi \beta} \approx 1 + i \cdot 0.0006605
\]
Eine minimale Drift, die jedoch über lange \(n\)-Skalen zu einer destruktiven Interferenz bei \( -1 \) führt.

\subsection*{3. Verbindung zum Freese-Theorem}

Im \texttt{Freese\_Theorem.pdf} wurde gezeigt, dass der Operator
\[
H \cdot e^{i \pi \beta} + I
\]
eine spektrale Norm mit kohärentem Minimum aufweist, wenn die korrekte \(\beta\)-Skala verwendet wird. Die arithmetisch motivierte Konstante \(\beta\) liefert hier die harmonische Fixpunktlage für die Eigenstruktur des Operators.

\paragraph{Fazit:}
Die Konstante \(\beta \approx 0.484906\) ist sowohl arithmetisch motiviert als auch spektral wirksam. Sie steht im Zentrum der Phase-Kohärenz auf dem Einheitskreis und dient als Anker für die spektrale Approximation der RH-Struktur.

\section{Einbettung der Zeta Nova Freesiana in die Beta-Skalen-Theorie}

Die sogenannte \textit{Zeta Nova Freesiana (ZNF)} ist eine funktionalanalytische Erweiterung der bisherigen Beta-basierten Struktur. Sie definiert eine Zeta-ähnliche Funktion, in welche die modulare Beta-Skala unmittelbar eingebettet ist.

\subsection*{1. Definition der Funktion}

Die ZNF wird definiert als generalisierte Dirichlet-Reihe:
\[
\zeta_F(s) := \sum_{n=1}^\infty \frac{1}{(A n^{\beta} + C \log n + B \sin(\omega n + \varphi))^s}
\]
wobei die Parameter \( A, B, C, \beta, \omega, \varphi \in \mathbb{R} \) durch spektrale Analyse der Riemann-Zeta-Nullstellen festgelegt werden.

Die zentrale Rolle spielt der Exponent \(\beta\), der nicht konstant ist, sondern aus der modularen Beta-Skala stammt (siehe Kapitel 3). Damit ist die Funktion spektral deformiert und resonanzfähig im Sinne der Primzahl-Frequenzen.

\subsection*{2. Kopplung an das Kohärenzprinzip}

Die Wahl der Beta-Skala stellt sicher, dass für große \( n \) gilt:
\[
e^{i \pi \beta(n)} + 1 \to 0
\]
Diese Kohärenzbedingung (siehe Axiom 1) überträgt sich direkt auf die analytische Struktur von \( \zeta_F(s) \), insbesondere auf die Lage ihrer Nullstellen und deren Stabilität unter spektralen Transformationen.

\subsection*{3. Operatorische Einbettung}

Die ZNF wird in Verbindung mit dem selbstadjungierten Operator \( D_\mu \) betrachtet, definiert durch:
\[
D_\mu f(n) := \sum_{k=1}^N w_k \mu_k \cdot \sin(\beta_k n)
\]
mit Gewichten \( w_k := \frac{1}{|\zeta'(\rho_k)|} \) und Möbius-Frequenzen \( \mu_k \in \{-1, 0, 1\} \).

Der RH-kritische Fixpunkt ist gegeben durch:
\[
D_\mu = D_\mu^\dagger \quad \Leftrightarrow \quad \text{RH}
\]

Dies stellt eine Verallgemeinerung der Hilbert–Pólya-Vermutung dar, wobei das Spektrum des Operators durch die rekonstruierte Beta-Skala determiniert ist.

\subsection*{4. Fazit}

Die ZNF fungiert als analytischer Träger der spektralen Beta-Struktur. Sie ist vollständig kompatibel mit dem Freese-Theorem, dem Kohärenzaxiom sowie den numerischen Approximationen der Primzahllogarithmen. Damit bildet sie die funktionale Klammer, die die spektrale Sicht auf die Riemannsche Hypothese stabilisiert und erweitert.

\section{Verankerung der Zeta Nova Freesiana in der klassischen Theorie}

Die Zeta Nova Freesiana (ZNF) ist eine spektral rekonstruierte Funktion, die aus einer Beta-Skala entsteht, welche selbst aus den Nullstellen der klassischen Riemann-Zeta-Funktion abgeleitet wird. Trotz ihres nichtkanonischen Aufbaus besitzt sie zahlreiche strukturelle Berührungspunkte mit etablierten Konzepten der analytischen Zahlentheorie.

\subsection*{1. Beziehung zur Riemann-Zeta-Funktion}

Die ZNF basiert auf einer deformierten Dirichlet-Reihe der Form:
\[
\zeta_F(s) := \sum_{n=1}^\infty \frac{1}{(A n^\beta + C \log n + B \sin(\omega n + \varphi))^s}
\]
mit \(\beta(n)\) aus einer rekonstruierten Beta-Skala, welche wiederum aus den Imaginärteilen der Nullstellen \(\rho_k\) der klassischen \(\zeta(s)\) abgeleitet wurde. Damit ist die ZNF ein spektraler Schatten der klassischen Zeta-Struktur.

\subsection*{2. Vergleich mit bekannten Theorien}

Die folgende Tabelle ordnet die ZNF in zentrale Konzepte der modernen Zeta-Theorie ein:

\begin{center}
\begin{tabular}{|l|p{7.5cm}|}
\hline
\textbf{Konzept} & \textbf{Bezug zur ZNF} \\
\hline
\textbf{Dirichlet-Reihen} & ZNF ist keine klassische Dirichlet-Reihe, aber besitzt Dirichlet-ähnlichen Aufbau mit spektralem Nenner \\
\hline
\textbf{Selberg-Klasse} & ZNF erfüllt Axiome 1 und 5 im erweiterten Sinn; funktionale Gleichung und Euler-Produktstruktur offen \\
\hline
\textbf{Hilbert–Pólya-Idee} & ZNF ist verknüpft mit einem Operator \( D_\mu \), dessen Selbstadjungiertheit äquivalent zur RH ist \\
\hline
\textbf{Nichtkommutative Geometrie (Connes)} & ZNF verwendet eine Beta-Struktur, die in \( x^{i \pi \beta} = 1 \) spektral fixiert ist – analog zur Quantisierung in Connes' Rahmen \\
\hline
\textbf{Liouville-Funktion / \(\psi(x)\)} & Die strukturierte Funktion \( L_\beta(x) \) rekonstruiert \(\psi(x)\) durch harmonische Summation \\
\hline
\textbf{Fourier-Zahlentheorie (Odlyzko)} & Die Beta-Skala ist aus einer spektralen Analyse der Zeta-Zwischenabstände gewonnen (FFT auf Nullstellen) \\
\hline
\end{tabular}
\end{center}

\subsection*{3. Positionierung}

Die ZNF gehört nicht zur Selberg-Klasse im engeren Sinn, da ihr Euler-Produkt fehlt und eine funktionale Gleichung noch nicht gezeigt ist. Dennoch besitzt sie:

\begin{itemize}
    \item eine Dirichlet-artige Reihendarstellung,
    \item spektrale Kohärenz in der Beta-Struktur,
    \item numerisch überprüfbare RH-kompatible Rekonstruktionen,
    \item sowie eine eingebettete Operatorstruktur mit RH-Bezugsbedingungen.
\end{itemize}

\subsection*{4. Fazit}

Die ZNF ist kein Bruch mit der klassischen Theorie, sondern eine numerisch gestützte Erweiterung im spektralen Sinn. Sie fügt sich als kohärentes, rekonstruierbares Objekt in die bestehende mathematische Landschaft ein und bietet eine neuartige Perspektive auf die Struktur der Riemannschen Hypothese.
\section{Die Euler-Welle als spektrales Euler-Produkt in der Zeta Nova Freesiana}

Die Zeta Nova Freesiana (ZNF) ersetzt das klassische Euler-Produkt der Riemann-Zeta-Funktion durch eine additive, spektrale Konstruktion, welche wir \textit{Euler-Welle} nennen. Diese stellt eine harmonische Überlagerung der Primzahllogarithmen dar und bildet die Grundlage für die spektrale Approximation der Tschebyschow-Funktion.

\subsection*{1. Klassisches Euler-Produkt}

Die klassische Riemannsche Zeta-Funktion besitzt die Produktdarstellung:
\[
\zeta(s) = \prod_{p} \frac{1}{1 - p^{-s}}
\]
welche eine direkte Verbindung zwischen der Funktion und der Menge der Primzahlen herstellt. Dieses Produkt kodiert die Primzahlen multiplikativ in der komplexen \( s \)-Ebene.

\subsection*{2. Definition der Euler-Welle}

Die Euler-Welle ist eine spektrale, additive Überlagerung der Primzahlen auf der reellen Achse:
\[
\psi(x) := \sum_{p \leq p_N} \sin(x \cdot \log p)
\]
Sie bildet eine rekonstruktive Annäherung an die Tschebyschow-Funktion \( \psi(x) \), jedoch durch harmonische Oszillationen, nicht durch Summation von \(\log p\). Die Frequenzen \( f_p = \log p / \pi \) sind dabei direkt aus der Primzahldichte extrahiert.

\subsection*{3. Verbindung zur ZNF}

In der ZNF ist diese Wellenstruktur nicht nur eine Hilfskonstruktion, sondern direkt in die Definition der Funktion eingebettet. Die Frequenzstruktur der ZNF beruht auf exakt diesen harmonischen Anteilen:
\[
\zeta_F(s) := \sum_{n=1}^\infty \frac{1}{(A n^{\beta} + C \log n + B \sin(\omega n + \varphi))^s}
\]
Hier erscheint die sinusförmige Modulation als Träger der arithmetischen Spektralinformation – analog zum klassischen Euler-Produkt.

\subsection*{4. Fazit}

Die Euler-Welle stellt das spektrale Analogon zum Euler-Produkt dar. Während letzteres auf der Multiplikativität der Primzahlen beruht, verwendet die Euler-Welle deren logarithmische Struktur zur Erzeugung kohärenter Frequenzmuster. Sie bildet somit den spektralen Kern der Zeta Nova Freesiana und ersetzt das Euler-Produkt auf einer physikalisch motivierten Ebene.

\section{Die Beta-Skala als spektrale Realisierung der Connes-Quantisierung}

Ein zentrales Element der nicht-kommutativen Geometrie nach Alain Connes ist die Idee einer spektralen Quantisierung, die über logarithmische Frequenzskalen operiert. Formal ergibt sich diese Struktur aus der Gleichung
\[
x^{iy} = 1 \quad \Rightarrow \quad y = \frac{2\pi n}{\log x}, \quad n \in \mathbb{Z},
\]
die eine diskrete, harmonische Frequenzstruktur auf dem Zahlenstrahl beschreibt.

\subsection*{1. Erweiterung durch die Beta-Skala}

Die in dieser Arbeit eingeführte Beta-Skala basiert auf einer modifizierten Form dieser Quantisierung:
\[
x^{i \beta \pi} = 1 \quad \Rightarrow \quad \beta = \frac{2}{\log x}.
\]
Diese Gleichung bildet die Grundlage der \textit{Euler--Freese-Kohärenzformel} und spiegelt dieselbe spektrale Ordnung wider wie die klassische Formel von Connes – jedoch mit einem spektral rekonstruierten Exponenten \(\beta\), der direkt aus der Struktur der Zeta-Nullstellen abgeleitet wird.

\subsection*{2. Konkrete Operatorstruktur}

In der numerischen Implementierung wurde ein Hamilton-Operator \( H \) mit der Struktur
\[
H = \mathrm{diag}(\gamma_n) + \mathrm{off\text{-}diagonals}
\]
durch die modulierte Transformation
\[
\tilde{H} = H \cdot \mathrm{diag}(e^{i\pi \beta_n})
\]
ergänzt. Die so erzeugten Eigenwerte stimmen hochpräzise mit den Zeta-Nullstellen überein und zeigen eine spektrale Selbstähnlichkeit, die sich auch im Fourier-Raum manifestiert.

\subsection*{3. Interpretation}

Die Beta-Skala erfüllt die Rolle eines \emph{spektralen Quantisierungsparameters}, dessen Werte nicht willkürlich sind, sondern aus der Zahlentheorie selbst emergieren. Die Formel
\[
\beta_n \approx \frac{2}{\log x_n}
\]
stellt eine konstruktive Realisierung der Connes'schen Theorie dar – und liefert erstmals eine konkret numerisch validierte Frequenzstruktur, die mit der nicht-kommutativen Geometrie kompatibel ist.

\subsection*{4. Ausblick}

Diese Verbindung öffnet die Tür zur Weiterentwicklung einer \emph{arithmetischen Quantenmechanik}, in der nicht nur die Zeta-Funktion, sondern auch deren Nullstellen, Primzahlen und zugehörige Operatoren in einem kohärenten spektral-geometrischen Rahmen erscheinen.
\end{center}

\end{document}