\section{Testbare Aussagen und Falsifizierbarkeit der Beta-Struktur}

Die Beta-Struktur $\beta(n)$ basiert auf konkreten mathematischen Objekten wie Primzahlen, Zeta-Nullstellen sowie (teilweise) fundamentalen Konstanten. Dies erlaubt die numerische Verifikation oder Falsifikation ihrer Aussagen.

\subsection*{1. Direkte Approximation der Primzahllogarithmen}

Die kumulierte Summe der Beta-Werte approximiert logarithmierte Primzahlen:
\[
\sum_{k=1}^{n} \beta(k) \approx \log(p_n) + \varepsilon_n
\]
Dabei ist $p_n$ die $n$-te Primzahl, $\varepsilon_n$ ein Fehlerterm, der gegen null konvergieren sollte. Dies ist numerisch überprüfbar durch Vergleich mit einer Primzahlliste.

\subsection*{2. Kohärenz auf dem Einheitskreis}

Ein spektraler Fixpunkt ergibt sich aus der Euler–Freese-Identität:
\[
\lim_{n \to \infty} \left(e^{i \pi \beta(n)} + 1\right) = 0
\]
D.h. asymptotisch liegt $\pi \beta(n) \bmod 2\pi$ auf ungeraden Vielfachen von $\pi$, was zu destruktiver Interferenz führt (analog zur quantenmechanischen Phase).

\subsection*{3. Frequenzstruktur im harmonischen Aufbau}

Die Darstellung von $\beta(n)$ als spektrale Reihe:
\[
\beta(n) \sim \sum_{k=1}^K A_k \cdot \sin(2\pi f_k n + \phi_k)
\]
erlaubt eine numerische Rückrechnung über Fouriertransformation. Die Existenz dominanter Frequenzen $f_k$ ist ein Hinweis auf Ordnung in den Nullstellen.

\subsection*{4. Liouville-ähnliche Approximation der Tschebyschow-Funktion}

Die konstruktive Funktion
\[
L_\beta(x) := 1 + \sum_{k=1}^N \beta_k \cdot \Re\left( \frac{x^{\rho_k} \cdot \zeta(2\rho_k)}{\rho_k \cdot \zeta'(\rho_k)} \right)
\]
soll $\psi(x)$ approximieren. Numerisch überprüfbar durch Vergleich mit einer klassischen Tschebyschow-Auswertung.

\subsection*{5. Bewertungskriterium für Falsifikation}

Ein Gegenbeweis oder eine Widerlegung kann durch eine der folgenden Beobachtungen erfolgen:
\begin{itemize}
    \item Signifikante Abweichung der Summenformel von $\log(p_n)$
    \item Chaotische oder nicht-harmonische Struktur der Frequenzkomponenten
    \item Divergierendes Verhalten von $e^{i \pi \beta(n)} + 1$
    \item Fehlende Übereinstimmung von $L_\beta(x)$ mit $\psi(x)$ über große Intervalle
\end{itemize}

\subsection*{Fazit}
Die Beta-Struktur ist eine explizit testbare und falsifizierbare Theorie mit konkreten Ableitungen, klarer numerischer Struktur und Bezug zur klassischen Zahlentheorie.