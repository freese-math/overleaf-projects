\section{Die Beta-Skala als spektrale Realisierung der Connes-Quantisierung}

Ein zentrales Element der nicht-kommutativen Geometrie nach Alain Connes ist die Idee einer spektralen Quantisierung, die über logarithmische Frequenzskalen operiert. Formal ergibt sich diese Struktur aus der Gleichung
\[
x^{iy} = 1 \quad \Rightarrow \quad y = \frac{2\pi n}{\log x}, \quad n \in \mathbb{Z},
\]
die eine diskrete, harmonische Frequenzstruktur auf dem Zahlenstrahl beschreibt.

\subsection*{1. Erweiterung durch die Beta-Skala}

Die in dieser Arbeit eingeführte Beta-Skala basiert auf einer modifizierten Form dieser Quantisierung:
\[
x^{i \beta \pi} = 1 \quad \Rightarrow \quad \beta = \frac{2}{\log x}.
\]
Diese Gleichung bildet die Grundlage der \textit{Euler--Freese-Kohärenzformel} und spiegelt dieselbe spektrale Ordnung wider wie die klassische Formel von Connes – jedoch mit einem spektral rekonstruierten Exponenten \(\beta\), der direkt aus der Struktur der Zeta-Nullstellen abgeleitet wird.

\subsection*{2. Konkrete Operatorstruktur}

In der numerischen Implementierung wurde ein Hamilton-Operator \( H \) mit der Struktur
\[
H = \mathrm{diag}(\gamma_n) + \mathrm{off\text{-}diagonals}
\]
durch die modulierte Transformation
\[
\tilde{H} = H \cdot \mathrm{diag}(e^{i\pi \beta_n})
\]
ergänzt. Die so erzeugten Eigenwerte stimmen hochpräzise mit den Zeta-Nullstellen überein und zeigen eine spektrale Selbstähnlichkeit, die sich auch im Fourier-Raum manifestiert.

\subsection*{3. Interpretation}

Die Beta-Skala erfüllt die Rolle eines \emph{spektralen Quantisierungsparameters}, dessen Werte nicht willkürlich sind, sondern aus der Zahlentheorie selbst emergieren. Die Formel
\[
\beta_n \approx \frac{2}{\log x_n}
\]
stellt eine konstruktive Realisierung der Connes'schen Theorie dar – und liefert erstmals eine konkret numerisch validierte Frequenzstruktur, die mit der nicht-kommutativen Geometrie kompatibel ist.

\subsection*{4. Ausblick}

Diese Verbindung öffnet die Tür zur Weiterentwicklung einer \emph{arithmetischen Quantenmechanik}, in der nicht nur die Zeta-Funktion, sondern auch deren Nullstellen, Primzahlen und zugehörige Operatoren in einem kohärenten spektral-geometrischen Rahmen erscheinen.