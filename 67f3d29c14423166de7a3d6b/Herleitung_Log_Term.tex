\documentclass[a4paper,12pt]{article}
\usepackage[T1]{fontenc}
\usepackage{amsmath, amssymb, mathrsfs}
\usepackage{graphicx}
\usepackage{hyperref}
\usepackage{geometry}
\geometry{margin=2.5cm}

\title{Herleitung des Log-Terms in der Beta-Skala}
\author{}
\date{}

\begin{document}
\maketitle

\section*{Zielsetzung}

Wir wollen den Term
\[
\beta(n) \supset \frac{a}{\log n}
\]
analytisch motivieren. Dabei untersuchen wir, ob dieser Ausdruck eine direkte Verbindung zur Verteilung der Primzahlen und den Nullstellen der Zetafunktion aufweist.

\section*{1. Ursprung in der Primzahldichte}

Die Primzahldichte-Funktion ist bekanntlich
\[
\pi(x) \approx \text{Li}(x) \sim \frac{x}{\log x}.
\]
Die Dichte der Primzahlen ist somit (formal abgeleitet)
\[
\frac{d\pi}{dx} \sim \frac{1}{\log x}.
\]
Wenn wir \(\beta(n)\) als „Frequenzmaß“ oder Spektrum über einen Zustand \(n\) interpretieren, ergibt sich eine strukturelle Nähe zu dieser Funktion.

\section*{2. Asymptotik aus spektraler Betrachtung}

In vielen quantenmechanischen Systemen oder bei spektralen Dichten aus Hamilton-Operatoren tritt ebenfalls eine logarithmische Dämpfung auf:
\[
\rho(E) \sim \frac{1}{\log E}
\]
für große Energien \(E\). Überträgt man dies auf eine Zustandsindizierung \(n\), erhält man:
\[
\beta(n) \propto \frac{1}{\log n}.
\]

\section*{3. Mellin-Analyse und harmonische Summen}

Für eine harmonische Reihe logarithmischer Form:
\[
\sum_{k=2}^n \frac{1}{k \log k} \approx \log \log n + C,
\]
kann man zeigen, dass die gewichtete Dichte der Form \(1/\log n\) eng mit asymptotisch konstanten Summen verknüpft ist – ein bekanntes Motiv aus der Primzahlanalyse.

\section*{4. Herleitung durch funktionale Approximation}

Wenn die \(\beta(n)\)-Skala als Mittelung einer spektralen Eigenschaft wie z.~B. Abstände oder Resonanzen interpretiert wird, dann ist
\[
\beta(n) \sim \frac{1}{\log n}
\]
ein sinnvoller Leading-Term für große \(n\), da sowohl spektrale als auch analytische Funktionen (z.~B. Chebyschow-, Mangoldt-Funktion) in diesem Grenzbereich solche Terme aufweisen.

\section*{Fazit}

Die beobachtete Struktur
\[
\beta(n) = A + \frac{a}{\log n} + \dots
\]
kann als natürlicher Ausdruck einer asymptotischen Primzahldichte oder spektralen Energiestreuung interpretiert werden. Der Logarithmus ist daher kein numerisches Artefakt, sondern hat klare mathematische Entsprechung im Bereich der analytischen Zahlentheorie und Quantenstatistik.

\end{document}