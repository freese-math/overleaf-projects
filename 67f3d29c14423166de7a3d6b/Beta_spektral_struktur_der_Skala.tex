\documentclass[a4paper,12pt]{article}
\usepackage[utf8]{inputenc}
\usepackage{amsmath,amssymb}
\usepackage{graphicx}
\usepackage{bm}
\usepackage{hyperref}
\usepackage{physics}
\usepackage{caption}
\usepackage{mathtools}

\title{Spektrale Struktur der Beta-Skala: \\ Empirisches Modell und theoretische Motivation}
\author{Autor: T.Freese}
\date{\today}

\begin{document}

\maketitle

\section{Einleitung}

In Teil 1 dieser Arbeit wurde ein empirisches Modell für die sogenannte Beta-Skala $\beta(n)$ aufgestellt, das eine bemerkenswert präzise Approximation im Bereich großer $n$ bietet. Das Modell basiert auf folgender analytischer Struktur:

\begin{equation}
\boxed{
\beta(n) = A + \frac{a}{\log n} + b \cos(2\pi f n + \varphi) + c \cdot n
}
\label{eq:betamodell}
\end{equation}

Ziel dieses Abschnitts ist es, die Struktur von Gleichung~\eqref{eq:betamodell} theoretisch zu motivieren und ihre Verbindung zu bekannten analytischen Objekten aus der Zahlentheorie und Spektralanalyse herzustellen.

\section{Komponentenanalyse des Modells}

\subsection*{(i) Der Logarithmische Term \boldmath$\frac{a}{\log n}$}

Dieser Ausdruck erinnert formal an klassische Abschätzungen in der Primzahldichte, etwa in der Form
\[
\pi(n) \sim \frac{n}{\log n},
\]
sowie an Korrekturterme im Zusammenhang mit der Chebyshev-Funktion $\theta(n)$ oder der Mangoldt-Funktion $\Lambda(n)$. Die logarithmische Abschwächung nahe $n \to \infty$ ist charakteristisch für Prozesse mit asymptotischer Regularisierung.

\subsection*{(ii) Der Cosinus-Term \boldmath$b \cos(2\pi f n + \varphi)$}

Die Existenz eines harmonischen Oszillationsterms über der natürlichen Zahlenskala $n$ ist außergewöhnlich und weist auf eine zugrunde liegende spektrale Struktur hin. Solche Terme treten typischerweise auf in:
\begin{itemize}
  \item der Fourieranalyse von periodischen Gitterfunktionen,
  \item quantenmechanischen Zustandsdichten,
  \item oder in der Paar-Korrelationsstruktur der Nullstellen der Riemannschen Zetafunktion (Montgomery-Odlyzko).
\end{itemize}

Insbesondere erinnert die Phase $\varphi$ an Verschiebungen in nichtkommutativen geometrischen Modellen oder die Modulationsstruktur von Theta-artigen Funktionen.

\subsection*{(iii) Der lineare Term \boldmath$c \cdot n$}

Dieser lineare Drift kompensiert die Residuenstruktur des Fits über größere Skalen. Er könnte als Effekt einer noch nicht vollständig eliminierbaren globalen Trendkomponente interpretiert werden, z.B. in Zusammenhang mit einem effektiven Massenspektrum.

\section{Hypothese: Verbindung zur Siegel-Theta-Funktion}

Der Cosinus-Term legt nahe, dass die zugrunde liegende Dynamik durch eine Transformation einer Theta-artigen Struktur erzeugt wurde. Die Siegel-Theta-Funktion besitzt eine Darstellung ihrer Mellin-Transformation in der Form

\begin{equation}
\mathcal{M}[\Theta(t)](s) \sim \cos\left(s \cdot \arctan\left(\frac{\omega}{\alpha}\right)\right) \cdot \Gamma(s),
\end{equation}

wobei $\omega$ und $\alpha$ Parameter der spektralen Dichte bezeichnen. Die Ähnlichkeit zum Cosinus-Term aus~\eqref{eq:betamodell} ist frappierend.

\section{Ausblick}

Im nächsten Abschnitt wird untersucht, ob das Modell~\eqref{eq:betamodell} als asymptotischer Ausdruck einer Mellin-Transformation einer geeigneten Funktion (z.B. einer Theta-Struktur) hergeleitet werden kann. Dies könnte nicht nur das empirische Modell theoretisch fundieren, sondern auch neue Einblicke in die spektrale Interpretation der $\beta(n)$-Werte liefern.

\end{document}