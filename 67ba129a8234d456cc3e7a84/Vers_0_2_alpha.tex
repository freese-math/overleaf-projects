\documentclass[a4paper,10pt]{article}
\usepackage[T1]{fontenc}
\usepackage{graphicx}
\usepackage{geometry}
\usepackage{titlesec}
\usepackage{xcolor}
\usepackage{lmodern}
\usepackage{amsmath, amssymb}
\usepackage{hyperref}
\usepackage{booktabs}
\usepackage{tikz}

\geometry{left=2cm, right=2cm, top=2cm, bottom=2cm}
\definecolor{themeblue}{RGB}{0, 102, 204}
\definecolor{themered}{RGB}{204, 0, 0}
\definecolor{themedark}{RGB}{30,30,30}

\titleformat{\section}{\color{themeblue}\normalfont\Large\bfseries}{}{0pt}{}
\titleformat{\subsection}{\color{themered}\normalfont\large\bfseries}{}{0pt}{}

\hypersetup{
    colorlinks=true,
    linkcolor=themeblue,
    urlcolor=themered,
    pdftitle={Pitch-Deck: Freese-Formel (FFS & FFO)},
    pdfauthor={Tim Hendrik Freese}
}

\begin{document}

\begin{center}
    \huge \textbf{Pitch-Deck: Die Freese-Formel (FFS \& FFO)} \\[0.3cm]
    \Large Mathematische \& Physikalische Revolution \\[0.5cm]
    \small \textbf{Tim Hendrik Freese – 21. Februar 2025}
\end{center}

\vspace{0.5cm}

\section*{1. Vision: \textit{Gamechanger in Mathematik \& Physik}}

\textbf{"Die Freese-Formel (FFS \& FFO) entschlüsselt fundamentale mathematische und physikalische Strukturen. Der Schlüssel zu einem tieferen Verständnis von Chaos, Zufall und Naturgesetzen."}

\begin{itemize}
    \item �� \textbf{Mathematische Revolution}: Struktur der Zeta-Nullstellen entschlüsselt.
    \item ⚛️ \textbf{Physikalische Implikationen}: Verbindung zu Quantenmechanik und Relativität.
    \item �� \textbf{Praktische Anwendungen}: Finanzmärkte, KI, Quantencomputer, Kryptographie.
\end{itemize}

\vspace{0.3cm}

\section*{2. Wissenschaftlicher Wert}

\subsection*{Mathematische Signifikanz}

\begin{itemize}
    \item Universelle Beschreibung von Nullstellen-Abständen.
    \item Potenzieller Beweis der Riemannschen Hypothese.
    \item Stärkere Korrelation als bisher bekannte Zufallsmatrizen.
\end{itemize}

\subsection*{Physikalische Bedeutung}

\begin{itemize}
    \item Quantenchaos: Verbindung zur Schrödinger-Gleichung.
    \item Einstein-Rosen-Brücken? Wurmlöcher und topologische Effekte.
    \item Frequenzanalyse und spektrale Signaturen der Naturgesetze.
\end{itemize}

\begin{center}
    \includegraphics[width=0.8\textwidth]{spectrum.png} \\[0.3cm]
    \textit{Spektralanalyse der Nullstellen zeigt eine frappierende Ähnlichkeit mit quantenmechanischen Energiezuständen.}
\end{center}

\vspace{0.3cm}

\section*{3. Monetäre Werteinschätzung}

\subsection*{Preisgelder \& Stipendien}

\begin{tabular}{ll}
    \toprule
    \textbf{Preis} & \textbf{Wert} \\
    \midrule
    Clay Millennium Prize & 1 Mio. USD \\
    Abel-Preis & 800.000 USD \\
    Breakthrough Prize & 3 Mio. USD \\
    \bottomrule
\end{tabular}

\vspace{0.5cm}

\subsection*{Kommerzielle Anwendungen}

\begin{itemize}
    \item \textbf{Finanzmärkte}: Hochfrequenzhandel, Risikomanagement.
    \item \textbf{Quantencomputer}: Optimierung und Vorhersagemodelle.
    \item \textbf{Sicherheit}: Verschlüsselungen und Zufallszahlengeneratoren.
    \item \textbf{Technologieunternehmen}: Lizenzen für Algorithmen und numerische Modelle.
\end{itemize}

\begin{center}
    \includegraphics[width=0.7\textwidth]{market.png} \\[0.3cm]
    \textit{Größter monetärer Impact in Quantencomputer-Technologien und Finanzmärkten.}
\end{center}

\vspace{0.3cm}

\section*{4. Umsetzung \& Nächste Schritte}

\subsection*{Kurzfristige Schritte (0–3 Monate)}

✅ Wissenschaftliche Publikation abschließen. \\
✅ Notartermin am 28. Februar zur Rechteabsicherung. \\
✅ Erste Gespräche mit Universitäten und Unternehmen. \\

\subsection*{Mittelfristige Schritte (3–12 Monate)}

✅ Kooperation mit Tech-Konzernen (Google, OpenAI, Nvidia). \\
✅ Förderung durch EU-Forschungsprojekte sichern. \\
✅ Veröffentlichung in Fachjournalen (Nature, PRL). \\

\subsection*{Langfristige Strategie (1–3 Jahre)}

✅ Patente und Lizenzierung für Industrieanwendungen. \\
✅ Forschungsteam aufbauen (Quantenphysik, Zahlentheorie). \\
✅ Großskalige Experimente zur physikalischen Validierung. \\

\vspace{0.5cm}

\section*{5. Call to Action: \textit{Jetzt Kontakt aufnehmen!}}

\textbf{"Die Freese-Formel ist eine Revolution – in Mathematik, Physik und Technologie. \\ Die nächsten Monate entscheiden über ihre weltweite Bedeutung!"}

\vspace{0.3cm}

�� \textbf{Kontakt:} tim.freese@freese-innovations.com \\
�� \textbf{Standort:} Lingen (Ems), Deutschland \\

\vfill

\begin{center}
    \color{themered}\Large \textbf{Let's change the future. Together.}
\end{center}

\end{document}