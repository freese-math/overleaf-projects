\documentclass[a4paper,12pt]{article}
\usepackage{amsmath,amssymb,amsthm,graphicx,hyperref,geometry}
\geometry{a4paper, margin=1in}
\usepackage{xcolor}
\hypersetup{colorlinks,linkcolor=blue,urlcolor=blue}

\title{\textbf{Mathematische Herleitung der Freese-Formel (FFS \& FFO)} \\ 
\large Inklusive Operator-Darstellung, Invarianten \& Hypothetische Naturkonstante f}
\author{Tim Hendrik Freese}
\date{\today}

\begin{document}
\maketitle
\tableofcontents
\newpage

\section{Einleitung}
Die folgende Herleitung beschreibt die **Freese-Formel Standard (FFS)** sowie ihre **oszillierende Variante (FFO)**. Dabei wird der aktuelle **numerisch optimierte Zustand** detailliert hergeleitet.  
Besonderes Augenmerk liegt auf möglichen Invarianten, einer möglichen **neuen Naturkonstante $f$**, Operator-Darstellungen und allgemeinen Erweiterungen.

\section{Definition der Freese-Formel (FFS \& FFO)}
\subsection{Standardform (FFS)}
Die Standardform der Freese-Formel beschreibt die Nullstellen-Abstände der Riemannschen Zeta-Funktion in geschlossener Form:
\begin{equation}
    \text{FFS}(n) = A \cdot n^{-1/2} + B \cdot n^{-1} + C
\end{equation}
mit den aktuell numerisch optimierten Parametern:
\begin{align*}
    A &= 25.953416, \\
    B &= -28.796961, \\
    C &= 0.529483.
\end{align*}
Der mittlere Fehler beträgt $0.185513$, was auf eine **sehr gute Approximation** hinweist.

\subsection{Oszillierende Variante (FFO)}
Die erweiterte Version beinhaltet eine oszillierende Korrektur:
\begin{equation}
    \text{FFO}(n) = A \cdot n^{-1/2} + B \cdot n^{-1} + C + w \cos(nw + \phi)
\end{equation}
mit den optimierten Parametern:
\begin{align*}
    A &= 25.967831, \quad B = -28.808266, \quad C = 0.529463, \\
    w &= 0.009999, \quad \phi &= 1.638759.
\end{align*}
Hier bleibt der Fehler fast identisch bei **$0.185596$**, was darauf hindeutet, dass die Oszillation nur eine marginale Verbesserung bringt.

\section{Ableitungen und Operator-Darstellung}
\subsection{Ableitungen der FFS}
\textbf{1. Ableitung} (Geschwindigkeit der Abstandsänderung):
\begin{equation}
    \frac{d}{dn} \text{FFS}(n) = -\frac{A}{2 n^{3/2}} - \frac{B}{n^2}
\end{equation}
\textbf{2. Ableitung} (Beschleunigung der Abstandsänderung):
\begin{equation}
    \frac{d^2}{dn^2} \text{FFS}(n) = \frac{3A}{4 n^{5/2}} + \frac{2B}{n^3}
\end{equation}

\subsection{Ableitungen der FFO}
Die Ableitungen der oszillierenden Variante beinhalten zusätzliche Terme:
\begin{align}
    \frac{d}{dn} \text{FFO}(n) &= -\frac{A}{2 n^{3/2}} - \frac{B}{n^2} - w^2 \sin(nw + \phi) \\
    \frac{d^2}{dn^2} \text{FFO}(n) &= \frac{3A}{4 n^{5/2}} + \frac{2B}{n^3} - w^3 \cos(nw + \phi).
\end{align}

\subsection{Operator-Darstellung}
Falls eine Operator-Darstellung existiert, könnte sie folgende Form haben:
\begin{equation}
    \hat{F} = A \cdot \hat{N}^{-1/2} + B \cdot \hat{N}^{-1} + C + w \cos(\hat{N} w + \phi)
\end{equation}
mit dem Operator $\hat{N}$, der die Nullstellen-Indizes beschreibt.

\section{Invarianten und Vermutete Naturkonstante f}
\subsection{Hypothese zur Naturkonstante f}
Ein auffälliges Muster in den Optimierungen ist das Auftreten einer universellen **Skalierungsgröße f**, die möglicherweise eine tiefere Verbindung zwischen Primzahlen und Zeta-Nullstellen darstellt.

Wir postulieren:
\begin{equation}
    f = \lim_{n \to \infty} \frac{\text{FFS}(n)}{n^{-1/2}}.
\end{equation}
Die aktuelle numerische Approximation deutet auf:
\begin{equation}
    f \approx 25.95.
\end{equation}

Falls $f$ tatsächlich eine Invariante ist, wäre sie als eine **fundamentale Konstante** der Nullstellenverteilung der Zeta-Funktion zu betrachten.

\section{Hypothetische Erweiterungen \& Verallgemeinerungen}
\subsection{Verallgemeinerte Operator-Darstellung}
Eine noch allgemeinere Form könnte folgende Gestalt annehmen:
\begin{equation}
    \hat{F} = A \cdot \hat{N}^{-\alpha} + B \cdot \hat{N}^{-\beta} + C + w \cos(\hat{N} w + \phi).
\end{equation}
Hierbei wären $\alpha, \beta$ freie Parameter, die möglicherweise numerisch bestimmt werden könnten.

\subsection{Differentialgleichung für die FFS/FFO}
Falls eine zugrunde liegende Differentialgleichung existiert, müsste sie die Form besitzen:
\begin{equation}
    \frac{d^2F}{dn^2} + P(n) \frac{dF}{dn} + Q(n) F = 0.
\end{equation}
Wobei $P(n)$ und $Q(n)$ noch zu bestimmen wären.

\section{Fazit \& Offene Fragen}
\begin{itemize}
    \item Ist die Oszillation in der FFO rein numerisch oder fundamental?
    \item Ist $f$ tatsächlich eine **neue Naturkonstante**?
    \item Welche Verbindung gibt es zur Riemannschen Vermutung?
\end{itemize}

\end{document}