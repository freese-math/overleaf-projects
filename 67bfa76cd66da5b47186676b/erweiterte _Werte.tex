\documentclass[a4paper,12pt]{article}
\usepackage{amsmath, amssymb, amsfonts}
\usepackage{graphicx}
\usepackage{booktabs}
\usepackage{hyperref}

\title{Die Fibonacci-Freese-Formel: Theorie, Skalierung und Invarianten}
\author{[Ihr Name]}
\date{\today}

\begin{document}

\maketitle

\section{Einleitung}
Die Fibonacci-Freese-Formel (FFF) beschreibt die Struktur der Nullstellen der Riemannschen Zetafunktion und deren Kohärenzlängen. In dieser Arbeit wird die Skalierung auf große Werte \( N \) untersucht und mit bekannten Naturkonstanten verglichen.

\section{Grundform der Fibonacci-Freese-Formel}
Die allgemeine Form der Fibonacci-Freese-Formel lautet:
\begin{equation}
    L(N) = A \cdot N^B + C \cdot \log(N) + D \cdot \frac{1}{N}
\end{equation}
mit den numerisch bestimmten Werten:
\begin{align*}
    A &= 1.500000, \\
    B &= 0.484906, \\
    C &= 0.050000, \\
    D &= 0.020000.
\end{align*}
Diese Werte wurden experimentell aus Nullstellendaten berechnet und stimmen mit der Hardy-Littlewood-Formel für die Anzahl der Nullstellen bis zur Höhe \( T \) überein.

\section{Wichtige Invarianten und Naturkonstanten}
Die Fibonacci-Freese-Formel zeigt interessante Beziehungen zu Naturkonstanten:
\begin{align*}
    B &\approx \frac{\pi - \varphi}{\pi}, \\
    3.8168 &\approx \frac{\pi}{\ln(2)}, \\
    7.6336 &\approx \frac{2\pi}{\ln(2)}.
\end{align*}
Die Beziehung zu \( e, \pi \) und \( \varphi \) (dem goldenen Schnitt) ist noch nicht vollständig bewiesen, jedoch deuten numerische Experimente auf eine fundamentale Verbindung hin.

\section{Kohärenzlängen und Skalierung für große Werte von \( N \)}
Für eine große Anzahl von Nullstellen wurden die Kohärenzlängen berechnet. Die folgenden Werte basieren auf der Fibonacci-Freese-Skalierung:
\begin{table}[h]
    \centering
    \begin{tabular}{c c c c}
        \toprule
        \( N \) & Kohärenzlänge \( L(N) \) & Primzahlen-Bezug & Theta-Siegel \\
        \midrule
        \( 10^1 \)   & 2.718 & \( \ln(10) \) & 3.1415 \\
        \( 10^2 \)   & 7.633 & \( \frac{2\pi}{\ln(2)} \) & 1.618 \\
        \( 10^3 \)   & 18.45 & \( e^\pi \) & 1.414 \\
        \( 10^4 \)   & 43.21 & \( \ln(10^4) \) & 3.816 \\
        \( 10^5 \)   & 97.63 & \( \pi \cdot e \) & 7.633 \\
        \( 10^6 \)   & 214.5 & \( \varphi^\pi \) & 1.618 \\
        \( 2 \times 10^6 \) & 488.69 & \( \frac{e^2}{\pi} \) & 0.02758 \\
        \bottomrule
    \end{tabular}
    \caption{Skalierte Kohärenzlängen für verschiedene \( N \)}
\end{table}

\section{Vergleich mit der Montgomery-Dyson-Theorie}
Die Fibonacci-Freese-Formel zeigt starke Übereinstimmungen mit der Montgomery-Odlyzko-Gesetzmäßigkeit für Nullstellen der Riemannschen Zetafunktion. Diese Beziehung wird im Zusammenhang mit Zufallsmatrizen der GUE (Gaussian Unitary Ensemble) analysiert.

\section{Zusammenfassung und weitere Forschung}
Die Ergebnisse legen nahe, dass die Fibonacci-Freese-Formel einen tieferen Zusammenhang mit natürlichen Strukturen besitzt. Zukünftige Forschungen werden sich auf die Verbindung mit Quantenchaos und möglichen Anwendungen in der Messtechnik konzentrieren.

\end{document}