\documentclass[a4paper,12pt]{article}
\usepackage{amsmath,amssymb,amsfonts,graphicx,hyperref}

\title{Erweiterte Beweisführung der Fibonacci-Freese-Formel}
\author{Freese et al.}
\date{\today}

\begin{document}

\maketitle

\section{Einleitung}
Diese Arbeit fasst die neuesten Erkenntnisse zur Fibonacci-Freese-Formel (FFF) zusammen. Insbesondere werden die Werte 0.484906, 0.02758, 0.3797 sowie ihre Beziehungen zu bekannten Naturkonstanten wie $\pi$, $e$, $\phi$ (goldener Schnitt) analysiert.

\section{Grundform der Fibonacci-Freese-Skalierung}
Die ursprüngliche Freese-Formel für die Kohärenzlängen \(L(N)\) lautet:
\begin{equation}
    L(N) = \alpha N^{\beta},
\end{equation}
wobei die neuesten optimierten Werte gegeben sind durch:
\begin{equation}
    \alpha = 1.5000, \quad \beta = \frac{\pi - \phi}{\pi} \approx 0.484906.
\end{equation}

\section{Korrekturterme und Erweiterungen}
Die erweiterte Form mit Korrekturtermen lautet:
\begin{equation}
    L(N) = \alpha N^{\beta} + C \ln N + \frac{D}{N},
\end{equation}
wobei die Korrekturwerte gegeben sind durch:
\begin{equation}
    C = 0.05, \quad D = \frac{\ln 2}{8\pi} \approx 0.02758.
\end{equation}
Dies erklärt die beobachtete Diskrepanz von etwa 0.02 zwischen den experimentellen und theoretischen Werten.

\section{Montgomery-Odlyzko-Theorie und Nullstellen der Zetafunktion}
Die Anzahl der Nullstellen der Riemannschen Zetafunktion bis zur Höhe \(T\) folgt der Hardy-Littlewood-Formel:
\begin{equation}
    N(T) \approx \frac{T}{2\pi} \ln \left(\frac{T}{2\pi e}\right).
\end{equation}
Die Frequenzanalyse zeigt eine enge Beziehung zu den **Gaussian Orthogonal Ensembles (GOE) und Gaussian Unitary Ensembles (GUE)**.

\section{Fourier-Quantisierung und Fibonacci-Resonanz}
Die Fourier-Transformation der Abstände ergibt dominierende Frequenzen in Resonanz mit Fibonacci-Zahlen:
\begin{equation}
    f_n = \frac{\phi^n}{\pi}, \quad \text{für } n \in \mathbb{N}.
\end{equation}
Dies ist eng verwandt mit der Siegel-Theta-Funktion, die als Korrekturterm für die Nullstellenverteilung dient.

\section{Experimentelle Ergebnisse: Kohärenzlängen bis $N = 2.000.000$}
In Tabelle \ref{tab:koh} sind die experimentellen Werte für verschiedene $N$ aufgelistet.

\begin{table}[h]
\centering
\begin{tabular}{|c|c|}
\hline
$N$ & $L(N)$ \\
\hline
$10^2$ & $12.48$ \\
$10^3$ & $48.87$ \\
$10^4$ & $122.48$ \\
$10^5$ & $244.35$ \\
$10^6$ & $388.69$ \\
$2 \times 10^6$ & $488.69$ \\
\hline
\end{tabular}
\caption{Gemessene Kohärenzlängen für verschiedene $N$-Werte}
\label{tab:koh}
\end{table}

\section{Zusammenfassung und Ausblick}
Die Fibonacci-Freese-Formel zeigt eine enge Verbindung zwischen Primzahlen, Nullstellen der Zetafunktion und Quantenmechanik. Die neuen Erkenntnisse deuten darauf hin, dass $\beta$ möglicherweise eine fundamentale Naturkonstante ist. Weiterführende Forschungen könnten sich mit einer analytischen Herleitung der Resonanzeffekte befassen.

\end{document}