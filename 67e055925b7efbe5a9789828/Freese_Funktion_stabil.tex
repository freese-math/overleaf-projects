\documentclass[12pt]{article}
\usepackage{amsmath, amssymb, amsfonts}
\usepackage{geometry}
\usepackage{physics}
\usepackage{graphicx}
\geometry{a4paper, margin=2.5cm}
\title{Die Freese-Zeta-Funktion \\[1ex] \large Eine strukturierte Erweiterung der Riemannschen Zeta-Funktion}
\author{Erstentwurf – Prime Zeta Pro}
\date{\today}

\begin{document}
\maketitle

\section*{1. Einführung}
Die \textbf{Freese-Zeta-Funktion} $\zeta_{\mathcal{F}}(s)$ ist eine neuartige, modulierte Dirichlet-Reihe, abgeleitet aus empirisch hochpräzise angepassten Zeta-Nullstellen über eine spektral strukturierte Formel der Form:

\begin{equation}
L(n) = A \cdot n^{\beta} + C \cdot \log n + D \cdot n^{-\gamma} \cdot \sin(\omega \log n + \phi)
\label{eq:freese_function}
\end{equation}

Diese sogenannte \textit{Freese-Funktion} bildet die Grundlage einer neuen Zeta-Variante.

\vspace{1em}

\section*{2. Definition der Freese-Zeta-Funktion}
\begin{equation}
\zeta_{\mathcal{F}}(s) := \sum_{n=1}^{\infty} \frac{1}{\left( L(n) \right)^s}
= \sum_{n=1}^{\infty} \frac{1}{\left( A n^{\beta} + C \log n + D n^{-\gamma} \sin(\omega \log n + \phi) \right)^s}
\end{equation}

Diese Funktion konvergiert für $\Re(s)$ ausreichend groß, analog zur klassischen Dirichlet-Reihe.

\vspace{1em}

\section*{3. Parameter (optimiert)}
Basierend auf GPU-gestützter PyTorch-Optimierung:

\begin{align*}
A &= 1.0585 \\
\beta &= 0.9547 \\
C &= 1.1096 \\
D &= 1.9548 \\
\gamma &= -0.7725 \\
\omega &= 0.2476 \\
\phi &= -0.6965
\end{align*}

\vspace{1em}

\section*{4. Zusammenhang zur Riemannschen Hypothese}
Die Funktion $L(n)$ wurde so konstruiert, dass $L(n) \approx \gamma_n$, also den imaginären Teilen der Nullstellen der klassischen Riemannschen Zetafunktion. Daraus ergibt sich für die Spurformel:

\begin{equation}
\sum_n f(L(n)) \approx \sum_n f(\gamma_n)
\end{equation}

Für $f(t) = e^{-st}$ ergibt sich eine alternative Darstellung des Operator-Traces.

\vspace{1em}

\section*{5. Operatorstruktur}
Die Freese-Funktion ist interpretierbar als Eigenwertstruktur eines Operators $\hat{\mathcal{O}}$:

\[
\hat{\mathcal{O}} = \hat{H}_\beta + \hat{B} + \hat{D} + \hat{T} + \hat{L}
\]

mit $\text{Tr}(\hat{\mathcal{O}}) \approx \zeta_{\mathcal{F}}(s)$.

\vspace{1em}

\section*{6. Fazit}
Die Freese-Zeta-Funktion stellt eine numerisch präzise Approximation der nichttrivialen Nullstellenstruktur dar und lässt sich als operatorbasierte, strukturierte Variante der klassischen Zeta-Funktion auffassen – mit enger Verbindung zur Riemannschen Hypothese.

\end{document}