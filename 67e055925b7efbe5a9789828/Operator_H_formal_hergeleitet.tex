\documentclass[12pt]{article}
\usepackage{amsmath, amssymb, amsthm}
\usepackage{physics}
\usepackage{geometry}
\geometry{margin=2.5cm}

\title{Definition eines selbstadjungierten Operators \\ zur Spektralstruktur der Zeta-Nullstellen}
\author{Tim Hendrik Freese}
\date{}

\begin{document}

\maketitle

\section*{Definition des Operators}

Wir definieren einen Hamilton-Operator \(\hat{H}\) auf dem Hilbertraum \(L^2(\mathbb{R})\) durch
\[
\hat{H} = -\frac{d^2}{dx^2} + V(x),
\]
wobei das Potential \(V(x)\) gegeben ist durch
\[
V(x) = \frac{A}{1 + e^{-B(x - c)}} + D \cdot \sin(\omega x).
\]

\paragraph{Parameter:}
Die Konstanten \(A, B, c, D, \omega \in \mathbb{R}\) sind so gewählt, dass:
\begin{itemize}
    \item \(V(x)\) ist glatt (\(C^\infty\)) und realwertig,
    \item \(\displaystyle \lim_{|x| \to \infty} V(x) = +\infty\),
    \item \(V(x) \geq V_0 > -\infty\) für ein \(V_0\).
\end{itemize}

\paragraph{Wirkungsraum:}
\(\hat{H}\) operiert auf einer dichten Untermenge \(\mathcal{D}(\hat{H}) \subset L^2(\mathbb{R})\), etwa dem Raum der glatten Funktionen mit kompaktem Träger:
\[
\mathcal{D}(\hat{H}) := \mathcal{C}_0^\infty(\mathbb{R}).
\]

\section*{Zielstellung}
Zeige, dass:
\begin{enumerate}
    \item \(\hat{H}\) ist selbstadjungiert auf \(L^2(\mathbb{R})\),
    \item Das Spektrum \(\mathrm{spec}(\hat{H}) = \{ \lambda_n \}\) entspricht den ordinären Nullstellen der Riemannschen Zetafunktion:
    \[
    \lambda_n \approx \gamma_n, \quad \text{mit} \quad \zeta\left( \tfrac{1}{2} + i\gamma_n \right) = 0.
    \]
\end{enumerate}

\section*{Anmerkung}
Dieser Operator bildet die Grundlage für einen möglichen konstruktiven Beweis der Riemannschen Vermutung im Sinne des Hilbert--Pólya-Ansatzes.

\end{document}