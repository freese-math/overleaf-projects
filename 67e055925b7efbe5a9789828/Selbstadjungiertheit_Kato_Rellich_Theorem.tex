\documentclass{article}
\usepackage{amsmath, amssymb}
\usepackage{physics}

\title{Selbstadjungiertheit des Operators \(\hat{H}\)}
\author{}
\date{}

\begin{document}

\maketitle

\section*{Definition des Operators}
Gegeben sei der Operator
\[
\hat{H} = -\frac{d^2}{dx^2} + V(x)
\quad \text{auf} \quad \mathcal{H} = L^2(\mathbb{R}),
\]
mit einem reellwertigen Potential der Form:
\[
V(x) = \frac{A}{1 + e^{-B(x - c)}} + D\sin(\omega x),
\]
wobei \(A, B, c, D, \omega \in \mathbb{R}\).

\section*{Ziel}
Zeige: \(\hat{H}\) ist selbstadjungiert auf einem geeigneten Definitionsbereich \(D(\hat{H})\subset L^2(\mathbb{R})\).

\section*{Argumentation}
\begin{itemize}
    \item Der freie Operator \(\hat{H}_0 = -\frac{d^2}{dx^2}\) ist auf \(C_0^\infty(\mathbb{R})\) dicht definiert und selbstadjungierbar.
    \item Das Potential \(V(x)\) ist eine glatte (\(C^\infty\)) und beschränkte Funktion.
    \item Nach dem \textbf{Kato-Rellich-Theorem} gilt: Ist \(V(x)\) relativform-beschränkt bezüglich \(\hat{H}_0\) mit Schranke kleiner als 1, so ist \(\hat{H} = \hat{H}_0 + V(x)\) selbstadjungiert.
\end{itemize}

\textbf{Fazit:} Da \(V(x)\) beschränkt ist, folgt die Selbstadjungiertheit von \(\hat{H}\).

\end{document}