\section{Optimierte Freese-Struktur als Zeta-Approximation}
Die analysierte Form der Nullstellenverteilung basiert auf einer erweiterten Freese-Funktion:

\[
L(N) = A N^\beta + C \log N + D N^{-\gamma} \cdot \sin(\omega \log N + \phi)
\]

Diese Gleichung approximiert die Zeta-Nullstellen mit hoher Präzision. Die Parameter ergeben eine harmonische, logarithmisch skalierte Struktur, welche durch die Beta-Frequenz getragen wird. Dabei ergibt sich eine phasenmodulierte Oszillation mit:

\begin{itemize}
  \item \textbf{Frequenz:} \( \omega \approx 0.248 \)
  \item \textbf{Exponent:} \( \beta \approx 0.955 \)
  \item \textbf{Phase:} \( \phi \approx -0.696 \)
\end{itemize}

Diese Struktur bildet die Grundlage für die Definition des Beta-Operators im nächsten Abschnitt.