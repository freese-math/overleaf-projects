\documentclass[12pt]{article}
\usepackage{amsmath, amssymb, graphicx}
\usepackage{physics}
\usepackage{hyperref}
\usepackage{geometry}
\geometry{a4paper, margin=2.5cm}
\title{Theta-Simulation der Zeta-Funktion mittels Spurformel}
\author{Prime Zeta Pro}
\date{\today}

\begin{document}

\maketitle

\section*{Abstract}
Wir präsentieren eine analytisch-numerische Simulation der Riemannschen Zeta-Funktion mittels modulierter Theta-Funktion. Durch Einbettung der Spurformel in ein spektrales Framework mit Beta-Korrektur gelingt es, die Struktur der Nullstellen über eine harmonische Theta-Modulation darzustellen.

\section{Grundlagen}
Die klassische Theta-Funktion wird definiert als
\begin{equation}
    \theta(t) = \sum_{n=-\infty}^{\infty} e^{-\pi n^2 t},
\end{equation}
für $t > 0$. Die modulierte Version der Theta-Funktion mit Beta-Skalierung lautet
\begin{equation}
    \Theta_{\beta}(t) = \sum_{n=1}^{\infty} e^{-\pi n^2 t} \cdot e^{i \beta \log(n)},
\end{equation}
wobei $\beta$ als spektraler Skalierungsparameter wirkt.

\section{Zeta-Spurformel über Theta}
Die Spurformel erlaubt die Approximation der Zeta-Funktion entlang der kritischen Linie $\Re(s) = \frac{1}{2}$ durch:
\begin{equation}
    \zeta(s) \approx \int_{0}^{\infty} \Theta_{\beta}(t) \cdot t^{\frac{s}{2} - 1} \dd{t},
\end{equation}
wobei $\beta = 0.914$ als optimaler Beta-Wert zur Stabilisierung der Resonanzstruktur verwendet wird.

\section{Numerische Resultate}
Die numerische Integration zeigt, dass die modifizierte Theta-Funktion $\Theta_{\beta}(t)$ sowohl im Betrag als auch in der Phase eine stabile Struktur liefert, welche mit der Verteilung der Zeta-Nullstellen konsistent ist.

\begin{figure}[h!]
    \centering
    \includegraphics[width=0.7\textwidth]{theta_simulation.png}
    \caption{Amplitude und Phase der modulierten Theta-Funktion $\Theta_{\beta}(t)$}
\end{figure}

\section{Fazit}
Die modifizierte Theta-Spurformel mit harmonischer Beta-Modulation bildet die spektrale Struktur der Zeta-Funktion effizient ab. Dies legt nahe, dass die Riemannsche Hypothese eine Konsequenz dieser spektralen Kohärenz sein könnte.

\end{document}