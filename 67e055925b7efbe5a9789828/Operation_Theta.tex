\documentclass[12pt]{article}
\usepackage{amsmath, amssymb}
\usepackage{geometry}
\geometry{a4paper, margin=2.5cm}
\title{Operation Theta\\[1ex] \large Modulare Struktur, Beta-Korrektur und Zeta-Verbindung}
\author{Prime Zeta Pro}
\date{\today}

\begin{document}
\maketitle

\section*{1. Die klassische Jacobi-Theta-Funktion}

Die Jacobi-Theta-Funktion (ohne Charakter) ist gegeben durch:
\begin{equation}
\theta(t) := \sum_{n=-\infty}^{\infty} e^{-\pi n^2 t}, \qquad t > 0
\end{equation}

Diese Funktion erfüllt die modulare Transformation:
\begin{equation}
\theta(t) = \frac{1}{\sqrt{t}} \, \theta\left( \frac{1}{t} \right)
\end{equation}

\vspace{1em}

\section*{2. Verbindung zur Riemannschen Zetafunktion}

Über die Theta-Funktion ergibt sich eine Darstellung der Zetafunktion:
\begin{equation}
\pi^{-s/2} \Gamma\left( \frac{s}{2} \right) \zeta(s) = \int_0^\infty \left( \theta(t) - 1 \right) t^{s/2} \, \frac{dt}{t}
\end{equation}

Dies ist die Grundlage für die Funktionalgleichung der Zeta-Funktion.

\vspace{1em}

\section*{3. Erweiterte Theta-Funktion mit Beta-Modulation}

Motiviert durch die empirische Struktur der Zeta-Nullstellen definieren wir eine modulierte Theta-Funktion:

\begin{equation}
\Theta_{\beta}(t) := \sum_{n=-\infty}^{\infty} 
\exp\left( -\pi n^2 t + i \beta \log(n^2 + 1) \right)
\label{eq:theta_beta}
\end{equation}

Hier moduliert der Parameter $\beta = \beta(t)$ das spektrale Verhalten.

\vspace{1em}

\section*{4. Ziel: Operatorbasierte Spurformel}

Die erweiterte Spurformel lautet:

\begin{equation}
\mathrm{Tr} \left( e^{-t \hat{H}_\Theta} \right) = \Theta_\beta(t)
\end{equation}

Daraus ergibt sich eine modifizierte Integralform der Zeta-Funktion:

\begin{equation}
\zeta_{\mathcal{F}}(s) := \int_0^\infty \left( \Theta_\beta(t) - 1 \right) t^{s/2} \frac{dt}{t}
\label{eq:zeta_freese_theta}
\end{equation}

Diese Darstellung enthält spektrale und phasische Informationen über die Zeta-Nullstellen.

\vspace{1em}

\section*{5. Interpretation}

Die Beta-modulierte Theta-Funktion $\Theta_\beta(t)$ kann als Trace eines spektralen Operators interpretiert werden, dessen Eigenwertestruktur die Nullstellenstruktur der Zeta-Funktion reflektiert. Die Phase $i \beta \log(n^2 + 1)$ wirkt als spinartige Frequenzmodulation im log-Raum.

\end{document}