\documentclass{article}
\usepackage{amsmath, amssymb}
\usepackage{graphicx}
\usepackage{physics}
\usepackage{hyperref}

\title{Theta-Modulierte Spurformel zur Zeta-Funktion}
\author{Prime Zeta Pro Framework}
\date{}

\begin{document}

\maketitle

\section*{1. Definition der modulierten Theta-Funktion}
Wir definieren die modulierte Theta-Funktion durch:
\[
\Theta_\beta(t) := \sum_{n=1}^{\infty} e^{-\pi n^2 t} \cdot e^{i \beta \log n} = \sum_{n=1}^{\infty} \frac{e^{-\pi n^2 t}}{n^{-i\beta}}
\]
Dabei erzeugt die Modulation mit \(\beta\) eine spektrale Verschiebung entlang des Frequenzraums.

\section*{2. Spurformel-Integration}
Die zugehörige Spurformel ergibt sich durch Mellin-Transformation:
\[
\zeta_F(s) := \int_0^\infty \Theta_\beta(t) \cdot t^{\frac{s}{2} - 1} \, dt
\]
Mit Austausch von Summation und Integration:
\[
\zeta_F(s) = \sum_{n=1}^{\infty} \frac{1}{n^{i\beta}} \int_0^\infty e^{-\pi n^2 t} t^{\frac{s}{2} - 1} \, dt
\]
Die innere Integral ist die Gamma-Funktion mit Substitution \(u = \pi n^2 t\):
\[
\int_0^\infty e^{-\pi n^2 t} t^{\frac{s}{2} - 1} dt = \frac{1}{(\pi n^2)^{s/2}} \Gamma\left( \frac{s}{2} \right)
\]
Somit folgt:
\[
\zeta_F(s) = \Gamma\left( \frac{s}{2} \right) \cdot \pi^{-s/2} \sum_{n=1}^{\infty} \frac{1}{n^{s - i\beta}} = \Gamma\left( \frac{s}{2} \right) \cdot \pi^{-s/2} \cdot \zeta(s - i\beta)
\]

\section*{3. Interpretation}
Diese Form erlaubt eine modulierte Fortsetzung der Riemannschen Zeta-Funktion:
\[
\zeta_F(s) \propto \zeta(s - i\beta)
\]
Sie zeigt explizit, wie sich die Modulation \(\beta\) auf die analytische Struktur und die Nullstellenverteilung auswirkt.

\section*{4. Anwendung}
Die Struktur ermöglicht eine Analyse der Nullstellenbewegung unter Variation von \(\beta\), insbesondere:
\begin{itemize}
  \item Stabilisierung auf der kritischen Linie \(\Re(s) = \frac{1}{2}\)
  \item Zusammenhang zur Euler-Freese-Identität \(e^{i \pi \beta} + 1 = 0\)
  \item Visualisierung des Theta-Spektrums und der Resonanzfrequenzen
\end{itemize}

\end{document}