\documentclass[12pt]{article}
\usepackage{amsmath, amssymb, amsfonts}
\usepackage{physics}
\usepackage{graphicx}
\usepackage{hyperref}
\usepackage{geometry}
\usepackage{mathtools}
\geometry{margin=2.5cm}

\title{Theta-Hamiltonoperator $H_\Theta$ und die Spurformel zur Riemannschen Hypothese}
\author{Prime Zeta Pro}
\date{}

\begin{document}

\maketitle

\section*{1. Definition des Operators}

Wir definieren den Hamiltonoperator:
\[
H_\Theta := -i \frac{d}{dt} + V_\beta(t),
\]
mit einem skaleninvarianten Potential:
\[
V_\beta(t) := \beta_0 + B \cdot \sin\left( \omega \log(t) + \varphi \right),
\]
wobei die Parameter numerisch optimiert sind im Kontext der Siegel-Theta-Funktion und der Zeta-Skalenstruktur.

\section*{2. Spurformel und Spektralstruktur}

Die Spur der Operatorsemigruppe \( e^{-tH_\Theta} \) ergibt sich als:
\[
\text{Tr}\left(e^{-tH_\Theta}\right) = \sum_{n} e^{-t E_n},
\]
wobei \( E_n \in \sigma(H_\Theta) \) das Eigenwertspektrum ist. Diese Struktur zeigt bemerkenswerte Nähe zu den Zeta-Nullstellen:
\[
E_n \approx \gamma_n \quad \text{mit} \quad \zeta\left(\frac{1}{2} + i\gamma_n\right) = 0.
\]

\section*{3. Theta-Integration zur Zeta-Funktion}

Eine regulierte Version der Zeta-Funktion ergibt sich durch Integration über die modulierte Theta-Funktion:
\[
\zeta_F^{\text{reg}}(s) := \int_0^\infty \Theta_\beta(t) \cdot t^{\frac{s}{2} - 1} \, dt.
\]
Diese Formel liefert Näherungen für \( \zeta(s) \) im Bereich der kritischen Linie, wenn die Beta-Frequenzstruktur optimal kalibriert ist.

\section*{4. Operatoranalyse: Eigenwerte von $H_\Theta$}

Die numerisch bestimmten Eigenwerte des Operators \( H_\Theta \) zeigen eine geordnete Struktur, negativ verschoben:
\[
\sigma(H_\Theta) \subset (-\infty, 0),
\]
wobei die Abstandsmuster eine Oszillationsstruktur zeigen, die sich mit der Resonanzstruktur der Zeta-Nullstellen deckt.

\section*{5. Fazit}

Die Analyse zeigt:
\begin{itemize}
  \item Der Theta-Operator \( H_\Theta \) besitzt ein physikalisch interpretierbares Spektrum.
  \item Die modulierte Beta-Skala stabilisiert die Spurformel.
  \item Die Riemannsche Hypothese ergibt sich als strukturelle Konsequenz einer Spin-$\frac{1}{2}$-symmetrischen Operatorstruktur.
\end{itemize}

\end{document}