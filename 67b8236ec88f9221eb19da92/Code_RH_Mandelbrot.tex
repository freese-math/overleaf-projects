\documentclass[a4paper,12pt]{article}
\usepackage{amsmath, amssymb, amsthm, graphicx, hyperref, geometry, listings}

\geometry{a4paper, margin=1in}

\title{Python-Code zur Analyse der Riemann-Nullstellen mit der Mandelbrotmenge}
\author{[Tim Hendrik Freese geboren am 26.02.1980 in Osnabrück]}
\date{Vor Ausführung in Google Colab: 21. Februar 2025, 06:30 Uhr, Lingen (Ems)}

\begin{document}

\maketitle

\section{Einleitung}
Dieses Dokument enthält den vollständigen Python-Code für eine **numerische Analyse** der Riemann-Nullstellen in Verbindung mit der Mandelbrotmenge.  
Ziel ist es, zu untersuchen, ob die Nullstellen der Zeta-Funktion in einer geometrischen Beziehung zur fraktalen Struktur der Mandelbrotmenge stehen.  

Die hier dokumentierte Version wurde vor der Ausführung in **Google Colab gesichert**, um die **zeitliche Priorität und meine Autorschaft** nachzuweisen.

\section{Python-Code}

\begin{lstlisting}[language=Python, caption=Python-Code zur Analyse der RH-Nullstellen in der Mandelbrotmenge]
# �� Importiere notwendige Bibliotheken
import numpy as np
import matplotlib.pyplot as plt
from scipy.special import zeta
from scipy.stats import gaussian_kde
from tqdm import tqdm

# ✅ Mandelbrot-Funktion
def mandelbrot(c, max_iter=100):
    z = c
    for n in range(max_iter):
        if abs(z) > 2:
            return n
        z = z**2 + c
    return max_iter

# ✅ Erzeuge die Mandelbrotmenge
def compute_mandelbrot(xmin, xmax, ymin, ymax, width, height, max_iter=100):
    x_vals = np.linspace(xmin, xmax, width)
    y_vals = np.linspace(ymin, ymax, height)
    mandelbrot_set = np.zeros((height, width))

    for i, y in enumerate(y_vals):
        for j, x in enumerate(x_vals):
            mandelbrot_set[i, j] = mandelbrot(complex(x, y), max_iter)
    
    return mandelbrot_set, x_vals, y_vals

# ✅ Riemann-Nullstellen laden
file_path = "/content/drive/MyDrive/zeros6.txt"  # Passe den Pfad an, falls nötig
nullstellen = np.loadtxt(file_path)

# ✅ Komplexe Nullstellen auf die Mandelbrotmenge projizieren
riemann_x = np.real(nullstellen)  # Realteil (sollte 0.5 sein)
riemann_y = np.imag(nullstellen)  # Imaginärteil

# ✅ Mandelbrotmenge berechnen (Ausschnitt nahe der kritischen Linie)
xmin, xmax, ymin, ymax = -2, 2, 0, 50
width, height = 1000, 500
mandelbrot_set, x_vals, y_vals = compute_mandelbrot(xmin, xmax, ymin, ymax, width, height, max_iter=100)

# ✅ Visualisierung
plt.figure(figsize=(10, 6))
plt.imshow(mandelbrot_set, extent=[xmin, xmax, ymin, ymax], cmap='inferno', aspect='auto')
plt.colorbar(label="Iterationen bis Divergenz")
plt.scatter(riemann_x, riemann_y, color='cyan', s=2, label="Riemann-Nullstellen")
plt.xlabel("Re(c)")
plt.ylabel("Im(c)")
plt.title("Mandelbrotmenge & Riemannsche Nullstellen")
plt.legend()
plt.show()
\end{lstlisting}

\section{Zeitstempel und Absicherung}
\begin{itemize}
    \item \textbf{Datum und Uhrzeit der Dokumentation:} 21. Februar 2025, 06:30 Uhr (vor Ausführung in Google Colab)
    \item \textbf{Ort:} Lingen (Ems), Deutschland
    \item \textbf{Zeugen:} Dietmar Freese, Tanja Freese, Dirk Freese, Wolfgang Tautorat, Radek Kolodziejczyk, Thomas Krieger, Merle Freese
    \item \textbf{Rechtsanwalt:} Michael Lito Schulte (seit dem 18.02.2025 in die Dokumentation involviert)
    \item \textbf{Notariatsreferenz:} Kanzlei Lindwehr, Notarin Sabrina Lindwehr
\end{itemize}

\section{Juristische Handlungsempfehlung}
Um die Autorschaft dieses Codes und seiner wissenschaftlichen Methodik rechtlich abzusichern, wird empfohlen:
\begin{enumerate}
    \item \textbf{Notarielle Beglaubigung dieses Dokuments mit Zeitstempel.}
    \item \textbf{Hinterlegung des Codes als digitale Schutzschrift.}
    \item \textbf{Klare Feststellung der menschlichen Urheberschaft (ohne KI-generierte Inhalte).}
    \item \textbf{Ggf. Eintragung des Codes als wissenschaftlicher Beitrag in relevanten Datenbanken.}
\end{enumerate}

\section{Unterschrift und notarielle Bestätigung}
\noindent Diese Verfügung tritt mit meiner eigenhändigen Unterschrift in Kraft.

\vspace{1.5cm}
\noindent \textbf{Unterschrift:} \\
\vspace{2cm}
\noindent [Tim Freese] \\
[Datum] \\
[Ort] \\

\vspace{2cm}
\noindent \textbf{Notarielle Bestätigung:} \\
\vspace{2cm}
\noindent Notar: [Sabrina Lindwehr] \\
\noindent Ort: [Lingen (Ems)] \\
\noindent Datum: [21.02.2025] \\

\end{document}