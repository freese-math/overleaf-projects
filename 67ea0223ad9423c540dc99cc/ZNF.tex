\documentclass[12pt]{article}
\usepackage[utf8]{inputenc}
\usepackage{amsmath, amssymb, amsfonts}
\usepackage{graphicx}
\usepackage{hyperref}
\usepackage{geometry}
\geometry{a4paper, margin=2.5cm}
\usepackage{lmodern}
\usepackage{physics}
\usepackage{xcolor}
\usepackage{caption}
\usepackage{tikz}
\usepackage{float}
\setlength{\parskip}{1em}
\setlength{\parindent}{0em}

\title{\textbf{Zeta Nova Freesiana: Spektralstruktur, Beta-Skala und Primzahlwellen}}
\author{Tim Hendrik Freese}
\date{März 2025}

\begin{document}

\maketitle

\section*{1. Einführung}

Die Zeta Nova Freesiana (ZNF) erweitert die klassische Riemannsche Zeta-Funktion durch spektrale, modulierte und operatorielle Komponenten. Sie verknüpft Primzahlsummen, Wellenfunktionen und analytische Struktur über einen neuen Zugriff:

\[
\zeta_F(s) := \sum_{n=1}^\infty \frac{1}{(A n^\beta + C \log n + B \sin(\omega n + \varphi))^s}
\]

Diese Funktion basiert auf der Idee, die harmonische Struktur der Zeta-Nullstellen durch eine Beta-Skala sowie eine modulierte Euler-Wellenfunktion zu rekonstruieren.

\section*{2. Euler-Wellenfunktion und Beta-Skala}

\subsection*{2.1 Definition der Euler-Welle}
Die modulierte Euler-Wellenfunktion lautet:
\[
\psi(x) = \sum_{p \leq p_N} \sin(x \log p)
\]
mit \(p\) Primzahlen. Sie zeigt Resonanzen bei ganzzahligen Vielfachen von \(\log p\) und entspricht einer überlagerten Frequenzstruktur.

\subsection*{2.2 Harmonische Parametrisierung:}
Die Beta-Werte \( \beta = \frac{m}{n} \) werden in eine Kreisbewegung eingebettet:
\[
H(\beta) = e^{i \pi \beta}
\]
und erzeugen spektrale Identitäten mit hoher Genauigkeit (Fehler \(\Delta < 10^{-4}\)).

\section*{3. Operatoransatz und spektrale Hypothese}

\subsection*{3.1 Der Operator \(\mathcal{D}_\mu\)}

Wir definieren:
\[
(\mathcal{D}_\mu f)(n) = \sum_{k=1}^{K} w_k \cdot \exp\left(-\frac{(n - \gamma_k)^2}{2\sigma^2} \right) \cdot f(n) + \lambda \sum_{d=1}^{n-1} \mu(d) \left\lfloor \frac{n}{d} \right\rfloor f(n)
\]

\textbf{Notation:}
\begin{itemize}
  \item \(\gamma_k\): Imaginärteile der Zeta-Nullstellen
  \item \(\mu(d)\): Möbius-Funktion
  \item \(w_k = \frac{1}{|\zeta'(\rho_k)|}\): spektrales Gewicht
\end{itemize}

\subsection*{3.2 Hypothese zur Selbstadjungiertheit}
Die Riemannsche Hypothese ist äquivalent zur Selbstadjungiertheit:
\[
\mathcal{D}_\mu = \mathcal{D}_\mu^\dagger \quad \text{in } \ell^2(\mathbb{N})
\]

\section*{4. Beta-Skala v5 und Fehleranalyse}

Die Beta-Skala ist eine geordnete Liste rationaler \(\beta\)-Werte, bei denen der Fehler in der Euler-Freese-Identität
\[
H(\beta) = e^{i\pi\beta} \overset{?}{=} \text{spektral gemessener Wert}
\]
minimiert wird.

Beispiel:
\begin{align*}
\beta &= \frac{7}{33300}, & \Delta &= 8.9 \times 10^{-5} \\
\beta &= \frac{1}{66600}, & \Delta &= 1.1 \times 10^{-4}
\end{align*}

\section*{5. Frequenzanalyse der Zeta-Strukturen}

Die Spektralanalyse zeigt dominante Frequenzen in \(\beta(n)\) und \(\psi(x)\) bei:

\begin{itemize}
  \item \(f = 0.001, 0.002, 0.007, \ldots\) in \(\beta(n)\)
  \item \(f = \log p / \pi\) in \(\psi(x)\)
\end{itemize}

Diese Überlappung begründet die strukturelle Kopplung von Primzahlschwingungen und Beta-Modulation.

\section*{6. Fazit und Ausblick}

Die Zeta Nova Freesiana interpretiert die Nullstellenstruktur der Zeta-Funktion über:

\begin{itemize}
  \item Spektrale Operatoren mit Beta-Frequenzen
  \item Harmonische Interferenz aus Primzahlen
  \item Fehleranalytische Optimierung der Zeta-Näherung
\end{itemize}

Ziel ist es, diese Formulierung zu einem vollständigen analytischen Beweis der Riemannschen Hypothese zu führen.

\vspace{1em}
\hrule
\vspace{1em}

\textbf{Kontakt:} Tim Hendrik Freese \\
\texttt{[E-Mail einsetzen]} \quad \texttt{[Optional: arXiv / GitHub]}

\end{document}