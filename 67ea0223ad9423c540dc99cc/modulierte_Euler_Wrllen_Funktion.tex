\documentclass[12pt]{article}
\usepackage{amsmath, amssymb, geometry, booktabs, graphicx, float, hyperref}
\geometry{a4paper, margin=2.5cm}
\title{Spektralanalyse der Euler-Wellenfunktion mit FFT-Frequenzvergleich}
\author{Zeta Nova Freesiana – März 2025}
\date{}

\begin{document}
\maketitle

\section*{1. Hintergrund}

Die modulierte Euler-Wellenfunktion ist definiert durch:
\[
\psi(x) = \sum_{p \leq p_N} \frac{1}{\sqrt{p}} \cdot \sin(x \log p)
\]
mit \(p\) Primzahlen und \(x\) als kontinuierlichem Parameter. Diese Konstruktion ist inspiriert von der expliziten Formel in der analytischen Zahlentheorie, in der die Verteilung der Primzahlen über oszillatorische Terme dargestellt wird.

\textbf{Ziel:} Frequenzanalyse von \(\psi(x)\) mittels FFT, um spektrale Resonanzen in der Primzahlstruktur offenzulegen.

\section*{2. Herleitung der Frequenzstruktur}

Da die Argumente der Sinusfunktionen \(\log p\) enthalten, ergibt sich für jede Primzahl eine charakteristische Frequenz:
\[
f_p := \frac{\log p}{2\pi}
\]
Dies motiviert die Interpretation von \(\psi(x)\) als Superposition harmonischer Oszillatoren, deren Frequenzen durch die Primzahlen definiert sind.

\section*{3. Verbindung zur Zeta-Funktion}

Die Riemannsche Zeta-Funktion besitzt eine Euler-Produktdarstellung:
\[
\zeta(s) = \prod_{p} \left(1 - p^{-s}\right)^{-1}
\]
Deren Logarithmus liefert eine Summe über Primzahlterme:
\[
\log \zeta(s) = \sum_{p} \sum_{k=1}^\infty \frac{1}{k p^{ks}}
\]
Diese Summe ist formal eng verwandt mit \(\psi(x)\), wobei hier anstelle von komplexen \(s\) die reale Modulation über \(x\) erfolgt. Die Oszillationen in \(\psi(x)\) spiegeln somit die Primzahlverteilung wider, wie sie auch in den Nullstellen der Zeta-Funktion auftreten.

\section*{4. Frequenzanalyse (FFT) – Vergleich}

\begin{table}[H]
\centering
\renewcommand{\arraystretch}{1.2}
\begin{tabular}{rrrr}
\toprule
\textbf{Primzahl $p$} & \(\log(p)/(2\pi)\) & \textbf{FFT-Frequenz} & \textbf{Abweichung $\Delta f$} \\
\midrule
2  & 0.11031780 & 0.11032000 & 0.00000220 \\
3  & 0.17484958 & 0.17485000 & 0.00000042 \\
5  & 0.25615000 & 0.25615000 & 0.00000000 \\
7  & 0.30970122 & 0.30970000 & 0.00000122 \\
11 & 0.38163689 & 0.38164000 & 0.00000311 \\
13 & 0.40822437 & 0.40822000 & 0.00000437 \\
17 & 0.45091991 & 0.45092000 & 0.00000009 \\
19 & 0.46862202 & 0.46862000 & 0.00000202 \\
23 & 0.49902940 & 0.49903000 & 0.00000060 \\
\bottomrule
\end{tabular}
\caption{Primzahlfrequenzen im Vergleich zu beobachteten FFT-Dominanzen.}
\end{table}

\section*{5. Interpretation}

Die hohe Präzision zwischen \(\log(p)/(2\pi)\) und den beobachteten Frequenzen in der FFT-Analyse legt nahe, dass \(\psi(x)\) tatsächlich als \emph{spektrale Signatur} der Primzahlen verstanden werden kann. Diese Übereinstimmungen bestätigen die mathematische Kohärenz zwischen Primzahlstruktur und harmonischer Wellenanalyse.

\section*{6. Ausblick: Operatorenstruktur}

Die Euler-Wellenfunktion lässt sich interpretieren als Eigenfunktion eines spektralen Operators \(\mathcal{H}_\psi\), in dem jede Frequenz \(f_p = \log(p)/(2\pi)\) ein diskretes Energieniveau repräsentiert:
\[
\mathcal{H}_\psi \psi(x) = \lambda_p \cdot \psi(x), \quad \lambda_p = f_p^2
\]
Ein solcher Operator könnte als quantenmechanisches Modell zur Rekonstruktion der Zeta-Nullstellen dienen — analog zur Spektralinterpretation der Riemannschen Hypothese (Hilbert-Pólya-Ansatz).

\section*{7. Weiterführende Schritte}

\begin{itemize}
  \item Erweiterung der Frequenzanalyse auf Millionen von Primzahlen
  \item Untersuchung der Interferenzstruktur der Beta-Skala und \(\psi(x)\)
  \item Verknüpfung der dominanten FFT-Frequenzen mit spektralen Peaks der Zeta-Nullstellen
  \item Modellierung eines Hamiltonoperators mit Eigenfrequenzen \(\log(p)/(2\pi)\)
\end{itemize}

\bigskip

\noindent\textbf{Kontakt:}\\
Tim Hendrik Freese\\
\texttt{[E-Mail-Adresse einsetzen]}\\
\texttt{[Zeta Nova Freesiana, März 2025]}

\end{document}