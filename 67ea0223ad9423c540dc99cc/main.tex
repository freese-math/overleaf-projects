\documentclass[12pt]{article}
\usepackage{amsmath, amssymb, physics, geometry, hyperref, graphicx}
\geometry{a4paper, margin=2.8cm}
\usepackage{lmodern}
\usepackage{xcolor}
\usepackage{mathrsfs}
\usepackage{enumitem}
\usepackage{tikz}

\title{\textbf{Die Euler-Wellenfunktion als spektrales Fundament der Zeta Nova Freesiana}}
\author{Tim Hendrik Freese}
\date{März 2025}

\begin{document}
\maketitle

\section*{1. Einführung}

Die modulierte Euler-Wellenfunktion
\[
\psi(x) = \sum_{p \leq p_N} \sin(x \log p)
\]
stellt eine harmonische Überlagerung aller Primzahllogarithmen dar. In dieser Konstruktion bildet sie die spektrale Trägerfunktion der \textit{Zeta Nova Freesiana} (ZNF), also jenes dynamischen Modells, das Nullstellen der Riemannschen Zeta-Funktion über eine modulationsgetriebene Interferenzstruktur erzeugt.

\vspace{1em}
Die Idee: Primzahlen erzeugen Schwingungen; über Interferenz, Frequenzfilter und harmonische Kohärenz entstehen daraus Nullstellenmuster und Fehlerresonanzen.

\section*{2. Beta-Skala und Kreiswellenphasen}

Die Beta-Skala analysiert exponentielle Fits der Form
\[
L(n) \approx \alpha \cdot n^{\beta(n)},
\]
wobei \(\beta\) als modulierende Exponentenskala interpretiert wird. Die zugehörige Phase auf dem Einheitskreis ergibt sich durch die Euler-Freese-Identität:
\[
H(\beta) = e^{i\pi \beta}.
\]
Diese Phase wirkt als Modulator der Primzahlfrequenzen. Besonders kleine Fehlerträger wie
\[
\beta = \frac{7}{33300}, \quad \beta = \frac{1}{66600}
\]
minimieren die Abweichung der harmonischen Struktur – vermutlich, weil sie mit Frequenzen der Form \(\log p / \pi\) interferieren.

\section*{3. Strukturelle Kopplung: \(\psi(x)\) und \(\beta\)-Skala}

\begin{itemize}[leftmargin=1.5em]
  \item \textbf{Frequenzresonanz:} Die Frequenzen in \(\psi(x)\) sind exakt \(\log p\), also mit \(\beta\)-Modulationen verknüpfbar. Kleinste \(\beta\)-Werte korrelieren mit Primzahlen hoher Ordnung.
  \item \textbf{Fehlerminimierung:} Die Beta-Skala minimiert systematisch die Abweichung \(\Delta(\beta)\) für bestimmte rational approximierte Werte – genau jene, die in \(\psi(x)\) durch harmonische Verstärkung dominieren.
  \item \textbf{Kohärenz:} Das Muster der \(\psi(x)\)-Wellen und die spektralen Peaks in \(\beta(n)\) stimmen in dominanten Frequenzen überein (siehe Fourier-Analysen).
\end{itemize}

\section*{4. Erweiterung: Operatorielle Interpretation}

Wir schlagen eine Verallgemeinerung als Integralform vor:
\[
Z_{\mathrm{ZNF}}(x) = \int_0^{\log p_N} \rho(\omega) \sin(x \omega) \, d\omega,
\]
wobei \(\rho(\omega)\) eine gewichtete spektrale Dichtefunktion ist, z.\,B. über die Möbius-Funktion oder Beta-Filter moduliert.

\section*{5. Fazit}

Die Euler-Wellenfunktion ist nicht nur ein Hilfsmittel, sondern ein struktureller Bestandteil der ZNF. Ihre Anwendung ermöglicht:
\begin{itemize}
  \item eine spektrale Analyse der Primzahlen als Wellenform,
  \item eine quantitative Interpretation der Beta-Skala als Phasenmodulator,
  \item eine Verbindung von Primzahlen, Nullstellen und Fehlerstruktur,
  \item und langfristig: einen spektralen Zugang zur Riemannschen Hypothese.
\end{itemize}

\vspace{1em}
\noindent\boxed{\text{Die Euler-Wellenfunktion ist das spektrale Rückgrat der Zeta Nova Freesiana.}}

\end{document}