\documentclass[a4paper,12pt]{article}
\usepackage{amsmath, amssymb, graphicx, float}
\usepackage{booktabs, siunitx}
\usepackage[margin=2.5cm]{geometry}
\usepackage{hyperref}
\usepackage{physics}

\title{Spektrale Analyse der modulierenden Euler-Wellenfunktion im Kontext der Zeta-Funktion und relativistischer Skalarfelder}
\author{Zeta Nova Freesiana Projekt}
\date{\today}

\begin{document}
\maketitle

\section*{1. Einführung}
Die modulierende Euler-Wellenfunktion
\[
\psi(x) = \sum_{p \leq p_N} \sin(x \log p)
\]
verknüpft die harmonischen Strukturen der Primzahlen direkt mit dem Fourierraum. Diese Wellenfunktion wurde in Verbindung mit rekonstruierten Fehlerstrukturen aus Zeta-Nullstellen untersucht. Über die Fourier-Analyse lassen sich dominante Frequenzen extrahieren, die eng mit \(\log(p)/(2\pi)\) für kleine Primzahlen korrespondieren.

\section*{2. Frequenzanalyse}
Die dominanten Frequenzen im Spektrum der \(\psi(x)\)-Welle lauten:
\begin{align*}
f_1 &= 0.25615 \approx \frac{\log(5)}{2\pi} \\
f_2 &= 0.17485 \approx \frac{\log(3)}{2\pi} \\
f_3 &= 0.45092 \approx \frac{\log(17)}{2\pi}
\end{align*}

Diese Übereinstimmung legt nahe, dass das Spektrum der \(\psi(x)\) tatsächlich Primzahlinformationen kodiert. Die folgende Tabelle vergleicht präzise:

\begin{table}[H]
\centering
\begin{tabular}{rccc}
\toprule
Primzahl \(p\) & \(\log(p)/(2\pi)\) & FFT-Frequenz & Abweichung \\
\midrule
5  & 0.25615000 & 0.25615000 & 0.00000000 \\
17 & 0.45091991 & 0.45092000 & 0.00000009 \\
3  & 0.17484958 & 0.17485000 & 0.00000042 \\
23 & 0.49902940 & 0.49903000 & 0.00000060 \\
\bottomrule
\end{tabular}
\caption{Vergleich von Primzahlfrequenzen mit FFT-Daten}
\end{table}

\section*{3. Zusammenhang mit relativistischen Feldern}
Gemäß der klassischen Feldtheorie wird ein Skalarfeld \(\phi(x^\mu)\) in der Raumzeit durch Ableitungen dargestellt:

\[
\partial_\mu \phi = \frac{\partial \phi}{\partial x^\mu}
\]

Dies entspricht Gleichung (4.14) aus dem Skript. Wenn man nun \(\psi(x)\) als übergeordnetes Skalarfeld interpretiert, das durch Primzahlfrequenzen moduliert ist, kann man folgende Deutung wagen:

\[
\phi(x) := \sum_{p} \sin\left(\frac{\log p}{2\pi} x\right)
\]

Mit dieser Deutung wird \(\phi(x)\) Lorentz-skalierbar, da \(\log p\) invariant und die Frequenzen reell sind. Damit ergibt sich ein physikalisch motivierter Zugang zur Untersuchung spektraler Eigenschaften von Primzahlen.

\section*{4. Perspektive: Verbindung zur Zeta-Funktion}
Die explizite Formel aus der Zahlentheorie (z.~B. Riemann-von-Mangoldt) zeigt, dass Primzahlen im Zusammenhang mit den Nullstellen der Zeta-Funktion stehen. Der Zusammenhang:

\[
\psi(x) = \sum_{p \leq x} \log p \sim x - \sum_{\rho} \frac{x^\rho}{\rho} - \dots
\]

wird durch die modulierte Euler-Wellenfunktion möglicherweise geometrisch visualisierbar. Das FFT-Spektrum stellt somit einen experimentell inspirierten Weg zur Analyse der Riemannschen Hypothese dar.

\section*{5. Ausblick}
Die Verbindung zwischen Primzahlspektren, Fehlerstrukturen der Zeta-Nullstellen und der Physik (Skalarfeld, Lorentztransformation) eröffnet neue Pfade zur rigorosen Formulierung der Riemannschen Hypothese über harmonische und spektrale Eigenschaften.

\end{document}