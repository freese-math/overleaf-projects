\documentclass[12pt]{article}
\usepackage{amsmath, amssymb, physics, geometry, graphicx}
\geometry{a4paper, margin=2.5cm}
\usepackage{lmodern}
\usepackage{xcolor}
\usepackage{hyperref}

\title{Spektral-arithmetischer Operatoransatz zur Riemannschen Hypothese}
\author{Tim Hendrik Freese}
\date{März 2025}

\begin{document}
\maketitle

\section*{1. Operatoransatz \boldmath$\mathcal{D}_\mu$}

Sei $\mathcal{D}_\mu$ ein Operator mit Wirkung auf $f : \mathbb{N} \to \mathbb{R}$, definiert durch:
\[
(\mathcal{D}_\mu f)(n) = \sum_{k=1}^{K} w_k \cdot \delta_\sigma(n - \gamma_k) \cdot f(n)
+ \lambda \sum_{d=1}^{n-1} \mu(d) \left\lfloor \frac{n}{d} \right\rfloor f(n)
\]

\textbf{Notation:}
\begin{itemize}
  \item $\rho_k = \sigma_k + i \gamma_k$: nicht-triviale Nullstellen der Riemannschen Zeta-Funktion
  \item $w_k = \frac{1}{|\zeta'(\rho_k)|}$: spektrales Gewicht
  \item $\delta_\sigma(n - \gamma_k) = \exp\left( -\frac{(n - \gamma_k)^2}{2 \sigma^2} \right)$: gaußförmige Approximation
  \item $\mu(d)$: Möbius-Funktion
  \item $\lambda$: Rückkopplungsfaktor
\end{itemize}

\section*{1.1 Spektrale Erweiterung}

Die Euler-Wellenfunktion
\[
\psi(x) = \sum_{p \leq p_N} \sin(x \log p)
\]
besitzt ein charakteristisches Fourier-Spektrum mit dominanten Frequenzen
\[
\omega_p = \frac{\log p}{2\pi}
\]
welche mit hoher Genauigkeit durch Lorentz-Peaks rekonstruiert werden können:
\[
L_p(\omega) = \frac{A_p}{(\omega - \omega_p)^2 + \gamma_p^2}
\]
Die modifizierte Operatorwirkung von $\mathcal{D}_\mu$ kann spektral formuliert werden als:
\[
(\mathcal{D}_\mu f)(n) =
\int_0^\infty \left( \sum_{p \leq p_N} w_p \cdot L_p(\omega) \right) \cdot e^{2\pi i n \omega} \cdot f(n) \, d\omega
+ \lambda \sum_{d=1}^{n-1} \mu(d) \left\lfloor \frac{n}{d} \right\rfloor f(n)
\]

\section*{2. Hypothese (informell)}

\textbf{Satz (RH-Selbstadjungiertheit):} \\
Die Riemannsche Hypothese gilt genau dann, wenn $\mathcal{D}_\mu = \mathcal{D}_\mu^\dagger$ auf einem geeigneten Hilbertraum $\mathcal{H}_\psi$.

\section*{3. Ziel der Zusammenarbeit}

\begin{itemize}
  \item Exakte Definition von $\mathcal{D}_\mu$ als Operator auf $\ell^2$ oder einem spektral modifizierten Raum
  \item Formale Untersuchung der Selbstadjungiertheit und Spektrum
  \item Konstruktion eines funktionalen Zusammenhangs zwischen Primzahlspektrum und Zeta-Nullstellen
\end{itemize}

\section*{4. Kontakt}
Tim Hendrik Freese\\
\texttt{[E-Mail-Adresse einsetzen]}\\
\texttt{[ggf. GitHub / arXiv-Link]}

\end{document}