\documentclass[12pt]{article}
\usepackage{amsmath, amssymb, physics, geometry, hyperref, graphicx, listings, mathtools}
\geometry{a4paper, margin=2.5cm}
\usepackage{lmodern}
\usepackage{xcolor}

\title{Spektral-arithmetischer Operatoransatz zur Riemannschen Hypothese}
\author{Tim Hendrik Freese}
\date{März 2025}

\begin{document}
\maketitle

\section*{1. Spektral motivierter Operator \boldmath$\mathcal{D}_\mu$}

Sei \(\mathcal{H} = \ell^2(\mathbb{N})\) der Hilbertraum quadratsummierbarer Folgen. Wir definieren den Operator:

\begin{equation}
(\mathcal{D}_\mu f)(n) = \sum_{k=1}^{K} w_k \cdot \exp\left(-\frac{(n - \gamma_k)^2}{2\sigma^2} \right) \cdot f(n) 
+ \lambda \sum_{d=1}^{n-1} \mu(d) \left\lfloor \frac{n}{d} \right\rfloor f(n)
\end{equation}

mit Parametern:
\begin{itemize}
  \item \(\rho_k = \sigma_k + i\gamma_k\): nicht-triviale Nullstellen von \(\zeta(s)\)
  \item \(w_k := \left| \zeta'(\rho_k) \right|^{-1}\): spektrale Gewichtung
  \item \(\mu(d)\): Möbius-Funktion (Dirichlet-Faltung)
  \item \(\lambda \in \mathbb{R}\): Kopplung an arithmetische Struktur
  \item \(\sigma > 0\): Breite der Nullstellen-Lokalisation
\end{itemize}

\paragraph{Ziel:} Untersuchung, ob die \textbf{Selbstadjungiertheit von \(\mathcal{D}_\mu\)} die Riemannsche Hypothese impliziert.

\section*{2. Kontext zur Euler-Wellenfunktion}

Die modulierte Euler-Welle
\[
\psi(x) := \sum_{p \leq P} \sin(x \log p)
\]
erzeugt ein Frequenzspektrum, dessen dominante Peaks mit \(\log(p)/(2\pi)\) korrelieren. Durch Fourier-Analyse ergibt sich:

\begin{equation}
\widehat{\psi}(f) \approx \sum_p \delta\left(f - \frac{\log p}{2\pi} \right)
\end{equation}

\noindent
Dies stützt die Interpretation von \(\mathcal{D}_\mu\) als spektral-arithmetischer Operator, da dessen Gauß-Terme exakt auf die imaginären Teile \(\gamma_k\) der Nullstellen ausgerichtet sind.

\section*{3. Hypothese (formalisiert)}

\paragraph{RH als Spektralbedingung:}
Es existiert ein dichter Definitionsbereich \(D \subset \mathcal{H}\), sodass:

\[
\mathcal{D}_\mu = \mathcal{D}_\mu^\dagger \iff \rho_k = \frac{1}{2} \pm i\gamma_k \text{ für alle } k
\]

Die Selbstadjungiertheit impliziert: alle \(\rho_k\) liegen auf der kritischen Linie \(\Re(s) = \frac{1}{2}\).

\section*{4. Numerische Exploration (Python)}

\noindent Vergleich der Skalarprodukte zur Testung:

\begin{lstlisting}[language=Python, basicstyle=\ttfamily\footnotesize, breaklines=true, backgroundcolor=\color{gray!10}]
# [siehe urspr\"ungliches Code-Listing im Haupttext]
\end{lstlisting}

Ziel ist: numerische Approximation von
\[
\langle \mathcal{D}_\mu f, g \rangle \stackrel{?}{=} \langle f, \mathcal{D}_\mu g \rangle
\]
für Testfunktionen \(f, g \in \ell^2\).

\section*{5. Weiteres Forschungsprogramm}

\begin{itemize}
  \item Definition eines Dichteoperators \(\rho_\psi(n) := |\psi(n)|^2\)
  \item Spektrale Analyse von \(\mathcal{D}_\mu\) mittels orthogonaler Zerlegung
  \item Approximation durch finite Matrixdarstellungen: \(\mathcal{D}_\mu \approx D_N\)
  \item Entwurf eines RH-Äquivalenzsatzes über die symmetrische Positivität von \(\mathcal{D}_\mu\)
\end{itemize}

\vspace{1em}

{7. Visuelle Struktur von Primzahlen und spektrale Interpretation}

Die nebenstehende Abbildung (Abb.~7.1) zeigt die zweidimensionale Punktverteilung von Primzahlen im quadratischen Gitter. Jeder rote Punkt markiert eine Primzahl \(p\), dargestellt in einer geeigneten Koordinatenabbildung \(f: \mathbb{N} \rightarrow \mathbb{Z}^2\). Auffällig ist die scheinbar chaotische, aber dennoch strukturierte Verteilung der Punkte.

\begin{figure}[h!]
\centering
% \includegraphics[width=0.65\textwidth]{62598F32-744C-4042-A669-BC4DB9156101.png}
\caption{Primzahlverteilung im zweidimensionalen Gitter}
\label{fig:primegrid}
\end{figure}

\subsection*{7.1 Fraktale und spektrale Signatur}

Wie bereits in Kapitel 5 ausgeführt, besitzt das \(\beta(n)\)-Signal dominante Frequenzen, die mit \(\log(p)/(2\pi)\) für Primzahlen \(p\) korrelieren. Die visuelle Struktur dieser Punktwolke lässt sich als Projektion dieser spektralen Ordnung interpretieren:

\begin{itemize}
  \item Lokale Ballungen entsprechen Resonanzzonen bestimmter harmonischer Frequenzen.
  \item Leere Regionen deuten auf destruktive Interferenz im Spektrum der Euler-Welle \(\psi(x)\) hin.
  \item Die scheinbare Fraktalität könnte mit der iterativen Selbstähnlichkeit der \(\zeta\)-Nullstellenstruktur verbunden sein.
\end{itemize}

\subsection*{7.2 Integration in die Operatorstruktur}

Die Operatorwirkung \(\mathcal{D}_\mu f(n)\) mit spektralem Gewicht \(w_k\) (vgl. Abschnitt 1) könnte als Projektion auf eine "räumlich codierte" Basis interpretiert werden:

\[
(\mathcal{D}_\mu f)(n) \approx \sum_k w_k \cdot \psi_k(n)
\]

mit \(\psi_k(n) = \exp(i n \log p_k)\) als Fourier-Basis harmonischer Primfrequenzen. Die Darstellung in Abb.~7.1 wäre dann ein Realteil-Plot über diese zusammengesetzten Wellenformen.

\subsection*{7.3 Fazit}

Die visuelle Struktur der Primzahlen liefert qualitative Bestätigung für die spektrale Resonanzhypothese. Ein Ziel künftiger Arbeiten ist es, die räumliche Dichtefunktion der Primzahlen aus dem Operator \(\mathcal{D}_\mu\) analytisch zu rekonstruieren – womöglich durch inverses Mapping des Frequenzspektrums auf eine visuelle Raumstruktur.

\boxed{\text{Primzahlmuster sind spektrale Interferenzen im Zahlengitter.}}

\end{document}