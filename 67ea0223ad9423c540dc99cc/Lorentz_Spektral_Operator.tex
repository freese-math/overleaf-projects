\documentclass[12pt]{article}
\usepackage{amsmath, amssymb, physics, geometry, hyperref, graphicx, mathrsfs}
\geometry{a4paper, margin=2.5cm}
\usepackage{lmodern}
\usepackage{xcolor}
\usepackage{microtype}

\title{\textbf{Spektral-arithmetischer Operatoransatz zur Riemannschen Hypothese}}
\author{Tim Hendrik Freese}
\date{März 2025}

\begin{document}
\maketitle

\section*{1. Definition des Operators \boldmath$\mathcal{L}_\mu$}

Sei \( f: \mathbb{N} \to \mathbb{R} \) eine quadratsummierbare Funktion, d.\,h.\ \( f \in \ell^2(\mathbb{N}) \). Wir definieren den spektral-arithmetischen Operator \(\mathcal{L}_\mu\) durch:

\begin{equation}
(\mathcal{L}_\mu f)(n) = \sum_{p \leq P} \frac{A_p}{(n - \omega_p)^2 + \gamma_p^2} \cdot f(n) + \lambda \sum_{d=1}^{n-1} \mu(d) \left\lfloor \frac{n}{d} \right\rfloor f(n)
\end{equation}

  \section*{1.1 Spektrale Erweiterung des Operators \boldmath$\mathcal{D}_\mu$}

Die Euler-Wellenfunktion
\[
\psi(x) = \sum_{p \leq p_N} \sin(x \log p)
\]
besitzt ein charakteristisches Fourier-Spektrum mit dominanten Frequenzen
\[
\omega_p = \frac{\log p}{2\pi}
\]
welche mit hoher Genauigkeit durch Lorentz-Peaks rekonstruiert werden können:
\[
L_p(\omega) = \frac{A_p}{(\omega - \omega_p)^2 + \gamma_p^2}
\]

Die modifizierte Operatorwirkung von \(\mathcal{D}_\mu\) kann dadurch spektral formuliert werden als:

\[
(\mathcal{D}_\mu f)(n) =
\int_0^\infty \left( \sum_{p \leq p_N} w_p \cdot L_p(\omega) \right) \cdot e^{2\pi i n \omega} \cdot f(n) \, d\omega
+ \lambda \sum_{d=1}^{n-1} \mu(d) \left\lfloor \frac{n}{d} \right\rfloor f(n)
\]

Hierbei ist \(w_p = \frac{1}{|\zeta'(\rho_p)|}\) das spektrale Gewicht, bestimmt über die Zeta-Nullstellen \(\rho_p\), welche den jeweiligen Primzahlresonanzen \(\omega_p\) entsprechen.

Die Selbstadjungiertheit dieses Operators auf einem geeigneten spektralen Hilbertraum \(\mathcal{H}_\psi\) bildet den Kern der folgenden Analyse.


\bigskip

\subsection*{Notation:}
\begin{itemize}
  \item \(p\): Primzahlindex, begrenzt durch festen Cutoff \(P\)
  \item \(\omega_p = \frac{\log p}{2\pi}\): charakteristische Frequenz der Primzahl \(p\)
  \item \(A_p\): FFT-Amplitude (empirisch gewonnen aus Euler-Wellenanalyse)
  \item \(\gamma_p\): Halbwertsbreite (z.\,B.\ \(\gamma_p \sim 10^{-4}\))
  \item \(\mu(d)\): Möbius-Funktion
  \item \(\lambda\): Rückkopplungsparameter
\end{itemize}

\bigskip

\section*{2. Spektrale Motivation}

Die Terme 
\[
\frac{1}{(n - \omega_p)^2 + \gamma_p^2}
\]
sind Realteile einer Lorentz-Funktion, die als Fourier-Transformierte exponentiell gedämpfter Schwingungen bekannt ist. Dies spiegelt exakt das Spektrum der Euler-Wellenfunktion wider, welche durch die modulierte Summe \(\sum_{p} \sin(x \log p)\) beschrieben wird. Der Operator \(\mathcal{L}_\mu\) implementiert somit eine Frequenzfilterung entlang der Primzahlstruktur.

\bigskip

\section*{3. Riemann-Hypothese und Selbstadjungiertheit}

\textbf{Hypothese:} Die Riemannsche Hypothese ist äquivalent zur Selbstadjungiertheit des Operators \(\mathcal{L}_\mu\) auf einem geeigneten Definitionsbereich \(D(\mathcal{L}_\mu) \subset \ell^2(\mathbb{N})\), d.\,h.

\begin{quote}
\textbf{RH-Satz:} \\
Die Gleichheit
\[
\mathcal{L}_\mu = \mathcal{L}_\mu^\dagger
\]
gilt genau dann, wenn alle nicht-trivialen Nullstellen der Riemannschen Zeta-Funktion auf der kritischen Geraden \(\Re(s) = \frac{1}{2}\) liegen.
\end{quote}

\bigskip

\section*{4. Ausblick und weitere Arbeit}

\begin{itemize}
  \item Definition eines dichten Kerns \( \mathcal{D} \subset \ell^2(\mathbb{N}) \)
  \item Untersuchung symmetrischer und selbstadjungierter Wirkung mittels Testfunktionen
  \item Operatorielle Spektralanalyse und numerische Approximation
  \item Vergleich mit bekannten Hamiltonoperatoren in der Quantenchaostheorie
\end{itemize}


\section*{Kontakt}
Tim Hendrik Freese\\
\texttt{[E-Mail-Adresse einsetzen]}\\
\texttt{[ggf. GitHub / arXiv-Link]}
\bigskip

\end{document}