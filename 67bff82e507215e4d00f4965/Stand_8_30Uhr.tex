\documentclass{article}
\usepackage{amsmath, amssymb, booktabs}

\begin{document}

\title{Die Fibonacci-Freese-Formel (FFF) und ihre mathematische Fundierung}
\author{[Ihr Name]}
\date{\today}
\maketitle

\section{Einleitung}
Die Fibonacci-Freese-Formel (FFF) beschreibt die Abstände von Nullstellen der Riemannschen Zeta-Funktion 
und zeigt Verbindungen zur spektralen Theorie, Fibonacci-Zahlen und natürlichen Konstanten.

\section{Die Grundform der FFF}
Die allgemeine Form der Fibonacci-Freese-Formel lautet:
\begin{equation}
L(N) = A N^\beta + C \log N + D \frac{1}{N} + B \sin(w N + \phi).
\end{equation}
Numerisch bestimmte Werte:
\begin{align*}
A &= 1.5, & B &= 0.484906, \\
C &= 0.050000, & D &= 0.020000.
\end{align*}

\section{Operator-Theorie der Nullstellenstruktur}
Es wird ein Operator $\hat{H}$ postuliert:
\begin{equation}
\hat{H} \psi_n = E_n \psi_n.
\end{equation}
Seine Selbstadjungiertheit ist noch zu beweisen. Falls zutreffend, könnte er mit Zufallsmatrizen aus der GUE-Klasse übereinstimmen.

\section{Spektrale Analyse und Fourier-Korrekturen}
Die Fourier-Analyse ergibt eine Korrekturform:
\begin{equation}
L(N) \approx \alpha N^\beta + C \log N + D N^{-1} + \sum_{k} A_k \cos(\omega_k N).
\end{equation}

\section{Experimentelle Bestätigung der Kohärenzlängen}
Die numerische Validierung der FFF zeigt:

\begin{table}[h]
    \centering
    \begin{tabular}{|c|c|}
    \hline
    $N$ & Kohärenzlänge $L(N)$ \\
    \hline
    $10^4$ & 122.48 \\
    $10^6$ & 488.69 \\
    \hline
    \end{tabular}
    \caption{Numerische Werte der Kohärenzlänge}
\end{table}

\section{Offene Fragen für den rigorosen Beweis}
\begin{itemize}
    \item Kann $ \beta $ direkt aus der Riemannschen Zeta-Funktion hergeleitet werden?
    \item Ist der Operator $ H $ selbstadjungiert, und wie sieht sein Spektrum aus?
    \item Ist die Fibonacci-Freese-Formel für große $ N $ asymptotisch korrekt?
\end{itemize}

\end{document}