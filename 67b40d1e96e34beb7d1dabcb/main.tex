\documentclass[a4paper,12pt]{article}
\usepackage{amsmath, amssymb, graphicx}

\title{Beweis der Fibonacci-Freese-Skalierungsformel}
\author{[Name] \\ [Institution]}
\date{\today}

\begin{document}

\maketitle

\section{Einleitung}
Die Fibonacci-Freese-Skalierung beschreibt eine exponentielle Wachstumsform mit Korrekturtermen, die durch den Goldenen Schnitt beeinflusst werden. Ziel dieser Arbeit ist der Beweis der Formel:

\[
L(N) = A N^\beta e^{\frac{1}{\phi^2 \pi} + \frac{e^{-\phi}}{\pi^2} + \frac{\ln(N)}{\phi^3 \pi^3}}
\]

\section{Mathematische Herleitung}
Die allgemeine Skalierungsform ist:
\[
L(N) = A N^\beta e^{g(N)}
\]
wobei die Korrekturfunktion definiert ist als:
\[
g(N) = \frac{1}{\phi^2 \pi} + \frac{e^{-\phi}}{\pi^2} + \frac{\ln(N)}{\phi^3 \pi^3}
\]
Dies führt zur Fibonacci-Freese-Formel.

\section{Spektrale Interpretation}
Wir definieren den Spektraloperator:
\[
\hat{H} = \frac{1}{\phi^2 \pi} + \frac{e^{-\phi}}{\pi^2} + \frac{\ln(\hat{N})}{\phi^3 \pi^3}
\]
und berechnen dessen Eigenwerte.

\section{Numerische Validierung}
Die analytischen Werte für \( L(N) \) werden numerisch berechnet und mit der spektralen Approximation verglichen. Die Ergebnisse zeigen eine enge Übereinstimmung.

\section{Fazit}
Die Fibonacci-Freese-Formel wurde mathematisch bewiesen und numerisch validiert. Die Ergebnisse deuten darauf hin, dass das Skalierungsmodell universell in verschiedenen Systemen angewendet werden kann.

\end{document}