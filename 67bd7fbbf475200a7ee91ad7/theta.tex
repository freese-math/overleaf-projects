# �� Installiere benötigte Bibliotheken (falls nicht vorhanden)
!pip install mpmath sympy numpy scipy matplotlib

# �� Importiere Bibliotheken
import numpy as np
import scipy.special as sp
import matplotlib.pyplot as plt
from scipy.optimize import curve_fit
from sympy import primerange
from google.colab import drive

# �� Mount Google Drive
drive.mount('/content/drive')

# �� Fibonacci-Zahlen (stabilisierte Berechnung)
def fibonacci_zahlen(n):
    fib = [1, 1]
    for i in range(2, n):
        fib.append(fib[-1] + fib[-2])
        if fib[-1] > 1e18:  # Verhindert Überlauf
            break
    return np.array(fib, dtype=np.float64)

# �� Primzahlen bis n
def primzahlen_bis(n):
    return np.array(list(primerange(1, n+1)), dtype=np.float64)

# �� Zeta-Nullstellen aus Datei laden (statt numerisch berechnen)
def zeta_nullstellen_n(n, filepath="/content/drive/MyDrive/zeros6.txt"):
    try:
        # Lade Nullstellen aus Datei
        zeta_zeros = np.loadtxt(filepath)
        return zeta_zeros[:n]  # Begrenze auf n Werte
    except Exception as e:
        print(f"⚠️ Fehler beim Laden der Datei: {e}")
        return np.array([])  # Falls Fehler, leere Liste zurückgeben

# �� Kohärenzlängen-Berechnung
def kohärenzlängen(werte):
    abstände = np.diff(werte)
    if len(abstände) == 0:
        return np.array([])
    return np.cumsum(abstände) / np.arange(1, len(abstände) + 1)

# �� Theta-Siegel-Funktion zur Gegenprüfung
def theta_siegel(x):
    return np.cos(np.pi * x) * sp.j0(x)  # Beispielhafte Form

# �� Potenzgesetz für Fit: L(N) = α * N^β
def potenzgesetz(x, alpha, beta):
    return alpha * x**beta

# �� Fitting-Funktion für Kohärenzlänge
def fit_kohärenzlänge(L_data):
    x_data = np.arange(1, len(L_data) + 1)
    valid_mask = (L_data > 0) & (np.isfinite(L_data))
    return curve_fit(potenzgesetz, x_data[valid_mask], L_data[valid_mask])

# �� Generiere Daten
N_max = 500  # Sicherheitsgrenze
fib_numbers = fibonacci_zahlen(N_max)
prime_numbers = primzahlen_bis(N_max)
zeta_numbers = np.abs(zeta_nullstellen_n(N_max))  # Lade Zeta-Nullstellen aus Datei

L_fib = kohärenzlängen(fib_numbers)
L_prime = kohärenzlängen(prime_numbers)
L_zeta = kohärenzlängen(zeta_numbers)

# �� Fit durchführen
popt_fib, _ = fit_kohärenzlänge(L_fib)
popt_prime, _ = fit_kohärenzlänge(L_prime)
popt_zeta, _ = fit_kohärenzlänge(L_zeta)

# �� Extrahiere Parameter
alpha_fib, beta_fib = popt_fib
alpha_prime, beta_prime = popt_prime
alpha_zeta, beta_zeta = popt_zeta

# �� Visualisierung: Kohärenzlängen + Theta-Siegel-Funktion
N_vals = np.linspace(1, N_max, 500)

plt.figure(figsize=(8,6))
plt.loglog(N_vals, potenzgesetz(N_vals, *popt_fib), 'b--', label=f'Fibonacci (β = {beta_fib:.3f})')
plt.loglog(N_vals, potenzgesetz(N_vals, *popt_prime), 'r-.', label=f'Primzahlen (β = {beta_prime:.3f})')
plt.loglog(N_vals, potenzgesetz(N_vals, *popt_zeta), 'g:', label=f'Zeta-Nullstellen (β = {beta_zeta:.3f})')

# �� Theta-Siegel-Funktion hinzufügen
theta_values = theta_siegel(N_vals)
plt.plot(N_vals, np.abs(theta_values), 'k-', label='Theta-Siegel-Funktion')

plt.scatter(np.arange(1, len(L_fib)+1), L_fib, color='blue', s=10, label='Daten Fibonacci')
plt.scatter(np.arange(1, len(L_prime)+1), L_prime, color='red', s=10, label='Daten Primzahlen')
plt.scatter(np.arange(1, len(L_zeta)+1), L_zeta, color='green', s=10, label='Daten Zeta-Nullstellen')

plt.xlabel("N")
plt.ylabel("Kohärenzlänge L(N)")
plt.title("Theta-Siegel-Funktion & Freese-Formel mit Zeta-Nullstellen aus Datei")
plt.legend()
plt.grid(True, which="both", linestyle="--")
plt.show()

# �� Ergebnisse ausgeben
print("\n�� Fit-Ergebnisse:")
print(f"�� Fibonacci: α = {alpha_fib:.5f}, β = {beta_fib:.5f}")
print(f"�� Primzahlen: α = {alpha_prime:.5f}, β = {beta_prime:.5f}")
print(f"�� Zeta-Nullstellen: α = {alpha_zeta:.5f}, β = {beta_zeta:.5f}")