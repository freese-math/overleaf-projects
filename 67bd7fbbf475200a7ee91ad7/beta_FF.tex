import numpy as np
import matplotlib.pyplot as plt
from scipy.optimize import curve_fit

# ⚙️ Numerische Konstanten
phi = (1 + np.sqrt(5)) / 2  # Goldener Schnitt
psi = (1 - np.sqrt(5)) / 2
sqrt_5 = np.sqrt(5)

# �� Funktion zur Berechnung stabilisierter Fibonacci-Zahlen
def fibonacci_zahlen(n):
    """Berechnet Fibonacci-Zahlen stabil ohne Overflow."""
    n = min(n, 1000)  # Sicherheitsgrenze gegen Overflow
    fib = [1, 1]
    for i in range(2, n):
        fib.append(fib[-1] + fib[-2])  # Rekursive Berechnung
    return np.array(fib, dtype=np.int64)

# �� Funktion zur Berechnung stabilisierter Primzahlen
def primzahlen_bis(n):
    """Berechnet Primzahlen bis n mit dem Sieb des Eratosthenes."""
    sieve = np.ones(n+1, dtype=bool)
    sieve[:2] = False
    for i in range(2, int(np.sqrt(n)) + 1):
        if sieve[i]:
            sieve[i*i:n+1:i] = False
    return np.where(sieve)[0]

# �� Funktion zur Berechnung der Riemann-Nullstellen (Simuliert)
def zeta_nullstellen_n(n):
    """Erzeugt eine synthetische Sequenz von Riemann-Zeta-Nullstellen (kritische Linie)."""
    return np.array([0.5 + 1j * t for t in range(1, n+1)])

# �� Berechnung der Kohärenzlängen mit stabiler Methode
def kohärenzlängen(werte):
    """Berechnet die Kohärenzlängen stabilisiert ohne NaN oder Overflow."""
    if len(werte) < 2:
        return np.array([])
    abstände = np.diff(werte)
    return np.cumsum(np.abs(abstände)) / np.arange(1, len(abstände) + 1)

# ✅ Potenzgesetz für Fitting
def potenzgesetz(x, alpha, beta):
    """Potenzgesetz-Funktion für Fit."""
    return alpha * x ** beta

# �� Logarithmierte Version zur besseren numerischen Stabilität
def potenzgesetz_log(log_x, log_alpha, beta):
    """Logarithmierte Version der Potenzgesetz-Funktion."""
    return log_alpha + beta * log_x

# �� Daten generieren (stabilisiert)
N_max = 1000  # Sicherheitsgrenze gegen Überlauf
fib_numbers = fibonacci_zahlen(N_max)
prime_numbers = primzahlen_bis(N_max)
zeta_numbers = np.abs(zeta_nullstellen_n(N_max))  # Absolutwert zur Stabilisierung

L_fib = kohärenzlängen(fib_numbers)
L_prime = kohärenzlängen(prime_numbers)
L_zeta = kohärenzlängen(zeta_numbers)

# �� Fit-Funktion mit Constraints gegen Nicht-Konvergenz
def fit_kohärenzlänge(L_data):
    """Führt robusten Fit mit Constraints durch."""
    x_data = np.arange(1, len(L_data) + 1)
    valid_mask = np.isfinite(L_data) & (L_data > 0)
    if np.sum(valid_mask) < 5:
        return (1, 0)  # Sicheres Default-Fallback
    return curve_fit(potenzgesetz, x_data[valid_mask], L_data[valid_mask], maxfev=5000)[0]

# �� Berechnung der FFF-Parameter
alpha_fib, beta_fib = fit_kohärenzlänge(L_fib)
alpha_prime, beta_prime = fit_kohärenzlänge(L_prime)
alpha_zeta, beta_zeta = fit_kohärenzlänge(L_zeta)

# �� Visualisierung der Ergebnisse
plt.figure(figsize=(8, 6))
N_vals = np.logspace(0.1, np.log10(N_max), 100)

plt.loglog(N_vals, potenzgesetz(N_vals, alpha_fib, beta_fib), 'b--', label=f'Fibonacci (β = {beta_fib:.3f})')
plt.loglog(N_vals, potenzgesetz(N_vals, alpha_prime, beta_prime), 'r-.', label=f'Primzahlen (β = {beta_prime:.3f})')
plt.loglog(N_vals, potenzgesetz(N_vals, alpha_zeta, beta_zeta), 'g:', label=f'Zeta-Nullstellen (β = {beta_zeta:.3f})')

plt.scatter(np.arange(1, len(L_fib) + 1), L_fib, color='blue', s=10, label='Daten Fibonacci')
plt.scatter(np.arange(1, len(L_prime) + 1), L_prime, color='red', s=10, label='Daten Primzahlen')
plt.scatter(np.arange(1, len(L_zeta) + 1), L_zeta, color='green', s=10, label='Daten Zeta-Nullstellen')

plt.xlabel(r'$N$')
plt.ylabel(r'Kohärenzlänge $L(N)$')
plt.title(r'�� Stabilisierte Kohärenzlängen \& Fits nach der Freese-Formel')
plt.legend()
plt.grid(True, which='both', linestyle='--')
plt.show()

# �� Ergebnis-Ausgabe
print("\n�� Fit-Ergebnisse (stabilisiert):")
print(f"�� Fibonacci: α = {alpha_fib:.5f}, β = {beta_fib:.5f}")
print(f"�� Primzahlen: α = {alpha_prime:.5f}, β = {beta_prime:.5f}")
print(f"�� Zeta-Nullstellen: α = {alpha_zeta:.5f}, β = {beta_zeta:.5f}")