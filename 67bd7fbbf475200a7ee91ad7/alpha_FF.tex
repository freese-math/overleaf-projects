import numpy as np
import matplotlib.pyplot as plt
from scipy.optimize import curve_fit

# �� Konstanten
phi = (1 + np.sqrt(5)) / 2  # Goldener Schnitt
psi = (1 - np.sqrt(5)) / 2
sqrt_5 = np.sqrt(5)

# �� Fibonacci-Zahlen stabil berechnen
def fibonacci(n):
    return np.round((phi**np.arange(n) - psi**np.arange(n)) / sqrt_5).astype(int)

# �� Primzahlen generieren (Sieb des Eratosthenes)
def primzahlen(n):
    sieve = np.ones(n, dtype=bool)
    sieve[:2] = False
    for i in range(2, int(np.sqrt(n)) + 1):
        if sieve[i]:
            sieve[i*i:n:i] = False
    return np.where(sieve)[0]

# �� Riemann-Nullstellen (nur zur Illustration)
def zeta_nullstellen_n(n):
    return np.linspace(1, n, n)  # Placeholder für echte Nullstellen

# �� Kohärenzlängen-Funktion
def kohärenzlängen(werte):
    abstände = np.diff(werte)
    if len(abstände) == 0:
        return np.array([])
    return np.cumsum(abstände) / np.arange(1, len(abstände) + 1)

# ✅ Potenzgesetz für Fitting
def potenzgesetz(x, alpha, beta):
    return alpha * x**beta

# �� Log-Transformation für Stabilität
def potenzgesetz_log(log_x, log_alpha, beta):
    return np.exp(log_alpha) * np.exp(beta * log_x)

# �� Daten generieren
N_max = 10000  # Größe der Zahlenmengen
fib_numbers = fibonacci(N_max)
prime_numbers = primzahlen(N_max)
zeta_numbers = zeta_nullstellen_n(N_max)

# �� Berechne Kohärenzlängen
L_fib = kohärenzlängen(fib_numbers)
L_prime = kohärenzlängen(prime_numbers)
L_zeta = kohärenzlängen(zeta_numbers)

# ✅ Logarithmische Fits für Stabilität
def fit_kohärenzlänge(L_data):
    x_data = np.arange(1, len(L_data) + 1)
    valid_mask = (L_data > 0)
    log_x = np.log(x_data[valid_mask])
    log_y = np.log(L_data[valid_mask])
    popt, _ = curve_fit(potenzgesetz_log, log_x, log_y, p0=[np.log(1), 0.5])
    return popt

# �� Berechne Fits
log_alpha_fib, beta_fib = fit_kohärenzlänge(L_fib)
log_alpha_prime, beta_prime = fit_kohärenzlänge(L_prime)
log_alpha_zeta, beta_zeta = fit_kohärenzlänge(L_zeta)

# ✅ Visualisierung
plt.figure(figsize=(8,6))
N_vals = np.arange(1, len(L_fib) + 1)

plt.loglog(N_vals, L_fib, 'bo', markersize=2, label="Daten Fibonacci")
plt.loglog(N_vals, L_prime, 'ro', markersize=2, label="Daten Primzahlen")
plt.loglog(N_vals, L_zeta, 'go', markersize=2, label="Daten Zeta-Nullstellen")

plt.loglog(N_vals, np.exp(potenzgesetz_log(np.log(N_vals), log_alpha_fib, beta_fib)), 'b--', label=f'Fibonacci (β = {beta_fib:.3f})')
plt.loglog(N_vals, np.exp(potenzgesetz_log(np.log(N_vals), log_alpha_prime, beta_prime)), 'r-.', label=f'Primzahlen (β = {beta_prime:.3f})')
plt.loglog(N_vals, np.exp(potenzgesetz_log(np.log(N_vals), log_alpha_zeta, beta_zeta)), 'g:', label=f'Zeta-Nullstellen (β = {beta_zeta:.3f})')

plt.xlabel("N")
plt.ylabel("Kohärenzlänge L(N)")
plt.title("�� Stabilisierte Kohärenzlängen & Fits nach der Freese-Formel")
plt.legend()
plt.grid(True)
plt.show()

# �� Ergebnisse ausgeben
print("\n�� Fit-Ergebnisse (stabilisiert):")
print(f"�� Fibonacci: α = {np.exp(log_alpha_fib):.5f}, β = {beta_fib:.5f}")
print(f"�� Primzahlen: α = {np.exp(log_alpha_prime):.5f}, β = {beta_prime:.5f}")
print(f"�� Zeta-Nullstellen: α = {np.exp(log_alpha_zeta):.5f}, β = {beta_zeta:.5f}")