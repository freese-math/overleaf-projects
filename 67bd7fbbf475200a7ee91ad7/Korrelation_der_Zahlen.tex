\documentclass{article}
\usepackage{booktabs} % Für schöne Tabellen
\usepackage{xcolor} % Für Farben

\begin{document}

\title{DefCon Ampel - Mathematische Stabilitätsbewertung}
\author{}
\date{}
\maketitle

\section*{Tabellarische Übersicht der Fit-Ergebnisse}

\begin{table}[h]
    \centering
    \rowcolors{2}{gray!15}{white} % Alternierende Farben
    \begin{tabular}{lccc}
        \toprule
        \textbf{Modell} & \textbf{α-Wert} & \textbf{β-Wert} & \textbf{Status} \\
        \midrule
        Fibonacci & 2.00000 & 2.00000 & \cellcolor{green!25} Stabil \\
        Primzahlen & 1.57461 & 0.27282 & \cellcolor{yellow!25} Unsicher \\
        Zeta-Nullstellen & 0.98374 & 0.00288 & \cellcolor{red!25} Kritisch \\
        \bottomrule
    \end{tabular}
    \caption{Fit-Ergebnisse der verschiedenen Modelle mit DefCon-Ampel}
    \label{tab:defcon}
\end{table}

\end{document}