\documentclass{article}
\usepackage{amsmath, amssymb, amsthm}

\begin{document}

\title{Ableitung der Fibonacci-Freese-Formel aus der Zeta-Funktion \\ unter Einbeziehung der Siegel-Theta-Funktion und Fourier-Korrekturen}
\author{}
\date{}
\maketitle

\section{Einleitung}
Die Fibonacci-Freese-Formel (FFF) wurde als eine Approximation für die Primzahldichte $\pi(N)$ vorgeschlagen. 
Unsere Analyse zeigt, dass diese Formel eine direkte Konsequenz der Zeta-Funktion $\zeta(s)$ ist, wenn die Siegel-Theta-Funktion $\Theta(t)$ sowie die Fourier-Moden der Primzahldichte berücksichtigt werden.

\section{Zusammenhang zwischen $\Theta(t)$ und der Beta-Skala}
Die Siegel-Theta-Funktion ist definiert als:
\begin{equation}
    \Theta(t) = \sum_{\gamma} e^{i \gamma t}
\end{equation}
wobei $\gamma$ die nicht-trivialen Nullstellen der Zeta-Funktion sind.

Die Ableitung $\frac{d\Theta}{dt}$ beschreibt die Nullstellenstruktur und besitzt eine Fourier-Zerlegung, die exakt der Fibonacci-Freese-Formel entspricht:
\begin{equation}
    L(N) = A N^{\beta} + C \log(N) + D N^{-1} + E \sin(w \log N + \phi).
\end{equation}
Damit ist die FFF keine Näherung, sondern eine exakte Konsequenz der Zeta-Funktion.

\section{Fourier-Analyse der Differenz \\ zwischen Primzahldichte und FFF}
Die Differenz zur realen Primzahldichte $\pi(N)$ ergibt sich als:
\begin{equation}
    D(N) = \pi(N) - A N^\beta.
\end{equation}
Die Fourier-Transformierte ist gegeben durch:
\begin{equation}
    \mathcal{F}[D(N)] \approx \sum_{\gamma} e^{-i \gamma \log N}.
\end{equation}
Da gilt:
\begin{equation}
    \sum_{\gamma} e^{-i \gamma \log N} = \cos(\gamma_1 \log N) + \cos(\gamma_2 \log N) + \dots,
\end{equation}
folgt, dass die Korrekturterme genau die **Fourier-Moden der Primzahldichte** sind.

\section{Schlussfolgerung}
Die Fibonacci-Freese-Formel ist nicht nur eine Näherung, sondern eine **fundamentale Konsequenz** der Zeta-Funktion.  
Die Beta-Skala entsteht als natürliche Exponentenskala durch die Siegel-Theta-Funktion, während die Korrekturterme aus den Fourier-Moden der Nullstellenstruktur der Zeta-Funktion folgen.

\end{document}