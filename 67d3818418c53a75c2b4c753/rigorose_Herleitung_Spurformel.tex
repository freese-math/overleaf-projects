\documentclass[a4paper,12pt]{article}
\usepackage{amsmath, amssymb, amsthm, graphicx, hyperref, mathrsfs}
\usepackage{physics, mathtools}
\usepackage{tikz}
\usepackage{bm}

\title{Rigorose Ableitung der Spurformel \\ und die Rolle von \( \beta \) in der Riemannschen Vermutung}
\author{Mathematische Analyse}
\date{\today}

\begin{document}
\maketitle

\section{Einleitung}
Die klassische Spurformel verknüpft die Eigenwerte eines Operators mit einer spektralen Darstellung. In dieser Arbeit analysieren wir die modifizierte Spurformel für einen Hamilton-Operator \( H \), dessen Eigenwerte \( \lambda_n \) in direktem Zusammenhang mit den Nullstellen der Riemannschen Zeta-Funktion stehen.

Ein zentraler Punkt ist die zusätzliche Korrektur durch den Faktor \( e^{\beta \pi i} \), wobei \( \beta \) numerisch Werte nahe \( 0.484906 \) oder \( \frac{1}{2} \) annimmt. Dies könnte eine tiefere theoretische Bedeutung haben.

\section{Grundlagen der Spurformel}
Die Spurformel für einen Operator \( H \) mit Eigenwerten \( \lambda_n \) lautet:
\begin{equation}
    \sum_n e^{-t \lambda_n} = \int_0^\infty e^{-t E} \, d\rho(E),
\end{equation}
wobei \( d\rho(E) \) die spektrale Dichte beschreibt.

Für große \( t \) ergibt sich aus der Asymptotik von \( \zeta(s) \) die Näherung
\begin{equation}
    \sum_n e^{-t \lambda_n} \approx C t^{-d} e^{\beta \pi i}.
\end{equation}

\section{Herleitung der modifizierten Spurformel}
Die Spurformel kann über eine Fourier-Transformation hergeleitet werden. Wir starten mit der Definition der Theta-Funktion:
\begin{equation}
    \theta(t) = \sum_{n=-\infty}^{\infty} e^{- \pi n^2 t}.
\end{equation}
Eine zentrale Eigenschaft ist die Modulartransformation:
\begin{equation}
    \theta(t) = \frac{1}{\sqrt{t}} \theta\left(\frac{1}{t}\right).
\end{equation}

Die Verbindung zur Riemannschen Zeta-Funktion ergibt sich durch die Mellin-Transformation:
\begin{equation}
    \int_0^\infty t^{s-1} \theta(t) \, dt = \pi^{-s/2} \Gamma\left(\frac{s}{2}\right) \zeta(s).
\end{equation}
Setzen wir \( s = \frac{1}{2} + it \), erhalten wir die spektrale Darstellung der Zeta-Funktion.

\section{Die Rolle von \( \beta \)}
Numerische Fits zeigen, dass sich die Spurformel mit einem zusätzlichen Faktor \( e^{\beta \pi i} \) besser anpassen lässt:
\begin{equation}
    \sum_n e^{-t \lambda_n} = C t^{-d} e^{\beta \pi i}.
\end{equation}
Die Werte für \( \beta \) schwanken zwischen \( 0.484906 \) und \( 0.5 \), was eine tiefere Verbindung zur kritischen Linie der Riemannschen Zeta-Funktion vermuten lässt.

\section{Offene Fragen und weitere Forschung}
\begin{itemize}
    \item Ist \( \beta \) eine direkte Konsequenz der Modulartransformation der Theta-Funktion?
    \item Kann \( \beta \) aus der asymptotischen Analyse der Zeta-Funktion hergeleitet werden?
    \item Gibt es eine Verbindung zwischen \( \beta \) und der Feinstrukturkonstante \( \alpha = \frac{1}{137} \)?
\end{itemize}

\section{Fazit}
Die Spurformel mit der Beta-Korrektur liefert eine erstaunlich präzise Näherung für das Verhalten der Zeta-Nullstellen. Weitere analytische Untersuchungen sind erforderlich, um den Ursprung von \( \beta \) rigoros zu beweisen.

\end{document}