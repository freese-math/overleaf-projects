\documentclass{article}
\usepackage{amsmath, amssymb, amsthm}
\usepackage{graphicx}
\usepackage{hyperref}

\title{Axiomatische Formulierung der Spurformel mit Beta-Korrektur}
\author{[Dein Name]}
\date{\today}

\theoremstyle{definition}
\newtheorem{axiom}{Axiom}
\newtheorem{hypothesis}{Hypothese}
\newtheorem{definition}{Definition}
\newtheorem{theorem}{Theorem}
\newtheorem{corollary}{Korollar}

\begin{document}

\maketitle

\section{Einleitung}

Die Spurformel spielt eine zentrale Rolle in der Spektraltheorie der Riemannschen Zeta-Funktion. 
In diesem Dokument wird eine **modifizierte Spurformel** vorgeschlagen, die einen Korrekturterm der Form \( e^{\beta \pi i} \) enthält. 
Wir untersuchen, ob diese Struktur aus grundlegenden Prinzipien folgt und welche Konsequenzen sich für die Riemannsche Vermutung (RH) ergeben.

\section{Axiomatische Grundlage}

\begin{axiom}[Spektrale Spurformel]
Es existiert eine Spurformel der Form
\begin{equation}
    Z(t) = \sum_n e^{- t \lambda_n},
\end{equation}
wobei \( \lambda_n \) die Eigenwerte eines geeigneten Operators sind.
\end{axiom}

\begin{axiom}[Euler-Identität]
Für jede komplexe Zahl \( \beta \) gilt:
\begin{equation}
    e^{\beta \pi i} = \cos(\beta \pi) + i \sin(\beta \pi).
\end{equation}
\end{axiom}

\begin{definition}[Modifizierte Spurformel]
Sei \( C, d \) reelle Konstanten. Wir definieren eine modifizierte Spurformel als:
\begin{equation}
    Z_{\beta}(t) = C t^{-d} e^{\beta \pi i}.
\end{equation}
\end{definition}

\begin{hypothesis}[Beta-Korrektur]
Die Korrekturgröße \( \beta \) folgt einer festen Regel:
\begin{equation}
    \beta = \frac{1}{2} + \Delta\beta.
\end{equation}
\end{hypothesis}

\begin{theorem}[RH als Grenzfall]
Falls \( \beta = \frac{1}{2} \), dann folgt unmittelbar die kritische Linie der Nullstellen von
\begin{equation}
    \zeta\left( \frac{1}{2} + it \right) = 0.
\end{equation}
\end{theorem}

\begin{corollary}[Mögliche Verbindung zur Feinstrukturkonstante]
Falls \( \Delta\beta \approx \frac{1}{137} \), existiert ein möglicher Zusammenhang zur Feinstrukturkonstante:
\begin{equation}
    \alpha \approx \frac{1}{137} \quad \Rightarrow \quad \beta = \frac{1}{2} + \frac{1}{137}.
\end{equation}
\end{corollary}

\section{Numerische Überprüfung}

Die numerische Approximation der Spurformel zeigt eine enge Übereinstimmung mit der Theorie für \( \beta \approx 0.484906 \). Die Ergebnisse unterstützen die Hypothese, dass ein **Korrekturterm** notwendig ist.

\section{Fazit}

Falls die modifizierte Spurformel mit \( e^{\beta \pi i} \) universell gilt, liefert dies eine **neue Perspektive** auf die analytische Fortsetzung der Spurformel und möglicherweise auf die Riemannsche Hypothese.

\end{document}