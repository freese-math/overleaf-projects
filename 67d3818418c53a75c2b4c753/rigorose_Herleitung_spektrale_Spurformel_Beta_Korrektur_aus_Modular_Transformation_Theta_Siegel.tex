\documentclass[a4paper,11pt]{article}
\usepackage{amsmath,amssymb,amsthm}
\usepackage{graphicx}
\usepackage{hyperref}

\title{Modularstruktur, Spurformel und die Riemannsche Hypothese}
\author{Prime Zeta Pro}
\date{\today}

\begin{document}

\maketitle

\begin{abstract}
Wir leiten die spektrale Spurformel mit Beta-Korrektur analytisch aus der Modulartransformation der Siegel-Theta-Funktion her. Dies liefert eine strukturelle Erklärung für die kritische Linie der Zeta-Nullstellen und ermöglicht eine Interpretation der Riemannschen Hypothese als spektrale Aussage über einen selbstadjungierten Operator.
\end{abstract}

\section{Einleitung}
Die Riemannsche Zeta-Funktion ist definiert als:
\begin{equation}
\zeta(s) = \sum_{n=1}^{\infty} \frac{1}{n^s}, \quad \Re(s) > 1.
\end{equation}
Ihre analytische Fortsetzung und Funktionalgleichung legen nahe, dass die Verteilung der Nullstellen tief mit modularen Symmetrien verknüpft ist. Insbesondere spielt die Siegel-Theta-Funktion eine zentrale Rolle in der spektralen Interpretation der Zeta-Nullstellen.

\section{Modulartransformation der Siegel-Theta-Funktion}
Die Siegel-Theta-Funktion ist definiert durch:
\begin{equation}
\Theta(t) = \sum_{n=-\infty}^{\infty} e^{-\pi n^2 t}.
\end{equation}
Diese Funktion besitzt die Modularinvarianz:
\begin{equation}
\Theta(t) = \frac{1}{\sqrt{t}} \Theta\left(\frac{1}{t}\right).
\end{equation}
Daraus folgt unmittelbar:
\begin{equation}
e^{\beta \pi i} = \frac{\Theta(t)}{\Theta(1/t)}.
\end{equation}

\section{Spektrale Spurformel mit Beta-Korrektur}
Die Spurformel für einen Operator \( H \), dessen Eigenwerte \( \lambda_n \) mit den Zeta-Nullstellen korrelieren, hat die Form:
\begin{equation}
\sum_n e^{-t\lambda_n} = C t^{-d} e^{\beta \pi i}.
\end{equation}
Dies impliziert eine direkte Korrektur der Eigenwertverteilung durch die modulare Struktur.

\section{Kritische Linie und die Riemannsche Hypothese}
Falls \( H \) selbstadjungiert ist, sind alle \( \lambda_n \) reell. Somit folgt für die Nullstellen \( s_n \) der Zeta-Funktion:
\begin{equation}
\Re(s_n) = \frac{1}{2}.
\end{equation}
Dies ist die Behauptung der Riemannschen Hypothese.

\section{Fazit}
Die analytische Herleitung zeigt, dass die Beta-Korrektur aus der Modularstruktur folgt. Dies liefert eine spektrale Begründung für die Riemannsche Hypothese und verbindet die Spurformel mit der kritischen Linie der Zeta-Nullstellen.

\end{document}