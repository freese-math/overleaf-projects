\documentclass[a4paper,12pt]{article}

% Wichtige Pakete
\usepackage{amsmath, amssymb, amsthm}  % Mathematische Symbole
\usepackage{graphicx}                  % Grafiken einbinden
\usepackage{hyperref}                   % Hyperlinks
\usepackage{physics}                    % Physikalische Operatoren
\usepackage{geometry}                   % Seitenränder anpassen
\geometry{left=2.5cm, right=2.5cm, top=3cm, bottom=3cm}

% Theorem-Umgebungen
\newtheorem{theorem}{Theorem}
\newtheorem{definition}{Definition}
\newtheorem{corollary}{Korollar}

\title{Die Freese-Formel und die Riemannsche Hypothese}
\author{Tim Hendrik Freese \\ \small Emsland, Deutschland}
\date{\today}

\begin{document}

\maketitle
\begin{abstract}
    In dieser Arbeit wird die \textbf{Freese-Formel} als neue Struktur in der Zahlentheorie vorgestellt. 
    Sie verknüpft Primzahlen, Nullstellen der Riemannschen Zeta-Funktion und spektrale Operatoren.
    Wir formulieren eine neue \textbf{Euler-Freese-Identität}, definieren einen \textbf{Hamilton-Operator}
    für die Nullstellenstruktur und untersuchen die topologische Struktur mit Betti-Zahlen.
    Die Ergebnisse deuten auf eine tiefere Verbindung zwischen der Riemannschen Hypothese und
    Quantenmechanik hin.
\end{abstract}

\tableofcontents

\section{Einleitung}
Die Riemannsche Hypothese (RH) ist eines der bedeutendsten ungelösten Probleme der Mathematik.
Sie besagt, dass alle nichttrivialen Nullstellen der Zeta-Funktion auf der kritischen Linie $\Re(s) = \frac{1}{2}$ liegen.
Diese Arbeit schlägt einen neuen Ansatz vor: Die \textbf{Freese-Formel} beschreibt eine universelle 
Struktur der Nullstellenverteilung mit einer fraktalen Skala.

\section{Mathematische Definitionen}
\subsection{Freese-Formel}
Die allgemeine Form der Freese-Formel ist:
\begin{equation}
    P(N) = A N^\beta + C \log N + D N^{-1}
\end{equation}
mit den Parametern:
\begin{itemize}
    \item $A, C, D$ - Fit-Parameter zur Anpassung an die Primzahlverteilung
    \item $\beta$ - Skalenexponent der Fibonacci-Struktur
\end{itemize}

\subsection{Euler-Freese-Identität}
Die klassische Euler-Identität lautet:
\begin{equation}
    e^{i\pi} + 1 = 0
\end{equation}
Durch eine Beta-Skala erweitert sich dies zu:
\begin{equation}
    e^{\beta \pi i} + 1 = \epsilon
\end{equation}
wobei $\beta$ numerisch optimiert wurde, sodass $\epsilon \approx 10^{-6}$.

\section{Spektraloperator und RH}
Die Nullstellen der Riemannschen Zeta-Funktion können als Spektrum eines Hamilton-Operators interpretiert werden:
\begin{equation}
    \hat{H} \psi_n = E_n \psi_n
\end{equation}
mit:
\begin{equation}
    \hat{H} = -\frac{d^2}{dx^2} + V(x)
\end{equation}
Das Potential $V(x)$ wird aus der Freese-Formel abgeleitet.

\section{Topologische Analyse}
Eine Analyse der Betti-Zahlen zeigt, dass die Nullstellen eine fraktale Struktur besitzen:
\begin{itemize}
    \item Bei $N = 50.000$ ist die Betti-Zahl $250$
    \item Bei $N = 1.000.000$ ist die Betti-Zahl $500$
    \item Bei $N = 2.000.000$ ist die Betti-Zahl $1000$
\end{itemize}

\section{Physikalische Interpretation}
Die Struktur erinnert an einen \textbf{Quantenresonator} oder eine kosmische Rotation:
\begin{itemize}
    \item Nullstellen als Resonanzfrequenzen
    \item Zusammenhang mit der Feinstrukturkonstante
\end{itemize}

\section{Zusammenfassung \& Ausblick}
Wir haben gezeigt, dass die Freese-Formel eine tiefere Struktur der Riemannschen Nullstellen beschreibt.
Zukünftige Forschungen sollten sich auf eine rigorose Herleitung der Operator-Eigenschaften fokussieren.

\end{document}