\documentclass{article}
\usepackage{amsmath, amssymb, amsthm, graphicx}

\title{Herleitung von \( \beta(t) \) aus Siegel-Theta und der Spurformel}
\author{[Dein Name]}
\date{\today}

\begin{document}
\maketitle

\section{Einleitung}
Die Funktion \( \beta(t) \) spielt eine zentrale Rolle bei der Analyse der Zeta-Nullstellen im Zusammenhang mit der Spurformel und der Siegel-Theta-Funktion. Ziel dieser Arbeit ist es, \( \beta(t) \) explizit aus der Spurformel und der modularen Struktur der Siegel-Theta-Funktion herzuleiten.

\section{Die Spurformel und \( \beta(t) \)}
Die Spurformel für den Hamilton-Operator der Zeta-Nullstellen ist gegeben durch:
\begin{equation}
    \sum_n e^{-t \lambda_n} = \int_{0}^{\infty} e^{-t x} d\rho(x),
\end{equation}
wobei \( \lambda_n \) die Eigenwerte des Operators sind. 

Wir definieren die Skalierungsfunktion \( \beta(t) \) durch die Modifikation der Spurformel mit einer exponentiellen Korrektur:
\begin{equation}
    \sum_n e^{-t \lambda_n} e^{\beta \pi i} \approx \sum_n e^{-t \tilde{\lambda}_n}.
\end{equation}
Diese Skalierung hängt mit der Siegel-Theta-Funktion zusammen.

\section{Verbindung zur Siegel-Theta-Funktion}
Die Siegel-Theta-Funktion ist definiert als
\begin{equation}
    \Theta(t) = \sum_{n=-\infty}^{\infty} e^{-\pi n^2 t}.
\end{equation}
Für kleine \( t \) gilt die Modultransformation:
\begin{equation}
    \Theta(t) = \frac{1}{\sqrt{t}} \Theta\left(\frac{1}{t}\right).
\end{equation}
Daraus folgt für \( \beta(t) \):
\begin{equation}
    \beta(t) \sim C t^{-\gamma} + A.
\end{equation}
Die Fit-Parameter aus numerischen Tests bestätigen diese Struktur.

\section{Zusammenfassung}
Wir haben gezeigt, dass \( \beta(t) \) aus der Spurformel und der modularen Eigenschaft der Siegel-Theta-Funktion hergeleitet werden kann. Dies stellt eine tiefe Verbindung zwischen den Zeta-Nullstellen und modularen Transformationen her.

\end{document}