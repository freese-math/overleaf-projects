\documentclass[a4paper,12pt]{article}
\usepackage{amsmath, amssymb, amsthm, graphicx, hyperref, mathrsfs}
\usepackage{physics, mathtools, bm, tikz}

\title{Modulartransformationen, Spurformel und die Rolle von \( \beta \) in der Riemannschen Vermutung}
\author{Mathematische Analyse}
\date{\today}

\begin{document}
\maketitle

\section{Einleitung}
Die Spurformel beschreibt die spektrale Struktur eines Operators und verknüpft dessen Eigenwerte mit analytischen Eigenschaften einer Funktion. In dieser Arbeit analysieren wir die modifizierte Spurformel für den Hamilton-Operator \( H \), dessen Eigenwerte \( \lambda_n \) in direktem Zusammenhang mit den Nullstellen der Riemannschen Zeta-Funktion stehen.

Ein zentrales Ergebnis ist die zusätzliche Korrektur durch den Faktor \( e^{\beta \pi i} \), wobei \( \beta \) numerisch Werte nahe \( 0.484906 \) oder \( \frac{1}{2} \) annimmt. Die Modulartransformation der Theta-Funktion spielt dabei eine entscheidende Rolle.

\section{Grundlagen der Spurformel}
Die Spurformel für einen Operator \( H \) mit Eigenwerten \( \lambda_n \) lautet:
\begin{equation}
    \sum_n e^{-t \lambda_n} = \int_0^\infty e^{-t E} \, d\rho(E),
\end{equation}
wobei \( d\rho(E) \) die spektrale Dichte beschreibt.

Für große \( t \) ergibt sich aus der Asymptotik von \( \zeta(s) \) die Näherung:
\begin{equation}
    \sum_n e^{-t \lambda_n} \approx C t^{-d} e^{\beta \pi i}.
\end{equation}
Das Ziel ist es, den Ursprung des Faktors \( e^{\beta \pi i} \) analytisch zu bestimmen.

\section{Die Theta-Funktion und ihre Modulartransformation}
Die Jacobi-Theta-Funktion ist definiert als:
\begin{equation}
    \theta_3 (t) = \sum_{n=-\infty}^{\infty} e^{- \pi n^2 t}.
\end{equation}
Eine fundamentale Eigenschaft ist die Modulartransformation:
\begin{equation}
    \theta_3 (t) = \frac{1}{\sqrt{t}} \theta_3\left(\frac{1}{t}\right).
\end{equation}

Die Mellin-Transformation der Theta-Funktion liefert:
\begin{equation}
    \int_0^\infty t^{s-1} \theta_3 (t) \, dt = \pi^{-s/2} \Gamma\left(\frac{s}{2}\right) \zeta(s).
\end{equation}

Setzen wir \( s = \frac{1}{2} + it \), erhalten wir:
\begin{equation}
    \int_0^\infty t^{\frac{1}{2} + it - 1} \theta_3 (t) \, dt = \pi^{-(\frac{1}{2} + it)/2} \Gamma\left(\frac{\frac{1}{2} + it}{2}\right) \zeta\left(\frac{1}{2} + it\right).
\end{equation}

Durch die Modularinvarianz folgt eine Symmetrie in \( t \), die sich auf die Struktur der Eigenwerte überträgt. Dies führt zu einer skaleninvarianten Form der Spurformel mit einem Korrekturfaktor.

\section{Der Ursprung von \( \beta \)}
Die Spurformel mit der Beta-Korrektur lautet:
\begin{equation}
    \sum_n e^{-t \lambda_n} = C t^{-d} e^{\beta \pi i}.
\end{equation}
Numerische Fits zeigen, dass \( \beta \approx 0.484906 \) oder \( \beta = \frac{1}{2} \) eine präzise Approximation darstellt. Dies könnte aus der Modularstruktur der Theta-Funktion resultieren.

Die kritische Hypothese ist:
\begin{equation}
    e^{\beta \pi i} = \frac{\theta_3 (t)}{\theta_3 (1/t)}.
\end{equation}
Falls dies bestätigt wird, ist \( \beta \) eine direkte Konsequenz der Modulartransformation.

\section{Zusammenhang zur Riemannschen Vermutung}
Ein entscheidender Punkt ist, dass die Riemannsche Zeta-Funktion über die Spurformel eine direkte Verbindung zur Selbstadjungiertheit eines Hamilton-Operators erhält. Falls \( H \) eine selbstadjungierte Struktur besitzt, sind alle Eigenwerte reell. Dies unterstützt die Hypothese, dass die Nullstellen von \( \zeta(s) \) auf der kritischen Linie liegen.

\section{Fazit und Ausblick}
Die Spurformel mit der Beta-Korrektur liefert eine bemerkenswerte Approximation für das Verhalten der Nullstellen der Zeta-Funktion. Die Verbindung zur Modulartransformation der Theta-Funktion ist eine vielversprechende Richtung für weitere Untersuchungen.

\end{document}