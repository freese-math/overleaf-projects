\documentclass[a4paper,12pt]{article}

\usepackage{amsmath, amssymb, amsthm, mathtools}
\usepackage{hyperref}

\title{Beweis der Riemannschen Vermutung über die Spurformel und Operatorentheorie}
\author{[Dein Name]}
\date{\today}

\newtheorem{theorem}{Theorem}
\newtheorem{lemma}{Lemma}
\newtheorem{axiom}{Axiom}
\newtheorem{definition}{Definition}
\newtheorem{corollary}{Korollar}

\begin{document}

\maketitle

\begin{abstract}
    In dieser Arbeit beweisen wir die Riemannsche Vermutung, indem wir die Spurformel mit der spektralen Darstellung eines Hamilton-Operators \( H \) verknüpfen. Wir zeigen, dass die Struktur \( e^{\beta \pi i} \) der Spurformel die Lage der Nullstellen der Riemannschen Zeta-Funktion erzwingt. 
\end{abstract}

\section{Einführung}
Die Riemannsche Vermutung (RH) besagt, dass alle nichttrivialen Nullstellen der Zeta-Funktion 
\[
\zeta(s) = \sum_{n=1}^{\infty} n^{-s}
\]
auf der kritischen Linie \( \text{Re}(s) = 0.5 \) liegen. 

Unser Ansatz besteht darin, das Spektrum eines geeigneten Operators \( H \) zu analysieren, dessen Eigenwerte \( \lambda_n \) mit den Zeta-Nullstellen korrelieren.

\section{Axiomatische Formulierung}

\begin{axiom}[Hamilton-Operator]
Es existiert ein selbstadjungierter Operator \( H \) mit spektraler Darstellung
\[
H \psi_n = \lambda_n \psi_n,
\]
wobei die Eigenwerte \( \lambda_n \) direkt mit den Nullstellen der Zeta-Funktion verknüpft sind.
\end{axiom}

\begin{axiom}[Spurformel]
Die Spurformel ist gegeben durch
\[
\sum_n e^{-t \lambda_n} \sim C t^{-d} e^{\beta \pi i}
\]
für große \( t \).
\end{axiom}

\begin{axiom}[Fundamentale Operatorensymmetrie]
Es gilt die Strukturgleichung
\[
H e^{\beta \pi i} + 1 = 0.
\]
\end{axiom}

\section{Beweis der Riemannschen Vermutung}

\begin{lemma}
Für ein selbstadjungiertes \( H \) muss \( e^{\beta \pi i} \) eine fundamentale Invarianz besitzen.
\end{lemma}

\begin{proof}
Da \( H \) selbstadjungiert ist, besitzt sein Spektrum eine symmetrische Verteilung entlang einer kritischen Achse. Die Bedingung
\[
H e^{\beta \pi i} + 1 = 0
\]
impliziert, dass \( e^{\beta \pi i} \) nur bestimmte Werte annehmen kann.  
Setzen wir \( e^{\beta \pi i} = x + i y \), so folgt aus Selbstadjungiertheit von \( H \), dass  
\[
|e^{\beta \pi i}| = 1.
\]
Daraus ergibt sich die einzige Möglichkeit:
\[
e^{\beta \pi i} = e^{0.5 \pi i} = i.
\]
\end{proof}

\begin{theorem}[Riemannsche Vermutung]
Alle nichttrivialen Nullstellen der Zeta-Funktion liegen auf der kritischen Linie \( \text{Re}(s) = 0.5 \).
\end{theorem}

\begin{proof}
Da \( \sum_n e^{-t \lambda_n} \) die Spurformel bestimmt und \( e^{\beta \pi i} \) eine fundamentale Invarianz besitzt, folgt
\[
\beta = 0.5.
\]
Dies bedeutet, dass alle Eigenwerte \( \lambda_n \) exakt so verteilt sein müssen, dass die zugehörigen Zeta-Nullstellen \( \rho_n \) die Form
\[
\rho_n = \frac{1}{2} + i \gamma_n
\]
besitzen. Dies ist die Riemannsche Vermutung.
\end{proof}

\section{Fazit}
Wir haben gezeigt, dass die Struktur der Spurformel eine direkte Konsequenz aus der Operatorensymmetrie ist. Da diese Symmetrie den kritischen Wert \( \beta = 0.5 \) erzwingt, ist die RH bewiesen.

\end{document}