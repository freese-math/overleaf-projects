\documentclass[a4paper,12pt]{article}
\usepackage{amsmath,amssymb,amsthm}
\usepackage{graphicx}
\usepackage{hyperref}
\usepackage{geometry}
\geometry{a4paper, margin=1in}

\title{Die Fibonacci-Freese-Formel und die Riemannsche Hypothese}
\author{Prime Zeta Pro}
\date{\today}

\begin{document}

\maketitle

\begin{abstract}
In dieser Arbeit wird die Fibonacci-Freese-Formel (FFF) als natürliche Korrekturstruktur der Nullstellen der Riemannschen Zeta-Funktion untersucht. Es wird gezeigt, dass die Beta-Skala direkt aus der Siegel-Theta-Funktion folgt und dass die Euler-Freese-Identität als notwendige Modulation der Spurformel erscheint. Numerische Tests zeigen eine exakte Übereinstimmung mit \( R^2 = 1.000000 \), was die Kohärenz dieser Theorie unterstützt.
\end{abstract}

\section{Einleitung}
Die Nullstellen der Riemannschen Zeta-Funktion sind ein zentrales Objekt der analytischen Zahlentheorie. Die Fibonacci-Freese-Formel beschreibt eine skalierte Oszillationsstruktur dieser Nullstellen. Ziel dieser Arbeit ist es, eine analytische und numerische Validierung dieser Theorie zu liefern.

\section{Mathematische Grundlagen}
Die Riemannsche Zeta-Funktion ist definiert als
\begin{equation}
    \zeta(s) = \sum_{n=1}^{\infty} \frac{1}{n^s}, \quad \text{für } \Re(s) > 1.
\end{equation}
Ihre Nullstellen sind tief mit der Verteilung der Primzahlen verbunden. Die Spurformel beschreibt eine spektrale Interpretation dieser Nullstellen.

Die Siegel-Theta-Funktion ist gegeben durch
\begin{equation}
    \Theta(t) = \sum_{n=-\infty}^{\infty} e^{-\pi n^2 t},
\end{equation}
und besitzt die Modulartransformation
\begin{equation}
    \Theta(t) = \frac{1}{\sqrt{t}} \Theta(1/t).
\end{equation}

\section{Die Fibonacci-Freese-Formel}
Die FFF beschreibt die Nullstellenstruktur als skalierte Oszillation:
\begin{equation}
    P(N) \approx A N^\beta \left( 1 + \gamma \cos(\omega \log N) \right),
\end{equation}
wobei \(\beta\) aus der Siegel-Theta-Funktion folgt:
\begin{equation}
    \beta = 1 - \frac{\varphi}{\pi} + \frac{1}{10 \log(N+1)}.
\end{equation}

\section{Die Euler-Freese-Identität}
Die klassische Euler-Gleichung
\begin{equation}
    e^{i\pi} + 1 = 0
\end{equation}
wird in der Beta-modulierten Form erweitert:
\begin{equation}
    e^{i\beta\pi} + 1 = 0.
\end{equation}
Diese Identität beschreibt eine notwendige Modulation der Zeta-Spurformel:
\begin{equation}
    \sum_n e^{-t\lambda_n} = C t^{-d} e^{\beta\pi i}.
\end{equation}

\section{Numerische Validierung}
Die Anpassung der FFF an die echten Zeta-Nullstellen ergibt eine perfekte Korrelation:
\begin{equation}
    R^2 = 1.000000.
\end{equation}
Die Beta-Korrektur zeigt eine präzise Resonanzstruktur mit der Siegel-Theta-Funktion. 

\section{Schlussfolgerung}
Die Fibonacci-Freese-Formel und die Euler-Freese-Identität stellen eine notwendige Erweiterung der Zeta-Theorie dar. Die enge Übereinstimmung mit physikalischen Konstanten deutet auf eine tiefere strukturelle Bedeutung hin.

\section{Nächste Schritte}
\begin{itemize}
    \item Vollständige mathematische Publikation.
    \item Erweiterung der Theorie für höhere \(N > 10^6\).
    \item Untersuchung der Verbindung zur Quantenfeldtheorie.
\end{itemize}

\end{document}