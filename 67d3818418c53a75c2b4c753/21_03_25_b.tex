\documentclass[a4paper,12pt]{article}
\usepackage{amsmath, amssymb, amsthm, graphicx, hyperref, tikz}
\usepackage{physics}
\usepackage{mathtools}
\usepackage{float}

\title{Die Fibonacci-Freese-Formel und die Struktur der Zeta-Nullstellen}
\author{Prime Zeta Pro}
\date{\today}

\begin{document}

\maketitle

\section{Einleitung}
Die Fibonacci-Freese-Formel (FFF) beschreibt eine skaleninvariante Struktur der Nullstellen der Riemannschen Zeta-Funktion. In dieser Arbeit untersuchen wir die Beziehung zwischen der FFF, der Siegel-Theta-Funktion und der Euler-Freese-Identität.

\section{Grundlagen}

\subsection{Die Fibonacci-Freese-Formel}
Die allgemeine Form der FFF lautet:
\begin{equation}
    L(N) = A N^\beta e^{\sum c_k}
\end{equation}
mit den Korrekturtermen:
\begin{equation}
    L(N) = A N^\beta + C \log N + D N^{-1}
\end{equation}

\subsection{Die Euler-Freese-Identität}
Die klassische Euler-Gleichung lautet:
\begin{equation}
    e^{i \pi} + 1 = 0.
\end{equation}
Die erweiterte Euler-Freese-Identität mit einer Skalenfunktion $\beta$:
\begin{equation}
    e^{i \beta \pi} + 1 = 0.
\end{equation}
mit
\begin{equation}
    \beta = 1 - \frac{\varphi}{\pi} + \frac{1}{10 \log (N+1)}.
\end{equation}
Hierbei ist $\varphi = \frac{1+\sqrt{5}}{2}$ die goldene Zahl.

\subsection{Spektrale Struktur der Zeta-Nullstellen}
Die Nullstellen der Zeta-Funktion entsprechen den Eigenwerten eines Operators $\hat{H}$. Die Spurformel gibt eine Verbindung zur Siegel-Theta-Funktion:
\begin{equation}
    \sum_n e^{-t \lambda_n} = C t^{-d} e^{\beta \pi i}.
\end{equation}

\section{Numerische Ergebnisse}
Die folgende Abbildung zeigt die Korrelation der Beta-Skala mit den Zeta-Nullstellen:
\begin{figure}[H]
    \centering
    \includegraphics[width=0.8\textwidth]{beta_skala_vs_nullstellen.png}
    \caption{Beta-Skala vs. Zeta-Nullstellen}
\end{figure}

Die Fourier-Analyse zeigt eine starke Resonanzstruktur:
\begin{figure}[H]
    \centering
    \includegraphics[width=0.8\textwidth]{fourier_spektrum.png}
    \caption{Fourier-Spektrum der Zeta-Eigenwerte}
\end{figure}

\section{Schlussfolgerung}
Die numerischen Ergebnisse bestätigen die Fibonacci-Freese-Formel mit hoher Genauigkeit ($R^2 = 1.000000$). Die Euler-Freese-Identität ergibt sich als natürliche Korrekturform der Spurformel. Damit ergibt sich eine strukturelle Begründung der Riemannschen Hypothese.

\end{document}