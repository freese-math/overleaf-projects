\documentclass{beamer}
\usepackage[utf8]{inputenc}
\usepackage{amsmath, amssymb}
\usepackage{graphicx}
\usepackage{hyperref}
\usepackage{tikz}
\usetheme{Madrid}

\title{Die Beta-Skala und die Spektralstruktur der Zeta-Funktion}
\subtitle{Ein quantisierter Zugang zur Riemannschen Hypothese}
\author{Freese Math Research Initiative}
\date{April 2025}

\begin{document}

\begin{frame}
  \titlepage
\end{frame}

\begin{frame}{Motivation}
  \begin{itemize}
    \item Die Riemannsche Hypothese (RH) bleibt eines der bedeutendsten ungelösten Probleme der Mathematik.
    \item Ziel: Rekonstruktiver Beweis über die \textbf{Beta-Skala} $\beta(n)$ aus der spektralen Struktur der Zeta-Nullstellen.
    \item Visualisierung zeigt hohe strukturelle Kohärenz zwischen Eigenwerten — ein Hinweis auf tiefere Ordnung.
  \end{itemize}
  \vfill
  \includegraphics[width=0.8\linewidth]{colorbands1.png} % Abb. 1
\end{frame}

\begin{frame}{Definition der Beta-Skala}
  \begin{block}{Spektrale Struktur}
    \[
      \beta(n) \approx \frac{A}{n} + \sum_{j=1}^K a_j \cdot \sin(2\pi f_j n + \varphi_j)
    \]
  \end{block}
  \vfill
  \includegraphics[width=0.45\linewidth]{beta3d_1.jpg}
  \includegraphics[width=0.45\linewidth]{beta3d_2.jpg}
\end{frame}

\begin{frame}{Mathematische Form: Spektraltriplett}
  \[
  \Theta(\lambda,k) = (\mathcal{A}(\lambda), \mathcal{H}(\lambda), D(\lambda,k))
  \]
  \begin{itemize}
    \item $\mathcal{A}(\lambda) := C^\infty(\mathbb{R}_+^*/\mu^\mathbb{Z})$
    \item $\mathcal{H} := L^2([\lambda^{-1},\lambda], d^*u)$
    \item $D(\lambda,k) := Q \circ D_0(\lambda) \circ Q$, \quad $D_0 := -i u \partial_u$
  \end{itemize}
  \includegraphics[width=0.8\linewidth]{spectral_triple.jpg}
\end{frame}

\begin{frame}{Quantisierungsbedingung}
  \begin{itemize}
    \item Frequenzdiskretisierung:
    \[
      \log x \in \frac{2\pi}{\zeta_n} \mathbb{Z}, \quad \text{bzw.} \quad x^{iy} = 1
    \]
    \item Resonanzkriterium für Stabilität der Eigenmoden.
  \end{itemize}
  \includegraphics[width=0.65\linewidth]{quantization_condition.jpg}
\end{frame}

\begin{frame}{Zeta-Cycles: Geometrische Interpretation}
  \[
    \text{Zeta-Cycle: } L = \log \mu \quad \text{mit nicht-dichtem Spektralraum in } L^2(C)
  \]
  \begin{itemize}
    \item Analog zu nichttrivialen homotopischen Schleifen.
    \item Frequenzspezifische Topologie im Skalenraum.
  \end{itemize}
  \includegraphics[width=0.65\linewidth]{zeta_cycle.jpg}
\end{frame}

\begin{frame}{Rekonstruktive Summenformel für $\psi_\beta(x)$}
  \[
    \psi_\beta(x) = x - 2 \Re\left( \sum_{\rho} \frac{x^\rho \cdot \log p}{\rho \cdot \zeta'(\rho)} \right), \quad \rho = \frac{1}{2} + i\gamma
  \]
  \begin{itemize}
    \item Exakte Näherung der Tschebyschow-Funktion durch die Beta-Skala.
    \item Validierung über über 2 Millionen Odlyzko-Zeta-Nullstellen.
  \end{itemize}
  \includegraphics[width=0.75\linewidth]{summary_sheet.jpg}
\end{frame}

\begin{frame}{Numerische Evidenz}
  \begin{itemize}
    \item FFT-Spektren zeigen Übereinstimmung mit $\log(p)/2\pi$.
    \item Lorentz-Fits bestätigen dominante Frequenzen.
    \item Projekt ist: \textbf{quellenoffen}, \textbf{juristisch gesichert}, \textbf{technisch überprüfbar}.
  \end{itemize}
  \vfill
  \texttt{github.com/freese-math/riemann-spectral-proof}
\end{frame}

\begin{frame}{Ausblick}
  \begin{itemize}
    \item Mathematische Formalisierung von Operator $D_\mu$ und Selbstadjungiertheit.
    \item Erweiterung auf L-Funktionen und Langlands-Zusammenhänge.
    \item Anwendung in Laserphysik (FFF), Frequenzmodulation, Spektralfilter.
  \end{itemize}
  \vfill
  Vielen Dank!
\end{frame}

\end{document}