%% Beispiel-Präsentation mit LaTeX Beamer im KIT-Design
%% entsprechend den Gestaltungsrichtlinien vom Februar 2025
%%
%% Siehe https://sdq.kastel.kit.edu/wiki/Dokumentvorlagen

%% Beispiel-Präsentation
\documentclass{sdqbeamer} 
 
%% Gruppenlogo, muss im Verzeichnis logos/ liegen
%% falls kein Gruppenlogo gewünscht, bitte \grouplogo{} aufrufen
%\grouplogo{mylogo} 

%% Gruppenname und Breite (Standard: 89 mm)
\groupname{KIT-Fakultät/-Institut/-Forschungsgruppe}
%\groupnamewidth{89mm}

% Beginn der Präsentation

\title[KIT-Beamer-Vorlage]{Präsentation mit \LaTeX{} Beamer im KIT-Design}
\subtitle{entsprechend den Gestaltungsrichtlinien von Februar 2025} 
\author[Erik Burger]{Dr.-Ing. Erik Burger}

\date[1.\,3.\,2025]{1. März 2025}

% Literatur 
 
\usepackage[citestyle=authoryear,bibstyle=numeric,hyperref,backend=biber]{biblatex}
\addbibresource{presentation.bib}
\bibhang1em

\usepackage{lipsum}

\begin{document}

%Titelseite
\begin{frame}[title white vertical, picture=images/palladio_bauplan]
	\titlepage
\end{frame}

\begin{frame}[title blue vertical, picture=images/palladio_bauplan]
	\titlepage
\end{frame}

\begin{frame}[title green horizontal, picture=images/palladio_bauplan, kitlogo=white]
\titlepage
\end{frame}

\begin{frame}[title white horizontal, picture=images/palladio_bauplan, kitlogo=white]
\titlepage
\end{frame}

%Inhaltsverzeichnis
\begin{frame}[tableofcontents=green]{Inhaltsverzeichnis}
	\tableofcontents
\end{frame}

\begin{frame}[tableofcontents=blue]{Inhaltsverzeichnis}
	\tableofcontents
\end{frame}

\section{Erster Abschnitt}
\subsection{Inhaltselemente}

\begin{frame}{Blöcke}{in den KIT-Farben}
	\begin{columns}
		\column{\kittwocolumns}
		\begin{greenblock}{Greenblock}
			Standard (\texttt{block})
        \end{greenblock}
		\column{\kittwocolumns}
		\begin{royalblueblock}{Royalblueblock}
			= \texttt{exampleblock}
        \end{royalblueblock}
		\column{\kittwocolumns}
		\begin{redblock}{Redblock}
			= \texttt{alertblock}
        \end{redblock}
	\end{columns}
	\begin{columns}
		\column{\kittwocolumns}
		\begin{grayblock}{Grayblock}
			Text
        \end{grayblock}
		\column{\kittwocolumns}
		\begin{lightgrayblock}{Lightgrayblock}
			Text
        \end{lightgrayblock}
		\column{\kittwocolumns}
		\begin{blueblock}{Blueblock}
			Text
        \end{blueblock}
	\end{columns}
	\begin{columns}
		\column{\kittwocolumns}
        \begin{brownblock}{Brownblock}
			Text
        \end{brownblock}
		\column{\kittwocolumns}
        \begin{purpleblock}{Purpleblock}
			Text
        \end{purpleblock}
		\column{\kittwocolumns}
        \begin{cyanblock}{Cyanblock}
			Text
        \end{cyanblock}
	\end{columns}
	\begin{columns}
		\column{\kittwocolumns}
        \begin{yellowblock}{Yellowblock}
			Text
        \end{yellowblock}
		\column{\kittwocolumns}
        \begin{lightgreenblock}{Lightgreenblock}
			Text
        \end{lightgreenblock}
		\column{\kittwocolumns}
        \begin{orangeblock}{Orangeblock}
			Text
        \end{orangeblock}
	\end{columns}
	\begin{columns}
		\column{\kittwocolumns}
		\begin{contentblock}{Contentblock}
			(farblos)
		\end{contentblock}
		\column{\kittwocolumns}
		\column{\kittwocolumns}
	\end{columns}
\end{frame}
	  
\begin{frame}{Auflistungen}
	Text
	\begin{itemize}
		\item Auflistung\\ Umbruch
		\item Auflistung
		\begin{itemize}
			\item Auflistung
			\item Auflistung
		\end{itemize}
	\end{itemize}
	\begin{enumerate}
		\item Aufzählung
		\item Aufzählung
		\item Aufzählung
	\end{enumerate}
\end{frame}

\begin{frame}{Spalten}
	\begin{columns}
		\column{\kitcolumn}	
		\begin{standardbox}
			Ich bin ein Blindtext.
		\end{standardbox}
		\column{\kitcolumn}	
		\begin{highlightbox}
			Ich bin ein Blindtext.
		\end{highlightbox}
		\column{\kitcolumn}	
		\begin{grayhighlightbox}
			Ich bin ein Blindtext.
		\end{grayhighlightbox}
		\column{\kitcolumn}	
		\begin{lightgrayhighlightbox}
			Ich bin ein Blindtext.
		\end{lightgrayhighlightbox}
		\column{\kitcolumn}	
		\begin{standardbox}
			Ich bin ein Blindtext.
		\end{standardbox}
		\column{\kitcolumn}	
		\begin{standardbox}
			Ich bin ein Blindtext.
		\end{standardbox}
	\end{columns}
	\vspace{1em}
	\begin{columns}
		\column{\kittwocolumns}	
		\begin{standardbox}
			Ich bin ein Blindtext.
		\end{standardbox}
		\column{\kittwocolumns}	
		\begin{highlightbox}
			Ich bin ein Blindtext.
		\end{highlightbox}
		\column{\kittwocolumns}	
		\begin{grayhighlightbox}
			Ich bin ein Blindtext.
		\end{grayhighlightbox}
	\end{columns}
	\vspace{1em}
	\begin{columns}
		\column{\kitthreecolumns}	
		\begin{standardbox}
			Ich bin ein Blindtext.
		\end{standardbox}
		\column{\kitthreecolumns}	
		\begin{highlightbox}
			Ich bin ein Blindtext.
		\end{highlightbox}
	\end{columns}
\end{frame}

\begin{frame}{Spalten}
	\begin{columns}
		\column{\kitfourcolumns}	
			\includegraphics[width=\linewidth, trim={0 2cm 0 2cm}, clip]{images/palladio_bauplan.jpg}
		\column{\kittwocolumns}	
			\begin{standardbox}
				Beschreibung
			\end{standardbox}

			\vspace{1em}

			\begin{highlightbox}
				Dies ist ein Bauplan der berühmten Villa Rotonda. 
			\end{highlightbox}

			\vspace{1em}

			\begin{grayhighlightbox}
				Foto: Klaus Krogmann
			\end{grayhighlightbox}
	\end{columns}
\end{frame}

\section{Spezialframes}
\begin{frame}[picture 66 vertical,picture=images/palladio_bauplan,kitlogo=black]{Folie mit Bild auf $\frac{2}{3}$ Größe}
	\lipsum[1][1-8]
\end{frame}

\begin{frame}[picture 50 vertical,picture=images/palladio_bauplan,kitlogo=black]{Folie mit Bild auf halber Größe}
	\lipsum[1][1-16]
\end{frame}

\begin{frame}[picture 33 vertical,picture=images/palladio_bauplan,kitlogo=white]{Folie mit Bild auf $\frac{1}{3}$ Größe}
	\begin{columns}
		\column{\kittwocolumns}	
		\lipsum[1][1-8]
		\column{\kittwocolumns}	
		\lipsum[1][1-8]
	\end{columns}
\end{frame}


\begin{frame}[picture vertical=20,picture=images/palladio_bauplan,kitlogo=white]{Folie mit Bild auf variabler Größe (20 \%)}
	\lipsum[1][1-16]
\end{frame}

\section{Titel-/Fußzeile}
\begin{frame}
        Bei Frames ohne Titel wird die Kopfzeile nicht angezeigt, und  
    der freie Platz kann für Inhalte genutzt werden.
\end{frame}

\begin{frame}[plain]
    Bei Frames mit Option \texttt{[plain]} werden weder Kopf- noch Fußzeile angezeigt.
\end{frame}

\begin{frame}[t]{Beispielinhalt}
    Bei Frames mit Option \texttt{[t]} werden die Inhalte nicht vertikal zentriert, sondern an der Oberkante begonnen.
\end{frame}


\begin{frame}{Beispielinhalt: Literatur}
    Literaturzitat: \cite{klare2021jss}
\end{frame}

\appendix
\beginbackup

\begin{frame}{Literatur}
\begin{exampleblock}{Backup-Teil}
    Folien, die nach \texttt{\textbackslash beginbackup} eingefügt werden, zählen nicht in die Gesamtzahl der Folien.
\end{exampleblock}

\printbibliography
\end{frame}

\section{Farben}
%% ----------------------------------------
%% | Test-Folie mit definierten Farben |
%% ----------------------------------------
\begin{frame}{Farbpalette}
\newcommand{\csq}{\strut\hskip1.2em}
\begin{tabular}{rccccccccccccc}
	& 100 & 90 & 80 & 70 & 60 & 50 & 40 & 30 & 25 & 20 & 15 & 10 & 5\\
% GREEN
	kit-green
	& \colorbox{kit-green100}{\csq}
	& \colorbox{kit-green90}{\csq}
	& \colorbox{kit-green80}{\csq}
	& \colorbox{kit-green70}{\csq}
	& \colorbox{kit-green60}{\csq}
	& \colorbox{kit-green50}{\csq}
	& \colorbox{kit-green40}{\csq}
	& \colorbox{kit-green30}{\csq}
	& \colorbox{kit-green25}{\csq}
	& \colorbox{kit-green20}{\csq}
	& \colorbox{kit-green15}{\csq}
	& \colorbox{kit-green10}{\csq}
	& \colorbox{kit-green5}{\csq}\\[.5em]
% BLUE
	kit-royalblue
	& \colorbox{kit-royalblue100}{\csq}
	& \colorbox{kit-royalblue90}{\csq}
	& \colorbox{kit-royalblue80}{\csq}
	& \colorbox{kit-royalblue70}{\csq}
	& \colorbox{kit-royalblue60}{\csq}
	& \colorbox{kit-royalblue50}{\csq}
	& \colorbox{kit-royalblue40}{\csq}
	& \colorbox{kit-royalblue30}{\csq}
	& \colorbox{kit-royalblue25}{\csq}
	& \colorbox{kit-royalblue20}{\csq}
	& \colorbox{kit-royalblue15}{\csq}
	& \colorbox{kit-royalblue10}{\csq}
	& \colorbox{kit-royalblue5}{\csq}\\[.5em]
% BLUE
	kit-blue
	& \colorbox{kit-blue100}{\csq}
	& \colorbox{kit-blue90}{\csq}
	& \colorbox{kit-blue80}{\csq}
	& \colorbox{kit-blue70}{\csq}
	& \colorbox{kit-blue60}{\csq}
	& \colorbox{kit-blue50}{\csq}
	& \colorbox{kit-blue40}{\csq}
	& \colorbox{kit-blue30}{\csq}
	& \colorbox{kit-blue25}{\csq}
	& \colorbox{kit-blue20}{\csq}
	& \colorbox{kit-blue15}{\csq}
	& \colorbox{kit-blue10}{\csq}
	& \colorbox{kit-blue5}{\csq}\\[.5em]
% RED
	kit-red
	& \colorbox{kit-red100}{\csq}
	& \colorbox{kit-red90}{\csq}
	& \colorbox{kit-red80}{\csq}
	& \colorbox{kit-red70}{\csq}
	& \colorbox{kit-red60}{\csq}
	& \colorbox{kit-red50}{\csq}
	& \colorbox{kit-red40}{\csq}
	& \colorbox{kit-red30}{\csq}
	& \colorbox{kit-red25}{\csq}
	& \colorbox{kit-red20}{\csq}
	& \colorbox{kit-red15}{\csq}
	& \colorbox{kit-red10}{\csq}
	& \colorbox{kit-red5}{\csq}\\[.5em]
% GREY
	kit-gray	
	& \colorbox{kit-gray100}{\csq}
	& \colorbox{kit-gray90}{\csq}
	& \colorbox{kit-gray80}{\csq}
	& \colorbox{kit-gray70}{\csq}
	& \colorbox{kit-gray60}{\csq}
	& \colorbox{kit-gray50}{\csq}
	& \colorbox{kit-gray40}{\csq}
	& \colorbox{kit-gray30}{\csq}
	& \colorbox{kit-gray25}{\csq}
	& \colorbox{kit-gray20}{\csq}
	& \colorbox{kit-gray15}{\csq}
	& \colorbox{kit-gray10}{\csq}
	& \colorbox{kit-gray5}{\csq}\\[.5em]
% Orange
	kit-orange
	& \colorbox{kit-orange100}{\csq}
	& \colorbox{kit-orange90}{\csq}
	& \colorbox{kit-orange80}{\csq}
	& \colorbox{kit-orange70}{\csq}
	& \colorbox{kit-orange60}{\csq}
	& \colorbox{kit-orange50}{\csq}
	& \colorbox{kit-orange40}{\csq}
	& \colorbox{kit-orange30}{\csq}
	& \colorbox{kit-orange25}{\csq}
	& \colorbox{kit-orange20}{\csq}
	& \colorbox{kit-orange15}{\csq}
	& \colorbox{kit-orange10}{\csq}
	& \colorbox{kit-orange5}{\csq}\\[.5em]
% lightgreen
	kit-lightgreen
	& \colorbox{kit-lightgreen100}{\csq}
	& \colorbox{kit-lightgreen90}{\csq}
	& \colorbox{kit-lightgreen80}{\csq}
	& \colorbox{kit-lightgreen70}{\csq}
	& \colorbox{kit-lightgreen60}{\csq}
	& \colorbox{kit-lightgreen50}{\csq}
	& \colorbox{kit-lightgreen40}{\csq}
	& \colorbox{kit-lightgreen30}{\csq}
	& \colorbox{kit-lightgreen25}{\csq}
	& \colorbox{kit-lightgreen20}{\csq}
	& \colorbox{kit-lightgreen15}{\csq}
	& \colorbox{kit-lightgreen10}{\csq}
	& \colorbox{kit-lightgreen5}{\csq}\\[.5em]
% Brown
	kit-brown
	& \colorbox{kit-brown100}{\csq}
	& \colorbox{kit-brown90}{\csq}
	& \colorbox{kit-brown80}{\csq}
	& \colorbox{kit-brown70}{\csq}
	& \colorbox{kit-brown60}{\csq}
	& \colorbox{kit-brown50}{\csq}
	& \colorbox{kit-brown40}{\csq}
	& \colorbox{kit-brown30}{\csq}
	& \colorbox{kit-brown25}{\csq}
	& \colorbox{kit-brown20}{\csq}
	& \colorbox{kit-brown15}{\csq}
	& \colorbox{kit-brown10}{\csq}
	& \colorbox{kit-brown5}{\csq}\\[.5em]
% Purple
	kit-purple
	& \colorbox{kit-purple100}{\csq}
	& \colorbox{kit-purple90}{\csq}
	& \colorbox{kit-purple80}{\csq}
	& \colorbox{kit-purple70}{\csq}
	& \colorbox{kit-purple60}{\csq}
	& \colorbox{kit-purple50}{\csq}
	& \colorbox{kit-purple40}{\csq}
	& \colorbox{kit-purple30}{\csq}
	& \colorbox{kit-purple25}{\csq}
	& \colorbox{kit-purple20}{\csq}
	& \colorbox{kit-purple15}{\csq}
	& \colorbox{kit-purple10}{\csq}
	& \colorbox{kit-purple5}{\csq}\\[.5em]
% Cyan
	kit-cyan
	& \colorbox{kit-cyan100}{\csq}
	& \colorbox{kit-cyan90}{\csq}
	& \colorbox{kit-cyan80}{\csq}
	& \colorbox{kit-cyan70}{\csq}
	& \colorbox{kit-cyan60}{\csq}
	& \colorbox{kit-cyan50}{\csq}
	& \colorbox{kit-cyan40}{\csq}
	& \colorbox{kit-cyan30}{\csq}
	& \colorbox{kit-cyan25}{\csq}
	& \colorbox{kit-cyan20}{\csq}
	& \colorbox{kit-cyan15}{\csq}
	& \colorbox{kit-cyan10}{\csq}
	& \colorbox{kit-cyan5}{\csq}\\[.5em]
\end{tabular}
\end{frame}
%% ----------------------------------------
%% | /Test-Folie mit definierten Farben |
%% ----------------------------------------
\backupend

\end{document}
