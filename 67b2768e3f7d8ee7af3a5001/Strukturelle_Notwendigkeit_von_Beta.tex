\documentclass[12pt]{article}
\usepackage[a4paper,margin=2.5cm]{geometry}
\usepackage{amsmath,amssymb,amsthm}
\usepackage{lmodern}
\usepackage{hyperref}
\usepackage{mathrsfs}

\title{Strukturelle Notwendigkeit der Beta-Skala in der Freese-Formel}
\author{Tim Hendrik Freese}
\date{März 2025}

\newtheorem{theorem}{Satz}
\newtheorem{definition}{Definition}
\newtheorem{remark}{Bemerkung}

\begin{document}
\maketitle

\section*{Hauptsatz: Strukturelle Notwendigkeit von \( \beta^* \)}

\begin{theorem}
Es existiert genau ein Wert
\[
\beta^* := \frac{1}{\pi}(\pi - \varphi), \quad \text{mit } \varphi = \frac{1 + \sqrt{5}}{2},
\]
für den alle drei folgenden Aussagen gleichzeitig erfüllt sind:
\begin{enumerate}
    \item \textbf{Asymptotische Abstandsstruktur:}\\
    Die Differenzfolge der nicht-trivialen Nullstellen \( \gamma_n \) der Zetafunktion erfüllt:
    \[
    L(n) := \gamma_{n+1} - \gamma_n \sim A n^{\beta^*} + \mathcal{O}(\log n)
    \]
    und für \( \beta \ne \beta^* \) ist der mittlere quadratische Fehler \( \varepsilon(n) \) nicht verschwindend.
    
    \item \textbf{Modulare Phaseninvarianz:}\\
    In der Spurformel
    \[
    \mathrm{Tr}(e^{-tH}) \sim t^{-\beta} e^{i\pi \beta}
    \]
    führt nur der Wert \( \beta = \beta^* \) zu einer kohärenten, interferenzfreien Phasenstruktur. Für andere \( \beta \) entstehen modulare Brechungen.
    
    \item \textbf{Funktionale Rekonstruktion:}\\
    Die Reihe
    \[
    \psi(x) \approx \sum_k \beta(k) \cos(\gamma_k \log x)
    \]
    konvergiert gleichmäßig auf kompakten Intervallen und approximiert die klassische Chebyschow-Funktion nur, wenn \( \beta = \beta^* \).
\end{enumerate}
\end{theorem}

\begin{remark}
Die Konstanz \( \beta^* \) erscheint somit in drei methodisch unabhängigen Kontexten:
\begin{center}
\begin{tabular}{|c|c|c|}
\hline
\textbf{Quelle} & \textbf{Form} & \textbf{Bedeutung} \\
\hline
\( L(n) \) & \( A n^{\beta} \) & Abstandsstruktur der Nullstellen \\
\hline
Spurformel & \( t^{-\beta} e^{i\pi \beta} \) & Modulare Invarianz in der Operatorphase \\
\hline
FFF \( \to \psi(x) \) & \( \sum \beta(k) \cos(\gamma_k \log x) \) & Funktionale Rekonstruktion der Primzahlinformation \\
\hline
\end{tabular}
\end{center}
Diese dreifache Redundanz impliziert strukturelle Notwendigkeit.
\end{remark}

\begin{remark}[Beweisidee]
\leavevmode
\begin{itemize}
    \item Für (1): Abstandsapproximation mittels asymptotischer Inversion der Hardy-Zählung und Fit-Minimierung.
    \item Für (2): Phasenbruch in Spurformel bei irrationaler Modulation von \( \beta \ne \beta^* \).
    \item Für (3): Fehlerdivergenz in der Fourier-Rekonstruktion bei \( \beta \ne \beta^* \).
\end{itemize}
\end{remark}

\end{document}