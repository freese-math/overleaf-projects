\documentclass[12pt]{article}
\usepackage[T1]{fontenc}
\usepackage[a4paper,margin=2.5cm]{geometry}
\usepackage{amsmath,amssymb}
\usepackage{lmodern}
\usepackage{hyperref}

\title{Positionspapier: Der Weg zur Freese-Funktion\\
\large Entwicklung, Stand und Perspektive}
\author{Tim Hendrik Freese}
\date{März 2025}

\begin{document}
\maketitle

\section*{1. Ausgangspunkt (WAR-Zustand)}

Die Theorie der Freese-Funktion begann mit der Beobachtung periodisch strukturierter Abstände zwischen den Nullstellen der Riemann-Zetafunktion.  
In der frühen Phase (2022–2023) standen Begriffe wie \emph{Kohärenzlänge}, \emph{Skalarwellenstruktur} und \emph{Frequenznetzwerke} im Vordergrund – intuitiv, visuell und spektral gedacht.

\section*{2. Entwicklungsschritt (IST-Zustand)}

Die analytische Reifung (2024–2025) führte zur Herleitung einer Fourier-ähnlichen Strukturform aus der Siegel-Theta-Funktion:
\[
\Theta(t) \sim A t^{\beta} + C \log t + D t^{-1} + E \sin(\omega \log t + \varphi),
\]
deren Ableitung konsistent mit empirischen Nullstellenabständen ist.  
Die \emph{Freese-Funktion} ergibt sich daraus nicht als Postulat, sondern als strukturierte Konsequenz.

\section*{3. Perspektive (SOLL-Zustand)}

Ziel ist die Einbettung dieser Strukturform in einen operatorischen Kontext:
\begin{itemize}
  \item Konstruktion eines Operators mit Spektrum \( \{ \gamma_n \} \)
  \item Interpretation der RH als Stabilitätsbedingung dieser Fourierstruktur
  \item Formalisierung eines „Kohärenzgesetzes nach Freese“
\end{itemize}

\section*{4. Kommunikation \& Dokumentation}

Die Theorie kann in zwei Formaten präsentiert werden:
\begin{itemize}
  \item \textbf{Formal-akademisch:} arXiv, Clay-Institute, mathematische Fachkreise
  \item \textbf{Didaktisch-populär:} Vorträge, Wettbewerbe, Schüler-Olympiade
\end{itemize}

\section*{5. Zusammenfassung}

Die Freese-Theorie entwickelt sich von einer numerisch inspirierten Hypothese zur mathematisch strukturtragenden Analyse.  
Zentral bleibt die Erkenntnis:  
\emph{Wenn die spektrale Ordnung in der Theta-Funktion kohärent ist, folgt RH als notwendige Stabilitätsbedingung.}

\end{document}