\documentclass[10pt]{article}
\usepackage[T1]{fontenc}
\usepackage[a4paper,margin=0.0cm]{geometry}
\usepackage{tikz}
\usetikzlibrary{positioning, arrows.meta, shapes.geometric}

\title{Entwicklung der Freese-Theorie: Vom intuitiven Ansatz zum kohärenten Strukturmodell}
\author{Tim Hendrik Freese}
\date{März 2025}

\begin{document}
\maketitle

\section*{Entwicklungsschema (Zeitleiste)}

\begin{center}
\begin{tikzpicture}[
  node distance=0.0cm and 0.1cm,
  every node/.style={font=\small},
  milestone/.style={rectangle, draw, fill=gray!10, align=center, minimum width=0.1cm, minimum height=3.0cm},
  arrow/.style={-{Latex[length=3mm]}, thick}
  ]

% Nodes
\node[milestone] (start) {Frühe Phase (2022–2023)\\„Kohärenzlänge“, Skalarwellen\\Intuition, visuelle Struktur};
\node[milestone, right=of start] (middle) {Strukturphase (2023–2024)\\Freese-Funktion, Beta-Skala\\Numerik, Fourieransatz};
\node[milestone, right=of middle]
% \node
(now) {Analytische Reifung (2024–2025)\\Siegel-Theta-Fourier\\Ableitung, Bias-Prüfung};
\node[milestone, right=of now] (future) {SOLL-Zustand\\Operator-Integration, RH-Stabilität\\Publikation, ArXiv};

% Arrows
\draw[arrow] (start) -- (middle);
\draw[arrow] (middle) -- (now);
\draw[arrow] (now) -- (future);

\end{tikzpicture}
\end{center}

\end{document}