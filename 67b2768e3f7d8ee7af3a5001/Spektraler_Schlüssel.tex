\documentclass[12pt]{article}
\usepackage{amsmath, amssymb, amsfonts}
\usepackage{physics}
\usepackage{graphicx}
\usepackage{bm}
\usepackage{mathrsfs}

\title{Frequenzstruktur der Zeta-Funktion aus der Funktionalen Gleichung}
\author{}
\date{}

\begin{document}
\maketitle

\section*{1. Ausgangspunkt: Funktionale Gleichung}
Die Riemannsche Zeta-Funktion erfüllt die Funktionale Gleichung:
\[
\pi^{-s/2} \Gamma\left(\frac{s}{2}\right)\zeta(s) = \pi^{-(1-s)/2} \Gamma\left(\frac{1 - s}{2}\right)\zeta(1 - s)
\]
Durch Umstellung erhält man eine Beziehung:
\[
\zeta(s) = \chi(s)\zeta(1 - s)
\quad \text{mit} \quad 
\chi(s) = \pi^{s - \frac{1}{2}} \frac{\Gamma\left( \frac{1 - s}{2} \right)}{\Gamma\left( \frac{s}{2} \right)}
\]

\section*{2. Betrachtung entlang der kritischen Linie}
Setze \( s = \frac{1}{2} + it \). Dann ergibt sich:
\[
\chi\left(\frac{1}{2} + it\right)
= \pi^{it} \frac{ \Gamma\left( \frac{1}{4} - \frac{it}{2} \right) }{ \Gamma\left( \frac{1}{4} + \frac{it}{2} \right) }
\]

Der Ausdruck rechts ist eine komplexe Funktion mit Norm 1 und reinem Phasenanteil:
\[
\chi\left( \frac{1}{2} + it \right) = e^{-2i\theta(t)}
\]
mit einer Winkel- oder Phasenfunktion \( \theta(t) \), die über die Gammafunktion oszilliert.

\section*{3. Ableitung einer Frequenzstruktur}
Für große \(t\) kann man die asymptotische Expansion der Gammafunktion verwenden:
\[
\arg \Gamma\left( \frac{1}{4} + \frac{it}{2} \right) \sim \frac{t}{2} \log \frac{t}{2\pi} - \frac{t}{2} + \cdots
\]
Daraus folgt:
\[
\theta(t) \sim \frac{t}{2} \log \frac{t}{2\pi} - \frac{t}{2}
\]

Diese Phase erzeugt eine Rotation auf dem Einheitskreis im komplexen Raum mit variabler Geschwindigkeit. Für bestimmte \(t_k\) (Nullstellen) ergibt sich eine quantisierte Drehung.

\section*{4. Beobachtung: Dominante Frequenz}
Numerisch wurde in der Fourier-Analyse der Nullstellen gefunden:
\[
\omega_{\text{dominant}} \approx \frac{\pi}{8}
\]

Dies entspricht exakt einer harmonischen Struktur mit periodischem Phasenbezug. Diese Frequenz ergibt sich als konstante Basisfrequenz einer stabilen Drehstruktur.

\section*{5. Schlussfolgerung}
Die Funktionale Gleichung erzwingt eine komplexe Rotation (Drehstruktur) entlang der kritischen Linie. Die Nullstellen der Zeta-Funktion sind auf dieser Linie angeordnet, weil nur dort diese Frequenzstruktur konsistent aufrechterhalten werden kann.

\end{document}