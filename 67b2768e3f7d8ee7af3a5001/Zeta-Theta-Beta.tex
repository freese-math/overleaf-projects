\documentclass[12pt]{article}
\usepackage[a4paper,margin=2.5cm]{geometry}
\usepackage{amsmath,amssymb,amsthm}
\usepackage{lmodern}
\usepackage{hyperref}
\usepackage{mathrsfs}

\title{Fundamentalsatz: Herleitung der Freese-Funktion aus der Siegel-Theta-Funktion}
\author{Tim Hendrik Freese}
\date{März 2025}

\newtheorem{theorem}{Satz}
\newtheorem{definition}{Definition}
\newtheorem{remark}{Bemerkung}

\begin{document}
\maketitle

\section*{Struktureller Ursprung der Freese-Funktion}

\begin{definition}[Siegel-Theta-Funktion]
Die Siegel-Theta-Funktion ist definiert als die kontinuierliche Argumentfunktion
\[
\Theta(t) := \arg \zeta\left( \tfrac{1}{2} + i t \right),
\]
deren Ableitung die spektrale Dichte der nicht-trivialen Nullstellen liefert:
\[
\frac{d\Theta}{dt} = \sum_n \delta(t - \gamma_n).
\]
\end{definition}

\begin{theorem}[Herleitung der Freese-Funktion]
Es existiert eine eindeutige strukturierte Fourierentwicklung der Form
\[
\Theta(t) \sim A \cdot t^{\beta} + C \cdot \log t + D \cdot t^{-1} + E \cdot \sin(\omega \log t + \varphi)
\]
mit \( \beta = \frac{1}{\pi}(\pi - \varphi) \), wobei \( \varphi = \frac{1 + \sqrt{5}}{2} \),  
deren Ableitung eine spektralkohärente Approximation der Nullstellendichte ergibt:
\[
\frac{d\Theta}{dt} \approx L(t) := A' t^{\beta - 1} + C' t^{-1} + E' \cos(\omega \log t + \varphi)
\]
und die mit der empirisch gefitteten Freese-Funktion für \( L(n) = \gamma_{n+1} - \gamma_n \) übereinstimmt.

\end{theorem}

\begin{remark}[Strukturelle Konsequenz]
Die dargestellte Fourierstruktur ist nur dann kohärent und divergenzfrei,
wenn sämtliche Nullstellen \( \rho_n = \tfrac{1}{2} + i \gamma_n \) auf der kritischen Linie liegen.

\textbf{Daher gilt:} Die Riemannsche Hypothese ist äquivalent zur Stabilität der Fourierentwicklung der Siegel-Theta-Funktion –  
und damit zur Gültigkeit der Freese-Funktion.
\end{remark}

\section*{Folgerung}
Die Freese-Funktion ist nicht konstruiert, sondern eine notwendige Konsequenz der spektralen Struktur von \( \zeta(s) \).  
Die Riemannsche Hypothese ist die Stabilitätsbedingung dieses spektralen Systems.

\end{document}