\documentclass[12pt]{article}
\usepackage[a4paper,margin=2.5cm]{geometry}
\usepackage{amsmath,amssymb}
\usepackage{lmodern}
\usepackage{graphicx}
\usepackage{float}
\usepackage{hyperref}

\title{Die empirische Kohärenzlänge der Nullstellen der Riemannschen Zetafunktion\\
\large Numerische Struktur, Fit-Verhalten und Stabilität der Beta-Skala}
\author{Tim Hendrik Freese}
\date{März 2025}

\begin{document}
\maketitle

\section*{1. Einleitung}

Bevor sich die Freese-Funktion als Fourierstruktur der Siegel-Theta-Ableitung analytisch herauskristallisierte, zeigte sich bereits in der empirischen Struktur der Nullstellendifferenzen ein auffälliges Muster.  
Dieses Zusatzpapier dokumentiert die wichtigsten numerischen Signaturen dieser sogenannten \emph{Kohärenzlänge} \( L(n) := \gamma_{n+1} - \gamma_n \).

\section*{2. Best-Fit: Potenzgesetz}

Eine zentrale Beobachtung war, dass sich \( L(n) \) hervorragend durch ein einfaches Potenzgesetz annähern lässt:
\[
L(n) \approx \alpha \cdot n^{\beta}
\]
Die besten globalen Fits (basierend auf > 1 Million Odlyzko-Nullstellen):

\begin{itemize}
  \item \( \alpha \approx 3.838 \)
  \item \( \beta \approx 0.2825 \)
\end{itemize}

\section*{3. Lokale Fits und Driftanalyse}

Bei lokaler Fensterung (Sliding Windows mit 1000–5000 Nullstellen) zeigt sich eine leichte Drift von \( \beta(n) \), abhängig von der Position \( n \).  
Die folgende Grafik zeigt die Werte von \( \beta(n) \) auf einem gleitenden Fit (Platzhalter):

\begin{figure}[H]
\centering
% \includegraphics[width=0.8\textwidth]{beta_drift_example.png}
\caption{Lokaler Fit der Exponenten \( \beta(n) \) auf Sliding Windows (fiktives Beispiel).}
\end{figure}

\section*{4. Vorschlag: Beta-Stabilitätsindex}

Wir definieren:

\textbf{Definition (Beta-Stabilität):}  
Die Funktion \( L(n) \) gilt als spektral stabil, wenn  
\[
\Delta \beta := \max_n \beta(n) - \min_n \beta(n) < \varepsilon
\]
für ein festes Fenstermaß \( W \) und gewähltes \( \varepsilon \) (z.~B. \( \varepsilon = 0.015 \)).

Dieser Index erlaubt:
\begin{itemize}
  \item Vergleich verschiedener Fit-Modelle
  \item Einschätzung der spektralen Konsistenz
  \item Basis für automatisierte Tests der FFF-Gültigkeit
\end{itemize}

\section*{5. Fazit}

Diese empirische Kohärenzlänge ist kein Artefakt, sondern ein erstes Anzeichen einer strukturellen Ordnung, die später in der Fourier-Zerlegung der Theta-Funktion analytisch verankert wurde.  
Sie bleibt ein wertvolles diagnostisches Werkzeug – sowohl für die Theoriekonstruktion als auch für deren Überprüfbarkeit.

\end{document}