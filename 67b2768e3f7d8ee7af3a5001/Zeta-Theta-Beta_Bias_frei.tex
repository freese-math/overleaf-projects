\documentclass[12pt]{article}
\usepackage[a4paper,margin=2.5cm]{geometry}
\usepackage{amsmath,amssymb,amsthm}
\usepackage{lmodern}
\usepackage{hyperref}
\usepackage{mathrsfs}

\title{Fundamentalsatz (revidiert): Strukturierte Fourier-Approximation der Siegel-Theta-Funktion}
\author{Tim Hendrik Freese}
\date{März 2025}

\newtheorem{theorem}{Satz}
\newtheorem{definition}{Definition}
\newtheorem{remark}{Bemerkung}

\begin{document}
\maketitle

\section*{Begriffliche Grundlage}

\begin{definition}[Siegel-Theta-Funktion]
Die Siegel-Theta-Funktion ist definiert durch
\[
\Theta(t) := \arg \zeta\left( \tfrac{1}{2} + i t \right),
\]
und steht in Zusammenhang mit der Verteilung der nicht-trivialen Nullstellen der Zetafunktion. Ihre formale Ableitung liefert:
\[
\frac{d\Theta}{dt} = \sum_n \delta(t - \gamma_n),
\]
wobei \( \gamma_n \) die Ordinaten der Nullstellen sind.
\end{definition}

\section*{Strukturvorschlag}

\begin{theorem}[Fourierstruktur unter spektraler Hypothese]
Unter der Annahme, dass alle Nullstellen der Zetafunktion auf der kritischen Linie liegen, lässt sich für große \( t \) eine Fourier-ähnliche Strukturfunktion
\[
\Theta(t) \sim A \cdot t^{\beta} + C \cdot \log t + D \cdot t^{-1} + E \cdot \sin(\omega \log t + \varphi)
\]
als konsistente Approximation angeben.  
Dabei ist
\[
\beta := \frac{1}{\pi}(\pi - \varphi), \quad \text{mit } \varphi = \frac{1 + \sqrt{5}}{2}.
\]
Die Ableitung dieser Struktur liefert eine Näherung an die lokale Nullstellendichte:
\[
\frac{d\Theta}{dt} \approx L(t) := A' t^{\beta - 1} + C' t^{-1} + E' \cos(\omega \log t + \varphi).
\]
\end{theorem}

\begin{remark}[Interpretation]
Diese Fourierstruktur ist konsistent mit numerischen Beobachtungen aus der Nullstellendifferenzfunktion \( L(n) := \gamma_{n+1} - \gamma_n \)  
und lässt sich als spektral motiviertes Modell zur Annäherung dieser Struktur interpretieren.  
Eine präzise mathematische Ableitung bleibt Gegenstand weiterer Analyse.

\textbf{Hinweis:} Die Gültigkeit dieser Darstellung setzt die kritische Linienlage der Nullstellen voraus. Umgekehrt könnte die Verletzung der RH zu einer destruktiven Instabilität dieser spektralen Form führen.
\end{remark}

\section*{Schlussfolgerung (vorsichtig formuliert)}
Die sogenannte Freese-Funktion ergibt sich in diesem Rahmen als spektrale Approximation, nicht als Axiom.  
Ihre Struktur legt eine kohärente Verbindung zwischen der spektralen Zeta-Analyse und der Nullstellenstatistik nahe, ohne diese jedoch vollständig zu beweisen.
\end{document}