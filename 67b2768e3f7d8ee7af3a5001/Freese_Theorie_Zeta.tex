\documentclass[12pt]{article}
\usepackage[utf8]{inputenc}
\usepackage[utf8]{inputenc}
\usepackage[utf8]{inputenc}
\usepackage[T1]{fontenc}
\usepackage[a4paper,margin=2.5cm]{geometry}
\usepackage{amsmath,amssymb,amsthm}
\usepackage{lmodern}
\usepackage{hyperref}
\usepackage{mathrsfs}

\title{Die Freese-Funktion als spektrale Strukturform der Siegel-Theta-Funktion\\
\large Ein funktionalanalytischer Zugang zur Nullstellenverteilung der Zetafunktion}
\author{Tim Hendrik Freese}
\date{März 2025}

\newtheorem{theorem}{Satz}
\newtheorem{definition}{Definition}
\newtheorem{remark}{Bemerkung}

\begin{document}
\maketitle

\section*{1. Einleitung}

Die Riemannsche Zetafunktion steht im Zentrum zahlentheoretischer Strukturfragen.  
Die in diesem Beitrag vorgestellte \emph{Freese-Funktion} ergibt sich als spektral strukturierte Approximation der Ableitung der Siegel-Theta-Funktion.  
Ziel ist keine Beweisführung der Riemannschen Hypothese (RH), sondern die Herausarbeitung einer kohärenten Strukturform, welche Nullstellenabstände, Spektralanalyse und rekonstruktive Formeln verbindet.

\section*{2. Spektrale Grundlage}

\begin{definition}[Siegel-Theta-Funktion]
Sei
\[
\Theta(t) := \arg \zeta\left( \tfrac{1}{2} + i t \right)
\]
die kontinuierlich fortgesetzte Argumentfunktion. Ihre formale Ableitung ergibt
\[
\frac{d\Theta}{dt} = \sum_n \delta(t - \gamma_n),
\]
wobei \( \gamma_n \) die Ordinaten der nicht-trivialen Nullstellen von \( \zeta(s) \) sind.
\end{definition}

\section*{3. Fourierstruktur als Modell}

\begin{theorem}[Spektralstruktur der Theta-Funktion]
Unter der Voraussetzung, dass alle Nullstellen auf der kritischen Linie liegen, lässt sich für \( t \gg 1 \) eine strukturierte Fourierapproximation formulieren:
\[
\Theta(t) \sim A t^{\beta} + C \log t + D t^{-1} + E \sin(\omega \log t + \varphi),
\]
mit dem strukturellen Exponenten
\[
\beta := \frac{1}{\pi}(\pi - \varphi), \quad \varphi = \frac{1 + \sqrt{5}}{2}.
\]
\end{theorem}

\begin{remark}
Die Ableitung dieser Struktur liefert eine Approximation der lokalen Nullstellendichte:
\[
\frac{d\Theta}{dt} \approx A' t^{\beta - 1} + C' t^{-1} + E' \cos(\omega \log t + \varphi),
\]
welche mit empirischen Nullstellendifferenzen \( L(n) = \gamma_{n+1} - \gamma_n \) kohärent ist.
\end{remark}

\section*{4. Interpretation und Geltung}

Die Freese-Funktion ist als strukturierte Approximation zu verstehen, nicht als analytisch bewiesene Identität.  
Die „Kohärenz“ ergibt sich aus:

\begin{itemize}
  \item Übereinstimmung mit rekonstruktiven \( \psi(x) \)-Formeln
  \item Fourierkompatibilität in der spektralen Analyse
  \item Stabilität unter Variation von Nullstellen – nur bei RH erhalten
\end{itemize}

\section*{5. Ausblick}

Der nächste Schritt ist die operatorische Einbettung der Freese-Struktur, etwa in Form eines selbstadjungierten Operators mit \( \gamma_n \) als Spektrum.  
Zusätzlich bieten sich Anwendungen in der rekonstruktiven Zahlentheorie (\(\psi(x)\), \(\Lambda(n)\)) an.

\end{document}