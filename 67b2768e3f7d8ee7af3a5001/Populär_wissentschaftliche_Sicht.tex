\documentclass[12pt]{article}
\usepackage[a4paper,margin=2.5cm]{geometry}
\usepackage{lmodern}
\usepackage[T1]{fontenc}
\usepackage{graphicx}
\usepackage{hyperref}
\usepackage{amsmath}

\title{Harmonien im Unsichtbaren\\
\large Wie Primzahlen, Frequenzen und die Freese-Formel miteinander tanzen}
\author{Tim Hendrik Freese}
\date{März 2025}

\begin{document}
\maketitle

\section*{1. Primzahlen – das Rätsel im Fundament}

Primzahlen sind wie die Atome der Mathematik – unteilbar, unregelmäßig, aber scheinbar chaotisch verteilt.  
Doch hinter diesem Chaos könnte ein tiefer Klang verborgen liegen.

\section*{2. Die Zeta-Funktion als Resonanzkörper}

Die Riemannsche Zetafunktion fasst alle Primzahlen in einer einzigen Formel zusammen. Ihre Nullstellen erzeugen eine geheimnisvolle Struktur, die wie ein Frequenzspektrum wirkt – ein kosmisches Klangbild.

\section*{3. Die Siegel-Theta-Funktion – das Ohr zur Zeta-Welt}

Mit
\[
\Theta(t) = \arg \zeta\left( \tfrac{1}{2} + i t \right)
\]
können wir dieses Frequenzbild lesen – als spektrale Kurve, als musikalische Landschaft.

\section*{4. Die Freese-Formel – eine spektrale Handschrift}

Aus dieser Kurve lässt sich eine Formel ableiten:
\[
\Theta(t) \approx A t^\beta + C \log t + D t^{-1} + E \sin(\omega \log t + \varphi)
\]
Die Struktur erinnert an musikalische Wellen – mit Frequenz, Dämpfung und Resonanz.

\section*{5. Eine Hypothese erwacht}

Wenn alle Nullstellen der Zeta-Funktion auf einer bestimmten Linie liegen – der „kritischen Linie“ – dann bleibt dieses Spektrum harmonisch.  
Das ist genau das, was die Riemannsche Hypothese behauptet.

\section*{6. Ausblick – ein neuer Blick auf Ordnung}

Die Freese-Formel ist kein Beweis, sondern eine Brücke.  
Eine Struktur, die Primzahlen, Wellen und Mathematik vereint – und vielleicht zeigt, dass Ordnung dort beginnt, wo das Unsichtbare zu klingen beginnt.

\end{document}