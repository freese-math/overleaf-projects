\documentclass[12pt]{article}
\usepackage[a4paper, margin=2.5cm]{geometry}
\usepackage{amsmath, amssymb, amsthm}
\usepackage{lmodern}
\usepackage{hyperref}

\title{Beweisrahmen zur asymptotischen Struktur der Nullstellenabstände der Zetafunktion}
\author{Tim Hendrik Freese}
\date{März 2025}

\newtheorem{theorem}{Satz}
\newtheorem{definition}{Definition}
\newtheorem{remark}{Bemerkung}

\begin{document}
\maketitle

\section*{I. Klassische Grundlagen}

\begin{theorem}[Hardy–Littlewood-Zählung]
Für \( T \to \infty \) gilt:
\[
N(T) = \#\left\{ \rho = \tfrac{1}{2} + i\gamma \in \mathbb{C} \,\middle|\, 0 < \gamma \le T,\ \zeta(\rho) = 0 \right\}
\sim \frac{T}{2\pi} \log\left( \frac{T}{2\pi} \right)
\]
\end{theorem}

\begin{theorem}[Asymptotik der Nullstellenpositionen]
Für \( n \to \infty \) gilt:
\[
\gamma_n \sim \frac{2\pi n}{\log n}
\]
\end{theorem}

\begin{theorem}[Asymptotischer Nullstellenabstand]
Für \( n \to \infty \) folgt aus Taylorentwicklung:
\[
L(n) := \gamma_{n+1} - \gamma_n
\sim \frac{2\pi}{\log n} \left( 1 - \frac{1}{\log n} \right)
\]
\end{theorem}

\section*{II. Strukturhypothese}

\begin{definition}[Freese-Strukturform]
Die Nullstellenabstände \( L(n) \) folgen asymptotisch einer Skala:
\[
L(n) \approx A n^\beta + C \log n + D n^{-1}
\quad \text{mit} \quad \beta := \frac{1}{\pi} (\pi - \varphi),
\quad \varphi := \frac{1 + \sqrt{5}}{2}
\]
\end{definition}

\begin{remark}
Diese Struktur erfüllt:
\begin{itemize}
    \item empirisch hohe Genauigkeit für \( n \gg 1 \)
    \item Stabilität gegenüber Fourier-Analyse
    \item Kompatibilität mit GOE/GUE-Statistik
    \item Fundament für spektral-arithmetische Rekonstruktionsformeln
\end{itemize}
\end{remark}

\section*{III. Perspektive}

Die Freese-Skala stellt eine strukturgetragene Hypothese dar, deren analytisch bewiesene Vorstufen die klassische Hardy–Littlewood-Theorie umfassen. Die eindeutige Skalenform legt den Grundstein für eine spektrale Interpretation der Riemannschen Hypothese und eine duale Sicht auf Primzahlen und Nullstellen.

\end{document}