\section{Zusätzliche technische Details zur Fibonacci-Quantisierung der Nullstellen}

Die Fibonacci-Quantisierung der Zeta-Nullstellen basiert auf der Skalenordnung:

\[
L(N) = \alpha \cdot N^{\beta}
\]

Numerische Berechnungen haben gezeigt, dass sich der Exponent \( \beta \) im Bereich von 0.42 stabilisiert.
Die kritische Vermutung ist, dass:

\[
\beta = \rho = 1 - \frac{\varphi}{\pi}
\]

Falls diese Beziehung zutrifft, folgt die Riemannsche Hypothese direkt.

\subsection{Verbindung zur Funktionalen Gleichung}

Die Funktionale Gleichung der Zeta-Funktion:

\[
\pi^{-s/2} \Gamma(s/2) \zeta(s) = \pi^{-(1-s)/2} \Gamma((1-s)/2) \zeta(1-s)
\]

impliziert eine Spiegelung um die kritische Linie. Durch eine Fourier-Transformation der Nullstellenabstände zeigt sich, dass die dominante Frequenz:

\[
\omega_{\text{dominant}} = 0.4033 \approx \frac{\pi}{8}
\]

Diese Drehstruktur zeigt, dass eine Nullstelle außerhalb von \( \Re(s) = 1/2 \) die gesamte Fourier-Symmetrie zerstören würde.

\subsection{Numerische Validierung}

Die Fourier-Analyse der realen Nullstellenverteilung zeigt:

- **Dominante Frequenz:** \( \pi/8 \)
- **Sekundäre Modulation:** \( \rho = 1 - \frac{\varphi}{\pi} \)

Das bedeutet, dass die Nullstellen durch eine **harmonische Fibonacci-Skalierung** angeordnet sind. Falls eine Nullstelle außerhalb der kritischen Linie existiert, wäre diese Ordnung zerstört.

\section{Zusammenfassung}

- Die Fibonacci-Quantisierung der Nullstellen folgt aus der Funktionalen Gleichung.
- Numerische Tests zeigen eine **Kohärenzlänge mit \( \rho \)-Skalenordnung**.
- Falls \( \beta = \rho \), ist die Riemannsche Hypothese bewiesen.

\textbf{Offene Frage:} Kann diese Relation analytisch aus der Zeta-Funktion hergeleitet werden?