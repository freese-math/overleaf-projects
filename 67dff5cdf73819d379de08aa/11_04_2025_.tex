\documentclass[11pt,a4paper]{article}
\usepackage[utf8]{inputenc}
\usepackage{amsmath, amssymb, amsthm}
\usepackage{graphicx}
\usepackage{geometry}
\usepackage{hyperref}
\geometry{margin=2.5cm}

\title{\textbf{Zeta Nova Freesiana (ZNF)}\\Ein spektraler Zugang zur Riemannschen Hypothese}
\author{Freese Math Research}
\date{\today}

\begin{document}

\maketitle

\section*{1. Motivation}
Die klassische Riemannsche Zeta-Funktion trägt tiefe Verbindungen zur Primzahlverteilung in sich. Mit der \emph{Zeta Nova Freesiana (ZNF)} schlagen wir einen alternativen, spektral fundierten Zugang zur Riemannschen Hypothese (RH) vor. Im Zentrum steht eine modulierte Exponentenfunktion \(\beta(n)\), deren innere Struktur eine spektrale Ordnung offenbart.

\section*{2. Die ZNF-Formel}
Die Grundform lautet:
\[
\zeta_F(s) = \sum_{n=1}^{\infty} \frac{1}{\left(A n^{\beta(n)} + C \log n + B \sin(\omega n + \varphi) \right)^s}
\]
mit einer analytisch kontrollierten Exponentenfunktion:
\[
\beta(n) = C_\beta \cos(\omega_\beta \log n + \varphi_\beta) + \alpha n^\delta + \beta_0
\]

\section*{3. Der spektrale Operator \( H_\beta \)}
Definiert als:
\[
H_\beta \psi(n) = \beta(n) \cdot \psi(n)
\]
mit spektraler Modulation über den Einheitskreis:
\[
\theta_n = \pi \cdot \beta(n) \Rightarrow e^{i \theta_n}
\]

\subsection*{Ableitung des Operators}
Die vollständige Ableitung ergibt sich zu:
\[
H'_\beta \psi(n) = \left( \frac{C_\beta \omega_\beta \sin(\omega_\beta \log n + \varphi_\beta)}{n} + \frac{\alpha \delta n^\delta}{n} \right) \cdot \psi(n) + \beta(n) \cdot \frac{d\psi(n)}{dn}
\]

\section*{4. Struktur der Kohärenzlänge}
Definiert durch:
\[
\Delta \beta(n) = \beta(n+1) - \beta(n)
\]
und empirisch gut angenähert durch:
\[
\Delta \beta(n) \approx A \cdot n^\gamma + C \cdot \sin(\omega \log n)
\]

\section*{5. Validierung}
\begin{itemize}
    \item \textbf{Korrelation mit Riemann-Nullstellen:} \(\rho_n \approx A n^\beta + C \log n\)
    \item \textbf{Spektralanalyse:} Fourier-Analyse der \(\Delta \beta(n)\) zeigt kohärente Frequenzstruktur
    \item \textbf{Numerische Resultate:} Pearson \(r \approx 0.999\), \( \text{MSE} \ll 1 \)
\end{itemize}

\section*{6. Fazit}
Die analytisch rekonstruierte Beta-Skala erweist sich als spektraler Träger der RH-Zahlenwelt. Der Operator \( H_\beta \) ist selbstadjungiert (in diskreter Form) und ermöglicht eine neue Interpretation der Riemannschen Hypothese als spektrales Phasenphänomen.

\vspace{0.5cm}
\hrule
\vspace{0.2cm}
\noindent \textbf{Anhänge:}
\begin{itemize}
    \item numerische Fits der Beta-Skala
    \item Visualisierung des Operators \( H_\beta \) im komplexen Spektrum
    \item Vergleich zu klassischen RH-Formulierungen (Hilbert–Pólya, Connes)
\end{itemize}

\end{document}