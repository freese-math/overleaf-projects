\documentclass[12pt]{article}
\usepackage{amsmath, amssymb, amsthm}
\usepackage{geometry}
\usepackage[utf8]{inputenc}
\geometry{a4paper, margin=2.5cm}

\title{\textbf{Axiomatischer Rahmen des Freese-Ansatzes zur Riemannschen Hypothese}}
\author{Freese et al.}
\date{\today}

\begin{document}
\maketitle

\section*{Einleitung}
Dieser axiomatische Rahmen beschreibt die Grundstruktur des Freese-Ansatzes zur Riemannschen Hypothese (RH), basierend auf spektraler Analyse der Zeta-Nullstellen, der Beta-Skala \(\beta(n)\), der Euler–Freese-Identität sowie der modifizierten Zetafunktion ZNF (Zeta Nova Freesiana). Ziel ist es, die zugrunde liegenden spektraltheoretischen Strukturen so zu formulieren, dass sie Grundlage einer vollständigen Beweisarchitektur sein können.

\section*{Axiom 1: Spektralstruktur der \(\beta\)-Skala}
Es existiert eine Folge \(\beta(n)\), die als spektral kohärente Modulationsreihe mit folgender Struktur dargestellt werden kann:
\[
\beta(n) = A + \frac{a}{\log n} + b \cos(2\pi f n + \phi) + c n
\]
Dabei sind \(A, a, b, f, \phi, c \in \mathbb{R}\), wobei \(f \approx 1 / (33 \cdot 249)\). Die Funktion \(\beta(n)\) ist aus den spektralen Eigenschaften des Operators mit Eigenwerten \(\rho_n\) (Zeta-Nullstellen) ableitbar.

\section*{Axiom 2: Euler–Freese-Kohärenz}
Die Phase \(\pi \beta(n)\) beschreibt einen spektralen Fixpunkt im Einheitskreis:
\[
e^{i\pi \beta(n)} + 1 = 0
\]
Dies ist äquivalent zur Aussage, dass \(\beta(n) \in 2\mathbb{Z} + 1\) für ideal kohärente Zustände.

\section*{Axiom 3: Struktur der ZNF (Zeta Nova Freesiana)}
Es existiert eine verallgemeinerte Zetafunktion der Form:
\[
\zeta_F(s) = \sum_{n=1}^\infty \frac{1}{\left(A n^\beta + C \log n + B \sin(\omega n + \phi)\right)^s}
\]
welche dieselben funktionalanalytischen Eigenschaften (analytische Fortsetzung, Funktionalgleichung, symmetrische Nullstellen) wie die klassische Riemannsche Zetafunktion aufweist.

\section*{Konsequenz: Spektralkohärenz erzwingt RH}
Die obigen Axiome führen zur Konsequenz, dass die Zeta-Nullstellen notwendigerweise auf der kritischen Linie liegen, da die β(n)-Modulation die einzige Form kohärenter spektraler Fixpunktstruktur erzeugt. Die Liouville-Rekonstruktion der klassischen \(\psi(x)\) mittels der β(n)-Modulierten Zeta-Spektren liefert eine numerisch identische Approximation.

\end{document}