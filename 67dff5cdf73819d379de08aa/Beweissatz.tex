\documentclass[11pt]{article}
\usepackage[utf8]{inputenc}
\usepackage{amsmath, amssymb}
\usepackage{graphicx}
\usepackage{physics}
\usepackage{caption}
\usepackage{geometry}
\geometry{a4paper, margin=2.5cm}

\title{Zur Struktur der freseschen Zeta-Funktion und deren spektraler Ordnung}
\author{Tim H. Freese}
\date{}

\begin{document}

\maketitle

\section*{1. Definition der freseschen Zeta-Funktion}

Wir definieren eine Zeta-artige Funktion \(\zeta_F(n)\), die oszillatorisch und dämpfend wirkt:

\begin{equation}
\zeta_F(n) := \sum_{k=1}^{n} \frac{(-1)^k}{k^{\beta}} \cdot \sin\left( \omega \log k + \varphi \right)
\end{equation}

Dabei sind:
\begin{itemize}
    \item \(\beta > 0\): Dämpfungsparameter (ähnlich wie in Dirichlet-Reihen)
    \item \(\omega \in \mathbb{R}\): Frequenzmodulation im logarithmischen Raum
    \item \(\varphi \in [0, 2\pi)\): Phasenverschiebung
\end{itemize}

Diese Funktion erzeugt eine über \(n\) abfallende, oszillierende Struktur, welche Kohärenzphänomene (z.B. der Zeta-Nullstellen) modellieren kann.

\section*{2. Vergleich zur Potenzform}

Die klassische Wachstumsform (z. B. für Kohärenzlängen) lautet:

\begin{equation}
L(n) = A n^{\beta} + x
\end{equation}

Sie beschreibt eine monotone Hüllkurve, z.\,B. für das Wachstum der Nullstellenverteilung oder Primzahldichte.

\vspace{1em}
\begin{figure}[h]
    \centering
    \includegraphics[width=0.9\textwidth]{vergleich-fresesche-zeta-vs-potenzform.png}
    \caption{Vergleich: Fresesche Zeta-Funktion vs. klassische Potenzform.}
\end{figure}

\section*{3. Satz: Spektrale Resonanzstruktur}

\textbf{Satz (Fresesche Resonanzstruktur):} \\
Für geeignete Parameter \((\beta, \omega, \varphi)\) beschreibt \(\zeta_F(n)\) eine oszillierende, dämpfende Struktur mit spektraler Ordnung. Diese steht in numerischer und analytischer Beziehung zur Nullstellenstruktur der Riemannschen Zeta-Funktion.

\textit{Beweisskizze:}
\begin{itemize}
    \item Die Funktion \(\zeta_F(n)\) enthält eine logarithmisch modulierte Frequenzstruktur.
    \item Der gewichtete sinusförmige Anteil erzeugt destruktive und konstruktive Interferenzen.
    \item Bei geeigneter Wahl von \((\beta, \omega)\) ergibt sich eine numerische Approximation an die Verteilung der \(\gamma_n\), den Imaginärteilen der Zeta-Nullstellen.
\end{itemize}

\section*{4. Schlussfolgerung}

Die fresesche Zeta-Funktion stellt eine Alternative bzw. Erweiterung der klassischen Zeta-Ansätze dar — mit Betonung auf spektrale, kohärente und resonante Strukturen. Ihre Nähe zur physikalischen Resonanztheorie macht sie geeignet für Anwendungen in Quantenmodellen, Operatorentheorie und Zahlenspektralanalyse.

\end{document}