\documentclass[a4paper,11pt]{article}
\usepackage[utf8]{inputenc}
\usepackage{amsmath, amssymb}
\usepackage{geometry}
\usepackage{lmodern}
\usepackage{hyperref}

\title{Beta-modulierte Zeta-Funktion \\ \large Eine Erweiterung der klassischen L-Reihe}
\author{Forschungsteam Zeta Nova Freesiana}
\date{\today}

\geometry{margin=2.5cm}

\begin{document}

\maketitle

\section*{1. Definition der Funktion}

Wir definieren die \textbf{Beta-modulierte Zeta-Funktion} \( \zeta_{F,\chi}(s) \) als:

\[
\zeta_{F,\chi}(s) = \sum_{n=1}^{\infty} \frac{\chi(n)}{n^s} \cdot \sin(\omega \log n + \phi)
\]

\noindent mit den Parametern:
\begin{itemize}
  \item \( \chi(n) \): ein Dirichlet-Charakter modulo \( q \)
  \item \( s = \sigma + it \in \mathbb{C} \): komplexes Argument
  \item \( \omega \in \mathbb{R}_{+} \): fundamentale Modulationsfrequenz, häufig gesetzt als \( \omega = \frac{1}{\varepsilon} \)
  \item \( \phi \in [0, 2\pi) \): feste Phasenverschiebung
\end{itemize}

\section*{2. Interpretation}

Diese Erweiterung erzeugt eine \textit{frequent modulierte Zeta-Reihe}, die harmonische Oszillationen auf logarithmischer Skala integriert. Die Modulation lässt sich als periodische \textbf{Skalenresonanz} interpretieren.

\subsection*{2.1 Spezielle Fälle}
Für \( \chi(n) = 1 \) (trivialer Charakter), reduziert sich \( \zeta_{F,\chi}(s) \) zu:

\[
\zeta_{F}(s) = \sum_{n=1}^{\infty} \frac{1}{n^s} \cdot \sin(\omega \log n + \phi)
\]

Diese Variante enthält bereits eine nichttriviale periodische Struktur, welche analysiert werden kann via Fourier-Transformationen oder Spektralanalyse.

\section*{3. Weiteres}

Die Struktur erlaubt Erweiterungen wie:
\[
\zeta_{F}(s) = \sum_{n=1}^{\infty} \frac{1}{n^s} \cdot \left(1 + \varepsilon \cdot \sin(\omega \log n + \phi)\right)
\]

\noindent wobei die harmonische Modulation als \textbf{Korrekturterm} zur klassischen Zeta-Reihe dient.

\vfill
\hrulefill

\noindent\footnotesize
Diese Definition ist Teil des Forschungsprojekts \emph{Zeta Nova Freesiana}, das die Verbindung von harmonischer Spektralanalyse und klassischen Zetafunktionen untersucht.

\end{document}