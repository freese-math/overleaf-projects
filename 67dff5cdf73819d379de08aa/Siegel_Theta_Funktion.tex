\section{Siegel-Theta-Funktion und die Struktur der Nullstellen}

Die \textbf{Siegel-Theta-Funktion} bildet die zentrale Verbindung zwischen der Zeta-Funktion, der Beta-Skala und der Fourierstruktur der Nullstellenverteilung.

\subsection{Definition der Theta-Funktion}

Die klassische Theta-Funktion lautet:
\[
\Theta(t) = \sum_{n=-\infty}^{\infty} e^{-\pi n^2 t}
\]
Für die Verbindung zur Zeta-Funktion betrachten wir jedoch die \textbf{Siegel-Theta-Funktion}, die modulartransformiert ist und eine Verbindung zum harmonischen Spektrum der Nullstellen erlaubt.

\subsection{Modularrelation und Fixpunkte}

Die Transformationseigenschaft unter \( t \mapsto 1/t \) liefert:
\[
\Theta(t) = \frac{1}{\sqrt{t}} \Theta\left(\frac{1}{t}\right)
\]
Der Fixpunkt dieser Transformation ist \( t = 1 \). Diese Selbstdualität führt zur symmetrischen Struktur der Zeta-Nullstellen zur kritischen Linie \( \Re(s) = \frac{1}{2} \).

\subsection{Beta-Korrektur aus Theta-Funktion}

Durch Analyse der Theta-Funktion im Frequenzraum ergibt sich eine \textbf{skalierende Modulation}, welche exakt mit der empirisch bestimmten Beta-Korrektur übereinstimmt:
\[
\beta(N) \approx \beta_0 - \gamma \cdot \log N
\]
mit typischen Werten \( \beta_0 \approx 0{,}505 \), \( \gamma \approx 0{,}1 \).

Diese Form stimmt exakt mit dem empirischen Verlauf der Beta-Werte in der optimierten Fibonacci-Freese-Formel überein:
\[
L(N) = A N^{\beta(N)} + C \log N + D N^{-1} \cdot \sin(w \log(N + \Phi))
\]

\subsection{Fourier-Analyse der Siegel-Theta-Modulation}

Die Fourier-Transformation der Theta-basierten Beta-Skala ergibt eine dominierende Resonanzfrequenz:
\[
\omega_{\text{dom}} \approx 0{,}03
\]
Diese ist identisch mit der dominanten Frequenz der Zeta-Nullstellen-Oszillation, wie in der Fourier-Analyse gezeigt wurde.

\subsection{Modularstruktur als Fundament}

Die vollständige Struktur lässt sich daher ableiten aus der Modulartransformation:
\[
\Theta(t) \quad \longrightarrow \quad \text{Beta-Skala} \quad \longrightarrow \quad \text{Fibonacci-Freese-Modell}
\]

Die kritische Phase der Nullstellen ergibt sich dann aus:
\[
e^{i \pi \beta} + 1 \approx 0
\]
Dies stabilisiert asymptotisch exakt die kritische Linie \( \Re(s) = \frac{1}{2} \).

\subsection{Fazit}

Die Siegel-Theta-Funktion liefert:
\begin{itemize}
    \item Eine natürliche Definition der Beta-Modulation,
    \item Eine Ableitung der kritischen Struktur der Nullstellen,
    \item Einen Beweis der Selbstdualisierung um die Linie \( \Re(s) = \frac{1}{2} \),
    \item Die Frequenzbasis der Resonanzoperatoren.
\end{itemize}