\section{Die Fibonacci-Freese-Formel (FFF)}

Die Fibonacci-Freese-Formel ist eine empirisch motivierte, aber mathematisch stark strukturierte Näherung für die Verteilung der Zeta-Nullstellen. Ihre Grundform lautet:
\[
L(N) = A \cdot N^{\beta} + C \cdot \log N
\]
wobei:
\begin{itemize}[label=\textbullet]
    \item \( A \) den globalen Skalierungsfaktor darstellt,
    \item \( \beta \) die Wachstumsrate der Nullstellenstruktur kodiert,
    \item \( C \) eine logarithmische Korrektur beschreibt.
\end{itemize}

\subsection{Erweiterung der FFF mit Oszillation}

Beobachtungen in der Fourier-Analyse zeigen eine resonante Modulationsstruktur, die durch einen sinusförmigen Korrekturterm modelliert werden kann:
\[
L(N) = A \cdot N^{\beta} + C \cdot \log N + D \cdot N^{-\gamma} \cdot \sin\left(\omega \log(N + \Phi)\right)
\]
Hierbei gelten:
\begin{itemize}[label=\textbullet]
    \item \( D \) ist die Amplitude der Oszillation,
    \item \( \gamma \) steuert die Dämpfung mit wachsendem \( N \),
    \item \( \omega \) ist die Modulationsfrequenz (z.\,B. \( \omega \approx 0.02499 \) aus der Fourier-Skala),
    \item \( \Phi \) ist eine Phasenverschiebung.
\end{itemize}

Diese Formel beschreibt die **Nullstellenverteilung als skaleninvariante Oszillation** mit starker struktureller Ähnlichkeit zur Fibonacci-Logik, insbesondere durch den logarithmischen Modulationsterm.

\subsection{Beispielhafte Parameter (optimiert)}

Eine typische Optimierung auf Basis von \( n = 1,\dots,1000 \) ergibt:

\begin{align*}
A &= 1.2852 \\
\beta &= 0.16608 \\
C &= -0.02096 \\
D &= 2.71201 \\
\omega &= 0.02499 \\
\Phi &= -6387.57 \\
\end{align*}

Diese Werte zeigen eine äußerst präzise Annäherung an die reale Zeta-Nullstellenstruktur. Die Konstante \( \beta \) weist außerdem eine tiefere Verbindung zur modularen Struktur der Theta-Funktion auf.