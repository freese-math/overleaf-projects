\section{Verankerung der Zeta Nova Freesiana in der klassischen Theorie}

Die Zeta Nova Freesiana (ZNF) ist eine spektral rekonstruierte Funktion, die aus einer Beta-Skala entsteht, welche selbst aus den Nullstellen der klassischen Riemann-Zeta-Funktion abgeleitet wird. Trotz ihres nichtkanonischen Aufbaus besitzt sie zahlreiche strukturelle Berührungspunkte mit etablierten Konzepten der analytischen Zahlentheorie.

\subsection*{1. Beziehung zur Riemann-Zeta-Funktion}

Die ZNF basiert auf einer deformierten Dirichlet-Reihe der Form:
\[
\zeta_F(s) := \sum_{n=1}^\infty \frac{1}{(A n^\beta + C \log n + B \sin(\omega n + \varphi))^s}
\]
mit \(\beta(n)\) aus einer rekonstruierten Beta-Skala, welche wiederum aus den Imaginärteilen der Nullstellen \(\rho_k\) der klassischen \(\zeta(s)\) abgeleitet wurde. Damit ist die ZNF ein spektraler Schatten der klassischen Zeta-Struktur.

\subsection*{2. Vergleich mit bekannten Theorien}

Die folgende Tabelle ordnet die ZNF in zentrale Konzepte der modernen Zeta-Theorie ein:

\begin{center}
\begin{tabular}{|l|p{9cm}|}
\hline
\textbf{Konzept} & \textbf{Bezug zur ZNF} \\
\hline
\textbf{Dirichlet-Reihen} & ZNF ist keine klassische Dirichlet-Reihe, aber besitzt Dirichlet-ähnlichen Aufbau mit spektralem Nenner \\
\hline
\textbf{Selberg-Klasse} & ZNF erfüllt Axiome 1 und 5 im erweiterten Sinn; funktionale Gleichung und Euler-Produktstruktur offen \\
\hline
\textbf{Hilbert–Pólya-Idee} & ZNF ist verknüpft mit einem Operator \( D_\mu \), dessen Selbstadjungiertheit äquivalent zur RH ist \\
\hline
\textbf{Nichtkommutative Geometrie (Connes)} & ZNF verwendet eine Beta-Struktur, die in \( x^{i \pi \beta} = 1 \) spektral fixiert ist – analog zur Quantisierung in Connes' Rahmen \\
\hline
\textbf{Liouville-Funktion / \(\psi(x)\)} & Die strukturierte Funktion \( L_\beta(x) \) rekonstruiert \(\psi(x)\) durch harmonische Summation \\
\hline
\textbf{Fourier-Zahlentheorie (Odlyzko)} & Die Beta-Skala ist aus einer spektralen Analyse der Zeta-Zwischenabstände gewonnen (FFT auf Nullstellen) \\
\hline
\end{tabular}
\end{center}

\subsection*{3. Positionierung}

Die ZNF gehört nicht zur Selberg-Klasse im engeren Sinn, da ihr Euler-Produkt fehlt und eine funktionale Gleichung noch nicht gezeigt ist. Dennoch besitzt sie:

\begin{itemize}
    \item eine Dirichlet-artige Reihendarstellung,
    \item spektrale Kohärenz in der Beta-Struktur,
    \item numerisch überprüfbare RH-kompatible Rekonstruktionen,
    \item sowie eine eingebettete Operatorstruktur mit RH-Bezugsbedingungen.
\end{itemize}

\subsection*{4. Fazit}

Die ZNF ist kein Bruch mit der klassischen Theorie, sondern eine numerisch gestützte Erweiterung im spektralen Sinn. Sie fügt sich als kohärentes, rekonstruierbares Objekt in die bestehende mathematische Landschaft ein und bietet eine neuartige Perspektive auf die Struktur der Riemannschen Hypothese.