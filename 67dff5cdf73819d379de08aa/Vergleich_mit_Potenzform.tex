\documentclass{article}
\usepackage{amsmath,amsfonts,graphicx}
\usepackage{booktabs}
\usepackage{geometry}
\geometry{margin=2.5cm}

\title{Vergleich: Fresesche Zeta-Funktion vs. Potenzform}
\author{}
\date{}

\begin{document}

\maketitle

\section*{Formale Gegenüberstellung}

\begin{table}[h!]
\centering
\renewcommand{\arraystretch}{1.4}
\begin{tabular}{@{}p{4.5cm} | p{5.8cm} | p{5.8cm}@{}}
\toprule
\textbf{Merkmal} & \textbf{Fresesche Zeta-Funktion} & \textbf{Potenzform $L(n) = A \cdot n^\beta + x$} \\
\midrule
\textbf{Formel} &
$\displaystyle \zeta_{F,\chi}(s) = \sum_{n=1}^\infty \frac{\chi(n)}{n^s} \cdot \sin(\omega \log n + \varphi)$ &
$\displaystyle L(n) = A \cdot n^\beta + x$ \\
\textbf{Mathematische Herkunft} & Verallgemeinerung der Dirichlet- und Zeta-Reihe mit Frequenzmodulation & Elementare Potenzfunktion \\
\textbf{Frequenzanteil} & Ja: $\sin(\omega \log n + \varphi)$ erzeugt Oszillation & Nein \\
\textbf{Phasenmodulation} & Ja, über $\varphi$ & Nein \\
\textbf{Arithmetische Einbettung} & Ja, über $\chi(n)$ (Dirichlet-Charakter) & Nein \\
\textbf{Verbindung zur Spektralanalyse} & Ja: Zeta-Nullstellen als Resonanzen & Nein \\
\textbf{Numerische Interpretation} & Modell für dynamische Zeta-Operatoren & Wachstumsmodell, empirisch \\
\textbf{Verwendungszweck} & Beweis der Riemannschen Hypothese, Operatorisches Modell & Approximation von Skalen, keine tiefere Zahlentheorie \\
\bottomrule
\end{tabular}
\caption{Struktureller Vergleich zwischen der freseschen Zeta-Funktion und einer klassischen Potenzfunktion}
\end{table}

\end{document}