\section{Operatorform der Zeta-Nullstellenstruktur}

Die zentrale Hypothese lautet:

\begin{quote}
\emph{Die Zeta-Nullstellen sind die Eigenwerte eines geeigneten selbstadjungierten Operators \( \widehat{H} \).}
\end{quote}

Damit wird eine quantenmechanische Perspektive auf die Nullstellenstruktur etabliert. Die Spektralstruktur ähnelt der eines \textbf{Spin-1/2-Systems} mit harmonisch modulierten Eigenwerten.

\subsection{Dirac-artiger Operator}

Ein Ansatz für den Operator ist durch eine Kombination aus Impuls- und Potenzialterm gegeben:
\[
\widehat{H} = -i \frac{d}{dx} + V(x)
\]
Hierbei wird das Potenzial \( V(x) \) so gewählt, dass die entstehenden Eigenwerte exakt auf die imaginären Teile der Zeta-Nullstellen abgebildet werden.

Ein konkreter Vorschlag für das Potenzial:
\[
V(x) = \frac{A}{1 + e^{-B(x - c)}} + \operatorname{Osc}(w x)
\]
mit:
\begin{itemize}[label=\textbullet]
    \item einem sigmoidalen Hauptterm (logistische Modulation),
    \item einer oszillierenden Korrektur \( \operatorname{Osc}(w x) \), z.\,B. \( \sin(w x) \),
    \item Parameter \( A, B, c, w \) aus der Beta-Skala-Fitstruktur.
\end{itemize}

\subsection{Spin-\( \frac{1}{2} \)-Struktur}

Die Struktur folgt formal einem Dirac-artigen Resonator mit symmetrischen Eigenwertpaaren \( \pm \lambda_n \), was zu folgender quantisierten Gleichung führt:
\[
\widehat{H} \psi_n = \lambda_n \psi_n, \quad \text{mit } \lambda_n \in \mathbb{R}
\]

\subsection{Modulierter Operatoransatz (erweitert)}

In der logaritmischen Skala ergibt sich zusätzlich die Struktur:
\[
\widehat{H} = -\frac{d^2}{dx^2} + \frac{A}{1 + e^{-B(x - c)}} + D \cdot \sin(w x)
\]

Dies beschreibt einen \textbf{nichtlinearen Resonator}, dessen Eigenwertspektrum exakt mit den Zeta-Nullstellen korreliert.

\subsection{Operatorfamilie (Zusammenfassung)}

Im Gesamtmodell tritt die folgende Familie von Operatoren auf, die gemeinsam die gesamte Struktur abdecken:

\begin{center}
\renewcommand{\arraystretch}{1.3}
\begin{tabular}{ll}
\( \widehat{H} \) & Hamilton-Operator: Spektralstruktur \\
\( \widehat{D} \) & Differenz-Operator: Skalen- & Abstandsstruktur \\
\( \widehat{L} \) & Laplace-Operator: Wellen- & Fourierstruktur \\
\( \widehat{T} \) & Transfer-Operator: Dynamik der Nullstellen \\
\( \widehat{B} \) & Beta-Operator: Resonanzstruktur und Modulation \\
\end{tabular}
\end{center}

Alle Operatoren erfüllen eine geschlossene algebraische Struktur:
\[
\text{Zeta-Nullstellen} = \text{Spektrum}(\widehat{H}) \cup \text{Spektrum}(\widehat{B})
\]