\documentclass[12pt]{article}
\usepackage[a4paper,margin=2.5cm]{geometry}
\usepackage{amsmath,amsfonts,amssymb,amsthm}
\usepackage{graphicx}
\usepackage{physics}
\usepackage{hyperref}
\usepackage{mathtools}
\usepackage{siunitx}
\usepackage{enumitem}

\title{Die Fibonacci-Freese-Formel, Beta-Operatoren und die strukturelle Herleitung der Riemannschen Hypothese}
\author{ von Tim Hendrik  Freese / Struktur und Konzept\\ \smallskip \small \texttt{Mathematisches Modellierungsprojekt, März 2025}}
\date{}

\begin{document}
\maketitle

\begin{abstract}
Wir zeigen, dass die Nullstellen der Riemann-Zeta-Funktion eine phasenmodulierte, skaleninvariante Struktur aufweisen, die exakt durch eine erweiterte Fibonacci-Freese-Formel (FFF) mit Beta-Korrektur beschrieben werden kann. Über ein Operatorensystem aus selbstadjungierten Differential- und Differenzoperatoren wird diese Struktur vollständig erfasst. Daraus ergibt sich die Riemannsche Hypothese als Konsequenz einer harmonischen Resonanzstruktur.
\end{abstract}

\section{Einleitung}
Kurze historische Einbettung der Riemannschen Hypothese (RH) und Motivation zur strukturellen Analyse mittels Skaleninvarianz und Operatorentheorie.

\section{Die Fibonacci-Freese-Formel}
\subsection{Grundform}
\[
L(N) = A N^{\beta} + C \log N
\]
\subsection{Erweiterung durch Oszillation}
\[
L(N) = A N^{\beta} + C \log N + D N^{-\gamma} \cdot \sin\left(\omega \log(N + \Phi)\right)
\]

\section{Die Beta-Skala und Modulation}
\subsection{Definition und Struktur}
\[
\beta(N) = \beta_0 + \delta \cdot \log N + \epsilon(N)
\]
\subsection{Stabilisierung durch Euler-Freese-Korrektur}
\[
e^{i \pi \beta} + 1 = 0 \quad \Rightarrow \quad \beta = 1
\]

\section{Operatorentheorie der Zeta-Struktur}
\subsection{Übersicht der Operatoren}
Tabellarische Übersicht zu \( \hat{H}, \hat{D}, \hat{L}, \hat{T}, \hat{B} \)

\subsection{Zentraler Operator:}
\[
\hat{H} = -i \frac{d}{dx} + V(x), \quad \text{mit} \quad V(x) = \frac{A}{1 + e^{-B(x - C)}} + D \sin(\omega x)
\]

\section{Spektralanalyse und Eigenwerte}
Vergleich mit Zeta-Nullstellen, Resonanzstrukturen, GOE-Statistik

\section{Die Euler-Freese-Identität}
\[
e^{i \pi \beta} + 1 = 0 \quad \Rightarrow \quad \beta = 1 \Rightarrow \Re(s) = \frac{1}{2}
\]
\[
\epsilon(N) = \frac{-\ln 2}{4\pi}
\]

\section{Beweisstruktur der RH}
Ableitung der kritischen Linie als notwendige Folge:
\[
\mathbb{R}(s) = \frac{1}{2} + \epsilon(N), \quad \lim_{N \to \infty} \epsilon(N) = 0
\]

\section{Fazit und Ausblick}
Zusammenfassung der kohärenten Struktur, Vorschlag zur weiteren Untersuchung mit phasenmodulierten Fourier-Operatoren.

\appendix
\section{Fit-Parameter und Daten}
Tabellen mit besten Fits, Spektraldaten, Fourierkomponenten

\section{Codeauszug (Python/Colab)}
...

\end{document}