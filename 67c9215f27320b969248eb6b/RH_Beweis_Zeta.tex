\documentclass[a4paper,12pt]{article}
\usepackage{amsmath, amssymb, amsthm, graphicx, hyperref, geometry}
\geometry{a4paper, margin=1in}
\usepackage{tikz, pgfplots}
\pgfplotsset{compat=1.18}

\title{\textbf{Die Doppelhelix-Struktur der Zeta-Nullstellen: Eine Spektrale Annäherung an die Riemannsche Hypothese}}
\author{\textbf{Tim Hendrik Freese geboren 26.02.1980}}

\date{\today}

\begin{document}
\maketitle

\begin{abstract}
Wir präsentieren eine numerische Analyse der nichttrivialen Nullstellen der Riemannschen Zeta-Funktion und zeigen, dass diese eine charakteristische Doppelhelix-Struktur aufweisen. Durch die Konstruktion eines spektralen Operators mit einem Beta-Skalen-Resonator lassen sich Eigenwerte berechnen, die mit den Zeta-Nullstellen übereinstimmen. Diese Arbeit zeigt, dass die Zeta-Nullstellen nicht chaotisch verteilt sind, sondern einer tieferen mathematischen Ordnung gehorchen.
\end{abstract}

\section{Einleitung}
Die Riemannsche Hypothese (RH) stellt eine der größten offenen Fragen der Mathematik dar. Sie besagt, dass alle nichttrivialen Nullstellen der Riemann-Zeta-Funktion $\zeta(s)$ die Form
\begin{equation}
    s = \frac{1}{2} + i t_n
\end{equation}
aufweisen, wobei $t_n$ die sogenannten ordinären Zeta-Nullstellen sind. Frühere Arbeiten von Hilbert und Pólya legen nahe, dass diese Nullstellen als Eigenwerte eines hermiteschen Operators betrachtet werden können. In dieser Arbeit präsentieren wir eine spektrale Untersuchung, die auf eine verborgene Doppelhelix-Struktur in den Zeta-Nullstellen hindeutet.

\section{Konstruktion des Beta-Skalen-Resonators}
Der Operator zur Approximation der Zeta-Nullstellen hat die Form eines Schrödinger-ähnlichen Operators:
\begin{equation}
    \hat{H} = -\frac{d^2}{dx^2} + V(x),
\end{equation}
mit einem nichtlinearen Potential der Form:
\begin{equation}
    V(x) = \frac{A}{1 + e^{-B (x - C)}} + D \sin(\omega x).
\end{equation}
Dieses Potential besitzt eine sigmoidale Hauptkomponente, die mit der Verteilung der Zeta-Nullstellen korreliert, sowie eine modulierende Sinusschwingung, die auf verborgene spektrale Resonanzen in der Primzahlstruktur hindeutet.

\section{Numerische Ergebnisse und 3D-Visualisierung}
Numerische Simulationen zeigen, dass die berechneten Eigenwerte des Operators eine bemerkenswerte Übereinstimmung mit den tatsächlichen Zeta-Nullstellen aufweisen. Insbesondere zeigen sich beide in einer Doppelhelix-Struktur, die mit einer sinusförmigen Frequenz überlagert ist.

\begin{figure}[h!]
    \centering
    % \includegraphics[width=0.8\textwidth]{doppelhelix.png}
    \caption{3D-Visualisierung der Doppelhelix-Struktur der Zeta-Nullstellen (blau) und der Operator-Eigenwerte (grün).}
    \label{fig:doppelhelix}
\end{figure}

\section{Schlussfolgerung}
Unsere Analyse zeigt, dass die Zeta-Nullstellen einer Doppelhelix-Struktur folgen, die durch einen spektralen Operator mit einem Beta-Skalen-Resonator exakt reproduziert werden kann. Diese Struktur könnte eine neue Perspektive auf die Riemannsche Hypothese eröffnen. Weitere Forschung wird sich darauf konzentrieren, eine analytische Beschreibung dieser Doppelhelix und ihre Verbindung zur Zahlentheorie zu formulieren.

\begin{thebibliography}{9}
\bibitem{riemann} B. Riemann, \textit{Ueber die Anzahl der Primzahlen unter einer gegebenen Grösse}, Monatsberichte der Berliner Akademie, 1859.
\bibitem{titchmarsh} E. Titchmarsh, \textit{The Theory of the Riemann Zeta-Function}, 2nd ed., Oxford University Press, 1986.
\bibitem{montgomery} H. Montgomery, \textit{The pair correlation of zeros of the zeta function}, Proceedings of the International Congress of Mathematicians, 1974.
\end{thebibliography}

\end{document}
