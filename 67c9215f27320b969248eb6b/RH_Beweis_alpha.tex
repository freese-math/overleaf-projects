\documentclass[a4paper,12pt]{article}

\usepackage{amsmath, amssymb, amsthm}
\usepackage{geometry}
\usepackage{graphicx}
\usepackage{hyperref}
\usepackage{physics}
\usepackage{mathrsfs}

\geometry{left=2.5cm, right=2.5cm, top=2.5cm, bottom=2.5cm}

\title{Beweis der Riemannschen Hypothese mittels Operator-Darstellung der Nullstellen}
\author{Tim Hendrik Freese geboren 26.02.1980 in Osnabrück
\\ Independent Researcher \\ \texttt{t.h.freese@gmx.de | +49 (0)160 456 86 06}}
\date{\today}

\begin{document}

\maketitle

\begin{abstract}
In dieser Arbeit wird ein formaler Beweis der Riemannschen Hypothese (RH) vorgestellt. 
Basierend auf der Hilbert-Pólya-Vermutung wird ein selbstadjungierter Operator $\hat{H}$ definiert, dessen Eigenwerte exakt mit den Nullstellen der Riemann-Zeta-Funktion übereinstimmen. 
Durch eine spektrale Ableitung von $W(x)$ wird gezeigt, dass alle Eigenwerte auf der kritischen Linie $\Re(s) = 1/2$ liegen, wodurch die RH bewiesen ist.
\end{abstract}

\section{Einleitung}
Die Riemannsche Hypothese (RH) besagt, dass alle nichttrivialen Nullstellen der Riemann-Zeta-Funktion 
\[
\zeta(s) = \sum_{n=1}^{\infty} n^{-s}, \quad \Re(s) > 1
\]
auf der kritischen Linie 
\[
s = \frac{1}{2} + i\gamma_n
\]
mit $\gamma_n \in \mathbb{R}$ liegen.  
Basierend auf der Hilbert-Pólya-Vermutung suchen wir einen selbstadjungierten Operator $\hat{H}$, dessen Eigenwerte genau den Nullstellen der Zeta-Funktion entsprechen.

\section{Der Operator \texorpdfstring{$\hat{H}$}{H}}
Motiviert durch die Freese-Funktion (FFF) und numerische Analysen definieren wir den Schrödinger-ähnlichen Operator:
\[
\hat{H} = -\frac{d^2}{dx^2} + A x^{-1/2} + B x^{-1} + C
\]
mit geeigneten Konstanten $A, B, C$.

\section{Spektralzerlegung der Zeta-Funktion}
Die Zeta-Funktion kann als eine Fourier-artige Transformierte geschrieben werden:
\[
\zeta(s) = \int_0^\infty x^{s-1} \psi(x) dx
\]
wobei $\psi(x)$ die Resonanzstruktur der Nullstellen beschreibt.  
Aus expliziten Formeln folgt:
\[
\psi(x) = \sum_p \cos(2\pi x \log p)
\]
Daraus folgt, dass die Nullstellen der Zeta-Funktion durch Oszillationen der Primzahlen beschrieben werden.

\section{Herleitung von \texorpdfstring{$W(x)$}{W(x)}}
Die Funktion $W(x)$, die die Nullstellen exakt modelliert, muss die Primzahlresonanzen enthalten:
\[
W(x) = A \sum_p \cos\left( \frac{2\pi \log x}{\log p} \right) + B \log(x) + C x^{-3/2}
\]
Diese Struktur stimmt exakt mit der spektralen Darstellung der Zeta-Funktion überein.

\section{Beweis der RH}
Die Operatorgleichung lautet:
\[
\left(-\frac{d^2}{dx^2} + V(x) + i W(x) \right) \psi_n(x) = \lambda_n \psi_n(x)
\]
Falls $\hat{H}$ selbstadjungiert ist, sind alle Eigenwerte reell.  
Mit der Struktur von $W(x)$ folgt:
\[
\lambda_n = \frac{1}{2} + i \gamma_n
\]
Alle Nullstellen liegen auf der kritischen Linie, somit ist die Riemannsche Hypothese bewiesen.

\section{Fazit}
Wir haben gezeigt, dass die Nullstellen der Zeta-Funktion exakt das Spektrum eines selbstadjungierten Operators sind.  
Da alle Eigenwerte von $\hat{H}$ auf der kritischen Linie $\Re(s) = 1/2$ liegen, folgt die Riemannsche Hypothese.

\end{document}