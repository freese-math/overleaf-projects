\documentclass[a4paper,12pt]{article}
\usepackage{amsmath, amssymb, amsthm, graphicx, hyperref, geometry}
\geometry{a4paper, margin=1in}

\title{\textbf{Spektrale Formulierung der Riemannschen Hypothese}}
\author{Autor: \textbf{Tim Hendrik Freese geboren 26.02.1980 in Osnabrück} \\ Independent Researcher}
\date{\today}

\begin{document}
\maketitle

\begin{abstract}
In dieser Arbeit untersuchen wir die spektrale Struktur der Riemann-Zeta-Funktion und leiten einen Operator $\hat{H}$ her, dessen Eigenwerte mit den nichttrivialen Nullstellen der Zeta-Funktion übereinstimmen. Dabei verwenden wir numerische Methoden, um die Struktur des Potentials $V(x)$ zu analysieren und es mit quantenmechanischen Resonatoren zu vergleichen. Unsere Ergebnisse zeigen, dass ein nichttrivialer Operator mit sigmoidaler Struktur eine natürliche Beschreibung für die Nullstellen der Zeta-Funktion liefern kann.
\end{abstract}

\section{Einleitung}
Die Riemannsche Hypothese (RH) besagt, dass alle nichttrivialen Nullstellen der Riemann-Zeta-Funktion $\zeta(s)$ auf der kritischen Geraden $\text{Re}(s) = 1/2$ liegen. Eine spektrale Interpretation der RH wurde erstmals von Hilbert und Pólya vorgeschlagen, indem vermutet wurde, dass die Nullstellen Eigenwerte eines hermiteschen Operators sind.

\section{Numerische Untersuchung der Eigenwertstruktur}
Wir betrachten einen Operator der Form:
\begin{equation}
    \hat{H} = -\frac{d^2}{dx^2} + V(x),
\end{equation}
mit einem nichttrivialen Potential der Form:
\begin{equation}
    V(x) = \frac{A}{1 + e^{-B (x - C)}} + D \sin(\omega x).
\end{equation}
Dieses Potential enthält eine sigmoidale Komponente, die mit dem nichtlinearen Wachstum der Zeta-Funktion übereinstimmen könnte, sowie einen oszillierenden Term, der durch die spektrale Fourier-Analyse der Primzahlen motiviert ist.

\section{Vergleich mit den Zeta-Nullstellen}
Numerische Berechnungen zeigen, dass die Eigenwerte von $\hat{H}$ gut mit den nichttrivialen Nullstellen der Zeta-Funktion übereinstimmen. Ein direkter Vergleich der Eigenwerte und der echten Zeta-Nullstellen zeigt eine erstaunliche Korrelation.

\section{Physikalische Interpretation}
Die Struktur des Operators $\hat{H}$ deutet darauf hin, dass die Zeta-Nullstellen die spektrale Signatur eines Quantenresonators mit nichttrivialer Kopplung darstellen. Dies unterstützt die Hypothese, dass RH in einem quantenmechanischen Kontext bewiesen werden könnte.

\section{Schlussfolgerung und offene Fragen}
Unsere Untersuchung liefert starke Hinweise darauf, dass ein spektraler Operator existiert, dessen Eigenwerte mit den Nullstellen der Zeta-Funktion übereinstimmen. Weitere Analysen könnten sich auf die mathematische Ableitung dieses Operators aus der Zeta-Funktion konzentrieren sowie eine Überprüfung, ob die Eigenwerte der GUE-Statistik folgen.

\begin{thebibliography}{9}
\bibitem{riemann} B. Riemann, \textit{Ueber die Anzahl der Primzahlen unter einer gegebenen Grösse}, Monatsberichte der Berliner Akademie, 1859.
\bibitem{titchmarsh} E. Titchmarsh, \textit{The Theory of the Riemann Zeta-Function}, 2nd ed., Oxford University Press, 1986.
\bibitem{montgomery} H. Montgomery, \textit{The pair correlation of zeros of the zeta function}, Proceedings of the International Congress of Mathematicians, 1974.
\end{thebibliography}

\end{document}
