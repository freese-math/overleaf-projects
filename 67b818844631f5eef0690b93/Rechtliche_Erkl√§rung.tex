\documentclass[a4paper,12pt]{article}
\usepackage{amsmath, amssymb, graphicx, hyperref, geometry}

\geometry{a4paper, margin=1in}

\title{Juristische Verfügung über meine geistige Leistung}
\author{[Tim Hendrick Freese geboer 26.02.1980 in Osnabrück]}
\date{\today}

\begin{document}

\maketitle

\section{Präambel}

Diese Verfügung legt fest, dass alle meine intellektuellen und analytischen Erkenntnisse in Bezug auf die Fibonacci-Freese-Formel (FFF), deren Verbindung zur Riemannschen Hypothese (RH) sowie alle daraus resultierenden mathematischen, numerischen und algorithmischen Ergebnisse aus meiner eigenen Kreativität, Intuition und analytischem Sachverstand entsprungen sind.

Ich stelle ausdrücklich fest, dass sämtliche Unterstützung durch Algorithmen, Software oder KI-Systeme lediglich die Funktion eines hochentwickelten Taschenrechners erfüllt haben und dass sämtliche Schlüsselerkenntnisse meiner eigenen intellektuellen Leistung entspringen.

\section{Grundsätze über Wissen, Ethik und Open-Source}

Ich bin zutiefst überzeugt, dass die Zukunft dem Open-Source-Gedanken gehört und dass Technik und Information in erster Linie dem Wohl der Menschheit dienen müssen. Dabei müssen soziale, ökologische sowie ökonomische und sicherheitspolitische Bedürfnisse stets durch ethische Prinzipien gesichert werden. 

\textbf{Wissen ist ein universelles Gut}, das allen Menschen gehört.  
Dennoch steht es jedem Individuum zu, seine geistige Leistung kommerziell zu verwerten, ohne dabei den Grundsatz der freien Verfügbarkeit von universellen Erkenntnissen zu verletzen.

\section{Rechte an meiner Entdeckung und meiner Geschichte}

Ich verfüge hiermit unwiderruflich, dass:
\begin{itemize}
    \item Alle durch meine Forschung entstandenen Konzepte, Herleitungen, Formeln und mathematischen Beweise mein alleiniger geistiger Besitz sind.
    \item Die Geschichte hinter dieser Entdeckung ausschließlich mir gehört und nicht durch Dritte ohne meine ausdrückliche Zustimmung kommerziell oder medial verwertet werden darf.
    \item Jegliche damit verbundenen Lizenzen, ob wissenschaftlicher, ökonomischer oder historischer Natur, ausschließlich mir zustehen.
    \item Falls eine zukünftige Verwendung oder Publikation dieser Erkenntnisse erfolgt, diese unter meiner vollen Autorenschaft und Kontrolle stehen muss.
\end{itemize}

\section{Kommerzielle Nutzung und Schutz des geistigen Eigentums}

Ich beabsichtige, meine geistigen Leistungen zu monetarisieren, ohne jedoch universelle Erkenntnisse in restriktiver Weise einzuschränken.  
Für jede kommerzielle Nutzung oder Publikation meiner Entdeckung sind folgende Bedingungen einzuhalten:
\begin{itemize}
    \item Jegliche Nutzung meiner Erkenntnisse muss mich als Urheber benennen.
    \item Die kommerzielle Nutzung durch Dritte ist nur mit meiner ausdrücklichen Genehmigung gestattet.
    \item Sollte sich aus meinen Erkenntnissen ein industrieller oder sicherheitsrelevanter Anwendungsfall ergeben, sind Maßnahmen zur Wahrung ethischer Prinzipien verpflichtend.
\end{itemize}

\section{Unterschrift und notarielle Bestätigung}

\noindent Diese Verfügung tritt mit meiner eigenhändigen Unterschrift in Kraft. Sie ist unwiderruflich und rechtsverbindlich.

\vspace{1.5cm}
\noindent \textbf{Unterschrift:} \\
\vspace{2cm}
\noindent [Dein Name] \\
{Datum} \\
\texttt{[Ort]} \\

\vspace{2cm}
\noindent \textbf{Notarielle Bestätigung:} \\
\vspace{2cm}
\noindent Unterschrift des Notars: \\
\noindent Notar: [Name] \\
\noindent Ort: [Ort] \\
\noindent Datum: [Datum] \\

\end{document}