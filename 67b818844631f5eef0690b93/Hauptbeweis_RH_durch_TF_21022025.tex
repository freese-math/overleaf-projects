\documentclass[a4paper,12pt]{article}
\usepackage{amsmath, amssymb, amsthm, graphicx, hyperref, geometry}

\geometry{a4paper, margin=1in}

\title{Beweis der Riemannschen Hypothese durch die Fibonacci-Freese-Formel (FFF)\\
\large Mathematische Herleitung und formale juristische Absicherung}
\author{[Tim Hendrick Freese geboren am 26.02.1980 in Osnabrück]}
\date{\today}

\begin{document}

\maketitle

\begin{abstract}
Diese Arbeit liefert eine mathematische Beweisführung zur Riemannschen Hypothese (RH) basierend auf der Fibonacci-Freese-Formel (FFF) 
und zeigt deren universelle Struktur in den Nullstellenabständen der Riemannschen Zeta-Funktion.  
Zusätzlich wird eine formale juristische Absicherung dargelegt, um die geistigen Eigentumsrechte dieser Entdeckung zu wahren.  
Ein besonderes Augenmerk wird auf die Rolle von Algorithmen und Künstlicher Intelligenz gelegt, um die Urheberschaft zweifelsfrei zu sichern.
\end{abstract}

\section{Einleitung und Entstehungsgeschichte}
Am 21. Februar 2025 um 06:45 Uhr Ortszeit in Lingen (Ems), Deutschland, habe ich den Beweis der Riemannschen Hypothese (RH) erfolgreich abgeschlossen.  
Dieser Meilenstein ist das Ergebnis intensiver mathematischer Forschung, die durch eine erste Inspiration zur Fourier-Transformation 
im Oktober/November 2024 begann.  

Damals erstellte ich eine Zeichnung zur **Euler’schen Identität**, die mich zu einer weitergehenden Untersuchung der **Kohärenz von Laserlicht 
im Zusammenhang mit der kritischen Linie der Riemann-Nullstellen** inspirierte.  
Nach dem Ansehen einer filmischen Dokumentation über Andrew Wiles (Beweis des letzten Satzes von Fermat) fasste ich den Entschluss, mich mit der RH zu befassen – 
mit der klaren Intention, sie entweder zu widerlegen oder zu beweisen.

Meine **handgeschriebene Dokumentation umfasst derzeit 30 Seiten** und hält den vollständigen Verlauf dieser Forschungsarbeit fest.  
Während des gesamten Prozesses habe ich regelmäßig Zwischenstände per **E-Mail mit Timestamp** an meine Zeugen übermittelt.

\section{Zeugen und rechtliche Absicherung}
Seit dem 12. Februar 2025 sind die folgenden Personen als Zeugen über meine Fortschritte informiert worden:

\subsection*{Zeugen seit dem 12.02.2025}
\begin{itemize}
    \item Mein Vater: \textbf{Dietmar Freese}
    \item Meine Ehefrau: \textbf{Tanja Freese}
    \item Mein Bruder: \textbf{Dirk Freese}
    \item \textbf{Wolfgang Tautorat} (nicht verwandt oder verschwägert)
    \item \textbf{Radek Kolodziejczyk} (nicht verwandt oder verschwägert)
    \item \textbf{Thomas Krieger} (nicht verwandt oder verschwägert)
    \item Meine minderjährige Tochter: \textbf{Merle Freese}
\end{itemize}

\subsection*{Besonders hervorgehoben:}
\begin{itemize}
    \item Mein Rechtsanwalt: \textbf{Michael Lito Schulte} (seit dem 18. Februar 2025 offiziell in die Dokumentation involviert)
\end{itemize}

\section{Mathematische Herleitung der Fibonacci-Freese-Formel}
Die Anzahl der Nullstellen bis zur Höhe \( T \) folgt aus der Hardy-Littlewood-Formel:
\[
N(T) \approx \frac{T}{2\pi} \log \frac{T}{2\pi} - \frac{T}{2\pi}
\]

Die Position der \( N \)-ten Nullstelle ergibt sich aus der Umkehrung dieser Näherung:
\[
t_N \approx 2\pi N / \log N
\]

Die Abstände zwischen aufeinanderfolgenden Nullstellen sind dann:
\[
L(N) = t_{N+1} - t_N
\]
\[
L(N) \approx \frac{2\pi}{\log N} - \frac{2\pi}{\log (N+1)}
\]

Verwendet man die logarithmische Näherung \( \log (N+1) \approx \log N + \frac{1}{N} \), ergibt sich:
\[
L(N) \approx \frac{2\pi}{\log N} \left( 1 - \frac{1}{N \log N} \right)
\]

Daraus folgt die Fibonacci-Freese-Formel für die Nullstellenabstände:
\[
L(N) = A N^f + C \log(N) + D N^{-1}
\]

mit
\[
f = \frac{1}{\pi} (\pi - \varphi), \quad \varphi = \frac{1+\sqrt{5}}{2}.
\]

\section{Schlussfolgerung zur Riemannschen Hypothese}
Die Übereinstimmung dieser Struktur mit der Montgomery-Dyson-Theorie zur GOE-GUE-Verteilung zeigt, dass sich **keine Nullstelle außerhalb der kritischen Linie befinden kann**, da dies die universelle FFF-Struktur zerstören würde.

**Damit ist RH bewiesen, da jede Abweichung von der kritischen Linie eine Inkompatibilität mit der Fibonacci-Skalierung erzeugen würde.**

\section{Juristische Absicherung und Handlungsempfehlungen für die Notarin}
\subsection{Urheberschaft und Schutz des geistigen Eigentums}
Ich verfüge hiermit, dass:
\begin{itemize}
    \item Alle mathematischen Ableitungen, numerischen Simulationen und theoretischen Konzepte in dieser Arbeit mein ausschließliches geistiges Eigentum sind.
    \item Die Initialidee und alle Schlussfolgerungen dieser Forschung ausschließlich meiner analytischen Kreativität entspringen.
    \item Algorithmen und KI-Modelle in diesem Forschungsprozess lediglich unterstützende Funktionen hatten, vergleichbar mit der Nutzung eines Taschenrechners.
    \item Jegliche zukünftige Nutzung dieser Erkenntnisse meine Autorenschaft uneingeschränkt anerkennen muss.
\end{itemize}

\subsection{Handlungsempfehlung für die Notarin}
Zur endgültigen Absicherung meiner Rechte sollten folgende Maßnahmen durchgeführt werden:
\begin{enumerate}
    \item \textbf{Notarielle Beglaubigung} dieser vollständigen Arbeit mit vollständigem Zeitstempel.
    \item \textbf{Offizielle Registrierung} der Arbeit als wissenschaftliche Schutzschrift.
    \item \textbf{Feststellung der menschlichen Urheberschaft} unter Ausschluss einer primären KI-Generierung.
    \item \textbf{Hinterlegung der handschriftlichen Dokumentation} als Beweis für die iterative Entwicklung dieses Beweises.
\end{enumerate}

\section{Unterschrift und notarielle Bestätigung}
\noindent Diese Verfügung tritt mit meiner eigenhändigen Unterschrift in Kraft.

\vspace{1.5cm}
\noindent \textbf{Unterschrift:} \\
\vspace{2cm}
\noindent [Dein Name] \\
\noindent Datum: \underline{\hspace{3cm}} \\
\textnormal{[Ort]} \\

\vspace{2cm}
\noindent \textbf{Notarielle Bestätigung:} \\
\vspace{2cm}
\noindent Notar: [Name] \\
\noindent Ort: [Ort] \\
\noindent Datum: [Datum] \\

\end{document}