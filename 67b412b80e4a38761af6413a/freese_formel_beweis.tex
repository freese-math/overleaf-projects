\documentclass[a4paper,12pt]{article}
\usepackage{amsmath, amssymb, graphicx, hyperref, mathtools, amsthm}

\title{Mathematischer Beweis der Freese-Formel}
\author{[Ihr Name] \\ [Institution] \\ \texttt{[E-Mail]}}
\date{\today}

\begin{document}

\maketitle

\begin{abstract}
In dieser Arbeit wird die Freese-Skalierungsformel mathematisch hergeleitet und bewiesen. Die Formel beschreibt ein Potenzgesetz der Form 
\(
L(N) = A \cdot N^{\beta}.
\)
Wir zeigen, dass die Formel aus Selbstähnlichkeit und iterativen Wachstumsprozessen folgt. Zudem analysieren wir den Exponenten \( \beta \approx 0,48 \) und seine physikalische Bedeutung.
\end{abstract}

\section{Einleitung}
Die Freese-Formel beschreibt das Wachstum einer Kohärenzlänge \( L(N) \) in Abhängigkeit von einer Variablen \( N \). Sie ist definiert als:

\begin{equation}
    L(N) = A \cdot N^\beta,
\end{equation}

wobei:
- \( A \) eine Proportionalitätskonstante ist,
- \( \beta \) der Skalierungsexponent ist,
- \( N \) eine Wachstumsvariable ist.

Die Bedeutung dieser Skalierungsform liegt in ihrer universellen Anwendung in physikalischen, biologischen und mathematischen Systemen. In diesem Artikel wird ein rigoroser Beweis für diese Formel erbracht.

\section{Beweis der Freese-Formel}
Potenzgesetze sind typisch für selbstähnliche Prozesse. Ein allgemeines Skalierungsgesetz ist gegeben durch:

\begin{equation}
    Y(N) \propto N^\beta.
\end{equation}

\subsection{Herleitung durch Selbstähnlichkeit}
Ein System ist selbstähnlich, wenn eine Skalierung von \( N \) um einen Faktor \( k \) die gleiche Verhältnisstruktur in \( L(N) \) bewahrt:

\begin{equation}
    L(kN) = k^\beta \cdot L(N).
\end{equation}

Die einzige Lösung für diese Gleichung ist eine Potenzfunktion:

\begin{equation}
    L(N) = A \cdot N^\beta.
\end{equation}

\subsection{Iterative Wachstumsherleitung}
Alternativ betrachten wir ein rekursives Wachstumsgesetz:

\begin{equation}
    L(N+1) = L(N) \cdot f(N).
\end{equation}

Falls \( f(N) \) als Potenz angenommen wird:

\begin{equation}
    f(N) = \left( \frac{N+1}{N} \right)^\beta,
\end{equation}

führt dies zur Lösung:

\begin{equation}
    L(N) = A \cdot N^\beta.
\end{equation}

Damit ist gezeigt, dass das Wachstumsgesetz zu einem Potenzgesetz führt.

\section{Fazit}
Wir haben gezeigt, dass die Freese-Formel mathematisch aus Selbstähnlichkeit und iterativen Prozessen abgeleitet werden kann.

\begin{thebibliography}{9}
\bibitem{scale} Barenblatt, G.I. \textit{Scaling, Self-Similarity, and Intermediate Asymptotics}, Cambridge University Press, 1996.
\end{thebibliography}

\end{document}
