\documentclass[12pt,a4paper]{article}
\usepackage{amsmath, amssymb, amsfonts}
\usepackage{graphicx}
\usepackage[utf8]{inputenc}
\usepackage{hyperref}
\usepackage{physics}
\usepackage{geometry}
\geometry{margin=2.5cm}
\usepackage{lmodern}
\usepackage{enumitem}

\title{Beweisrahmen der Riemannschen Hypothese im Kontext der Beta-Skala}
\author{Projekt Freesiana}
\date{\today}

\begin{document}

\maketitle

\section*{Einleitung: Ein neuer Zugang zur Riemannschen Hypothese}

Die Riemannsche Zetafunktion
\[
\zeta(s) = \sum_{n=1}^{\infty} \frac{1}{n^s}
\]
ist eine zentrale Funktion der analytischen Zahlentheorie. Ihre nichttrivialen Nullstellen liegen — gemäß der berühmten Riemannschen Hypothese — alle auf der kritischen Geraden $\Re(s) = \frac{1}{2}$. Trotz enormer Fortschritte in der numerischen Verifikation und theoretischen Entwicklung ist die Hypothese bis heute unbewiesen.

\subsection*{Motivation: Die Freesiana-Zetafunktion und die Beta-Skala}

Wir schlagen eine neue spektrale Struktur auf Basis der sogenannten \emph{Beta-Skala} vor, die aus der Optimierung von Kohärenzlängen im Zusammenhang mit Nullstellen der Zetafunktion hervorgegangen ist. Dabei entsteht die modifizierte Funktion
\[
\zeta_F(s) = \sum_{n=1}^{N_{\max}} \frac{1}{L(n)^s}
\]
mit $L(n)$ als \textbf{Freesiana-Kohärenzlängenfunktion}, welche asymptotisch einem Ausdruck der Form
\[
L(n) \sim A n^\beta + a \left( \frac{1}{\log(n)} \right) + b e^{-\pi}
\]
folgt, mit empirisch bestimmten Parametern $A, \beta, a, b$.

Die Funktion $\zeta_F(s)$ konvergiert langsamer als die klassische Riemann-Zetafunktion, zeigt jedoch bemerkenswerte strukturelle Parallelen, insbesondere in Bezug auf spektrale Resonanzerwartungen und Nullstellenverhalten. Ein besonders stabiler Wert ergibt sich etwa für $\zeta_F(2)$:
\[
\zeta_F(2) \approx 0.28213232454651299276563112896950833091809552541373
\]

\subsection*{Strukturelle Perspektive und Operatorzugang}

Basierend auf der diskreten Skalenstruktur definieren wir einen Dirac-ähnlichen Operator auf dem Beta-Raum:
\[
H_\beta = i \varepsilon \frac{\Delta}{\Delta \beta}
\]
und formulieren eine eigene Dirac-Gleichung auf diskreter Skala:
\[
(i H_\beta - m)\psi(\beta) = 0
\]
in Matrixform:
\[
\mathcal{H}_{\text{Dirac}} =
\begin{pmatrix}
0 & H_\beta \\
H_\beta & 0
\end{pmatrix}
\psi(\beta)
= m \psi(\beta)
\]
Die Eigenwerte dieser Struktur zeigen eine chirale Symmetrie, analog zur kontinuierlichen Theorie, was auf ein spektral topologisches Ordnungsprinzip im Raum der Nullstellen hindeutet.

\subsection*{Zielsetzung des Rahmens}

Dieser Beweisrahmen verfolgt das Ziel, den Zusammenhang zwischen:
\begin{itemize}[itemsep=2pt]
  \item der diskreten Beta-Struktur,
  \item der spektralen Interpretation der Zetafunktion
  \item und topologischen Modellen (wie Dirac-Strings, Monopolen, Spin-Eis-Systemen)
\end{itemize}
in ein kohärentes Modell zu überführen, das die Grundidee der Riemannschen Hypothese auf eine neue mathematisch-physikalische Ebene hebt.

\vspace{1em}
\textbf{Nächste Kapitel:}
\begin{enumerate}
    \item Mathematische Definition der Beta-Skala
    \item Konstruktion der Freesiana-Zetafunktion
    \item Operatorstruktur und numerische Diagonalisierung
    \item Vergleich mit klassischen Nullstellen (Odlyzko, LMFDB)
    \item Topologische Interpretation (Eichtheorien, Strings)
    \item Diskussion der Clay-Konformität und Aussagekraft
\end{enumerate}

\end{document}