\documentclass[12pt]{article}
\usepackage{amsmath,amssymb,amsthm}
\usepackage{geometry}
\geometry{a4paper, margin=2.5cm}
\usepackage{hyperref}

\title{Strukturelle Stabilisierung der kritischen Linie durch fraktale Spektralkohärenz}
\author{}
\date{}

\newtheorem{theorem}{Theorem}
\newtheorem{lemma}{Lemma}

\begin{document}

\maketitle

\section*{Strukturelles Theorem zur Stabilität der kritischen Linie}

\begin{theorem}[Fraktale Kohärenz und kritische Linie]
Sei \(\hat{D}\) ein gewichteter Skalenoperator im MOS-System gemäß
\[
\hat{D} = \sum_{n=1}^\infty w_n \cdot D_n, \quad \text{mit } w_n \in \mathbb{R}_+, \quad D_n := N^{-\alpha_n} \cdot \Delta_n,
\]
und sei \(\hat{H}\) ein selbstadjungierter Hamiltonoperator mit Spektrum
\[
\text{Spec}(\hat{H}) = \{ t_n \in \mathbb{R} \mid \zeta(\tfrac{1}{2} + i t_n) = 0 \}.
\]
Dann stabilisiert \(\hat{D}\) durch fraktale Spektralkohärenz die Folge \((t_n)\) auf der kritischen Linie \(\Re(s) = \tfrac{1}{2}\), falls folgende Bedingungen erfüllt sind:
\begin{enumerate}
  \item Die Fouriertransformation \(\text{FFT}(\Delta t_n)\) weist eine logarithmisch skalierte Peak-Struktur auf,
  \item die Wavelet-Transformation \(\text{CWT}(\beta(t))\) zeigt skaleninvariante Resonanzbänder,
  \item und die Ableitungen \(\beta''(t)\) sind asymptotisch konstant (bzw. fraktal-stabilisiert).
\end{enumerate}
\textbf{Folgerung:}  
Die kritische Linie ist spektral invariant unter \(\hat{D}\), d.h.  
\[
\hat{D} \cdot \{t_n\} \sim \text{selbstähnliche Skalenstruktur} \quad \Rightarrow \quad \Re(\rho_n) = \tfrac{1}{2}.
\]
\end{theorem}

\section*{Beweisidee (heuristisch)}
Die Wirkung von \(\hat{D}\) als kohärent skalierender Operator erzeugt eine deterministische Frequenzordnung in den Nullstellenabständen. Die empirisch beobachtete FFT- und Wavelet-Struktur ist exakt jene, die ein solches Operatorenspektrum erzeugen würde. Die Skaleninvarianz der Struktur (insbesondere bei \(\beta(t)\) und \(\beta''(t)\)) ist nur erklärbar durch eine zugrunde liegende spektrale Selbstähnlichkeit.

\textbf{Physikalischer Hintergrund:}  
Die Darstellung \(\hat{H} \sim \sum n^{-3}\) weist auf ein quantisiertes, resonantes Spektrum hin, analog zur Siegel-Theta-Funktion und bekannten Quantenresonatoren. Dies stützt die strukturelle Notwendigkeit einer fraktal-stabilisierenden Transformation entlang der kritischen Linie.

\end{document}