\documentclass[12pt]{article}
\usepackage{amsmath, amssymb, physics, geometry}
\geometry{a4paper, margin=2.5cm}

\title{Spektralprojektion durch den Operator \(\mathcal{T}_\beta\)}
\author{Tim Hendrik Freese}
\date{April 2025}

\begin{document}
\maketitle

\section*{1. Definition des Operators \(\mathcal{T}_\beta\)}

Sei \(\hat{\beta}(\omega)\) das Frequenzspektrum einer rekonstruierten spektralen Wellenfunktion \(\beta(n)\), so definieren wir den Projektionsoperator \(\mathcal{T}_\beta\) durch:

\begin{equation}
(\mathcal{T}_\beta \hat{\beta})(\omega) := \sum_{p \leq P} w_p \cdot \exp\left( -\frac{(\omega - \log(p)/2\pi)^2}{2\sigma^2} \right) \cdot \hat{\beta}(\omega)
\end{equation}

\textbf{Notation:}
\begin{itemize}
    \item \(p\): Primzahl
    \item \(w_p\): Gewichtung (z.\,B. \(w_p = 1/\sqrt{p}\) oder konstant)
    \item \(\sigma\): Standardabweichung der gaußschen Glocke
    \item \(\omega\): Frequenzvariable im reellen Spektrum
\end{itemize}

\section*{2. Invarianzsatz (Spektrale Fixierung durch Primstruktur)}

\textbf{Satz:}  
Ist die Funktion \(\hat{\beta}(\omega)\) spektral auf \(\omega_p = \log(p)/2\pi\) konzentriert, so ist \(\hat{\beta}\) eine Eigenfunktion von \(\mathcal{T}_\beta\) mit approximativem Eigenwert \(\lambda \approx \sum w_p\).

\textit{Beweisidee:}  
Da jede Gaußfunktion lokal zentriert bei \(\omega_p\) wirkt, bleibt \(\hat{\beta}(\omega)\) bei Anwendung von \(\mathcal{T}_\beta\) invariant bis auf eine Skalierung durch \(\lambda\). Damit gilt:

\[
\mathcal{T}_\beta \hat{\beta} \approx \lambda \cdot \hat{\beta}
\]

Dies zeigt, dass die primzahlmodulierten Frequenzen spektral stabil unter \(\mathcal{T}_\beta\) sind.

\section*{3. Numerische Illustration (Python/Colab)}

Ein Python-Skript berechnet \(\hat{\beta}(\omega)\) via FFT und wendet \(\mathcal{T}_\beta\) auf diskrete Frequenzwerte an. Die dominanten Peaks bei \(\log(p)/2\pi\) können numerisch mit den Primzahlen \(p \leq 100\) korreliert werden.

\textbf{Plot:} \texttt{FFT} von \(\beta(n)\) vor und nach Anwendung von \(\mathcal{T}_\beta\), Vergleich mit \(\log(p)/2\pi\).

\end{document}