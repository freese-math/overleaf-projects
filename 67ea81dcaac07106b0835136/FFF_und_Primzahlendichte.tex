\documentclass[12pt]{article}
\usepackage{amsmath,amssymb}
\usepackage{graphicx}
\usepackage{lmodern}
\usepackage[margin=2.5cm]{geometry}

\title{Erweiterte Fibonacci--Freese-Formel zur Approximation der Primzahldichte}
\author{Tim Hendrik Freese}
\date{\today}

\begin{document}

\maketitle

\section*{Einleitung}

Die vorliegende Arbeit präsentiert eine strukturierte Erweiterung der klassischen Fibonacci--Freese-Formel (FFF), welche zur präzisen Approximation der Primzahlzählfunktion $\pi(N)$ dient. Die Formel basiert auf einer harmonischen Analyse der Zeta-Nullstellenstruktur, ergänzt durch logarithmische und oszillatorische Terme. Ihre spektrale Stabilität und die numerische Präzision deuten auf eine tiefere Ordnung in der Primzahldichte hin.

\vspace{1em}

\section*{Theorem: Erweiterte FFF-Approximation der Primzahldichte}

\textbf{Aussage.}
Sei $\pi(N)$ die Primzahlzählfunktion. Dann lässt sich für $N \in \mathbb{N}$ die Funktion $\pi(N)$ mit hoher numerischer Präzision durch die folgende 6-Parameter-Funktion $L_{\text{FFF}}(N)$ approximieren:

\begin{equation}
L_{\text{FFF}}(N) = A N^{B} + C \log N + \frac{D}{N} + \delta \cdot \sin\left( \omega \log N + \phi \right)
\end{equation}

\textbf{Mit den numerisch optimierten Parametern:}
\begin{align*}
A &= 0.591818, \quad
B = 0.825225, \quad
C = -18.712763, \\
D &= 1.517683, \quad
\delta = -161.634302, \quad
\omega = -0.123339, \quad
\phi = 0.011209
\end{align*}

\textbf{Fehlermaße:}
\begin{itemize}
  \item Mittlere absolute Abweichung: $\text{MAE} = 0.5618$
  \item Mittlere quadratische Abweichung: $\text{MSE} = 0.5028$
\end{itemize}

\vspace{1em}

\section*{Visualisierung}

\begin{center}
\includegraphics[width=\textwidth]{vergleich_fff_pi.png}
\end{center}

\vspace{1em}

\section*{Schlussfolgerung}

Die erweiterte FFF bildet eine kohärente Brücke zwischen analytischer Zahlentheorie, spektraler Operatorstruktur und der beobachtbaren Primzahlverteilung. Sie liefert eine präzise Näherung mit stark reduzierter Abweichung und weist auf eine zugrunde liegende spektrale Ordnung hin, die tief mit der Riemannschen Zeta-Funktion verknüpft ist.

\end{document}