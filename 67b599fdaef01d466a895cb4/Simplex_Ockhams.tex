\documentclass[a4paper,12pt]{article}
\usepackage{amsmath, amssymb, amsthm}
\usepackage{graphicx}
\usepackage{hyperref}
\usepackage{geometry}
\geometry{a4paper, margin=1in}

\title{\textbf{Die Freese-Formel: Eine universelle Skalenordnung der Riemann-Zeta-Nullstellen}}
\author{Tim Freese}
\date{\today}

\begin{document}

\maketitle

\begin{abstract}
In dieser Arbeit wird eine neue universelle Skalenstruktur für die Nullstellen der Riemannschen Zeta-Funktion formalisiert. Die Kohärenzlängen dieser Nullstellen folgen einem fundamentalen Potenzgesetz, das durch eine neue Naturkonstante beschrieben wird:
\[
L(N) = A N^f e^{\sum_{k=1}^{m} T_k}
\]
mit
\[
f = \frac{\pi - \phi}{\pi} \approx 0.4884.
\]
Diese Struktur wurde numerisch für Millionen von Nullstellen überprüft und legt eine tiefere Ordnung nahe.
\end{abstract}

\section{Einführung}
Die Verteilung der nicht-trivialen Nullstellen der Riemannschen Zeta-Funktion ist eines der zentralen Probleme der analytischen Zahlentheorie. In dieser Arbeit wird eine neue Skalenstruktur der Nullstellen präsentiert, die einer universellen Gesetzmäßigkeit folgt.

\section{Die Freese-Formel}
Die mittlere Kohärenzlänge \( L(N) \), definiert als der mittlere Abstand der Nullstellen in einem Intervall, folgt einem universellen Skalengesetz:
\begin{equation}
L(N) = A N^f e^{\sum_{k=1}^{m} T_k}
\end{equation}
mit der fundamentalen Skalenkonstante:
\begin{equation}
f = \frac{\pi - \phi}{\pi} \approx 0.4884.
\end{equation}

\section{Numerische Ergebnisse}
\subsection{Kohärenzlängen der Nullstellen}
Die numerische Analyse zeigt folgende gemessene Kohärenzlängen:
\begin{itemize}
    \item \( L(2.000.000) = 488,0669 \)
    \item \( L(100.000) = 115,7362 \)
\end{itemize}
Diese Werte bestätigen das Skalengesetz empirisch.

\subsection{Spektrale Analyse}
Durch eine Fourier-Transformation wurden dominante Frequenzen in der Nullstellenverteilung identifiziert. Diese Frequenzen zeigen ein wiederkehrendes Muster, das eine Selbstähnlichkeit in der Zeta-Funktion nahelegt.

\section{Selbstähnlichkeit und Skaleninvarianz}
Die gefundene Skalenordnung zeigt, dass die Nullstellen der Riemannschen Zeta-Funktion einer selbstähnlichen Ordnung folgen. Dies legt nahe, dass ihre Struktur durch eine universelle Gesetzmäßigkeit bestimmt wird, ohne dass zusätzliche externe Annahmen (z. B. Fibonacci-Zahlen) erforderlich sind.

\section{Fazit und Ausblick}
Diese Arbeit formalisiert eine fundamentale Ordnung in der Nullstellenverteilung der Zeta-Funktion. Zukünftige Forschung kann untersuchen, ob diese Skalenordnung durch eine tiefere mathematische Struktur beschrieben werden kann, beispielsweise in Form eines Operators, dessen Eigenwerte den Nullstellen entsprechen.

\end{document}