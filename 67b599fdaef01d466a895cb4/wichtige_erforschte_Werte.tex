\documentclass[a4paper,12pt]{article}
\usepackage{amsmath, amssymb, amsthm, geometry, booktabs, array}
\geometry{a4paper, margin=2.5cm}
\usepackage{graphicx}
\usepackage{hyperref}

\title{\textbf{Dokumentation der Fibonacci-Freese-Formel: \\ Alle bekannten Konstanten, Frequenzen und Invarianten}}
\author{Tim Freese}
\date{19. Februar 2025}

\begin{document}

\maketitle

\section{Einleitung}
Dieses Dokument enthält eine vollständige Auflistung aller numerisch bestimmten Konstanten, Frequenzen und Invarianten, die im Rahmen der Fibonacci-Freese-Formel und deren Anwendung auf die Nullstellen der Riemannschen Zeta-Funktion identifiziert wurden.

\section{Bekannte mathematische Konstanten}
\begin{table}[h]
    \centering
    \renewcommand{\arraystretch}{1.4}
    \begin{tabular}{l c l}
        \toprule
        \textbf{Name} & \textbf{Symbol} & \textbf{Wert} \\
        \midrule
        Goldener Schnitt & \( \varphi \) & \( \frac{1+\sqrt{5}}{2} \approx 1.6180339887 \) \\
        Fibonacci-Freese-Skalenexponent & \( f \) & \( \frac{\pi - \varphi}{\pi} \approx 0.4884 \) \\
        Universelle Potenzskala & \( \beta \) & \( 0.4884 \) \\
        Natürliche Logarithmusbasis & \( e \) & \( 2.7182818284 \) \\
        \(\pi\) & \( \pi \) & \( 3.1415926535 \) \\
        Euler-Mascheroni-Konstante & \( \gamma \) & \( 0.5772156649 \) \\
        Riemannsche Zeta-Funktion an \( s=2 \) & \( \zeta(2) \) & \( \frac{\pi^2}{6} \approx 1.64493 \) \\
        \bottomrule
    \end{tabular}
    \caption{Mathematische Konstanten der Fibonacci-Freese-Formel}
\end{table}

\section{Bekannte Hauptfrequenzen der Nullstellenstruktur}
Die Spektralanalyse zeigt, dass die Zeta-Nullstellen einer dominanten Frequenzstruktur folgen. Die zehn wichtigsten Hauptfrequenzen sind:

\begin{table}[h]
    \centering
    \renewcommand{\arraystretch}{1.4}
    \begin{tabular}{c c}
        \toprule
        \textbf{Rang} & \textbf{Frequenz} \\
        \midrule
        1 & \( 2.4986 \times 10^{-4} \) \\
        2 & \( -1.9989 \times 10^{-4} \) \\
        3 & \( 1.9989 \times 10^{-4} \) \\
        4 & \( -1.4991 \times 10^{-4} \) \\
        5 & \( 1.4991 \times 10^{-4} \) \\
        6 & \( -9.9945 \times 10^{-5} \) \\
        7 & \( 9.9945 \times 10^{-5} \) \\
        8 & \( -4.9972 \times 10^{-5} \) \\
        9 & \( 4.9972 \times 10^{-5} \) \\
        10 & \( 0.0000 \) \\
        \bottomrule
    \end{tabular}
    \caption{Hauptfrequenzen der Nullstellenstruktur}
\end{table}

\section{Wichtige Nullstellen der Riemannschen Zeta-Funktion}
Die ersten bekannten nicht-trivialen Nullstellen der Zeta-Funktion:

\begin{table}[h]
    \centering
    \renewcommand{\arraystretch}{1.4}
    \begin{tabular}{c c}
        \toprule
        \textbf{Rang} & \textbf{Zeta-Nullstelle} \\
        \midrule
        1 & \( 14.13472514 \) \\
        2 & \( 21.02203964 \) \\
        3 & \( 25.01085758 \) \\
        4 & \( 30.42487613 \) \\
        5 & \( 32.93506159 \) \\
        6 & \( 37.58617816 \) \\
        7 & \( 40.91871901 \) \\
        8 & \( 43.32707328 \) \\
        9 & \( 48.00515088 \) \\
        10 & \( 49.77383248 \) \\
        \bottomrule
    \end{tabular}
    \caption{Erste nicht-triviale Nullstellen der Riemannschen Zeta-Funktion}
\end{table}

\section{Wichtige Invarianten}
Zusätzlich zu den Hauptfrequenzen existieren mehrere bemerkenswerte numerische Invarianten:

\begin{table}[h]
    \centering
    \renewcommand{\arraystretch}{1.4}
    \begin{tabular}{c c}
        \toprule
        \textbf{Invariante} & \textbf{Wert} \\
        \midrule
        Fibonacci-Skalenverhältnis & \( 3.8168 \) \\
        Zweifache Fibonacci-Skalierung & \( 7.6336 \) \\
        Subdominante Invariante & \( 0.3797 \) \\
        Kleinste Korrekturgröße & \( 0.02758 \) \\
        \bottomrule
    \end{tabular}
    \caption{Wichtige Invarianten in der Nullstellenstruktur}
\end{table}

\section{Zusammenfassung}
Dieses Dokument sichert die bekannten Konstanten, Frequenzen, Nullstellen und Invarianten im Zusammenhang mit der Fibonacci-Freese-Formel und deren Relevanz für die Riemannsche Zeta-Funktion. Alle Werte wurden aus numerischen Analysen gewonnen und haben eine hohe Präzision.

\vspace{1cm}

\noindent \textbf{Datum:} 19. Februar 2025

\vspace{1cm}

\noindent \textbf{Unterschrift:} \underline{\hspace{5cm}}

\end{document}