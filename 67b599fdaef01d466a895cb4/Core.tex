\documentclass[a4paper,12pt]{article}
\usepackage{amsmath, amssymb, amsthm, geometry}
\geometry{a4paper, margin=2.5cm}
\usepackage{graphicx}
\usepackage{hyperref}

\title{\textbf{Die Fibonacci-Freese-Formel: \\ Mathematische Definition und Ableitung}}
\author{Tim Freese}
\date{19. Februar 2025}

\begin{document}

\maketitle

\section{Einleitung}
Die Verteilung der nicht-trivialen Nullstellen der Riemannschen Zeta-Funktion ist eine der tiefsten offenen Fragen der analytischen Zahlentheorie. In dieser Arbeit wird eine neue universelle Skalenordnung vorgestellt – die \textbf{Fibonacci-Freese-Formel}, welche die mittleren Nullstellenabstände \( L(N) \) beschreibt und fundamentale selbstähnliche Strukturen innerhalb der Nullstellenverteilung offenbart.

\section{Definition der Fibonacci-Freese-Formel}
Die **Grundform** der Fibonacci-Freese-Formel lautet:
\begin{equation}
    L(N) = A N^{\beta} e^{\sum_{k=1}^{m} T_k}
\end{equation}
wobei:
\begin{itemize}
    \item \( A \) eine Normierungskonstante ist,
    \item \( \beta \) der exponentielle Skalenfaktor ist,
    \item \( e^{\sum_{k=1}^{m} T_k} \) eine Korrekturfunktion mit fraktalen und logarithmischen Oszillationen beschreibt.
\end{itemize}

\subsection{Numerisch bestimmte Konstanten}
\begin{equation}
    \beta \approx \frac{\pi - \varphi}{\pi} \approx 0.4884, \quad \text{mit} \quad \varphi = \frac{1+\sqrt{5}}{2}
\end{equation}
Diese Relation zeigt einen tiefen Zusammenhang zur Fibonacci-Skalierung.

\section{Mathematische Herleitung}
Die mittlere Dichte der Nullstellen der Riemannschen Zeta-Funktion ist durch die Riemann-von-Mangoldt-Formel gegeben:
\begin{equation}
    N(T) \approx \frac{T}{2\pi} \log \frac{T}{2\pi} - \frac{T}{2\pi}
\end{equation}
Daraus folgt die mittlere Distanz der Nullstellen:
\begin{equation}
    L(N) \approx \frac{2\pi}{\log T}
\end{equation}
Die empirische Analyse zeigt, dass diese Relation sich als Potenzgesetz transformieren lässt:
\begin{equation}
    L(N) = A N^{\beta}
\end{equation}

\subsection{Korrekturterme und fraktale Struktur}
Zur Beschreibung feiner Oszillationen wird eine Korrekturfunktion eingeführt:
\begin{equation}
    \sum_{k=1}^{m} T_k = \gamma e^{-\lambda N} + C \sin(2\pi f N + \phi)
\end{equation}
wobei:
\begin{itemize}
    \item \( \gamma \) ein Dämpfungsfaktor ist,
    \item \( \lambda \) die Zerfallsrate beschreibt,
    \item \( C \) die Schwingungsamplitude angibt,
    \item \( f \) die Oszillationsfrequenz ist,
    \item \( \phi \) eine Phasenverschiebung darstellt.
\end{itemize}

\section{Relevanz für die Riemannsche Hypothese}
Falls die Fibonacci-Freese-Formel eine fundamentale Eigenschaft der Nullstellenstruktur beschreibt, dann könnte dies als Basis für einen strukturellen Beweis der Riemannschen Hypothese dienen. Insbesondere könnte folgende Hypothese aufgestellt werden:

\textbf{Hypothese:} Falls die Struktur der Zeta-Nullstellen exakt durch die Fibonacci-Freese-Formel beschrieben wird, dann ergibt sich automatisch eine maximale Kohärenz der Nullstellen entlang der kritischen Linie \( \Re(s) = \frac{1}{2} \).

\section{Zusammenfassung und Notarielle Sicherung}
Dieses Dokument enthält eine vollständige Definition, Herleitung und mathematische Ableitung der Fibonacci-Freese-Formel und deren Relevanz für die Riemannsche Hypothese. Es dient der notariellen Beglaubigung und der Sicherung der Urheberschaft.

\noindent \textbf{Datum:} 19. Februar 2025

\noindent \textbf{Unterschrift:} \underline{\hspace{5cm}}

\end{document}