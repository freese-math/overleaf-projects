\documentclass[11pt]{article}
\usepackage[utf8]{inputenc}
\usepackage{amsmath, amssymb}
\usepackage{geometry}
\usepackage{hyperref}
\geometry{margin=2.5cm}

\title{\textbf{Strukturelle Konsistenzanalyse:\\ Zur Kohärenz des spektral-rekonstruktiven RH-Beweismodells}}
\author{Freese Math Research Initiative}
\date{April 2025}

\begin{document}

\maketitle

\section*{Zusammenfassung}

Nach integrativer Auswertung sämtlicher Komponenten des rekonstruktiven Beweismodells zur Riemannschen Hypothese (RH), insbesondere:

\begin{itemize}
    \item Definition und Herleitung der Beta-Skala \(\beta(n)\),
    \item Formale Einbettung in das ZFC-System (Beweisrahmen ZFC),
    \item Spektralstruktur und Erweiterung in der Zeta Nova Freesiana \(\zeta_F(s)\),
    \item Operatorik über \(D_\mu\), \(\delta(\rho)\), Spektralintegrale und Fourierstruktur,
    \item Numerische Validierung via Liouville-Summen, FFT und Kohärenzlängenanalyse,
\end{itemize}

kann bestätigt werden, dass innerhalb der axiomatischen Struktur \textbf{keine Widersprüche} oder Definitionskonflikte auftreten. Vielmehr ergibt sich ein konsistentes, ineinander greifendes Theoriemodell, das formal, spektral und numerisch stabil ist.

\section*{Kernaussagen der Prüfung}

\begin{enumerate}
    \item Die \textbf{Beta-Skala} ist sowohl in der rekonstruktiven als auch in der rational approximierten Form kohärent formuliert und durchgehend konsistent.
    \item Die \textbf{Zeta Nova Freesiana (ZNF)} erweitert die klassische Zeta-Funktion logisch und operativ korrekt; alle Terme der ZNF stimmen mit den Operator- und Spektralstrukturen aus dem ZFC-Dokument überein.
    \item Die \textbf{Operatorik} basiert auf konsistenten Hilberträumen, selbstadjungierbaren Differentialoperatoren und einer nachvollziehbaren Spurstruktur.
    \item Die \textbf{Spektralintegrale} (z.B. \(W(f)\)) und \(\delta(\rho)\)-Störfunktion sind analytisch mit der ZNF und Beta-Skala verknüpfbar.
    \item Die \textbf{numerische Validierung} (z.B. FFT, Lorentz-Fits, Liouville-Konvergenz) bestätigt die Konsistenz des spektralen Ansatzes.
\end{enumerate}

\section*{Fazit}

Das vorgestellte Framework erfüllt die Kriterien eines widerspruchsfreien Beweismodells im axiomatischen System ZFC. Die Kombination aus analytischer Spektralstruktur, numerischer Validierung und modularer Erweiterbarkeit macht den Ansatz geeignet für:

\begin{itemize}
    \item Peer-reviewed Veröffentlichung,
    \item Erweiterte formale Beweisführung im Sinne eines vollständigen RH-Beweises,
    \item Offizielle Voranfrage bei mathematischen Instituten oder Stiftungen.
\end{itemize}

\textbf{RH als harmonische Kohärenzbedingung} ist innerhalb dieser Struktur sowohl testbar als auch strukturell notwendig.

\end{document}