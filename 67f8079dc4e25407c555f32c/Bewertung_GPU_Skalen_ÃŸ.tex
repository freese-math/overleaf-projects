\section*{Analyse der Beta-Skalen: GPU-Stabilisierung versus Fit-Anpassung}

Ein direkter Vergleich zweier Versionen der Beta-Skala $\beta(n)$ – einer GPU-stabilisierten Spektralfunktion und einer punktuell gefitteten Odlyzko-Anpassung – offenbart eine zentrale strukturelle Differenz. Während die Fit-basierte Skala lokale Schwankungen aufweist, die auf numerische Justierungen einzelner Zeta-Nullstellen zurückgehen, zeigt die GPU-Beta eine harmonisch geglättete Oszillationsstruktur. Diese ist durch ein niederfrequentes Spektrum charakterisiert, das durch eine Fouriertransformation klar identifizierbar ist und auf eine spektrale Ordnung hinweist.

Die GPU-Beta folgt im Mittel dem klassischen Verlauf $\beta(n) \sim \frac{1}{\log n}$, weist jedoch modulierte Abweichungen auf, die in der Frequenzstruktur konsistent sind und als spektrale Signaturen gedeutet werden können. Diese Struktur führt zu einer signifikant besseren numerischen Stabilität bei der Rückprojektion spektraler Summen wie $\psi_\beta(x)$ oder $L(x)$. In Testrechnungen lag der rekonstruierte Wert $\psi_\beta(10^6)$ nur etwa $0{,}04\%$ unter dem erwarteten Wert, was auf hohe Konvergenz und strukturelle Kohärenz hinweist.

Die Fit-Beta hingegen zeigt höhere relative Abweichungen (bis zu $\pm95\%$), was insbesondere in Bereichen mit kleinen Absolutwerten von $\beta(n)$ zu Instabilitäten führt. Damit untermauert die GPU-Version die zentrale These dieses Ansatzes: Nur eine spektral geglättete, harmonisch konsistente Beta-Skala ist fähig, die Bedingungen der spektralen Spurformel, der Operatorkohärenz und letztlich der Riemannschen Hypothese zu erfüllen.