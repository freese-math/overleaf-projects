import numpy as np
import matplotlib.pyplot as plt
from scipy.fftpack import fft

# Simulierte Daten für Zeta-Nullstellen
n_values = np.arange(1, 5000)
zeta_nullstellen = 2 * np.pi * n_values / np.log(n_values)

# Fourier-Analyse der Nullstellen-Abstände
frequenzen = np.fft.rfftfreq(len(zeta_nullstellen))
spektrum = np.abs(np.fft.rfft(zeta_nullstellen))

plt.figure(figsize=(10, 5))
plt.plot(frequenzen, spektrum, color='red', label="Zeta-Nullstellen")
plt.xlabel("Frequenz")
plt.ylabel("Spektrale Stärke")
plt.legend()
plt.title("Fourier-Analyse: Zeta-Nullstellen")
plt.savefig("fourier_spektrum.png")

# Abstandsverteilung der Primzahlen
primzahlen = np.loadtxt("primzahlen1mio.txt")[:5000]
primzahl_abstaende = np.diff(primzahlen)

plt.figure(figsize=(10, 5))
plt.plot(n_values[:4999], primzahl_abstaende, color='black', label="Echte Primzahlabstände")
plt.xlabel("Index n")
plt.ylabel("Abstand zwischen Primzahlen")
plt.legend()
plt.title("Vergleich der Primzahlabstände")
plt.savefig("primzahlabstaende.png")