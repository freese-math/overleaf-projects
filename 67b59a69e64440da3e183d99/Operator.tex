\documentclass[a4paper,12pt]{article}

\usepackage{amsmath, amssymb, amsthm}
\usepackage{geometry}
\geometry{a4paper, margin=1in}
\usepackage{hyperref}

\title{Die Freese-Skalierungsformel und eine neue Operator-Struktur für die Riemannsche Zetafunktion}
\author{Tim Freese}
\date{\today}

\begin{document}

\maketitle

\begin{abstract}
Diese Arbeit präsentiert eine neue Operatorstruktur für die Nullstellen der Riemannschen Zetafunktion.  
Die Freese-Formel beschreibt eine neue Skalenordnung der Nullstellenabstände und wird durch den Operator \( \hat{H} \) definiert:

\[
\hat{H} = \frac{1}{\varphi 2\pi} + e^{-\varphi} \frac{1}{\pi^2} + \ln(\hat{N}) \frac{1}{\varphi^3 \pi^3}
\]

Erste numerische Tests zeigen eine Übereinstimmung zwischen den Eigenwerten von \( \hat{H} \) und den Nullstellen der Zetafunktion.  
Diese Struktur könnte eine neue Perspektive auf die Riemannsche Hypothese bieten.

\end{abstract}

\section{Einleitung}
Die Riemannsche Hypothese (RH) ist eines der zentralen ungelösten Probleme der Mathematik.  
In dieser Arbeit wird eine neue Skalenordnung für die Nullstellen der Zetafunktion vorgeschlagen:

\[
L(N) = \alpha \cdot N^f
\]

wobei \( f \) eine neue fundamentale Naturkonstante ist.

\section{Operatorformulierung}
Die Nullstellenstruktur kann durch den Operator \( \hat{H} \) beschrieben werden:

\[
\hat{H} = \frac{1}{\varphi 2\pi} + e^{-\varphi} \frac{1}{\pi^2} + \ln(\hat{N}) \frac{1}{\varphi^3 \pi^3}
\]

Dieser Operator erzeugt eine neue spektrale Ordnung für die Nullstellenabstände.

\section{Numerische Evidenz}
Erste Tests zeigen eine signifikante Übereinstimmung der Eigenwerte von \( \hat{H} \) mit den Nullstellen der Zetafunktion.  
Weitere Analysen werden durchgeführt.

\section{Fazit}
Diese Arbeit stellt eine neue Operatorstruktur für die Riemannsche Zetafunktion vor.  
Weitere Forschung wird sich darauf konzentrieren, eine analytische Beweisführung für RH auf Basis dieser Struktur zu entwickeln.

\end{document}