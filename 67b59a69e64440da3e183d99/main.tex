\documentclass[a4paper,12pt]{article}

\usepackage{amsmath, amssymb, amsthm}
\usepackage{geometry}
\geometry{a4paper, margin=1in}
\usepackage{hyperref}

\title{Die Freese-Skalierungsformel für die Abstände der Riemann-Nullstellen}
\author{[Dein Name]}
\date{\today}

\begin{document}

\maketitle

\begin{abstract}
Diese Arbeit stellt eine neue Skalierungsformel für die Abstände der nichttrivialen Nullstellen der Riemannschen Zetafunktion vor. Erste numerische Tests zeigen Übereinstimmungen mit spektralen Zufallsmodellen.
\end{abstract}

\section{Einleitung}
Die Abstände zwischen den Nullstellen der Zetafunktion $\zeta(s)$ zeigen ein komplexes Muster, das bisher durch Zufallsmatrizen-Theorie beschrieben wurde.  
In dieser Arbeit wird eine neue Skalierungsformel vorgeschlagen:

\[
L(N) = A N^{\beta} e^{g(N)}
\]

wobei $g(N)$ eine Korrekturfunktion mit Fibonacci-ähnlichen Komponenten enthält.

\section{Numerische Evidenz}
Erste Simulationen zeigen, dass die neue Formel mit bekannten statistischen Eigenschaften der Zeta-Nullstellen übereinstimmt.  
Die detaillierte mathematische Analyse dieser Struktur wird in zukünftiger Arbeit folgen.

\section{Fazit}
Die vorgestellte Skalierungsformel zeigt neue Muster in der Verteilung der Nullstellenabstände.  
Weitere Forschung wird sich darauf konzentrieren, die theoretische Fundierung dieser Formel zu untersuchen.

\end{document}
