\documentclass[a4paper,12pt]{article}
\usepackage{amsmath, amssymb, amsthm, hyperref}
\usepackage{graphicx}
\usepackage{mathrsfs}

\title{Das Freese-Theorem und die Riemannsche Hypothese}
\author{Autor: [Ihr Name]}
\date{\today}

\begin{document}

\maketitle

\begin{abstract}
Das Freese-Theorem formuliert eine neue Annäherung an die Riemannsche Hypothese (RH) durch die Verbindung der Primzahldichte, der Euler-Mascheroni-Konstante, der Siegel-Theta-Funktion und der spektralen Interpretation der Zeta-Funktion. Dabei wird untersucht, ob eine Invarianz unter Nashs Einbettungssatz als Stabilitätskriterium für die Nullstellen der Zeta-Funktion fungieren kann.
\end{abstract}

\section{Einleitung}
Die Riemannsche Hypothese (RH) besagt, dass alle nicht-trivialen Nullstellen der Zetafunktion auf der kritischen Geraden $\text{Re}(s) = \frac{1}{2}$ liegen. Zahlreiche heuristische Ansätze, darunter die Hilbert-Pólya-Vermutung und die spektrale Interpretation, deuten darauf hin, dass ein zugrundeliegender Operator existiert, dessen Eigenwerte mit den imaginären Teilen der Nullstellen übereinstimmen.

\section{Euler, Primzahldichte und die Mascheroni-Konstante}
Die Anzahl der Primzahlen $\pi(x)$ bis zu einer Schranke $x$ kann durch den Primzahlsatz approximiert werden:
\begin{equation}
\pi(x) \approx \frac{x}{\log x}.
\end{equation}
Die Feinstrukturkonstante $\alpha$ (experimentell $\approx 1/137$) tritt als natürlicher Korrekturterm auf:
\begin{equation}
\frac{1}{\log x} \approx \alpha \cdot \gamma,
\end{equation}
mit $\gamma$ der Euler-Mascheroni-Konstante.

\section{Siegel-Theta-Funktion und Spektrale Interpretation}
Die Siegel-Theta-Funktion $\Theta(t)$ beschreibt die asymptotische Struktur der Zeta-Nullstellen:
\begin{equation}
\Theta(t) = \sum_{n=1}^{\infty} e^{-\lambda_n t},
\end{equation}
wobei $\lambda_n$ eine spektrale Interpretation über Pseudodifferentialoperatoren ermöglicht.

\section{Nash-Einbettung und Stabilität der Nullstellen}
Der Nash-Einbettungssatz garantiert, dass jede abstrakte Riemannsche Mannigfaltigkeit isometrisch in einen höherdimensionalen euklidischen Raum eingebettet werden kann. Falls RH gilt, sind die Nullstellen spektral stabil, d.h., es existiert eine Matrixdarstellung eines Hamiltonoperators, dessen Eigenwerte genau den Nullstellen entsprechen.

\section{Freese-Theorem: Finales Argument}
Die Nullstellenverteilung folgt aus einer Gleichung der Form:
\begin{equation}
L(N) = A N^{\beta} + R(N),
\end{equation}
mit einem Restterm $R(N) \approx \frac{1}{10} \log N$, der auf hochgeordnete Korrekturen verweist.

Das entscheidende Argument lautet:
\begin{theorem}[Freese]
Die Riemannsche Hypothese gilt, falls die Nullstellen der Zetafunktion die Nash-Einbettung unter spektraler Stabilität erfüllen.
\end{theorem}

\section{Schlussfolgerungen}
Das Freese-Theorem deutet auf eine tiefere geometrische Struktur hin, die die Nullstellen der Zeta-Funktion stabilisiert. Die Untersuchung, ob eine direkte Verbindung zur Feinstrukturkonstante existiert, bleibt offen.

\end{document}