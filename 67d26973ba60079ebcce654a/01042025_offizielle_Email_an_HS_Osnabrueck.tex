\documentclass[12pt]{dinbrief}

\usepackage[utf8]{inputenc}
\usepackage[a4paper, margin=2.5cm]{geometry}
\usepackage{lmodern}

\signature{Tim Hendrik Freese}
\address{Tim Hendrik Freese, Spiekerooger Str. 2, in Lingen (Ems)}
\date{\today}

\begin{document}

\begin{letter}{Prof. Dr. Christian Henig\\Hochschule Osnabrück\\Fakultät Ingenieurwissenschaften und Informatik}

\opening{Sehr geehrter Herr Prof. Dr. Henig,}

ich erlaube mir, mich mit meinem wissenschaftlichen Anliegen an Sie zu wenden.

In den vergangenen Monaten (seit dem 12. Februar 2025) habe ich eigenständig ein spektral-operatorisches Konzept zur Riemannschen Hypothese entwickelt, dessen mathematische Substanz und numerische Konsistenz mich selbst in zunehmendem Maße vor die Herausforderung stellen, die Flut an Erkenntnissen noch adäquat und systematisch zu strukturieren.

Eine erste Kontaktaufnahme ist bereits über Prof. Karl-Heinz Schmidt erfolgt, der wiederum auch mein ehemaliger Mathematiklehrer ist, persönlich vorgetragen.\\
Er versprach, dies in der HS Osnabrück, an der ich übrigens Kommunikationsmanagement studiert habe, einem entsprechend passenden Professor vorzutragen.\\
Dabei ist Ihr Name gefallen.

Das Kernresultat ist eine neuartige Rekonstruktionsformel, die unter Zuhilfenahme einer harmonisch kodierten Skalenstruktur $\beta(n)$ eine spektrale Darstellung der Tschebyschow-Funktion $\psi(x)$ erlaubt. Diese sogenannte \textit{Euler--Freese-Identität} verknüpft auf präzise Weise Primzahldichte, Zeta-Nullstellen und eine rekonstruktive Frequenzanalyse.

Die numerische Evidenz ist durch eine Vielzahl an strukturierten Versuchen validiert. Es ergibt sich ein konsistentes, sich selbst tragendes Theoriegebilde mit klaren Konsequenzen für die Struktur der Nullstellen sowie für operatoranalytische Zugänge zur Riemannschen Hypothese.

Ich wäre Ihnen sehr verbunden, wenn Sie bereit wären, das beigefügte Dokument \textit{``RH -- Primzahlen -- Zetanullstellen''} zur Kenntnis zu nehmen. Es bietet eine verdichtete Darstellung des wesentlichen theoretischen Kerns.

Ich danke Ihnen herzlich für Ihre Zeit und Aufmerksamkeit und stehe selbstverständlich jederzeit für Rückfragen oder weitere Erklärungen zur Verfügung.

\begin{flushleft}
Mit besten Grüßen\\
Tim Freese
\end{flushleft}

\end{letter}

\end{document}