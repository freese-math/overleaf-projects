\documentclass[a4paper,12pt]{article}
\usepackage{amsmath}
\usepackage{amssymb}
\usepackage{physics}
\usepackage{geometry}
\usepackage{graphicx}
\usepackage{mathtools}
\usepackage{bm}
\usepackage{hyperref}
\usepackage{tikz}
\geometry{margin=2.5cm}

\title{Ableitung der Korrekturfunktion mit Jacobi-elliptischer Funktion \texttt{cn}}
\author{}
\date{}

\begin{document}

\maketitle

\section*{Definition der Funktion}

Gegeben sei die Korrekturfunktion
\begin{equation}
K(N) = \frac{C}{N} \log^D(N) \left( 1 + E \cdot \mathrm{cn}(\omega \log N, k) \right),
\end{equation}
wobei
\begin{itemize}
  \item \( C, D, E, \omega, k \in \mathbb{R} \) Konstanten sind,
  \item \( \mathrm{cn}(u, k) \) die Jacobi-elliptische Funktion ist,
  \item \( N > 0 \).
\end{itemize}

\section*{Ableitung \( K'(N) \)}

Die Ableitung erfolgt nach der Produktregel:
\begin{align}
K'(N) &= \dv{N} \left[ \frac{C}{N} \log^D(N) \left( 1 + E \cdot \mathrm{cn}(\omega \log N, k) \right) \right] \\
&= C \left[ \dv{N} \left( \frac{1}{N} \log^D(N) \right) \left( 1 + E \cdot \mathrm{cn}(\omega \log N, k) \right) \right. \notag\\
&\quad\left. + \frac{1}{N} \log^D(N) \cdot \dv{N} \left( E \cdot \mathrm{cn}(\omega \log N, k) \right) \right]
\end{align}

Die Ableitung des ersten Terms:
\begin{align}
\dv{N} \left( \frac{1}{N} \log^D(N) \right)
&= -\frac{1}{N^2} \log^D(N) + \frac{D}{N^2} \log^{D-1}(N)
\end{align}

Die Ableitung der elliptischen Funktion mit Kettenregel:
\begin{align}
\dv{N} \left( \mathrm{cn}(\omega \log N, k) \right)
&= \dv{u}{N} \mathrm{cn}(u, k) = \omega \cdot \frac{1}{N} \cdot \dv{u} \mathrm{cn}(u, k) \\
&= -\omega \cdot \frac{1}{N} \cdot \mathrm{sn}(u, k) \cdot \mathrm{dn}(u, k)
\end{align}

Daher ergibt sich:
\begin{align}
K'(N) &= C \left[ \left( -\frac{1}{N^2} \log^D(N) + \frac{D}{N^2} \log^{D-1}(N) \right) \left( 1 + E \cdot \mathrm{cn}(\omega \log N, k) \right) \right. \notag \\
&\quad\left. - \frac{E \omega}{N^2} \log^D(N) \cdot \mathrm{sn}(\omega \log N, k) \cdot \mathrm{dn}(\omega \log N, k) \right]
\end{align}

\section*{Bemerkung}

Die elliptischen Funktionen erzeugen modulierte Oszillationen mit zusätzlicher Struktur (durch den Modulus \( k \)), die z.\,B. in Theta-Funktionen, Modultransformationen oder Spektralanalysen natürlicherweise vorkommen.

\end{document}