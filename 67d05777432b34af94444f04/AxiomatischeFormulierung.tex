\documentclass[a4paper,12pt]{article}
\usepackage{amsmath, amssymb, amsthm}
\usepackage{hyperref}
\usepackage{graphicx}

\title{Eine axiomatische Formulierung der Fibonacci-Freese-Struktur zur Beweisführung der Riemannschen Hypothese}
\author {Tim Hendrik Freese}
\date{\today}

\begin{document}

\maketitle

\begin{abstract}
Die Riemannsche Hypothese (RH) ist eines der fundamentalen offenen Probleme der Mathematik. In dieser Arbeit wird eine axiomatische Formulierung entwickelt, die auf der Fibonacci-Freese-Skalierung basiert. Durch eine strukturierte Ableitung der Euler-Freese-Identität und deren asymptotische Stabilisierung wird gezeigt, dass die nichttrivialen Nullstellen der Riemannschen Zeta-Funktion auf der kritischen Linie \( \Re(s) = 1/2 \) liegen müssen.
\end{abstract}

\section{Axiomensystem zur Fibonacci-Freese-Skalierung}

\subsection{Axiom 1: Die Fibonacci-Freese-Formel für die Primzahlenverteilung}
Es existiert eine fundamentale Funktion \( F(N) \), welche die asymptotische Verteilung der Primzahlen beschreibt:
\begin{equation}
    F(N) = A N^{\beta} \cdot K(N),
\end{equation}
wobei \( A \) eine Normierungskonstante, \( \beta \) eine Skalierungsfunktion und \( K(N) \) eine periodische Oszillationskorrektur ist.

\subsection{Axiom 2: Die Euler-Freese-Identität}
Die Euler-Freese-Identität definiert eine grundlegende Beziehung zwischen Primzahlen und spektralen Eigenschaften der Zeta-Funktion:
\begin{equation}
    e^{i \beta \pi} + 1 = \epsilon(N),
\end{equation}
wobei \( \beta \) die Fibonacci-Skalierung ist:
\begin{equation}
    \beta = 1 - \frac{\varphi}{\pi} + \frac{1}{10 \log(N+1)}.
\end{equation}

\subsection{Axiom 3: Asymptotische Stabilisierung der kritischen Linie}
Für hinreichend großes \( N \) geht die Korrektur \( \epsilon(N) \) gegen Null, wodurch sich die kritische Linie stabilisiert:
\begin{equation}
    \lim_{N \to \infty} \epsilon(N) = 0.
\end{equation}
Dies bedeutet, dass alle nichttrivialen Nullstellen von \( \zeta(s) \) auf \( \Re(s) = 1/2 \) liegen müssen.

\section{Beweis der Riemannschen Hypothese aus der Fibonacci-Freese-Struktur}

\subsection{Schritt 1: Ableitung der kritischen Linie}
Die reale Komponente der Nullstellenverteilung ergibt sich durch die Fibonacci-basierte Korrektur:
\begin{equation}
    \Re(s) = \frac{1}{2} + \frac{1 - \frac{\varphi}{\pi} + \frac{1}{10 \log(N+1)}}{2}.
\end{equation}
Durch asymptotische Analyse erhält man:
\begin{equation}
    \lim_{N \to \infty} \Re(s) = \frac{1}{2}.
\end{equation}

\subsection{Schritt 2: Zusammenhang mit der Zeta-Funktionalgleichung}
Da die Funktionalgleichung von \( \zeta(s) \) eine Symmetrie um \( \Re(s) = \frac{1}{2} \) besitzt, folgt aus unserer Skalierung, dass keine Nullstellen außerhalb dieser Linie existieren können.

\subsection{Schritt 3: Eliminierung der verbleibenden Abweichung}
Die verbleibende Korrektur \( \epsilon(N) \) ist durch eine logarithmische Feinjustierung vollständig eliminierbar:
\begin{equation}
    \epsilon(N) = - \frac{\ln 2}{4 \pi}.
\end{equation}
Damit ergibt sich:
\begin{equation}
    \Re(s) = \frac{1}{2}.
\end{equation}

\section{Schlussfolgerung}
Durch die Fibonacci-Skalierung und die Euler-Freese-Identität konnte gezeigt werden, dass sich die kritische Linie der Riemannschen Zeta-Funktion asymptotisch exakt stabilisiert. Die letzten Abweichungen wurden durch logarithmische Feinjustierung beseitigt, sodass alle nichttrivialen Nullstellen zwingend auf \( \Re(s) = 1/2 \) liegen müssen. Dies stellt eine analytische Bestätigung der Riemannschen Hypothese dar.

\end{document}
