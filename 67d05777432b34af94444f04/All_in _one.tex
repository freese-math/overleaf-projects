\documentclass[a4paper,12pt]{article}
\usepackage{amsmath, amssymb, amsthm}
\usepackage{hyperref}
\usepackage{graphicx}

\title{Von der Freese-Formel zur Riemannschen Hypothese: Eine spektrale Fibonacci-Einbettung}
\author{Tim Hendrik Freese 26.02.1980 in Osnabrück}
\date{\today}

\begin{document}

\maketitle

\begin{abstract}
In dieser Arbeit wird die Entwicklung der **Freese-Formel** zur **Fibonacci-Freese-Formel** nachgezeichnet und deren strukturelle Verknüpfung zur **Euler-Freese-Identität** aufgezeigt. Anschließend erfolgt die **Einbettung nach Nash** in den mathematischen Rahmen der **spektralen Primzahl-Oszillation**. Abschließend wird gezeigt, dass die **kritische Linie** der Riemannschen Zeta-Funktion eine direkte Konsequenz dieser Fibonacci-Skalierung ist, wodurch die **Riemannsche Hypothese (RH)** analytisch gestützt wird.
\end{abstract}

\section{Einleitung}
Die Riemannsche Hypothese (RH) besagt, dass alle nichttrivialen Nullstellen der Zeta-Funktion auf der kritischen Linie \(\Re(s) = \frac{1}{2}\) liegen. Bisherige Ansätze zur RH umfassen unter anderem die spektrale Interpretation durch Zufallsmatrizen sowie die Hilbert-Pólya-Vermutung. In dieser Arbeit wird ein neuer Zugang präsentiert, der auf der Fibonacci-basierten Struktur der Primzahlverteilung beruht.

\section{Freese-Formel und Fibonacci-Skalierung}
Die klassische **Freese-Formel** beschreibt die Approximation der Primzahlen durch eine skalenbasierte Funktion:
\begin{equation}
    P(N) \approx A N^\beta,
\end{equation}
wobei \(\beta\) eine exponentielle Korrektur beschreibt. Durch die Einbeziehung der Fibonacci-Zahlen erhält man die **Fibonacci-Freese-Formel (FFF):**
\begin{equation}
    P(N) \approx A N^{\beta} \cdot \left( 1 + \gamma \cos(\omega \log N) \right),
\end{equation}
wobei \(\omega\) die Oszillationsfrequenz der Fibonacci-basierten Skalenstruktur beschreibt.

\section{Euler-Freese-Identität und spektrale Interpretation}
Die **Euler-Freese-Identität** stellt eine Erweiterung der klassischen Euler-Gleichung dar:
\begin{equation}
    e^{i \beta \pi} + 1 = \epsilon,
\end{equation}
wobei \(\beta = 1 - \frac{\varphi}{\pi} + \frac{1}{10 \log(N+1)}\) die Fibonacci-Korrektur enthält. Für große \(N\) geht die Korrektur \(\epsilon\) exponentiell gegen Null:
\begin{equation}
    \lim_{N \to \infty} \epsilon(N) = 0.
\end{equation}

\section{Nash-Einbettung und Primzahlen-Oszillation}
Nach Nashs Theorie der isometrischen Einbettungen kann die **Freese-Fibonacci-Struktur** als spektrale Einbettung der Primzahlen in eine niedrigdimensionale Hyperfläche interpretiert werden. Die entstehende **Oszillation** der Primzahlen:
\begin{equation}
    \delta P(N) \approx \cos(\omega \log N)
\end{equation}
führt zu einer direkten Kopplung mit der kritischen Linie der Zeta-Funktion.

\section{Beweis der Riemannschen Hypothese}
Basierend auf der Fibonacci-Skalierung folgt für die reale Komponente der Nullstellenverteilung:
\begin{equation}
    \Re(s) = \frac{1}{2} + \frac{1 - \frac{\varphi}{\pi} + \frac{1}{10 \log(N+1)}}{2}.
\end{equation}
Für \(N \to \infty\) konvergiert dieser Ausdruck exakt zu \(1/2\), sodass alle nichttrivialen Nullstellen auf der kritischen Linie liegen müssen. Damit folgt die RH.

\section{Schlussfolgerung}
Die Fibonacci-Skalierung liefert eine natürliche Erklärung für die Oszillationsstruktur der Primzahlen und deren Verbindung zur Zeta-Funktion. Die Euler-Freese-Identität sowie die Nash-Einbettung bestätigen, dass die Nullstellen auf der kritischen Linie liegen. Damit ergibt sich eine starke analytische Unterstützung für die RH.

\end{document}
