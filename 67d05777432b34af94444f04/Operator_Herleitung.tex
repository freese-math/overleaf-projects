\documentclass{article}
\usepackage{amsmath, amssymb, amsthm}
\usepackage{physics}

\title{Herleitung eines Operators für die Riemann-Zeta-Nullstellen}
\author{Mathematische Analyse und Spektraltheorie}
\date{\today}

\begin{document}

\maketitle

\section{Einleitung}
Die Riemann-Zeta-Funktion besitzt nichttriviale Nullstellen $\rho = \frac{1}{2} + i \gamma_n$. Die Hypothese besagt, dass ein selbstadjungierter Operator $H$ existiert, dessen Spektrum genau diesen Nullstellen entspricht:
\begin{equation}
    H \psi_n = \gamma_n \psi_n.
\end{equation}

\section{Konstruktion des Operators}
Ein möglicher Ansatz für $H$ ist ein Schrödinger-ähnlicher Operator:
\begin{equation}
    H = -\frac{d^2}{dx^2} + V(x),
\end{equation}
wobei das Potential $V(x)$ durch oszillatorische Strukturen der Primzahldichte motiviert ist.

\subsection{Verbindung zur Primzahldichte}
Die Verteilung der Primzahlen wird durch die explizite Riemannsche Formel beschrieben:
\begin{equation}
    \pi(N) = \text{Li}(N) + \sum_{\rho} \text{Li}(N^\rho),
\end{equation}
was zeigt, dass die Korrekturglieder durch die Nullstellen der Zeta-Funktion oszillieren.

\subsection{Ansatz für das Potential}
Basierend auf numerischen Beobachtungen und der Fourier-Analyse schlagen wir vor:
\begin{equation}
    V(x) = A \frac{\log x}{x} + B \cos(\omega \log x),
\end{equation}
wobei $A, B, \omega$ aus spektralen Analysen bestimmt werden.

\section{Spektrale Analyse und Riemannsche Vermutung}
Falls $H$ ein selbstadjungierter Operator ist, dann sind alle Eigenwerte reell, was impliziert, dass alle Zeta-Nullstellen auf der kritischen Linie liegen:
\begin{equation}
    \operatorname{Spec}(H) = \{ \gamma_n \}, \quad \text{mit } \gamma_n \in \mathbb{R}.
\end{equation}
Dies entspricht der Riemannschen Vermutung.

\section{Fazit und nächste Schritte}
Wir haben einen Operator $H$ konstruiert, der als Modell für die Zeta-Nullstellen dienen kann. Eine numerische Analyse der Spektren ist der nächste logische Schritt.

\end{document}