\documentclass[a4paper,12pt]{article}
\usepackage[utf8]{inputenc}
\usepackage{amsmath, amssymb}
\usepackage{graphicx}
\usepackage{geometry}
\usepackage{caption}
\usepackage{float}
\geometry{margin=2.5cm}

\title{Analyse der Euler-Freese-Transformation}
\author{Prime Zeta Pro}
\date{\today}

\begin{document}

\maketitle

\section*{Zusammenfassung}

Die vorliegende Analyse untersucht die komplexe Transformation
\[
H(n) := \exp(i \pi \beta(n)),
\]
basierend auf einer skalenbasierten Modulation \(\beta(n)\), wie sie aus der Euler-Freese-Struktur hervorgeht. Ziel ist die Untersuchung des spektralen und phasischen Verhaltens dieser Transformation im Kontext möglicher Zeta-artiger Identitäten.

\section{Verteilung der Phasen auf dem Einheitskreis}

Zur Visualisierung der Phasenverteilung wurde für \( n = 1, \dots, N \) der Winkel
\[
\theta_n := \arg(H(n)) = \pi \beta(n) \mod 2\pi
\]
berechnet und histogrammisch ausgewertet.

\begin{figure}[H]
    \centering
    \includegraphics[width=0.8\textwidth]{phase_histogramm.png} % <- Bild ggf. umbenennen
    \caption{Verteilung der Phasen \( \arg(H(n)) \) auf dem Einheitskreis. Die Gleichverteilung deutet auf spektrale Neutralität hin.}
\end{figure}

\paragraph{Ergebnis:} Die Phasen \(\theta_n\) sind im Intervall \([-\pi, \pi]\) gleichverteilt. Dies weist auf eine vollständige phasische Rotation auf dem Einheitskreis hin. Das System zeigt keine bevorzugte Richtung, was auf eine pseudorandomisierte, aber strukturell kohärente Eigenschaft der Skala \(\beta(n)\) schließen lässt.

\section{Vergleich zur klassischen Euler-Identität}

Während bei der klassischen Gleichung
\[
e^{i\pi} + 1 = 0
\]
eine konstante Phase vorliegt, beschreibt die Euler-Freese-Transformation eine variable Rotation:
\[
H(n) = e^{i\pi\beta(n)}.
\]
Diese erzeugt eine Sequenz komplexer Zahlen auf dem Einheitskreis mit stochastisch anmutender Dynamik.

\begin{figure}[H]
    \centering
    \includegraphics[width=0.45\textwidth]{euler_freese_circle.png}
    \caption{Darstellung einzelner \( H(n) \) auf dem Einheitskreis.}
\end{figure}

\section{Ausblick}

Basierend auf der gleichverteilten Phasenstruktur kann vermutet werden, dass die Euler-Freese-Transformation \emph{unitäre Eigenschaften} besitzt. Dies legt nahe, dass
\[
\sum_{n=1}^N H(n) \rightarrow 0 \quad \text{für } N \to \infty
\]
unter Annahme von struktureller Kohärenz in \(\beta(n)\). Weitere Analysen (Spektren, Autokorrelation, Matrixmodelle) sind geplant.

\end{document}