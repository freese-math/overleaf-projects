\documentclass[a4paper,12pt]{article}
\usepackage{amsmath, amssymb, graphicx}
\usepackage{mathtools}
\usepackage{booktabs}
\usepackage{caption}
\usepackage{tikz}
\usepackage{geometry}
\geometry{margin=2.5cm}

\title{Test der Euler-Freese-Identität}
\author{Beta-Skalenanalyse und empirische Evaluation}
\date{}

\begin{document}

\maketitle

\section*{1. Einleitung}
Die \textbf{Euler-Freese-Identität} beschreibt eine additive Zerlegung einer Beta-Funktion $\beta(n)$ und eines Korrekturterms $\varepsilon(n)$, sodass ihre Summe eine konstante Skalensignatur erzeugt. Diese Idee basiert auf der Konstruktion harmonischer Skalen in der Zahlentheorie, insbesondere mit Bezug zu Primzahlen, Zeta-Nullstellen und asymptotischen Näherungen zur Eins.

\section*{2. Definition der Komponenten}
Sei $n \in \mathbb{N}$ und
\begin{align*}
    \beta(n) &= n^{-\beta}, \quad \text{mit } \beta \approx 0{,}273 \\
    \varepsilon(n) &= 1 - \beta(n)
\end{align*}
Dann ergibt sich die Summe:
\begin{align*}
    S(N) &= \sum_{n=1}^{N} \left( \beta(n) + \varepsilon(n) \right) = \sum_{n=1}^{N} 1 = N
\end{align*}

\noindent Die Gesamtstruktur erfüllt also
\[
\sum_{n=1}^{N} \left( \beta(n) + \varepsilon(n) \right) \sim N
\quad \Rightarrow \quad
\frac{1}{N} \sum_{n=1}^{N} \left( \beta(n) + \varepsilon(n) \right) \to 1
\]

\section*{3. Empirischer Test}
Die folgende Grafik zeigt die normierte partielle Summe im Vergleich zur idealen Konvergenz gegen $1$:

\begin{figure}[h!]
    \centering
    \includegraphics[width=0.9\textwidth]{212DC004-97A2-449E-B335-B482A31D3F28.png}
    \caption{Normierter Test der Euler-Freese-Identität: $\frac{1}{S(N)} \sum_{n=1}^N (\beta(n) + \varepsilon(n)) \to 1$}
\end{figure}

\section*{4. Interpretation}
\begin{itemize}
    \item Die empirische Kurve folgt exakt einer linearen Normierung.
    \item Das bestätigt, dass $\beta(n) + \varepsilon(n)$ pro $n$ konstant $= 1$ ergibt.
    \item Die Summenstruktur erzeugt damit eine perfekt skalierte Reihe, obwohl $\beta(n)$ eine Potenzfunktion und $\varepsilon(n)$ eine Nichtlinearität enthält.
\end{itemize}

\section*{5. Ausblick}
Die Euler-Freese-Struktur könnte ein Kandidat für:
\begin{itemize}
    \item harmonische Strukturen über Primzahldichten
    \item asymptotische Summendarstellungen über Zeta-Zerlegungen
    \item Operatorformulierung im Sinne eines Eigenwertproblems mit $\hat{H} \psi = \beta(n) \psi$
\end{itemize}

\end{document}