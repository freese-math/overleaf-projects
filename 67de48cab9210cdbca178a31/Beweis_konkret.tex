\documentclass[12pt]{article}
\usepackage[utf8]{inputenc}
\usepackage{amsmath,amssymb}
\usepackage{graphicx}
\usepackage{float}
\usepackage{caption}
\usepackage{xcolor}
\usepackage{geometry}
\usepackage{hyperref}
\geometry{margin=2.5cm}

\title{\textbf{Zur Struktur der Beta-Skala und dem Hamilton-Operator der Riemannschen Vermutung}}
\author{Ein Forschungsansatz zur Euler-Freese-Identität}
\date{\today}

\begin{document}

\maketitle

\begin{abstract}
Dieses Dokument untersucht die Verbindung zwischen der Beta-Skala $ \beta $, der Euler-Freese-Identität $ -1 = e^{i \pi \beta} $, und einer möglichen Operatorstruktur $ H $ zur numerischen Bestätigung der Riemannschen Vermutung. Fourier- und Wavelet-Analysen des Fehlerterms sowie geometrische Deutungen auf dem Einheitskreis und der Riemann-Spirale werden berücksichtigt.
\end{abstract}

\section{Der H-Operator}
Wir definieren den diskreten Hamilton-Operator $H$ als Matrix mit den ersten $N$ Zeta-Nullstellen $\gamma_n$ auf der Hauptlinie:

\[
H := \begin{bmatrix}
\gamma_1 & 1 & 0 & \cdots & 0 \\
1 & \gamma_2 & 1 & \cdots & 0 \\
0 & 1 & \gamma_3 & \cdots & 0 \\
\vdots & \vdots & \vdots & \ddots & 1 \\
0 & 0 & 0 & 1 & \gamma_N
\end{bmatrix}
\]

Die Matrix ist selbstadjungiert ($H^\dagger = H$), da sie symmetrisch mit reellen Einträgen ist. Ihre Eigenwerte $\lambda_k$ sind real und können mit \texttt{SciPy} numerisch bestimmt werden.

\section{Fehleranalyse zur Euler-Freese-Identität}
Der Fehlerbegriff ergibt sich durch den Unterschied zwischen den Eigenwerten $\lambda_k$ und den ursprünglichen Zeta-Nullstellen $\gamma_k$:

\[
\text{Fehler}_k := \lambda_k - \gamma_k
\]

\begin{figure}[H]
\centering
\includegraphics[width=0.75\textwidth]{example-image} % Placeholder image
\caption{Fourier-Spektrum des Fehlerterms – dominante Frequenzen zeigen Periodizitäten.}
\end{figure}

\begin{figure}[H]
\centering
\includegraphics[width=0.75\textwidth]{wavelet_analysis.png}
\caption{Wavelet-Spektrum des Fehlerterms – skalenabhängige Struktur.}
\end{figure}

Die charakteristischen Frequenzen und die starke Korrelation zur $e$-Struktur deuten auf eine tieferliegende Symmetrie hin (z.\,B. ln(3)-Modulation).

\section{Euler-Freese-Konstruktion auf dem Einheitskreis}
Die Euler-Freese-Identität wird geometrisch über die Formel

\[
H = -1 = e^{i \pi \beta}
\]

beschrieben. Für $\beta = 1$ erhalten wir $H = +1$, für $\beta = \frac{1}{2}$ ergibt sich $H = -1$, und für weitere rationale Näherungen (z.\,B. $\beta = \frac{7}{33300}$) entstehen Punkte nahe $-1$.

\begin{figure}[H]
\centering
\includegraphics[width=0.6\textwidth]{euler_freese_circle.png}
\caption{Einheitskreis mit komplexen Punkten $H = e^{i \pi \beta}$ für verschiedene $\beta$-Werte.}
\end{figure}

\section{Fazit und Ausblick}
Die Konstruktion eines Operators $H$, dessen Spektrum mit den Zeta-Nullstellen korreliert, eröffnet eine neue Perspektive auf die Riemannsche Vermutung. Die geringe Fehlernorm und die hochgradige Korrelation von $\rho_n \approx \lambda_n$ legen eine tiefere mathematische Struktur nahe – möglicherweise auf Basis der Euler-Freese-Geometrie. 

\vspace{0.5cm}
\noindent Weitere Fragen:
\begin{itemize}
    \item Ist $H$ ein physikalischer Operator in einem geeigneten Hilbertraum?
    \item Kann $\beta$ als Quantenzahl gedeutet werden?
    \item Ist die Riemannsche Vermutung durch ein Stabilitätsprinzip in der Spektralanalyse zugänglich?
\end{itemize}

\end{document}