\documentclass{article}
\usepackage{amsmath, amssymb, tikz}
\usepackage{graphicx}

\title{Riemann-Fläche der Beta-Skala und die Euler-Freese-Identität}
\author{}
\date{}

\begin{document}

\maketitle

\section*{Einleitung}

Die Beta-Skala, entwickelt zur Approximation der Nullstellen der Riemannschen Zeta-Funktion, weist eine charakteristische Struktur im Einheitskreis auf, wenn man sie über die Transformation
\[
H(\beta) = e^{i\pi\beta}
\]
abbildet. Diese Abbildung ist periodisch, was die Definition einer verzweigten Riemann-Fläche erfordert.

\section*{Definition der Riemann-Fläche}

Sei \( \mathcal{R}_\beta \) die Riemann-Fläche, die allen \( \beta \in \mathbb{C} \) eine eindeutige Phase zuordnet:
\[
\mathcal{R}_\beta := \left\{ (\beta, z) \in \mathbb{C} \times \mathbb{C} \;\middle|\; z = e^{i\pi\beta + 2\pi i n}, n \in \mathbb{Z} \right\}
\]

Diese Struktur ermöglicht eine eindeutige Fortsetzung über alle \(\beta\), auch bei mehrfachen Durchläufen des Einheitskreises.

\section*{Visualisierung}

Ein Punkt \( H(\beta_n) \) wird im Einheitskreis visualisiert, wobei die Tiefe der Spirale (Radius) der Größe des Fehlers in der Zeta-Rekonstruktion entspricht.

\begin{center}
    \includegraphics[width=0.8\textwidth]{spirale.png}
\end{center}

\section*{Schlussfolgerung}

Die Betrachtung der Beta-Skala auf einer geeigneten Riemann-Fläche offenbart ein tieferes geometrisches Verständnis für die Entstehung von Fehlern und Resonanzen in der numerischen Rekonstruktion. Eine vollständige Beschreibung könnte in ein kohärentes Modell der RH münden.

\end{document}