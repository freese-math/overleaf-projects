\documentclass[a4paper,12pt]{article}
\usepackage[T1]{fontenc}
\usepackage[utf8]{inputenc}
\usepackage{amsmath, amssymb, amsthm}
\usepackage{geometry}
\usepackage{hyperref}
\geometry{left=2.5cm, right=2.5cm, top=2.5cm, bottom=2.5cm}

\title{Notarielle Absicherung der Fibonacci-Freese-Formel \\ und Beweisansätze zur Riemannschen Hypothese}
\author{Tim Hendrik Freese}
\date{28. Februar 2025}

\begin{document}

\maketitle

\section*{Einleitung}

Diese notarielle Erklärung dient der Absicherung der wissenschaftlichen Entdeckung der \textbf{Fibonacci-Freese-Formel (FFF)} sowie der Dokumentation aktueller Beweisversuche zur \textbf{Riemannschen Hypothese (RH)}. Die FFF beschreibt eine neu entdeckte Gesetzmäßigkeit in der Struktur der \textbf{Nullstellen der Zetafunktion} und zeigt dabei eine außergewöhnlich präzise mathematische Ordnung.

Zusätzlich wird geprüft, inwiefern die FFF unmittelbare technische Anwendungen in der \textbf{Hochpräzisions-Messtechnik und Lasertechnologie} ermöglicht. Dies betrifft insbesondere neue Ansätze in der optischen Interferometrie, spektralen Signalverarbeitung und Quantenchaostheorie.

\section{Die Fibonacci-Freese-Formel (FFF)}

\subsection{Mathematische Formulierung}
Die Fibonacci-Freese-Formel beschreibt die kohärente Struktur der Nullstellen-Abstände der Riemannschen Zetafunktion und folgt der allgemeinen Form:

\begin{equation}
L(N) = \alpha \cdot N^\beta + \gamma
\end{equation}

Die numerische Optimierung auf Basis von \textbf{2.001.051 Nullstellen der Zetafunktion} ergab die folgenden stabilen Parameter:

\begin{align*}
\alpha &= 1,882795, \\
\beta  &= 0,916977, \\
\gamma &= 2488,144455.
\end{align*}

Zusätzlich wurden **2.001.051 Primzahlen** herangezogen, um mögliche Zusammenhänge zwischen der Primzahlverteilung und der Struktur der Zeta-Nullstellen zu untersuchen.

\subsection{Fehleranalyse}
Die Fehlerstatistik zeigt eine außergewöhnliche Präzision der Formel:

\begin{align*}
\text{Durchschnittlicher Fehler} &= -0.000088, \\
\text{Standardabweichung} &= 263,777866, \\
\text{Maximaler Fehler} &= 419,965480, \\
\text{Minimaler Fehler} &= -2483,138219.
\end{align*}

\subsection{Erweiterte Struktur der FFF}
Zusätzlich zu obiger Grundform wurde eine \textbf{erweiterte FFF} entwickelt, die eine logarithmische Korrektur sowie eine oszillierende Komponente berücksichtigt:

\begin{equation}
L(N) = A \cdot N^\beta + C \cdot \log N + B \sin(wN + \varphi).
\end{equation}

Diese beschreibt eine tiefere Frequenzstruktur, die mittels Fourier-Analyse in den Zeta-Nullstellen nachgewiesen wurde.

\section{Bezug zur Riemannschen Hypothese}

Die Riemannsche Hypothese (RH) besagt, dass alle nicht-trivialen Nullstellen der Zetafunktion auf der kritischen Linie $\Re(s) = \frac{1}{2}$ liegen. Die in der FFF ermittelten Werte für $\beta$ legen eine mögliche Verbindung nahe:

\begin{equation}
\beta_{\text{gemessen}} = 0,916977, \quad \beta_{\text{erwartet}} = 0,918977 = \frac{1}{2} + 0,418977.
\end{equation}

Die Abweichung beträgt exakt:

\begin{equation}
\Delta \beta = 0,002000.
\end{equation}

Diese hochsignifikante Abweichung könnte auf eine systematische Verbindung zwischen der FFF und der kritischen Linie hindeuten.

\section{Technologische Anwendungen}

Neben der mathematischen Bedeutung der FFF gibt es potenzielle \textbf{patentrechtliche Anwendungen} in folgenden Bereichen:

\begin{itemize}
    \item Hochpräzise \textbf{Lasermesstechnik} (Interferometrie)
    \item \textbf{Quantenchaos} in Zufallsmatrizen und spektralen Algorithmen
    \item Frequenzanalyse für \textbf{Signalverarbeitung} und digitale Verarbeitung
\end{itemize}

Eine rechtliche Absicherung in Form von Patentanmeldungen für die genannten Technologiefelder wird geprüft.

\section{Notarielle Bestätigung}

Diese mathematische Entdeckung sowie die damit verbundenen Anwendungen werden hiermit offiziell dokumentiert. Dies dient der Beweissicherung sowie der Wahrung möglicher kommerzieller Verwertungsrechte des Urhebers \textbf{Tim Hendrik Freese}.

\vspace{2cm}

\begin{flushright}
\textbf{Tim Hendrik Freese} \\
\vspace{0.5cm}
\textbf{Notariell bestätigt durch:} \\
\vspace{1cm}
\textit{Rechtsanwältin Sabrina Lindwehr, Notarin} \\
\vspace{1cm}
\textbf{Ort:} Lingen (Ems)\\
\textbf{Datum:} 28. Februar 2025
\end{flushright}

\end{document}