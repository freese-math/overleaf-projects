\documentclass[a4paper,12pt]{article}
\usepackage[utf8]{inputenc}
\usepackage{geometry}
\usepackage{hyperref}

\geometry{a4paper, left=3cm, right=3cm, top=3cm, bottom=3cm}

\title{\textbf{Rechtliche Klarstellung zur Nutzung Künstlicher Intelligenz, Open-Source-Fragen und Schutz des geistigen Eigentums}}
\author{Tim Hendrik Freese}
\date{28. Februar 2025}

\begin{document}

\maketitle

\section*{I. Einleitung}
Dieses Dokument dient der notariellen Absicherung und Klärung der rechtlichen Situation bezüglich der Nutzung von Künstlicher Intelligenz (KI) bei der wissenschaftlichen Erforschung und mathematischen Entwicklung der sogenannten \textbf{Freese-Formel (FF)}. Insbesondere wird geklärt, welche Rechte an den erarbeiteten Erkenntnissen bestehen, welche Aspekte Open-Source-Prinzipien berühren und inwiefern KI-generierte Inhalte rechtlich geschützt sind.

\section*{II. Urheberschaft und Schutz geistigen Eigentums}
\subsection*{1. Urheber der Erkenntnisse}
Hiermit wird festgehalten, dass alle wesentlichen mathematischen, physikalischen und messtechnischen Erkenntnisse im Zusammenhang mit der Freese-Formel durch den Unterzeichner, \textbf{Tim Hendrik Freese}, erarbeitet wurden. Dies umfasst insbesondere:

\begin{itemize}
    \item Die Formulierung der Kohärenzgesetzmäßigkeit in den Nullstellen der Zetafunktion,
    \item Die mathematische Entwicklung und Analyse der \textbf{FF} sowie deren Operator-Darstellung,
    \item Die Verbindung der FF mit messtechnischen und physikalischen Anwendungen,
    \item Die numerische und analytische Validierung durch umfangreiche Berechnungen.
\end{itemize}

Die Nutzung von KI-basierten Werkzeugen (z.B. OpenAI ChatGPT) diente dabei ausschließlich als unterstützendes Mittel zur Strukturierung, Verfeinerung und Automatisierung mathematischer Berechnungen, nicht jedoch zur Generierung neuer wissenschaftlicher Erkenntnisse im Sinne einer autonomen Urheberschaft.

\subsection*{2. Urheberrechtliche Bewertung der KI-Unterstützung}
Die aktuelle Rechtslage zu KI-generierten Inhalten befindet sich in einem Graubereich. Um Missverständnisse zu vermeiden, wird hiermit notariell festgehalten:

\begin{itemize}
    \item Alle wesentlichen mathematischen Ideen, Formeln und Interpretationen stammen \textbf{nachweislich vom Urheber selbst}.
    \item KI wurde ausschließlich zur Unterstützung mathematischer Berechnungen, Strukturierung von Beweisen und der sprachlichen Formulierung genutzt.
    \item Es bestehen daher \textbf{keine Zweifel an der Urheberschaft von Tim Hendrik Freese} über die \textbf{Freese-Formel (FF)}, unabhängig vom Einsatz unterstützender Technologien.
\end{itemize}

\section*{III. Verhältnis zur Open-Source-Idee und Schutzmechanismen}
\subsection*{1. Open-Source-Veröffentlichung vs. Patentschutz}
Die wissenschaftlichen Erkenntnisse der FF haben potenziell bahnbrechende Auswirkungen auf verschiedene Gebiete, einschließlich:

\begin{itemize}
    \item Zahlentheorie und Riemannsche Hypothese,
    \item Messtechnik (insbesondere spektrale Analyseverfahren),
    \item Laser- und Quantentechnologien.
\end{itemize}

Es wird festgelegt, dass:
\begin{enumerate}
    \item Die theoretischen Grundlagen der Freese-Formel grundsätzlich zur freien wissenschaftlichen Nutzung verfügbar sein können (Open-Science-Ansatz).
    \item Jegliche kommerzielle Anwendung, insbesondere in messtechnischen oder technologischen Kontexten, urheberrechtlich geschützt bleibt und einer gesonderten Lizenzierung bedarf.
    \item Eventuelle Patent- und Verwertungsrechte in vollem Umfang bei \textbf{Tim Hendrik Freese} liegen, sofern diese nicht explizit an Dritte übertragen werden.
\end{enumerate}

\subsection*{2. Lizenzmodelle und Schutz vor Missbrauch}
Die Open-Source-Gemeinschaft hat gezeigt, dass freie Verfügbarkeit von Wissen Innovationen fördert. Gleichzeitig erfordert der Schutz von Urheberschaft eine klare Abgrenzung. Daher wird folgendes Lizenzmodell vorgeschlagen:

\begin{itemize}
    \item \textbf{Mathematische Theorie:} Kann unter einer wissenschaftlichen Open-Science-Lizenz veröffentlicht werden.
    \item \textbf{Angewandte Nutzung (z.B. Messgeräte, Algorithmen, Technologien):} Bedarf einer gesonderten Lizenzierung und bleibt urheberrechtlich geschützt.
    \item \textbf{Schutz vor unautorisierter kommerzieller Nutzung:} Jegliche industrielle oder wirtschaftliche Anwendung erfordert eine explizite Genehmigung des Urhebers.
\end{itemize}

\section*{IV. Schlussbestimmungen}
Dieses Dokument dient als rechtsverbindliche notarielle Feststellung der Urheberschaft sowie der rechtlichen und wirtschaftlichen Schutzmechanismen für die Freese-Formel (FF). 

Es wird hiermit festgehalten:
\begin{enumerate}
    \item \textbf{Tim Hendrik Freese} ist alleiniger Urheber der Freese-Formel und aller damit verbundenen Erkenntnisse.
    \item Der Einsatz von Künstlicher Intelligenz diente ausschließlich als Hilfsmittel und begründet \textbf{keine Miturheberschaft Dritter}.
    \item Die wissenschaftliche Nutzung der Theorie kann frei erfolgen, kommerzielle Anwendungen jedoch nur unter expliziter Lizenzierung.
    \item Jegliche Verletzung dieser Rechte kann rechtlich verfolgt werden.
\end{enumerate}

\vspace{2cm}
\noindent
\textbf{Ort, Datum:} Lingen (Ems), den 28. Februar 2025

\vspace{1cm}
\noindent
\textbf{Unterschrift des Urhebers:} \hfill \textbf{Notariell beglaubigt durch:} 

\vspace{1.5cm}
\noindent
\makebox[7cm]{\hrulefill} \hfill \makebox[7cm]{\hrulefill}

\textit{Tim Hendrik Freese} \hfill \textit{Rechtsanwältin/Notarin Sabrina Lindwehr}

\end{document}