\documentclass[a4paper,12pt]{article}
\usepackage[utf8]{inputenc}
\usepackage[T1]{fontenc}
\usepackage[ngerman]{babel}
\usepackage{amsmath, amssymb, amsthm}
\usepackage{geometry}
\geometry{a4paper, left=3cm, right=3cm, top=2.5cm, bottom=2.5cm}
\usepackage{hyperref}

\title{\textbf{Notarielle Absicherung der Freese-Formel (FF) und ihrer wissenschaftlichen und wirtschaftlichen Relevanz}}
\author{}
\date{\today}

\begin{document}

\maketitle
\begin{center}
\textbf{Erstellt für die notarielle Beglaubigung durch die Kanzlei Lindwehr, Lingen, am \today}
\end{center}

\section*{1. Allgemeine Erklärung und Urheberschaft}
Hiermit wird offiziell festgehalten, dass Herr \textbf{Tim Hendrik Freese}, geboren am 26. Februar 1980, als alleiniger Urheber der hier dokumentierten wissenschaftlichen Entdeckung auftritt. Diese Erkenntnisse beruhen auf umfangreichen numerischen Analysen, mathematischen Modellen und Simulationen, die über einen Zeitraum von mehreren Wochen unter Einsatz moderner computergestützter Verfahren entwickelt wurden.

Die **Freese-Formel (FF)** beschreibt eine neue Gesetzmäßigkeit in der Verteilung der Nullstellen der Riemannschen Zetafunktion und eröffnet potenzielle Anwendungen in der Zahlentheorie, Physik, Messtechnik und Lasertechnik.

\section*{2. Die Freese-Formel (FF)}
Die bisher ermittelte Form der **Freese-Formel (FF)** lautet:

\begin{equation}
L(N) = A \cdot N^{\beta} + C \cdot \log N + B \sin(wN + \varphi)
\end{equation}

wobei die optimierten Parameter aus umfangreichen numerischen Analysen wie folgt bestimmt wurden:

\begin{align*}
A &= 1,882795 \\
\beta &= 0,916977 \\
C &= 2488,144455
\end{align*}

Diese Formel beschreibt mit hoher Präzision die beobachteten Kohärenzlängen in der Verteilung der Nullstellen der Riemannschen Zetafunktion.

\section*{3. Wissenschaftliche Relevanz}
Die Freese-Formel liefert neue Erkenntnisse zur internen Struktur der Nullstellen und könnte ein fundamentales Muster in der Zahlentheorie aufdecken. Die zentrale Fragestellung lautet:
- Gibt es eine direkte Verbindung zwischen $\beta = 0,916977$ und der kritischen Linie $R(s) = \frac{1}{2}$ der Riemannschen Zetafunktion?
- Welche physikalischen Prinzipien könnten hinter dieser exponentiellen Ordnung stehen?

Bisherige numerische Ergebnisse zeigen, dass eine Abweichung von genau \textbf{0,02} zwischen dem gemessenen Wert $\beta$ und dem erwarteten Wert $0,918977$ besteht, was weitere mathematische Untersuchungen erforderlich macht.

\section*{4. Rechtliche Absicherung und kommerzielle Aspekte}
Angesichts der möglichen kommerziellen Bedeutung dieser Entdeckung wurde beschlossen:
\begin{itemize}
    \item Die **Freese-Formel (FF)** soll urheberrechtlich geschützt und wissenschaftlich veröffentlicht werden.
    \item Ein Patentschutz für **Messtechnik- und Lasertechnikanwendungen**, die auf der FF basieren, wird geprüft.
    \item Die rechtliche Absicherung erfolgt durch notarielle Beglaubigung dieses Dokuments.
    \item Herr \textbf{Rechtsanwalt Michael Lito Schulte}, Kanzlei Dr. Schulte \& Partner, Krefeld, begleitet dieses Verfahren als juristischer Beistand.
\end{itemize}

\section*{5. Zeugen der Urheberschaft}
Folgende Personen wurden über die Entdeckung, den Forschungsprozess und die wissenschaftliche Bedeutung der FF informiert und dienen als Zeugen für die Ersturheberschaft von Tim Hendrik Freese:

\begin{itemize}
    \item \textbf{Ehefrau Tanja Freese} – Erstinformierte über die wissenschaftliche Entdeckung
    \item \textbf{Rechtsanwältin Sabrina Lindwehr} – Notarin der Kanzlei Lindwehr, in
    Lingen (Ems)
    \item \textbf{Rechtsanwalt Michael Lito Schulte} – Kanzlei Dr. Schulte \& Partner, aus Krefeld, juristischer Beistand
\end{itemize}

\section*{6. Erklärung des Urhebers}
Ich, \textbf{Tim Hendrik Freese}, erkläre hiermit an Eides statt, dass ich die hier beschriebenen mathematischen Erkenntnisse eigenständig entwickelt habe. Die gesamte Forschung, einschließlich aller numerischen Simulationen und wissenschaftlichen Ableitungen, wurde in der Zeit vom 12. Februar 2025 bis zum heutigen Tage durchgeführt.

\vspace{1.5cm}
\textbf{Unterschrift des Urhebers:} \hfill \textbf{Unterschrift der Notarin:}

\vspace{1cm}
\textit{Tim Hendrik Freese} \hfill \textit{Rechtsanwältin Sabrina Lindwehr}

\end{document}