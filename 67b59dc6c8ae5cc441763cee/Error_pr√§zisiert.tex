\documentclass[a4paper,12pt]{article}
\usepackage{amsmath, amssymb, graphicx, hyperref}

\title{Notarielle Dokumentation: Fibonacci-Freese-Formel und Nullstellen der Riemannschen Zeta-Funktion}
\author{[Ihr Name]}
\date{\today}

\begin{document}

\maketitle

\section{Einleitung}
Diese Dokumentation hält den experimentellen Stand der Untersuchungen zur Fibonacci-Freese-Formel (FFF) im Kontext der Nullstellen der Riemannschen Zeta-Funktion fest. Insbesondere wird das Verhalten der exponentiellen Skalenordnung für unterschiedliche Nullstellenbereiche untersucht.

\section{Theoretische Grundlage}
Die Fibonacci-Freese-Formel postuliert eine universelle Skalenordnung für die Nullstellenabstände der Riemannschen Zeta-Funktion:
\begin{equation}
L(N) \sim A \cdot N^f + C \cdot \log(N) + D \cdot N^{-1} + B \cdot J_0(\omega N + \phi),
\end{equation}
wobei $N$ die Sequenz der Nullstellen beschreibt, und die Parameter $A, f, C, D, B, \omega, \phi$ empirisch bestimmt werden.

\section{Experimentelle Ergebnisse}

\subsection{Datenquelle}
Die Nullstellen stammen aus einer externen Quelle (\url{https://beta.lmfdb.org/riemann-zeta-zeros/data/}) und wurden in der Datei gespeichert:

\begin{verbatim}
/content/drive/MyDrive/zeros6.txt
\end{verbatim}

Insgesamt wurden $2.000.000$ Nullstellen geladen.

\subsection{Messwerte der FFF-Sequenz}

Die folgende Tabelle zeigt die für verschiedene $N$-Werte ermittelten Exponenten:

\begin{center}
\begin{tabular}{|c|c|}
\hline
$N$ & $f_{\text{gemessen}}$ \\
\hline
100 & 0.981223 \\
1000 & 0.998148 \\
10000 & -4.246332 \\
50000 & \textbf{Fehlgeschlagen (nan)} \\
100000 & 0.365041 \\
250000 & 0.140458 \\
500000 & 0.164259 \\
1000000 & 0.166981 \\
\hline
\end{tabular}
\end{center}

\subsection{Fehlermeldungen und Instabilitäten}

Ab $N \geq 50.000$ traten numerische Probleme auf:

\begin{itemize}
    \item \textbf{Fehlgeschlagene Fits} ab $N = 50.000$.
    \item \textbf{Instabile Werte} für größere $N$.
    \item \textbf{Numerische Artefakte} durch exponentielles Wachstum.
\end{itemize}

\section{Relevanter Python-Code zur Berechnung}

\begin{verbatim}
import numpy as np
import matplotlib.pyplot as plt
from scipy.optimize import curve_fit

# Daten laden
file_path = "/content/drive/MyDrive/zeros6.txt"
nullstellen = np.loadtxt(file_path)
abstaende = np.diff(nullstellen)
N_values = np.arange(1, len(abstaende) + 1)

# FFF-Rekursionsformel mit Log-Korrektur
def fibonacci_freese_extended(N):
    log_fff_seq = np.zeros(N)
    log_fff_seq[:2] = np.log([1, 1])  # Startwerte

    for i in range(2, N):
        log_fff_seq[i] = np.log1p(np.expm1(log_fff_seq[i-1]) + np.expm1(log_fff_seq[i-2]))

    return log_fff_seq

# Exponentenschätzung
def estimate_exponent(x, y):
    def power_law(x, A, f):
        return A * x**f

    params, _ = curve_fit(power_law, x, y, p0=[1, 0.5])
    return params[1]

# Tests mit verschiedenen N-Werten
test_values = [100, 1000, 10000, 50000, 100000, 250000, 500000, 1000000]
results = []

for N in test_values:
    log_fff_seq = fibonacci_freese_extended(N)
    f_exp = estimate_exponent(np.arange(1, len(log_fff_seq) + 1), log_fff_seq)
    results.append((N, f_exp))

for N, f_exp in results:
    print(f"�� Berechnung für N = {N}...")
    print(f"�� Geschätzter Exponent f aus der FFF-Reihe (mit Log-Korrektur): {f_exp:.6f}")
\end{verbatim}

\section{Fragestellung und Interpretation}

\subsection{Offene Fragen}
\begin{enumerate}
    \item Warum divergiert die Exponenten-Berechnung für große $N$?
    \item Warum sind die Werte für kleine $N$ stabil, aber bei großen $N$ unstabil?
    \item Gibt es eine unerkannte oszillierende Struktur, die die Skalenordnung beeinflusst?
    \item Sind numerische Artefakte oder logarithmische Korrekturterme eine Erklärung?
\end{enumerate}

\subsection{Verbindung zur Riemannschen Hypothese}
Eine Hypothese ist, dass sich die beobachteten Instabilitäten aus der überlagerten Doppelhelix-Struktur der Nullstellen ergeben, die sich auf der kritischen Linie $s = \frac{1}{2} + it$ befinden.

Die Frage bleibt offen:
\begin{center}
\textbf{Ist dies ein numerisches Artefakt oder ein tieferer Hinweis auf die Struktur der Nullstellen?}
\end{center}

\end{document}