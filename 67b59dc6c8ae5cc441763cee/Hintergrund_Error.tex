\documentclass[a4paper,12pt]{article}
\usepackage{amsmath, amssymb, graphicx, hyperref, listings}

\title{Notarielle Dokumentation: Fibonacci-Freese-Formel und Nullstellen der Riemannschen Zeta-Funktion}
\author{[Ihr Name]}
\date{\today}

\begin{document}

\maketitle

\section{Einleitung}
Diese notarielle Dokumentation sichert den aktuellen Stand der Untersuchung der Fibonacci-Freese-Formel (FFF) ab. Die Hypothese ist, dass die Abstände zwischen den Nullstellen der Riemannschen Zeta-Funktion eine **universelle Skalenordnung** aufweisen, die mit einer Potenzgesetz-Struktur beschrieben werden kann.

\section{Mathematische Grundlagen}
Die Fibonacci-Freese-Formel (FFF) zur Beschreibung der Nullstellenabstände lautet:
\begin{equation}
L(N) \sim A \cdot N^f + C \cdot \log(N) + D \cdot N^{-1} + B \cdot J_0(\omega N + \phi),
\end{equation}
wobei \( f \) der gesuchte Exponent ist. Eine alternative Formulierung unter Verwendung einer log-transformierten Version der Fibonacci-Sequenz wird derzeit untersucht.

\section{Datenquelle und Methodik}

\subsection{Datenquelle}
Die verwendeten Nullstellen stammen aus der Online-Datenbank \url{https://beta.lmfdb.org/riemann-zeta-zeros/data/} und wurden als Datei verarbeitet:

\begin{verbatim}
/content/drive/MyDrive/zeros6.txt
\end{verbatim}

Es wurden insgesamt **2.000.000 Nullstellen** geladen.

\subsection{Python-Code zur Analyse}
\textbf{Wichtiger Python-Code zur Reproduktion der Ergebnisse:}

\begin{lstlisting}[language=Python, basicstyle=\ttfamily\small, breaklines=true]
import numpy as np
import matplotlib.pyplot as plt
from scipy.optimize import curve_fit

# Daten laden
file_path = "/content/drive/MyDrive/zeros6.txt"
nullstellen = np.loadtxt(file_path)
abstaende = np.diff(nullstellen)
N_values = np.arange(1, len(abstaende) + 1)

# Log-transformierte Fibonacci-Freese-Sequenz
def fibonacci_freese_extended(N):
    log_fff_seq = np.zeros(N)
    log_fff_seq[:2] = np.log([1, 1])  # Startwerte

    for i in range(2, N):
        log_fff_seq[i] = np.log1p(np.expm1(log_fff_seq[i-1]) + np.expm1(log_fff_seq[i-2]))

    return log_fff_seq

# Exponentenschätzung
def estimate_exponent(x, y):
    def power_law(x, A, f):
        return A * x**f

    params, _ = curve_fit(power_law, x, y, p0=[1, 0.5])
    return params[1]

# Tests mit verschiedenen N-Werten
test_values = [100, 1000, 10000, 50000, 100000, 250000, 500000, 1000000]
results = []

for N in test_values:
    try:
        log_fff_seq = fibonacci_freese_extended(N)
        f_exp = estimate_exponent(np.arange(1, len(log_fff_seq) + 1), log_fff_seq)
        results.append((N, f_exp))
        print(f"�� Berechnung für N = {N}... �� Geschätzter Exponent: {f_exp:.6f}")
    except Exception as e:
        print(f"⚠️ Fehler bei N = {N}: {e}")
\end{lstlisting}

\section{Experimentelle Ergebnisse}

Die folgenden Messungen wurden für verschiedene Werte von \( N \) durchgeführt:

\begin{table}[h!]
\centering
\begin{tabular}{|c|c|}
\hline
\( N \) & \( f_{\text{gemessen}} \) \\
\hline
100 & 0.981223 \\
1000 & 0.998148 \\
10000 & -4.246332 \\
50000 & \textbf{Fehlgeschlagen (nan)} \\
100000 & 0.365041 \\
250000 & 0.140458 \\
500000 & 0.164259 \\
1000000 & 0.166981 \\
\hline
\end{tabular}
\caption{Ergebnisse der Fibonacci-Freese-Exponentenschätzung}
\end{table}

\section{Fehlermeldungen und Instabilitäten}
\subsection{Numerische Probleme}
Ab \( N \geq 50.000 \) traten numerische Probleme auf:

\begin{itemize}
    \item \textbf{Overflow-Effekte}: Die Berechnung wurde instabil und erzeugte `nan`-Werte.
    \item \textbf{Fit-Probleme}: Das Curve-Fitting-Verfahren schlug ab einer bestimmten Grenze fehl.
    \item \textbf{Skalenbrüche}: Ab einem gewissen \( N \) wurde \( f(N) \) nicht mehr konsistent.
\end{itemize}

\subsection{Alternative Hypothesen}
Es gibt mehrere mögliche Erklärungen:

\begin{itemize}
    \item **Numerische Instabilitäten**: Verbesserung durch alternative Fit-Methoden wie Bayesianische Fits oder Regularisierung.
    \item **Fraktale Eigenschaften**: Untersuchung der Skalenstruktur durch Multifraktalanalyse.
    \item **Physikalische Interpretation**: Falls \( f(N) \) tatsächlich nicht konstant ist, könnte dies eine emergente Eigenschaft der Zeta-Funktion sein.
\end{itemize}

\section{Zusammenhang mit der Riemannschen Hypothese (RH)}
Falls \( f(N) \) eine **universelle Struktur** der Nullstellen beschreibt, sollte dieser Wert für große \( N \) stabil sein. Die Instabilität legt nahe:

\begin{itemize}
    \item **Übergangsphänomene**: Möglicherweise ist \( f(N) \) erst für extrem große \( N \) stabil.
    \item **Multifraktale Effekte**: Eine Wavelet-Analyse könnte zeigen, ob es eine tiefere mathematische Ordnung gibt.
    \item **Zusammenhang mit Zufallsmatrizen**: Vergleich mit GOE-Statistik könnte aufzeigen, ob \( f(N) \) eine zufallsstatistische Eigenschaft hat.
\end{itemize}

\section{Zusammenfassung der offenen Fragen}
\begin{enumerate}
    \item Warum divergiert \( f(N) \) für große \( N \)? $\Rightarrow$ Test mit Log-Transformationen.
    \item Ist die Bessel-Korrektur notwendig? $\Rightarrow$ Vergleich mit Cosinus-Modulation.
    \item Gibt es eine tiefere theoretische Begründung? $\Rightarrow$ Vergleich mit der asymptotischen Entwicklung der Zeta-Funktion.
    \item Ist der Zusammenhang mit Zufallsmatrizen real? $\Rightarrow$ Vergleich mit GOE-Spektren.
\end{enumerate}

\section{Ausblick}
Nächste Schritte:
\begin{itemize}
    \item Log-transformierte Fits testen.
    \item Alternative Fit-Strategien implementieren.
    \item Multifraktale Analyse der Zeta-Nullstellen.
\end{itemize}

\end{document}