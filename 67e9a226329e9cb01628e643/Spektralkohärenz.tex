\section{Spektralkohärenzstruktur in $\beta(n)$ und Verbindung zur Skalenkonstante}

Die spektrale Analyse der Sliding-Exponenten $\beta(n)$ ergibt ein dominantes Frequenzband bei
\[
f^* \approx 0{,}01,
\]
welches durch eine signifikante Amplitudenverstärkung auffällt. Dieses Frequenzband lässt sich mit einer gebrochenen Skalenkonstante verknüpfen:

\begin{equation}
\beta_{\text{krit}} := \frac{\log 3}{\log 6} \approx 0{,}484906,
\end{equation}

welche in früheren Arbeiten als universeller Skalenexponent postuliert wurde. Die Nähe zu $\frac{1}{2}$ verweist auf eine gebrochene, aber nahezu symmetrische Struktur entlang der kritischen Linie $\operatorname{Re}(s) = \frac{1}{2}$ in der komplexen Ebene.

\subsection*{Spektralmodell der Skalenanregung}

Wir postulieren ein spektrales Anregungssignal $S(n)$, das sich über eine gewichtete Fourierreihe der Form
\[
S(n) = \sum_{k=1}^{N} \hat{\beta}(f_k) \cdot \cos(2\pi f_k n + \phi_k)
\]
rekonstruieren lässt. Dabei bezeichnet $\hat{\beta}(f_k)$ die Spektralkomponente bei Frequenz $f_k$, welche in unserem Fall bei $f^* \approx 0{,}01$ ein globales Maximum aufweist.

\subsection*{Interpretation als Skalenresonanz}

Diese Struktur lässt sich als eine dynamische Kopplung zwischen:

\begin{itemize}
\item der skalenlogarithmischen Struktur der Nullstellenverteilung (über $L(n)$ und $\beta(n)$),
\item und der spektral rekonstruierten Tschebyschow-Funktion $\psi(x)$
\end{itemize}

interpretieren. Die beobachtete Frequenz $f^*$ wirkt somit wie ein Resonator in der log-periodischen Skalenstruktur und legt den Grundstein für ein Operator-basiertes Modell.