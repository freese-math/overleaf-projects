\section{Gemeinsames spektrales Modell}

Beide Funktionen, $\beta(n)$ und $\psi(x)$, lassen sich als Realisierungen eines zugrundeliegenden spektralen Stimulus $S(f)$ interpretieren, der durch die Verteilung der nichttrivialen Nullstellen der Riemannschen Zeta-Funktion bestimmt wird.

Wir postulieren die Struktur:

\begin{equation}
S(f) = w_\beta(f) \cdot \widehat{\beta}(f) + w_\psi(f) \cdot \widehat{\psi}(f)
\end{equation}

Hierbei bezeichnet:
\begin{itemize}
  \item $\widehat{\beta}(f)$: normiertes Spektrum der Sliding-Fit-Exponenten,
  \item $\widehat{\psi}(f)$: normiertes Spektrum der Tschebyschow-Funktion,
  \item $w_\beta(f), w_\psi(f)$: spektrale Gewichtsfunktionen.
\end{itemize}

\subsection*{Spezialfall: Harmonische Kopplung}

Im Falle einer dominanten Kohärenz bei Frequenz $f_0 = 0{,}001$, setzen wir:

\begin{equation}
w_\beta(f) = \delta(f - f_0), \qquad w_\psi(f) = \delta(f - f_0)
\end{equation}

Dann ergibt sich:

\begin{equation}
S(f_0) = \widehat{\beta}(f_0) + \widehat{\psi}(f_0)
\end{equation}

Diese Gleichung beschreibt die additive Überlagerung beider Spektren an der gemeinsamen Grundfrequenz und bildet den Kern der spektralen Kohärenzstruktur.