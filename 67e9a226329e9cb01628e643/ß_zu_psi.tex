\section{Spektralkohärenz zwischen \texorpdfstring{$\beta(n)$}{beta(n)} und \texorpdfstring{$\psi(x)$}{psi(x)}}

Zur Untersuchung einer möglichen strukturellen Kopplung zwischen der exponentiellen Abstandsstruktur $L(n)$ der Zeta-Nullstellen und der Tschebyschow-Funktion $\psi(x)$ wurde eine Fourier-Spektralanalyse durchgeführt. Grundlage sind die normierten Amplitudenspektren zweier Funktionen:

\begin{itemize}
  \item $\beta(n)$: lokal gefitteter Exponent aus Sliding-Window-Modellen $L(n) \approx \alpha n^{\beta(n)}$ über $2{.}001{.}052$ Zeta-Nullstellen (Odlyzko).
  \item $\psi(x)$: spektral rekonstruierte Tschebyschow-Funktion über die explizite Formel mit $\sim 10{,}000$ nichttrivialen Nullstellen.
\end{itemize}

\subsection*{Ergebnisse der Spektralanalyse}

\begin{itemize}
  \item Die \textbf{Pearson-Korrelation} der normierten Spektren beträgt:
  \[
  r = 0{,}7707
  \]
  \item Die \textbf{Cosinus-Ähnlichkeit} im Frequenzraum ergibt:
  \[
  \cos(\theta) = 0{,}7989
  \]
  \item Beide Spektren teilen einen \textbf{dominanten Grundfrequenzpeak} bei:
  \[
  f = 0{,}00100
  \]
\end{itemize}

\subsection*{Interpretation}

Die identische Grundfrequenz in beiden Spektren wird als Ausdruck einer \textbf{universellen Kohärenzlängenskala} gedeutet. Diese könnte die spektrale Organisation der Nullstellenstruktur sowie deren makroskopische Wirkung auf arithmetische Funktionen wie $\psi(x)$ koordinieren.

Die Sliding-Fit-Analyse der Funktion $L(n)$ ergibt über die gesamte Datenbasis einen stabilen Mittelwert:

\begin{equation}
\boxed{\bar{\beta(n)} = -0{,}127625}
\end{equation}

Dieser Wert stimmt mit der theoretisch motivierten Annahme einer logarithmischen Kohärenzstruktur überein.

\subsection*{Schlussfolgerung}

Die hohe strukturelle Übereinstimmung im niederfrequenten Spektralbereich, gepaart mit einer exakten Grundfrequenzkoinzidenz, bildet die Grundlage der \textbf{Spektralkohärenzhypothese}:

\begin{quote}
Die spektrale Struktur von $\beta(n)$ und $\psi(x)$ ist kohärent im Sinne einer gemeinsamen Skalenfrequenz. Diese Kohärenz ist Ausdruck der zugrundeliegenden Nullstellenstruktur der Riemannschen Zeta-Funktion.
\end{quote}