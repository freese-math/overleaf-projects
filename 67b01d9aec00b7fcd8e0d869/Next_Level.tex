\documentclass[11pt]{article}
\usepackage[utf8]{inputenc}
\usepackage{amsmath,amssymb,amsthm}
\usepackage{geometry}
\usepackage{hyperref}
\usepackage{graphicx}
\usepackage{enumitem}
\geometry{margin=2.5cm}

\title{\textbf{Spektralintegrale Verbindung zwischen der Freese-Funktion \(\delta(\rho)\)\\ und dem Weil-Funktional \(W(f)\)}}
\author{Freese Math Research Initiative}
\date{April 2025}

\newtheorem{theorem}{Theorem}
\newtheorem{definition}{Definition}
\newtheorem{proposition}{Proposition}
\newtheorem{remark}{Remark}

\begin{document}

\maketitle

\begin{abstract}
Wir untersuchen einen integraltheoretischen Zusammenhang zwischen der spektralen Störfunktion \(\delta(\rho)\), wie sie im rekonstruktiven Ansatz zur Riemannschen Hypothese von Freese verwendet wird, und dem Weil-Funktional \(W(f)\), das in der Spurformulierung von Connes \& Consani eine zentrale Rolle spielt. Durch eine gewichtete Darstellung der Fourier-transformierten Testfunktionen wird eine spektrale Integralform vorgeschlagen, die beide Konzepte strukturell vereint. Dies eröffnet neue Perspektiven für eine Positivitätsformulierung der Riemannschen Hypothese über frequenzmodulierte Eigenräume.
\end{abstract}

\section{Einleitung}

Die Riemannsche Hypothese (RH) ist äquivalent zur Aussage, dass alle nichttrivialen Nullstellen \(\rho\) der Zeta-Funktion auf der kritischen Linie \(\Re(\rho) = \tfrac{1}{2}\) liegen. Zwei moderne Zugänge zur RH operieren dabei auf unterschiedlichen Ebenen:

\begin{itemize}[topsep=3pt]
    \item Das \textbf{Freese-Theorem} verwendet die \textit{Beta-Skala} \(\beta(n)\), um über spektrale Modulationen eine harmonische Struktur in den Nullstellen zu rekonstruieren.
    \item Die \textbf{Weil-Positivität} bei Connes \& Consani (2021) formuliert RH als Nichtnegativität des Spurfunktionals \(W(f)\) auf dem komprimierten Sonin-Raum.
\end{itemize}

In dieser Arbeit formulieren wir eine Proposition, die beide Konzepte über ein Integral verknüpft.

\section{Grundlagen}

\begin{definition}[Freese-Störfunktion]
Die spektrale Störfunktion \(\delta(\rho)\) misst die Abweichung der harmonischen Ordnung der Zeta-Nullstellen von ihrer idealen Frequenzresonanz:
\[
\delta(\rho) := \mathrm{Si}(\rho \log x) - \mathrm{Si}\left(\frac{1}{\rho} \log x\right),
\]
wobei \(\mathrm{Si}(z) := \int_0^z \frac{\sin t}{t} \, dt\) die Sinusintegralfunktion ist.
\end{definition}

\begin{definition}[Weil-Funktional]
Für eine glatte, kompakt unterstützte Testfunktion \(f(u)\) auf \(\mathbb{R}_+^*\) mit Mellin-Transformierter \(\hat{f}(\rho)\) ist das Weil-Funktional definiert als
\[
W(f) := \sum_\rho \hat{f}(\rho),
\]
wobei sich die Summe über alle nichttrivialen Nullstellen \(\rho\) der Riemann-Zeta-Funktion erstreckt.
\end{definition}

\section{Hauptresultat}

\begin{proposition}[Spektralintegrale Verknüpfung]
Sei \(f(u)\) eine glatte Testfunktion mit Fourier-Transformierter \(\hat{f}(\rho)\), und sei \(\delta(\rho)\) wie oben definiert. Dann gilt formal:
\[
W(f) \approx \int_{\mathcal{C}} \delta(\rho) \cdot \omega(\rho) \, d\rho,
\]
wobei \(\omega(\rho) := \frac{\hat{f}(\rho)}{\delta(\rho)}\), und \(\mathcal{C}\) eine geeignete symmetrische Kurve in der kritischen Ebene ist.
\end{proposition}

\begin{remark}
Die Funktion \(\delta(\rho)\) wirkt als lokales Frequenzmaß, \(\hat{f}(\rho)\) als globaler Testimpuls. Die Verknüpfung zeigt, dass destruktive Frequenzinterferenzen außerhalb von \(\Re(\rho) = \tfrac{1}{2}\) zugleich \(\delta(\rho) \neq 0\) und \(W(f) < 0\) induzieren würden.
\end{remark}

\section{Diskussion und Ausblick}

Diese Integralformel erlaubt eine rekonstruktive Spurformel im Sinne eines harmonisch gewichteten Operatorraums. Perspektiven umfassen:

\begin{itemize}[topsep=3pt]
    \item Kopplung der Freese’schen Beta-Skala an Schwartz-Testfunktionen.
    \item Formale Operatoridentitäten in nichtkommutativen Spektraltripletts.
    \item Anwendung auf generalisierte L-Funktionen mit Theta-Charakter.
\end{itemize}

\section*{Danksagung}

Wir danken der Freese Math Research Initiative für den Impuls zur strukturellen Verbindung spektraler Zugänge.

\bibliographystyle{plain}
\begin{thebibliography}{9}
\bibitem{connes2021} A. Connes, C. Consani: \textit{Weil positivity and Trace formula}, arXiv:2107.03561 (2021).
\bibitem{freese2025} T. Freese: \textit{The Freese Theorem and Beta-Scale Oscillations}, internal research note, 2025.
\end{thebibliography}

\end{document}