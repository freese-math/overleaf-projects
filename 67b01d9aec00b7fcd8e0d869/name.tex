\documentclass[11pt]{article}
\usepackage{amsmath, amssymb}
\usepackage{graphicx}
\usepackage{physics}
\usepackage{hyperref}
\usepackage{geometry}
\geometry{a4paper, margin=2.5cm}

\title{Spektrale Rekonstruktion der Tschebyschow-Funktion mittels Beta-Gewichtung}
\author{}
\date{}

\begin{document}

\maketitle

\section*{1. Ziel und Erwartung}

Ziel ist es, eine spektrale Rekonstruktion der Tschebyschow-Funktion $\psi(x)$ auf Basis einer gewichteten Zeta-Summenformel zu realisieren:

\[
L(x) = 1 + \sum_{k=1}^{N} \frac{x^{\rho_k} \cdot \beta(k) \cdot \zeta(2\rho_k)}{\rho_k \cdot \zeta'(\rho_k)}
\]

Die Funktion $\psi(x)$ ist definiert als:

\[
\psi(x) := \sum_{n \le x} \Lambda(n) = \sum_{p^m \le x} \log(p)
\]

Die Annäherung $L(x) \approx \psi(x)$ wird erwartet, wenn die $\beta(k)$-Skala korrekt konstruiert ist und mit der spektralen Struktur der Riemannschen Zeta-Funktion harmoniert.

\section*{2. Beobachtung im Plot}

\begin{itemize}
    \item Die \textbf{schwarze Kurve} zeigt $\psi(x)$ als Referenz.
    \item Die \textbf{blaue Kurve} zeigt $L(x)$ mit $\beta(k)$ aus Frequenzextraktion.
\end{itemize}

\textbf{Beobachtung:} Bei geeigneter Wahl von $\beta(k)$ nähern sich beide Kurven sichtbar an. Die Oszillationen sind vergleichbar, jedoch nicht identisch. Dies lässt Rückschlüsse auf die spektrale Plausibilität der Konstruktion zu.

\section*{3. Konvergenztest}

Ein Konvergenzindikator der Reihe ist:

\[
S(n) = \sum_{k=1}^{n} \frac{|\beta(k)|}{\gamma_k}
\]

Dabei ist $\rho_k = \frac{1}{2} + i\gamma_k$ eine Nullstelle der Zeta-Funktion.

\textbf{Interpretation:} Wenn $S(n)$ sublinear wächst (z.\,B. logarithmisch), ist die Reihe konvergent. Im numerischen Experiment ist dies gegeben, was auf eine robuste Struktur der $\beta(k)$-Skala hinweist.

\section*{4. Begriffsklärung und Interpretation}

In der Herleitung wurden Begriffe aus der Physik übernommen. Für mathematische Strenge müssen diese präzise gefasst werden:

\begin{itemize}
    \item \textbf{Frequenz:} $\gamma_k = \operatorname{Im}(\rho_k)$, interpretiert als Spektralanteil.
    \item \textbf{Amplitude:} Betrag der spektralen Gewichtung $\beta(k)$.
    \item \textbf{Spektrum:} Eigenwertspektrum eines Operators, z.\,B. der Hamilton-Operator im quantenspektralen Ansatz.
\end{itemize}

\section*{5. Fazit}

Die Formel (76) mit rekonstruierter $\beta(k)$-Skala bildet $\psi(x)$ sichtbar annähernd nach. Konvergenz, spektrale Struktur und Visualisierung sprechen für die Validität des Ansatzes. Eine rigorose analytische Einbettung der $\beta$-Konstruktion in klassische Theorien (Guinand-Weil, Explicit Formula) ist ein folgender Schritt.

\end{document}