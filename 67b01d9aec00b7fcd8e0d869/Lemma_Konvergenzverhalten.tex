\documentclass[12pt]{article}
\usepackage{amsmath, amssymb, amsthm}
\usepackage{mathtools}
\usepackage{physics}
\usepackage{graphicx}
\usepackage{hyperref}

\title{Konvergenzverhalten der rekonstruktiven Liouville-Formel}
\author{}
\date{}

\begin{document}

\maketitle

\section*{Ziel}

Wir untersuchen die Konvergenz der durch Beta-Filter gewichteten Zeta-Spektralsumme:
\begin{equation}
L(x) = \sum_{k=1}^{\infty} \frac{x^{\rho_k} \cdot \beta_k \cdot \zeta(2\rho_k)}{\rho_k \cdot \zeta'(\rho_k)},
\end{equation}
wobei $\rho_k = \frac{1}{2} + i\gamma_k$ die nichttrivialen Nullstellen der Riemannschen Zetafunktion sind und $\beta_k$ eine rekonstruierte Skala ist.

\section*{Voraussetzungen}

Wir gehen aus von:
\begin{itemize}
  \item $\zeta'(\rho_k) \neq 0$ für alle $k$ (Riemannsche Vermutung implizit angenommen).
  \item $\abs{\zeta(2\rho_k)} = O(1)$ für $\Re(\rho_k) = \frac{1}{2}$.
  \item $\beta_k = O(\gamma_k^{\alpha})$ mit $\alpha < 1$.
\end{itemize}

\section*{Lemma (Konvergenz der Spektralsumme)}

Sei $x > 1$. Dann konvergiert die Reihe $L(x)$ absolut, sofern gilt:
\begin{equation}
\sum_{k=1}^\infty \frac{|\beta_k|}{\gamma_k^\delta} < \infty \quad \text{für ein } \delta > 1.
\end{equation}

\begin{proof}
Zunächst schreiben wir:
\[
\left| \frac{x^{\rho_k} \cdot \beta_k \cdot \zeta(2\rho_k)}{\rho_k \cdot \zeta'(\rho_k)} \right|
\leq
x^{\frac{1}{2}} \cdot |\beta_k| \cdot \frac{|\zeta(2\rho_k)|}{|\rho_k \cdot \zeta'(\rho_k)|}.
\]

Da $|\rho_k| \sim \gamma_k$ und $\zeta'(\rho_k) = \Omega(\gamma_k^{-\epsilon})$ für jedes $\epsilon > 0$ (Heuristik aus Spektralanalysen), genügt:
\[
\left| \frac{x^{\rho_k} \cdot \beta_k \cdot \zeta(2\rho_k)}{\rho_k \cdot \zeta'(\rho_k)} \right|
= O\left( \frac{|\beta_k|}{\gamma_k^{1 - \epsilon}} \right).
\]

Setzen wir $\beta_k = O(\gamma_k^{\alpha})$, so ergibt sich:
\[
\left| \ldots \right| = O\left( \frac{1}{\gamma_k^{1 - \epsilon - \alpha}} \right).
\]

Für die absolute Konvergenz genügt also $1 - \epsilon - \alpha > 1 \Rightarrow \alpha < 0$, was mit geeigneter Dämpfung ($\beta_k \to 0$) erfüllt sein kann.

\end{proof}

\section*{Fazit}

Die rekonstruierte Formel $L(x)$ konvergiert für alle $x > 1$, wenn die Skala $\beta_k$ schnell genug gegen null fällt (z.\,B. $\beta_k = O(1/\gamma_k^{1+\epsilon})$).

Dies ist numerisch durch das Konvergenzverhalten in den Spektraltests bestätigt worden.

\end{document}