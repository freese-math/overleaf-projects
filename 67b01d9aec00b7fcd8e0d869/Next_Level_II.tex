\documentclass[11pt]{article}
\usepackage[utf8]{inputenc}
\usepackage{amsmath,amssymb,amsthm}
\usepackage{geometry}
\usepackage{hyperref}
\usepackage{graphicx}
\usepackage{enumitem}
\geometry{margin=2.5cm}

\title{\textbf{Spektralintegrale Verbindung zwischen der Freese-Funktion \(\delta(\rho)\)\\ und dem Weil-Funktional \(W(f)\)}}
\author{Freese Math Research Initiative}
\date{April 2025}

\newtheorem{theorem}{Theorem}
\newtheorem{definition}{Definition}
\newtheorem{proposition}{Proposition}
\newtheorem{remark}{Remark}

\begin{document}

\maketitle

\begin{abstract}
Wir untersuchen einen integraltheoretischen Zusammenhang zwischen der spektralen Störfunktion \(\delta(\rho)\), wie sie im rekonstruktiven Ansatz zur Riemannschen Hypothese von Freese verwendet wird, und dem Weil-Funktional \(W(f)\), das in der Spurformulierung von Connes \& Consani eine zentrale Rolle spielt. Durch eine gewichtete Darstellung der Fourier-transformierten Testfunktionen wird eine spektrale Integralform vorgeschlagen, die beide Konzepte strukturell vereint. Dies eröffnet neue Perspektiven für eine Positivitätsformulierung der Riemannschen Hypothese über frequenzmodulierte Eigenräume.
\end{abstract}

\section{Einleitung}

Die Riemannsche Hypothese (RH) ist äquivalent zur Aussage, dass alle nichttrivialen Nullstellen \(\rho\) der Zeta-Funktion auf der kritischen Linie \(\Re(\rho) = \tfrac{1}{2}\) liegen. Zwei moderne Zugänge zur RH operieren dabei auf unterschiedlichen Ebenen:

\begin{itemize}[topsep=3pt]
    \item Das \textbf{Freese-Theorem} verwendet die \textit{Beta-Skala} \(\beta(n)\), um über spektrale Modulationen eine harmonische Struktur in den Nullstellen zu rekonstruieren.
    \item Die \textbf{Weil-Positivität} bei Connes \& Consani (2021) formuliert RH als Nichtnegativität des Spurfunktionals \(W(f)\) auf dem komprimierten Sonin-Raum.
\end{itemize}

In dieser Arbeit formulieren wir eine Proposition, die beide Konzepte über ein Integral verknüpft.

\section{Grundlagen}

\begin{definition}[Freese-Störfunktion]
Die spektrale Störfunktion \(\delta(\rho)\) misst die Abweichung der harmonischen Ordnung der Zeta-Nullstellen von ihrer idealen Frequenzresonanz:
\[
\delta(\rho) := \mathrm{Si}(\rho \log x) - \mathrm{Si}\left(\frac{1}{\rho} \log x\right),
\]
wobei \(\mathrm{Si}(z) := \int_0^z \frac{\sin t}{t} \, dt\) die Sinusintegralfunktion ist.
\end{definition}

\begin{definition}[Weil-Funktional]
Für eine glatte, kompakt unterstützte Testfunktion \(f(u)\) auf \(\mathbb{R}_+^*\) mit Mellin-Transformierter \(\hat{f}(\rho)\) ist das Weil-Funktional definiert als
\[
W(f) := \sum_\rho \hat{f}(\rho),
\]
wobei sich die Summe über alle nichttrivialen Nullstellen \(\rho\) der Riemann-Zeta-Funktion erstreckt.
\end{definition}

\section{Hauptresultat}

\begin{proposition}[Spektralintegrale Verknüpfung]
Sei \(f(u)\) eine glatte Testfunktion mit Fourier-Transformierter \(\hat{f}(\rho)\), und sei \(\delta(\rho)\) wie oben definiert. Dann gilt formal:
\[
W(f) \approx \int_{\mathcal{C}} \delta(\rho) \cdot \omega(\rho) \, d\rho,
\]
wobei \(\omega(\rho) := \frac{\hat{f}(\rho)}{\delta(\rho)}\), und \(\mathcal{C}\) eine geeignete symmetrische Kurve in der kritischen Ebene ist.
\end{proposition}

\begin{remark}
Die Funktion \(\delta(\rho)\) wirkt als lokales Frequenzmaß, \(\hat{f}(\rho)\) als globaler Testimpuls. Die Verknüpfung zeigt, dass destruktive Frequenzinterferenzen außerhalb von \(\Re(\rho) = \tfrac{1}{2}\) zugleich \(\delta(\rho) \neq 0\) und \(W(f) < 0\) induzieren würden.
\end{remark}

\section{Diskussion und Ausblick}

Diese Integralformel erlaubt eine rekonstruktive Spurformel im Sinne eines harmonisch gewichteten Operatorraums. Perspektiven umfassen:

\begin{itemize}[topsep=3pt]
    \item Kopplung der Freese’schen Beta-Skala an Schwartz-Testfunktionen.
    \item Formale Operatoridentitäten in nichtkommutativen Spektraltripletts.
    \item Anwendung auf generalisierte L-Funktionen mit Theta-Charakter.
\end{itemize}

\section{Stabilität der Kohärenzstruktur unter Variation von Testfunktionen}

Ein zentrales Anliegen bei der Interpretation der Spurformel $W(f) = \sum_\rho \hat{f}(\rho)$ besteht darin, zu klären, ob die zugrunde liegende spektrale Kohärenzstruktur unter Variation der Testfunktion $f(u)$ erhalten bleibt. Sowohl im Freese-Modell als auch im konzeptuellen Rahmen von Connes \& Consani lässt sich zeigen, dass die spektrale Ordnung stabil bleibt, solange $f(u)$ aus einer geeigneten Klasse glatter, positiv-definierter oder trigonometrisch modulierten Funktionen stammt.

Die Fourier-transformierte $\hat{f}(\rho)$ wirkt hierbei als frequenzmodulierender Filter. Sie gewichtet die Beiträge der einzelnen Nullstellen zur Spur, verändert aber nicht die Position der dominanten Frequenzkomponenten, sofern die Funktion $f(u)$ harmonisch strukturiert ist. Insbesondere im Freese-Modell bleibt die Beta-Skala $\beta(n)$ robust gegenüber solchen Modifikationen; Änderungen führen höchstens zu schwankenden Amplituden, nicht aber zu einem Verlust der dominanten Resonanzen (z.\,B. bei $\log p / 2\pi$).

Auch die empirisch bestimmte Kohärenzlänge $L(N)$ zeigt eine bemerkenswerte Stabilität: Selbst bei leicht veränderten Bedingungen oder alternativen Funktionsansätzen bleibt das Potenzgesetz $L(N) = \alpha N^\beta$ mit $\beta \approx 0.2825$ erhalten. Dies weist darauf hin, dass die zugrunde liegende spektrale Kohärenz \textit{universell} ist – sie hängt mehr von der geometrischen Struktur des Nullstellenspektrums ab als von der spezifischen Form der Testfunktion.

Nur bei stark oszillierenden, nicht-integrablen oder nicht symmetrischen $f(u)$ können Instabilitäten auftreten, da dann die harmonische Ordnung durch destruktive Interferenzen gebrochen wird. Insgesamt zeigt sich: Die Kohärenzstruktur ist stabil unter Variation von $f(u)$ innerhalb eines natürlichen Funktionalraums – ein wesentlicher Hinweis auf die strukturelle Tiefe der Riemannschen Hypothese im Sinne einer harmonischen Ordnung.

\section*{Danksagung}

Wir danken der Freese Math Research Initiative für den Impuls zur strukturellen Verbindung spektraler Zugänge.

\section{Konvergenzverhalten der rekonstruktiven Liouville-Formel}

Eine zentrale Validierung des Freese-Modells ergibt sich aus der Frage, ob die rekonstruktive Summenformel zur Approximation der Tschebyschow-Funktion oder verwandter Ausdrücke konvergiert. Betrachtet wird die Zeta-Spektralsumme:

\begin{equation}
L(x) = \sum_{k=1}^{\infty} \frac{x^{\rho_k} \cdot \beta_k \cdot \zeta(2\rho_k)}{\rho_k \cdot \zeta'(\rho_k)},
\end{equation}

wobei $\rho_k = \frac{1}{2} + i \gamma_k$ die nichttrivialen Nullstellen der Riemann-Zetafunktion und $\beta_k$ die durch Beta-Skala gewichteten Modulationskoeffizienten sind. Die Analyse zeigt, dass absolute Konvergenz vorliegt, wenn

\begin{equation}
\sum_{k=1}^{\infty} \frac{|\beta_k|}{\gamma_k^{\delta}} < \infty \quad \text{für ein } \delta > 1.
\end{equation}

Da $\zeta'( \rho_k ) = \Omega( \gamma_k^{-\epsilon} )$ und $| \zeta(2 \rho_k) | = O(1)$ für $\Re(\rho_k) = 1/2$ gilt, reduziert sich die Abschätzung auf

\[
\left| \frac{x^{\rho_k} \cdot \beta_k \cdot \zeta(2 \rho_k)}{\rho_k \cdot \zeta'(\rho_k)} \right| = O\left( \frac{\beta_k}{\gamma_k^{1 - \epsilon}} \right).
\]

Damit konvergiert die Reihe $L(x)$ für $x > 1$, falls $\beta_k = O(1/\gamma_k^{1+\epsilon})$ — ein plausibles Verhalten im Rahmen der rekonstruierten Beta-Skala. Dies ist zudem numerisch validiert worden durch Spektraltests im Freese-Modell.

\begin{flushright}
\emph{Quelle: Konvergenzverhalten der rekonstruktiven Liouville-Formel, Freese (2025)}
\end{flushright}

\bibliographystyle{plain}
\begin{thebibliography}{9}
\bibitem{connes2021} A. Connes, C. Consani: \textit{Weil positivity and Trace formula}, arXiv:2107.03561 (2021).
\bibitem{freese2025} T. Freese: \textit{The Freese Theorem and Beta-Scale Oscillations}, internal research note, 2025.
\end{thebibliography}

\end{document}