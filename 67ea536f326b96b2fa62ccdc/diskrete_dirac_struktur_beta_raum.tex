
\documentclass[a4paper,12pt]{article}
\usepackage{amsmath,amsfonts,amssymb}
\usepackage{graphicx}
\usepackage{geometry}
\geometry{margin=2.5cm}

\title{Diskrete Dirac-Struktur auf dem Beta-Raum}
\author{}
\date{}

\begin{document}

\maketitle

\textbf{Diskrete Dirac-Struktur auf dem Beta-Raum}

Wir definieren einen Operator $H_\beta$ als eine diskrete Version eines Impulsoperators entlang eines Skalenraums von Beta-Werten:
\[
H_\beta = i\varepsilon \frac{\Delta}{\Delta \beta}
\]
wobei $\varepsilon$ ein diskretes Skalenquantum ist und $\frac{\Delta}{\Delta \beta}$ der zentrale Differenzenoperator ist. Daraus ergibt sich eine antisymmetrische Matrixstruktur.

Die Dirac-Gleichung auf diesem Raum lautet:
\[
\left( i H_\beta - m \right)\psi(\beta) = 0
\]
In zweikomponentiger Matrixform schreiben wir:
\[
\mathcal{H}_{\text{Dirac}} =
\begin{pmatrix}
0 & H_\beta \\
H_\beta & 0
\end{pmatrix}
\]

Die Eigenwerte dieser Matrix zeigen eine spektrale Symmetrie um Null, ein Hinweis auf eine chirale Struktur analog zur kontinuierlichen Dirac-Theorie. Die zugehörigen Eigenvektoren kodieren mögliche Zustände auf dem Beta-Raum.

\end{document}
