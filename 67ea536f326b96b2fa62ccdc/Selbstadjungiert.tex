\documentclass[a4paper,12pt]{article}
\usepackage{amsmath,amssymb,amsthm}
\usepackage{mathtools}
\usepackage{physics}
\usepackage{graphicx}
\usepackage{hyperref}

\title{Spektraloperator \( D_\mu \) und seine Selbstadjungiertheit}
\author{Eigene Forschung}
\date{}

\begin{document}
\maketitle

\section*{Definition des Operators}

Sei \( \mu(n) \) eine strukturtragende Gewichtung, z.\,B.\ basierend auf der Beta-Skala \( \beta(n) \). Dann definieren wir den Operator \( D_\mu \) auf dem Funktionsraum \( \mathcal{H} = \ell^2(\mathbb{N}) \) durch
\[
(D_\mu f)(n) := \mu(n) \cdot f(n)
\]
für alle \( f \in \mathcal{H} \). Dieser Operator ist diagonal in der natürlichen Basis \( \{e_n\} \) gegeben durch \( e_n(k) = \delta_{nk} \).

\section*{Selbstadjungiertheit}

Zur Überprüfung der Selbstadjungiertheit betrachten wir zwei Testfunktionen \( f, g \in \mathcal{H} \), zum Beispiel:
\[
f(n) = \sin\left(\frac{2\pi n}{N}\right), \quad g(n) = \cos\left(\frac{4\pi n}{N}\right)
\]
Dann gilt:
\[
\braket{D_\mu f, g} = \sum_{n=1}^{N} \mu(n) f(n) g(n), \quad
\braket{f, D_\mu g} = \sum_{n=1}^{N} f(n) \mu(n) g(n)
\]
Da beide Summen identisch sind, folgt:
\[
\braket{D_\mu f, g} = \braket{f, D_\mu g} \quad \Rightarrow \quad D_\mu = D_\mu^\dagger
\]
Der Operator ist also \textbf{selbstadjungiert}.

\section*{Spektralkonstruktion von \( \eta(n) \)}

Wir definieren die spektrale Erweiterung durch:
\[
\eta(n) := \beta(n) + \theta(n)
\]
wobei \( \theta(n) \) eine strukturierte Korrekturfunktion beschreibt, die durch das Spektrum des Operators \( D_\mu \) erklärbar ist.

\textbf{Beobachtung:} Numerisch ergibt sich:
\begin{itemize}
    \item Die diagonale Wirkung \( \mu(n) \) ist mit der Beta-Skala hoch korreliert.
    \item Die Ergänzung zu \( \eta(n) \) besitzt eine Oszillationsstruktur, die durch spektrale Eigenschaften erklärbar ist.
\end{itemize}

\section*{Ausblick}

Da \( D_\mu \) selbstadjungiert ist, können weitere Untersuchungen zur Spektraldichte, möglichen Eigenfunktionen und zur Verbindung mit Zeta-Strukturen (via Fourier-Operatoren) erfolgen.

\end{document}