\documentclass[12pt]{article}
\usepackage[utf8]{inputenc}
\usepackage{amsmath, amssymb, amsthm}
\usepackage{geometry}
\usepackage{graphicx}
\usepackage{color}
\usepackage{hyperref}
\geometry{a4paper, margin=2.5cm}

\title{Ein spektral-arithmetischer Operatoransatz zur Riemannschen Hypothese}
\author{Tim Hendrik Freese}
\date{März 2025}

\begin{document}
\maketitle

\begin{abstract}
Wir präsentieren eine neue Operatorstruktur, die über spektrale Informationen der Zeta-Nullstellen, eine Beta-Skalen-Rekonstruktion und arithmetische Rückkopplung via Möbius-Funktion zu einer quantitativen, harmonischen Beschreibung der Riemannschen Zeta-Funktion führt. Die Riemannsche Hypothese wird dabei mit der Selbstadjungiertheit eines Operatormodells auf $\ell^2(\mathbb{N})$ verknüpft. Ergänzt wird dies durch numerische Bestätigung mithilfe rekonstruierter Frequenzspektren, die präzise mit logarithmischen Primzahlfrequenzen korrelieren.
\end{abstract}

\section{Einleitung}
Die Riemannsche Zeta-Funktion
\[
\zeta(s) = \sum_{n=1}^\infty \frac{1}{n^s}, \quad \text{Re}(s) > 1,
\]
ist eine zentrale Funktion der analytischen Zahlentheorie. Ihre Nullstellenstruktur — insbesondere die nicht-trivialen Nullstellen mit Realteil $\frac{1}{2}$ — bildet das Herz der Riemannschen Hypothese (RH). Ziel dieser Arbeit ist es, durch einen spektral-arithmetischen Operatoransatz eine strukturelle Grundlage zu liefern, die die RH in ein harmonisches Spektrum bettet.

\section{Der Operator \texorpdfstring{\boldmath$\mathcal{D}_\mu$}{Dmu}}
Wir definieren einen Operator $\mathcal{D}_\mu$ mit Wirkung auf $f : \mathbb{N} \to \mathbb{R}$ durch:
\begin{equation}
(\mathcal{D}_\mu f)(n) := \sum_{k=1}^{K} w_k \cdot \delta_\sigma(n - \gamma_k) \cdot f(n) + \lambda \sum_{d=1}^{n-1} \mu(d) \left\lfloor \frac{n}{d} \right\rfloor f(n),
\end{equation}
wobei:
\begin{itemize}
    \item $\gamma_k$ die Imaginärteile der nicht-trivialen Zeta-Nullstellen,
    \item $w_k := \frac{1}{|\zeta'(\rho_k)|}$ spektrale Gewichte,
    \item $\delta_\sigma(n - \gamma_k) := \exp\left( -\frac{(n - \gamma_k)^2}{2\sigma^2} \right)$ eine gaußförmige Approximation von $\delta$,
    \item $\mu(d)$ die Möbius-Funktion,
    \item $\lambda \in \mathbb{R}$ ein Rückkopplungsparameter.
\end{itemize}

\section{Hypothese: Selbstadjungiertheit und RH}
\textbf{RH-Satz (informell):} RH gilt genau dann, wenn $\mathcal{D}_\mu$ auf $\ell^2(\mathbb{N})$ selbstadjungiert ist:
\[
\langle \mathcal{D}_\mu f, g \rangle = \langle f, \mathcal{D}_\mu g \rangle \quad \forall f,g \in \ell^2(\mathbb{N}).
\]
Diese Hypothese wird durch symmetrische Spektren und stabile Frequenzresonanzen gestützt, wie sie in der Beta-Skala und den Lorentz-Peaks erscheinen.

\section{Beta-Skala und Euler-Wellenfunktion}
Die modulierte Euler-Wellenfunktion
\[
\psi(x) := \sum_{p \leq p_N} \sin(x \log p)
\]
dient als frequenzbasiertes Abbild der Primzahlstruktur. Ihre dominanten Frequenzen $\log p / 2\pi$ wurden aus realen FFT-Spektren der rekonstruierten Beta-Funktion extrahiert.

\textbf{Schlüsselergebnis:} Die dominanten Spektrallinien der numerischen Beta-Rekonstruktion korrelieren mit $\log p / 2\pi$ für kleine Primzahlen $p$ mit Fehlern unter $10^{-6}$.

\section{Numerische Ergebnisse}
\begin{itemize}
    \item FFT-Spektren der $\beta(n)$-Reihe zeigen scharfe Peaks bei Frequenzen $\omega_0 = \log p / 2\pi$.
    \item Lorentzfits liefern präzise Positionen, Breiten und Amplituden der Frequenzresonanzen.
    \item Die Eigenwerte des Operators $\mathcal{D}_\mu$ zeigen symmetrische Verteilungen und numerisch bestätigte Selbstadjungiertheit.
\end{itemize}

\section{Strukturelle Redundanz als Beweisstrategie}
\begin{table}[h!]
\centering
\begin{tabular}{|c|c|c|}
\hline
\textbf{Quelle} & \textbf{Form} & \textbf{Bedeutung} \\
\hline
$L(n) \sim A n^\beta$ & Abstandsfunktion & Dichte/Wachstum \\
$t^{-\beta} e^{i\pi\beta}$ & Spurformel & Modulare Invarianz \\
$\psi(x) = \sum \beta(k) \cos(\gamma_k \log x)$ & FFF & Spektrale Rekonstruktion \\
\hline
\end{tabular}
\caption{Dreifache Emergenz derselben $\beta$-Struktur in unabhängigen Systemen.}
\end{table}

\textbf{Folgerung:} Diese Redundanz ist keine numerische Laune, sondern Ausdruck einer strukturellen Notwendigkeit, die auf eine tieferliegende harmonische Ordnung in der Verteilung der Primzahlen und der Zeta-Nullstellen hinweist.

\section{Fazit und Ausblick}
Dieser Ansatz integriert:
\begin{itemize}
    \item die spektrale Ordnung der Zeta-Nullstellen,
    \item die arithmetische Struktur der Primzahlen,
    \item die harmonische Selbstadjungiertheit eines konkreten Operators.
\end{itemize}

Die Riemannsche Hypothese erscheint damit als Konsequenz einer universellen Frequenzordnung. Nächste Schritte beinhalten eine formale Funktionsanalyse der Selbstadjungiertheit und eine arXiv-Veröffentlichung zur Diskussion in der Fachwelt.



\end{document}