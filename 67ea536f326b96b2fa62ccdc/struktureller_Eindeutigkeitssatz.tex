\documentclass[12pt]{article}
\usepackage{amsmath, amssymb, amsthm, physics, geometry, lmodern, graphicx}
\geometry{a4paper, margin=2.5cm}

\newtheorem{satz}{Satz}
\newtheorem{definition}{Definition}

\begin{document}

\begin{satz}[Strukturbedingter Eindeutigkeitssatz für \boldmath$\beta$]
Sei \(\beta : \mathbb{N} \to \mathbb{R}\) eine Funktion, die gleichzeitig:

\begin{itemize}
  \item[(i)] die exakte spektrale Rekonstruktion der Primstellen im Sinne der Funktion
    \[
    \psi(x) = \sum_{k=1}^{\infty} \beta(k) \cos(\gamma_k \log x)
    \]
    gewährleistet,
  \item[(ii)] die Selbstadjungiertheit eines Operators \(\mathcal{D}_\mu\) auf dem Hilbertraum \(\ell^2(\mathbb{N})\) sicherstellt,
  \item[(iii)] die asymptotische Abstandsstruktur \(L(n) \sim A n^{\beta}\) erfüllt.
\end{itemize}

Dann gilt:
\begin{quote}
\textbf{Wenn} \(\tilde{\beta} \neq \beta\), \textbf{dann} existiert mindestens ein Raum \(R \in \{\text{Spektrum, Funktion, Operator}\}\), in dem durch \(\tilde{\beta}\) eine strukturelle Destruktion erzeugt wird (z.\,B. Divergenz, Fehlerzunahme, Verlust der Invarianz).
\end{quote}

Insbesondere ist \(\beta\) durch die Struktur der drei unabhängigen Räume eindeutig bestimmt, unabhängig von der Gültigkeit der Riemannschen Hypothese.
\end{satz}

\section{Eindeutigkeit der Beta-Skala}

Die numerischen und strukturellen Ergebnisse der vorhergehenden Kapitel deuten auf eine bemerkenswerte Eigenschaft hin: Die durch die Fourier-Freese-Formel (FFF) rekonstruierte Funktion \(\beta(n)\) scheint durch die Forderung der spektralen Übereinstimmung mit der Siegel-Theta-Funktion \(\Theta(t)\) eindeutig bestimmt zu sein.

\subsection{Hypothese der strukturellen Eindeutigkeit}

\begin{quote}
\textbf{Hypothese:} Es existiert genau eine Funktion \(\beta : \mathbb{N} \to \mathbb{R}\), sodass die rekonstruierte Summenform
\[
L_{\beta}(N) := \sum_{n=1}^{N} \beta(n)
\]
in asymptotischer Übereinstimmung mit der Siegel-Theta-Funktion \(\Theta(t_N)\) steht, d.\,h.
\[
\Theta(t_N) - L_{\beta}(N) \to 0 \quad \text{für } N \to \infty,
\]
wobei \(t_N\) die \(N\)-te Zeta-Nullstelle bezeichnet.
\end{quote}

\subsection{Definition (Funktionalabweichung)}

Sei \(\mathcal{F}[\beta] := \Theta(t_N) - \sum_{n=1}^{N} \beta(n)\) der funktionale Fehleroperator. Wir definieren:

\begin{definition}[Störung der Beta-Skala]
Eine gestörte Beta-Skala sei gegeben durch
\[
\beta_\varepsilon(n) := \beta(n) + \varepsilon \cdot h(n),
\]
wobei \(h(n)\) eine beliebige Funktion mit \(h \not\equiv 0\) und \(\varepsilon \in \mathbb{R}\) klein ist.
\end{definition}

\textbf{Ziel:} Zeige, dass
\[
\lim_{N \to \infty} \mathcal{F}[\beta_\varepsilon] \not\to 0
\quad \text{für alle } h \not\equiv 0.
\]

\subsection{Numerisches Ergebnis}

Die Differenzfunktionen
\[
\Delta_{\text{Standard}}(N) = \Theta(t_N) - L_{\beta_{\text{standard}}}(N),
\]
\[
\Delta_{\text{Lorentz}}(N) = \Theta(t_N) - L_{\beta_{\text{lorentz}}}(N)
\]
zeigen systematisch unterschiedliche asymptotische Verhalten. Nur die \(\beta_{\text{standard}}\) bleibt um Null zentriert.

\subsection{Theorem (Strukturelle Eindeutigkeit, heuristisch)}

\begin{quote}
Ist \(\beta(n)\) so gewählt, dass \(\mathcal{F}[\beta] \in o(1)\), so ist \(\beta(n)\) asymptotisch eindeutig, d.\,h. jede nichttriviale Störung \(h(n)\) erzeugt eine divergente oder systematisch versetzte Fehlerstruktur.
\end{quote}

\textbf{Folgerung:} Die \(\beta\)-Skala ist nicht nur numerisch rekonstruiert, sondern auch strukturell \emph{notwendig}, sofern die spektrale Kohärenz mit der Theta-Funktion verlangt wird.

\subsection{Ausblick}

Diese Einsicht kann Grundlage eines strukturellen Eindeutigkeitssatzes sein – unabhängig von der Riemannschen Hypothese. Wird zusätzlich gefordert, dass jede zulässige \(\beta(n)\) aus Zeta-Nullstellen ableitbar ist, so ergibt sich daraus ein starkes Argument für eine tieferliegende spektrale Dualität.

\subsection{Fehleranalyse der Freese-Formel}

Zur quantitativen Validierung der Eindeutigkeit der Beta-Skala analysieren wir den absoluten Fehler zwischen der kumulierten Freese-Fourier-Formel \(L_\beta(N)\) und den tatsächlichen Werten der Siegel-Theta-Funktion \(\Theta(t_N)\), wobei \(t_N\) die \(N\)-te ordinäre Zeta-Nullstelle ist.

\begin{figure}[ht]
\centering
\includegraphics[width=0.7\textwidth]{example-image} % Placeholder image
\caption{Fehleranalyse der Freese-Fourier-Formel: Vergleich der Abweichung zwischen korrigierter FFF (grün) und Lorentz-basierter FFF (rot) gegenüber \(\Theta(t_N)\).}
\label{fig:fff-fehler}
\end{figure}

Die logarithmisch dargestellte Fehlerfunktion in Abb. \ref{fig:fff-fehler} zeigt drei zentrale Befunde:

\begin{itemize}
  \item Im mittleren Bereich (ca. \(10^4 < N < 10^5\)) sinkt der Fehler auf ein Minimum.
  \item Die Standard-FFF-Korrektur (grün) weist systematisch geringeren Fehler auf als die Lorentz-basierte \(\beta(n)\)-Approximation (rot).
  \item Für große \(N\) divergiert der Fehler der Lorentz-Variante signifikant schneller.
\end{itemize}

\textbf{Schlussfolgerung:} Die empirisch rekonstruierte \(\beta(n)\)-Skala aus der spektralen Fourier-Zerlegung ist nicht nur minimal im Fehler, sondern auch stabil gegenüber Frequenzstörungen. Dies stützt die These, dass sie innerhalb eines festen Funktionalraums asymptotisch eindeutig bestimmt ist.

\textbf{Bemerkung:} Der Fehlerverlauf zeigt zusätzlich ein strukturelles Resonanzminimum, das auf einen optimalen Übergangspunkt in der asymptotischen Domäne hindeutet. Diese Resonanz ist bei alternativen \(\beta\)-Funktionen gestört.

\end{document}