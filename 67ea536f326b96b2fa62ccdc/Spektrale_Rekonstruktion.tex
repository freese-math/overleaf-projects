\documentclass[12pt]{article}
\usepackage{amsmath,amssymb,amsthm}
\usepackage{mathtools}
\usepackage{physics}
\usepackage{bm}
\usepackage{graphicx}
\usepackage{geometry}
\geometry{a4paper, margin=2.5cm}
\usepackage{hyperref}

\title{Spektraloperatorische Rekonstruktion der Chebyshev-Funktion \texorpdfstring{$\psi(x)$}{ψ(x)}}
\author{Eigene Forschung – Stand: \today}
\date{}

\begin{document}
\maketitle

\section*{1. Ziel}
Ziel ist die Konstruktion eines linearen Operators, der die klassische Chebyshev-Funktion $\psi(x)$ aus den nichttrivialen Nullstellen der Riemannschen Zetafunktion $\rho_k = \frac{1}{2} + i\gamma_k$ spektral rekonstruiert.

\section*{2. Ausgangspunkt: Explizite Formel}
Die klassische explizite Formel lautet:
\[
\psi(x) = x - \sum_{\rho} \frac{x^\rho}{\rho} - \log(2\pi) - \frac{1}{2} \log(1 - x^{-2})
\]

\section*{3. Numerische Rekonstruktion}
Ausgangspunkt für die numerische Rekonstruktion ist:
\[
L(x) := \sum_{k=1}^N \frac{x^{\rho_k}}{\rho_k \zeta'(\rho_k)} \cdot \beta_k
\]
mit geeigneten Gewichtungskoeffizienten $\beta_k$, z.\,B.\ normalisiert $\beta_k := \frac{1}{N}$ oder angepasst an spektrale Profile.

\section*{4. Operatorformulierung}
Wir definieren einen linearen Operator $\mathcal{T}_\zeta$, der auf eine Testfunktion $f$ wirkt:
\[
(\mathcal{T}_\zeta f)(x) := \sum_{k=1}^{N} \frac{x^{\rho_k}}{\rho_k \zeta'(\rho_k)} \cdot \langle f, \phi_k \rangle
\]
Dabei sind $\phi_k$ Eigenfunktionen oder spektrale Basismoden im Kontext der Zeta-Analyse.

\section*{5. Integraloperator}
Formal lässt sich $\mathcal{T}_\zeta$ auch als Integraloperator schreiben:
\[
(\mathcal{T}_\zeta f)(x) = \int_0^\infty K(x,y) f(y) \, dy
\]
mit dem spektralen Kern
\[
K(x,y) := \sum_{k=1}^N \frac{y^{\rho_k}}{\rho_k \zeta'(\rho_k)} \cdot \delta(x - y)
\]
Dies beschreibt eine lokal gewichtete Projektion im Sinne der spektralen Dichte.

\section*{6. Spektrale Interpretation}
Die Struktur erinnert an einen Hamilton-Operator:
\[
H := \text{diag}(\gamma_k) + \text{Shift-Operator}
\]
Die Eigenwerte des Operators $H$ stimmen empirisch mit den Imaginärteilen der $\rho_k$ überein. Dies motiviert die Sichtweise auf $\mathcal{T}_\zeta$ als quantisierten Spektraloperator.

\section*{7. Zielstruktur}
Gesucht ist ein vollständiger Operator $\mathcal{T}_\psi$, sodass:
\[
\mathcal{T}_\psi \mathbf{1} = \psi(x)
\]
bzw. als Projektion auf Eigenfunktionen:
\[
\psi(x) = \sum_{k} \lambda_k \phi_k(x)
\quad \text{mit} \quad \lambda_k := \frac{1}{\rho_k \zeta'(\rho_k)}
\]

\section*{8. Weiteres Vorgehen}
\begin{itemize}
  \item Untersuchung der Selbstadjungiertheit von $\mathcal{T}_\zeta$
  \item Vergleich mit klassischer $\psi(x)$ via Fehleranalyse
  \item Zusammenhang mit der Euler-Freese-Struktur
  \item Ziel: Vollständiger spektraltheoretischer Beweisansatz zur RH
\end{itemize}

\end{document}