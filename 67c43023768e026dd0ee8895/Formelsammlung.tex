\documentclass[a4paper,12pt]{article}
\usepackage{amsmath, amssymb, amsthm, mathtools}
\usepackage{physics}
\usepackage{graphicx}
\usepackage{xcolor}
\usepackage{hyperref}

\hypersetup{
    colorlinks=true,
    linkcolor=blue,
    urlcolor=blue,
    pdftitle={Mathematische Formelsammlung der Freese-Formel},
    pdfpagemode=FullScreen,
}

\title{Mathematische Formelsammlung der Freese-Formel}
\author{Zusammenstellung aller bekannten Versionen}
\date{\today}

\begin{document}

\maketitle
\tableofcontents
\newpage

\section{Grundlagen der Freese-Formel}

Die Freese-Formel beschreibt die Abstände $\Delta_n$ zwischen aufeinanderfolgenden nichttrivialen Nullstellen der Riemannschen Zeta-Funktion. Sie basiert auf der Asymptotik der Nullstellenverteilung und empirischen Anpassungen an große Nullstellendatensätze.

\subsection{Freese-Formel Standard (FFS)}
Die Standard-Form der Freese-Formel lautet:
\begin{equation}
    \Delta_n = A n^{-1/2} + B n^{-1} + C.
\end{equation}
Numerische Fits liefern folgende Werte:
\begin{align*}
    A &\approx 25.953416, \\
    B &\approx -28.796961, \\
    C &\approx 0.529483.
\end{align*}

\subsection{Freese-Formel Oszillierend (FFO)}
Um feinere Schwankungen zu berücksichtigen, wurde ein oszillatorischer Korrekturterm eingeführt:
\begin{equation}
    \Delta_n = A n^{-1/2} + B n^{-1} + C + w \cos(n w + \phi).
\end{equation}
Mit den angepassten Parametern:
\begin{align*}
    A &\approx 25.967831, \\
    B &\approx -28.808266, \\
    C &\approx 0.529463, \\
    w &\approx 0.009999, \\
    \phi &\approx 1.638759.
\end{align*}

\section{Ableitungen und Asymptotik}

Die erste Ableitung der Standard-Form lautet:
\begin{equation}
    \frac{d}{dn} \Delta_n = -\frac{A}{2 n^{3/2}} - \frac{B}{n^2}.
\end{equation}
Die zweite Ableitung:
\begin{equation}
    \frac{d^2}{dn^2} \Delta_n = \frac{3A}{4 n^{5/2}} + \frac{2B}{n^3}.
\end{equation}

\section{Operator-Form der Freese-Formel}

In Operator-Darstellung wird die Freese-Formel als Operator $\hat{F}$ definiert:
\begin{equation}
    \hat{F} = A \hat{N}^{-1/2} + B \hat{N}^{-1} + C + w \cos(\hat{N} w + \phi).
\end{equation}
Hierbei ist $\hat{N}$ der Zahloperator mit den Eigenwerten:
\begin{equation}
    \hat{N} |n\rangle = n |n\rangle.
\end{equation}

\section{Effektives Potential und Schrödinger-Form}

Das effektive Potential aus der Operator-Form ist gegeben durch:
\begin{equation}
    V(x) = A x^{-1/2} + B x^{-1} + C.
\end{equation}
Die zugehörige Schrödinger-Gleichung nimmt die Form an:
\begin{equation}
    \left[ -\frac{d^2}{dx^2} + V(x) \right] \psi_n(x) = E_n \psi_n(x).
\end{equation}
Numerisch bestimmte Eigenwerte ergeben die Frequenzen:
\begin{equation}
    E_n \sim \left( n + \frac{1}{2} \right)^\beta.
\end{equation}

\section{Verbindung zur Riemannschen Hypothese}

Die Freese-Formel und ihr Spektrum könnten Hinweise auf eine tiefere mathematische Struktur liefern. Die Hypothese von Hilbert-Pólya besagt, dass es einen selbstadjungierten Operator $\hat{H}$ gibt, dessen Eigenwerte mit den imaginären Teilen der Nullstellen der Zeta-Funktion übereinstimmen:
\begin{equation}
    \hat{H} | \psi_n \rangle = \gamma_n | \psi_n \rangle.
\end{equation}
Ein Kandidat für einen solchen Operator könnte in der Form
\begin{equation}
    \hat{H} = -\frac{d^2}{dx^2} + V(x)
\end{equation}
gegeben sein, wobei $V(x)$ aus der Freese-Formel abgeleitet werden kann.

\section{Numerische Konstanten und offene Fragen}

Eine weitere offene Frage ist, ob die Konstante
\begin{equation}
    f = \lim_{n\to\infty} \frac{\text{FFS}(n)}{n^{-1/2}}
\end{equation}
eine fundamentale Naturkonstante für die Nullstellenverteilung darstellt.

\section{Zusammenfassung}

Die Freese-Formel bietet eine präzise Annäherung an die Abstände der nichttrivialen Nullstellen der Riemannschen Zeta-Funktion. Die Einführung eines Operators $\hat{F}$ und die Ableitung eines effektiven Potentials $V(x)$ eröffnen neue Wege zur formalen Untersuchung der Nullstellenstruktur. Ein rigoroser Beweis der Riemannschen Vermutung könnte möglicherweise über die Selbstadjungiertheit eines solchen Operators erfolgen.

\end{document}