\documentclass[a4paper,12pt]{article}
\usepackage{amsmath,amssymb,amsthm}
\usepackage{graphicx}

\begin{document}

\title{Fraktal-Operator \( D \) in der Fibonacci-Freese-Formel}
\author{}
\date{}
\maketitle

\section{Einführung}
Der Operator \( D \) ist eine fraktale Korrektur zur Fibonacci-Freese-Formel (FFF):

\[
L(N) = A N^\beta + C \log N + D N^{-1}
\]

mit der empirisch bestimmten Näherung:

\[
D \approx \frac{1}{\delta} \left( \frac{1}{N} \right)
\]

wobei \( \delta = 4.669201609 \) die Feigenbaum-Konstante ist.

\section{Operator-Darstellung}
Der Operator \( D \) wird als fraktaler Schrödinger-Operator dargestellt:

\[
\hat{D} = \sum_{n=1}^{\infty} \frac{e^{-\lambda n}}{n^\gamma} \cdot \hat{H}
\]

mit:

\[
\hat{H} = \frac{1}{\pi^2} + e^{-\frac{1}{\pi^2}} \sum_{n=1}^{\infty} \frac{1}{n^3}
\]

\section{Bedeutung für die Riemannsche Hypothese}
Falls die Zeta-Nullstellen einer fraktalen Ordnung folgen, könnte die RH durch eine Skaleninvarianz erklärt werden.

\end{document}