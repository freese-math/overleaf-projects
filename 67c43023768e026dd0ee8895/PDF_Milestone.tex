\documentclass[a4paper,12pt]{article}

% Pakete für Mathematik und Grafiken
\usepackage{amsmath, amssymb, amsthm}
\usepackage{graphicx}
\usepackage{xcolor}
\usepackage{hyperref}
\usepackage{booktabs}
\usepackage{listings}
\usepackage{geometry}
\geometry{a4paper, left=25mm, right=25mm, top=25mm, bottom=25mm}

\title{Die Freese-Funktion und die Riemannsche Hypothese}
\author{Analyse und Beweisstrategie}
\date{\today}

\begin{document}

\maketitle

\begin{abstract}
In diesem Dokument wird eine umfassende Analyse der Freese-Funktion (FFF) durchgeführt. Es wird eine Verbindung zur Riemannschen Zeta-Funktion hergestellt, numerische Analysen durchgeführt und ein spektraler Operator \( H \) zur formalen Beweisführung untersucht.
\end{abstract}

\section{Einleitung}
Die Riemannsche Hypothese (RH) ist eine der tiefgreifendsten offenen Fragen der Mathematik. Die Freese-Funktion wurde entwickelt, um eine alternative Beschreibung der Nullstellen der Zeta-Funktion zu liefern. Ziel ist es, eine direkte Verbindung zur Zeta-Funktion \(\zeta(s)\) aufzuzeigen.

\section{Numerische Analyse}

\subsection{Fourier-Analyse der Nullstellenabstände}
Die Fourier-Transformation der Nullstellenabstände zeigt signifikante Frequenzspitzen:

\begin{itemize}
    \item Dominante Frequenz \( f_1 \approx 0.0000005 \), Spektrale Stärke \( 1.87 \times 10^5 \)
    \item Zweite dominante Frequenz \( f_2 \approx 0.0000015 \), Spektrale Stärke \( 1.18 \times 10^5 \)
\end{itemize}

\begin{figure}[h]
    \centering
    \includegraphics[width=0.8\textwidth]{fourier_analysis.png}
    \caption{Fourier-Analyse der Nullstellenverteilung}
\end{figure}

\subsection{Wavelet-Analyse der Oszillationen}
Die Wavelet-Transformation mit der mexh-Wavelet zeigt selbstähnliche Muster:

\begin{figure}[h]
    \centering
    \includegraphics[width=0.8\textwidth]{wavelet_analysis.png}
    \caption{Wavelet-Analyse der Oszillationen}
\end{figure}

\section{Operator-Hypothese und Zeta-Funktion}

\subsection{Definition des Operators \( H \)}
Wir definieren einen spektralen Operator \( \hat{H} \), dessen Eigenwerte die Nullstellenstruktur der Zeta-Funktion abbilden:

\begin{equation}
    \hat{H} \psi_N = \lambda_N \psi_N
\end{equation}

mit

\begin{equation}
    \hat{H} = \frac{1}{\phi^2 \pi} + e^{-1/\pi^2} \sum_{n=1}^{\infty} \frac{1}{n^3}
\end{equation}

\subsection{Verbindung zur Zeta-Funktion}
Die Hypothese lautet, dass \( \hat{H} \) als Transformation von \( \zeta(s) \) beschrieben werden kann:

\begin{equation}
    \hat{H} \approx \sum_{n=1}^{\infty} \frac{1}{n^s}
\end{equation}

\begin{figure}[h]
    \centering
    \includegraphics[width=0.6\textwidth]{operator_structure.png}
    \caption{Operator-Struktur der Zeta-Funktion}
\end{figure}

\section{Zusammenfassung und nächste Schritte}
Die Ergebnisse zeigen eine starke Verbindung zwischen der Freese-Funktion und der Riemannschen Zeta-Funktion. Zur rigorosen Beweisführung wird eine analytische Formulierung der Operator-Hypothese benötigt.

\begin{itemize}
    \item Numerische Simulationen stützen die Hypothese.
    \item Die nächste Phase beinhaltet eine analytische Ableitung.
    \item Eine Publikation zur Diskussion mit Experten wird vorbereitet.
\end{itemize}

\section*{Referenzen}
\begin{itemize}
    \item Riemannsche Zeta-Funktion: \url{https://www.mathematik.uni-muenchen.de}
    \item Fourier-Analyse: \url{https://www.physik.uni-jena.de}
    \item Wissenschaftliche Arbeiten aus Würzburg und München
\end{itemize}

\end{document}