\documentclass[a4paper,12pt]{article}
\usepackage{amsmath,amssymb,amsthm}
\usepackage{graphicx}

\begin{document}

\title{Fraktale Operatorstruktur und rigorose Herleitung von \( D = 0.8 \) }
\author{}
\date{}
\maketitle

\section{Einleitung}
Die Fibonacci-Freese-Formel (FFF) beschreibt eine neue Gesetzmäßigkeit in der Struktur der Riemannschen Zeta-Nullstellen:

\[
L(N) = A N^\beta + C \log N + D N^{-1}
\]

Numerische Experimente ergaben:
\[
D = 0.8 = \frac{4}{5}
\]

Diese Korrektur könnte aus einer fundamentalen Selbstähnlichkeit der Nullstellenverteilung folgen.

\section{Herleitung über spektrale Dichte}
Falls \( D \) aus der Eigenwertverteilung der Zeta-Nullstellen folgt, dann ist er durch eine spektrale Integraldarstellung definiert:

\[
D = \int_{0}^{\infty} \rho(\lambda) \cdot f(\lambda) \, d\lambda
\]

mit einer Gewichtsfunktion \( f(\lambda) \), die einer fraktalen Skalierung folgt.

\section{Numerische Approximation über Reihenentwicklung}
Falls \( D \) durch eine logarithmische Korrektur dargestellt wird, dann folgt:

\[
D = \frac{\sum_{n=1}^{\infty} \frac{1}{n^{\gamma}}}{\log N}
\]

Numerisch erhalten wir für \( \gamma = 3 \):

\[
\sum_{n=1}^{\infty} \frac{1}{n^3} \approx 0.8
\]

Daher ergibt sich:

\[
D = \frac{\zeta(3)}{\log N}
\]

\section{Fraktale Topologie und Betti-Zahl}
Falls die Betti-Zahl eine Selbstähnlichkeit beschreibt, dann folgt:

\[
\beta_1 = \sum_{n=1}^{\infty} D^n
\]

Für \( D = 0.8 \):

\[
\beta_1 = \sum_{n=1}^{\infty} (0.8)^n \approx 148
\]

\section{Fazit}
Die Struktur von \( D = 0.8 \) könnte eine fundamentale Verbindung zur topologischen Struktur der Riemannschen Zetafunktion beschreiben. 

\end{document}