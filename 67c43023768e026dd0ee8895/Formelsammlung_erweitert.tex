\documentclass[a4paper,12pt]{article}
\usepackage{amsmath,amssymb,amsthm}
\usepackage{graphicx}

\begin{document}

\title{Mathematische Herleitung der Fibonacci-Freese-Formel}
\author{}
\date{}
\maketitle

\section{Hardy-Littlewood-Formel}
Die Anzahl der Nullstellen bis zur Höhe $T$ folgt aus der Hardy-Littlewood-Formel:
\[
N(T) \approx \frac{T}{2\pi} \log \frac{T}{2\pi} - \frac{T}{2\pi}
\]

\section{Position der Nullstellen}
\[
t_N \approx \frac{2\pi N}{\log N}
\]

\section{Abstände zwischen Nullstellen}
\[
L(N) = t_{N+1} - t_N = \frac{2\pi}{\log(N+1)} - \frac{2\pi}{\log N}
\]

Für große $N$ gilt:
\[
\log(N+1) \approx \log N + \frac{1}{N}
\]
und somit:
\[
L(N) \approx \frac{2\pi}{\log N} \left( 1 - \frac{1}{\log N} \right)
\]

\section{Fibonacci-Freese-Formel}
\[
L(N) = A N^\beta + C \log(N) + D N^{-1}
\]

mit den empirisch bestimmten Parametern:
\[
A = 1.8828, \quad \beta = 0.91698, \quad C = 2488.1446, \quad D = 0.00555
\]

\section{Erweiterung der Formel}
\[
L(N) = A N^\beta + C \log(N) + D N^{-1} + B \sin(w \log N + \varphi)
\]

\section{Beta-Korrektur}
\[
\beta_{\text{gemessen}} = 0.916977, \quad \beta_{\text{erwartet}} = 0.918977 = \frac{1}{2} + 0.418977
\]

\[
\Delta \beta = 0.002
\]

\section{Operator-Form}
\[
\hat{L} = \hat{H} N^\beta + C \log N + D N^{-1}
\]

\section{Quantenmechanische Interpretation}
\[
\hat{H} = \frac{1}{\pi^2} + e^{-\frac{1}{\pi^2}} \sum_{n=1}^{\infty} \frac{1}{n^3}
\]

\section{Schlussfolgerung zur Riemannschen Hypothese}
Falls jede Abweichung von der kritischen Linie die FFF-Struktur zerstören würde, ist RH bewiesen.

\end{document}