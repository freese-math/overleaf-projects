\documentclass{article}
\usepackage{amsmath, amssymb}

\begin{document}

\title{Mathematische Herleitung der Fibonacci-Freese-Formel}
\author{Tim Hendrik Freese}
\date{2025}
\maketitle

\section{Definition der Fibonacci-Freese-Formel}
Die allgemeine Form der FFF lautet:
\[
L(N) = A \cdot N^\beta + C \cdot \log N + D \cdot N^{-1} + B \sin(w \log N + \varphi)
\]
mit den Parametern:
\[
A = 1.8828, \quad \beta = 0.91698, \quad C = 2488.1446, \quad D = 0.00555, \quad w = 0.08, \quad \varphi = -9005.7583
\]

\section{Mathematische Herleitung}
\subsection{Nullstellen der Zetafunktion}
Nach Hardy-Littlewood ergibt sich:
\[
N(T) \approx \frac{T}{2\pi} \log \frac{T}{2\pi} - \frac{T}{2\pi}
\]

Daraus folgt für die Nullstellenabstände:
\[
L(N) \approx \frac{2\pi}{\log N} \left(1 - \frac{1}{N \log N}\right)
\]

\subsection{Operator-Formulierung}
Es existiert ein Operator \( \hat{L} \) mit:
\[
\hat{L} \psi_n = \lambda_n \psi_n
\]

Der Schrödinger-Ansatz mit potentiellem Zusammenhang:
\[
\hat{H} \psi_n = \left( -\frac{d^2}{dx^2} + V(x) \right) \psi_n = E_n \psi_n
\]

Das effektive Potential:
\[
V(x) \approx x^{0.8}
\]

\subsection{Verbindung zu Primzahlen}
Primzahldichte:
\[
\pi(N) \sim \frac{N}{\log N}
\]

Abstände:
\[
L_p(N) \approx \log N + \frac{1}{\log N}
\]

\subsection{Betti-Zahl und Fraktale Strukturen}
\[
\beta = \frac{1}{2} + 0.48 - \frac{1}{10} = 0.91698
\]

\section{Schlussfolgerung zur Riemannschen Hypothese}
\textbf{Falls diese Struktur universell ist, kann keine Nullstelle außerhalb der kritischen Linie existieren.}

\end{document}