\documentclass[a4paper,12pt]{article}
\usepackage{amsmath,amssymb,amsthm}
\usepackage{graphicx}

\begin{document}

\title{Fraktale Betti-Zahlen und der Operator \( D = 0.8 \)}
\author{}
\date{}
\maketitle

\section{Einleitung}
Die Fibonacci-Freese-Formel (FFF) beschreibt die Nullstellenstruktur der Riemannschen Zetafunktion mit einer fraktalen Korrektur:

\[
L(N) = A N^\beta + C \log N + D N^{-1}
\]

mit der numerischen Bestimmung:

\[
D = 0.8 = \frac{4}{5}
\]

Diese fraktale Struktur steht in Zusammenhang mit einer topologischen Ordnung, die durch die Betti-Zahl beschrieben wird:

\[
\beta_1 = 148
\]

\section{Operator-Darstellung von \( D \)}
Die Korrekturkomponente \( D \) kann durch eine fraktale Selbstähnlichkeit dargestellt werden:

\[
\hat{D} = \left( \frac{1}{N} \right)^{\frac{4}{5}} \sum_{k=1}^{\infty} \frac{e^{-\lambda k}}{k^{\gamma}} \cdot \hat{H}
\]

mit dem Operator:

\[
\hat{H} = \frac{1}{\pi^2} + e^{-\frac{1}{\pi^2}} \sum_{n=1}^{\infty} \frac{1}{n^3}
\]

\section{Bedeutung für die Riemannsche Hypothese}
Falls die Zeta-Nullstellen eine fraktale Betti-Topologie besitzen, dann könnte dies die Ordnung der kritischen Linie erklären.

\end{document}