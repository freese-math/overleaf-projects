\documentclass[a4paper,12pt]{article}
\usepackage{amsmath, amssymb}

\title{Von der Fibonacci-Freese-Formel zur Operator-Darstellung}
\author{Mathematische Ableitung}
\date{\today}

\begin{document}
\maketitle

\section{Einleitung}
Die Fibonacci-Freese-Formel (FFF) beschreibt die Nullstellenstruktur der Riemannschen Zeta-Funktion mit einer hohen numerischen Präzision. In diesem Dokument wird gezeigt, wie die Operator-Darstellung der FFF hergeleitet werden kann.

\section{Grundform der Fibonacci-Freese-Formel}
Die Formel lautet:
\begin{equation}
L(N) = A N^\beta + C \ln N.
\end{equation}
Mit den numerisch bestimmten Werten:
\begin{equation}
\beta = \frac{1}{8}, \quad A = 2.818, \quad C = 2488.1446.
\end{equation}

\section{Fraktale Struktur der Nullstellen}
Die Ableitung des effektiven Potentials ergibt:
\begin{equation}
V(x) = A x^{-0.875} + C \ln x.
\end{equation}
Dies entspricht einer modifizierten Schrödinger-Gleichung:
\begin{equation}
\hat{L} \psi = E \psi.
\end{equation}

\section{Schlussfolgerung}
Die Ableitung zeigt, dass die Nullstellenstruktur der Zeta-Funktion als Eigenwerte eines fraktalen Schrödinger-Operators beschrieben werden kann. Dies könnte eine direkte analytische Verbindung zur Riemannschen Hypothese liefern.

\end{document}