\documentclass[12pt]{article}
\usepackage{amsmath,amssymb}
\usepackage{graphicx}
\usepackage{tikz}
\usepackage{geometry}
\usepackage{hyperref}
\usepackage{float}
\geometry{a4paper, margin=2.5cm}

\title{Ein rigoroser Zugang zur Euler-Freese-Identität}
\author{Tim Freese}
\date{\today}

\begin{document}

\maketitle

\begin{abstract}
Die sogenannte \textbf{Euler-Freese-Identität} erweitert die klassische Euler-Gleichung $e^{i\pi} + 1 = 0$ um einen kontinuierlichen Parameter $\beta \in \mathbb{R}$. Wir analysieren die Struktur der Funktion
\[
H(\beta) := e^{i \pi \beta}
\]
auf dem Einheitskreis in der komplexen Ebene und zeigen, dass $|H(\beta)| = 1$ für alle $\beta \in \mathbb{R}$ gilt. Weiterhin betrachten wir spezielle Werte von $\beta$, die in der Freese-Reihe auftreten, und untersuchen ihre geometrischen sowie analytischen Eigenschaften.
\end{abstract}

\section{Die klassische Euler-Identität}

Die bekannte Euler-Identität lautet:
\[
e^{i \pi} + 1 = 0
\]
Sie stellt eine tiefgreifende Verbindung zwischen den mathematischen Konstanten $e$, $i$, $\pi$, $0$ und $1$ her. Dabei gilt allgemein:
\[
e^{i \theta} = \cos(\theta) + i \sin(\theta)
\]
Insbesondere für $\theta = \pi$ ergibt sich:
\[
e^{i\pi} = -1
\]

\section{Verallgemeinerung: \texorpdfstring{$H(\beta)$}{H(beta)}}

Die Euler-Freese-Identität betrachtet die Funktion:
\[
H(\beta) := e^{i\pi \beta}
\]
Dies definiert eine stetige Bewegung entlang des Einheitskreises, wobei $\beta$ als skalarer Rotationsparameter wirkt. Es gilt für alle $\beta \in \mathbb{R}$:
\[
|H(\beta)| = |e^{i\pi\beta}| = 1
\]
Da $\cos(\pi \beta)^2 + \sin(\pi \beta)^2 = 1$, ist $H(\beta)$ stets ein Punkt auf dem Einheitskreis.

\section{Beweis der Identität auf dem Einheitskreis}

\subsection*{Satz:}
Für alle $\beta \in \mathbb{R}$ gilt:
\[
H(\beta) = e^{i\pi\beta} \in \mathbb{C}, \quad \text{mit } |H(\beta)| = 1
\]

\textbf{Beweis:}
Aus der Euler-Formel folgt:
\[
H(\beta) = \cos(\pi\beta) + i\sin(\pi\beta)
\]
Dann ergibt sich:
\[
|H(\beta)|^2 = \cos^2(\pi\beta) + \sin^2(\pi\beta) = 1
\]
also:
\[
|H(\beta)| = 1 \quad \forall\, \beta \in \mathbb{R}
\]
\hfill$\blacksquare$

\section{Analyse ausgewählter Werte}

\begin{table}[H]
\centering
\begin{tabular}{|c|c|c|c|}
\hline
$\beta$ & $H(\beta)$ & Betrag $|H|$ & Phase $\arg(H)$ \\
\hline
$1$     & $-1$             & $1$ & $\pi$ \\
$1/2$   & $i$              & $1$ & $\pi/2$ \\
$1/33$  & $\approx -0.995 + 0.095i$ & $1$ & $\approx 3.05$ \\
$1/137$ & $\approx -0.9997 + 0.023i$ & $1$ & $\approx 3.12$ \\
$7/33300$ & $\approx -1 + 0.00066i$ & $1$ & $\approx 3.141$ \\
$\ln 3$ & $\approx 0.952 - 0.305i$ & $1$ & $\approx -0.309$ \\
\hline
\end{tabular}
\caption{Spezielle Werte der Funktion $H(\beta)$}
\end{table}

\section{Geometrische Interpretation}

\begin{figure}[H]
\centering
\begin{tikzpicture}[scale=3]
% Einheitskreis
\draw[thick] (0,0) circle(1);
\draw[->] (-1.2,0) -- (1.2,0) node[right] {$\Re$};
\draw[->] (0,-1.2) -- (0,1.2) node[above] {$\Im$};

% Punkte einzeichnen
\foreach \b/\c/\lab in {1/red/{$\beta=1$}, 0.5/blue/{$\beta=1/2$}, 0.0212/green/{$\beta=1/33$}, 0.0073/orange/{$\beta=1/137$}} {
    \draw[fill=\c] ({cos(180*\b)}, {sin(180*\b)}) circle(0.015);
    \node at ({1.2*cos(180*\b)}, {1.2*sin(180*\b)}) {\lab};
}
\end{tikzpicture}
\caption{Punkte $H(\beta)$ auf dem Einheitskreis für ausgewählte $\beta$}
\end{figure}

\section{Ausblick}

Die Euler-Freese-Identität könnte als skalierbare Variante der Euler-Formel angesehen werden, die über die Beta-Skala dynamisch mit Nullstellen der Riemannschen Zetafunktion korrespondiert. Weitere Untersuchungen können den Zusammenhang mit der modularen Struktur der Zeta-Nullstellen und deren Operatorstruktur aufdecken.

\end{document}