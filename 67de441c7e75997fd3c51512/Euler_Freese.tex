\documentclass[12pt]{article}
\usepackage{amsmath,amssymb}
\usepackage{tikz}
\usepackage{pgfplots}
\usepackage{animate}
\usepackage{csvsimple}
\usepackage{geometry}
\geometry{margin=2.5cm}
\pgfplotsset{compat=1.17}

\title{Dynamische Visualisierung der Euler-Freese-Identität}
\author{Prime Zeta Pro}
\date{\today}

\begin{document}

\maketitle

\section*{Animation auf dem Einheitskreis}

Die folgende Animation zeigt den Punkt 
\[
H(\beta) = e^{i\pi\beta}
\]
auf dem Einheitskreis für Werte $\beta \in [0,2]$. Der Punkt bewegt sich kontinuierlich gegen den Uhrzeigersinn:

\begin{center}
\begin{animateinline}[poster=first, controls, loop]{10}
\multiframe{41}{i=0+1}{
\begin{tikzpicture}[scale=2.5]
  \draw[->] (-1.2,0) -- (1.2,0) node[right] {$\Re$};
  \draw[->] (0,-1.2) -- (0,1.2) node[above] {$\Im$};
  \draw[thick] (0,0) circle(1);
  \def\b{\i*0.05}
  \fill[red] ({cos(180*\b)},{sin(180*\b)}) circle(0.03);
  \draw[dashed,gray] (0,0) -- ({cos(180*\b)},{sin(180*\b)});
  \node at (1.0,1.0) {$\beta = \b$};
\end{tikzpicture}
}
\end{animateinline}
\end{center}

\section*{Datenanalyse: Beta-Skala (aus CSV)}

Als nächstes analysieren wir numerische Beta-Werte, wie sie z.B. aus einer Datei `beta_skala_v5.csv` geladen werden könnten. Die Datei muss dabei zwei Zeilen enthalten: erste Zeile = Zeta-Nullstellen, zweite Zeile = Beta-Werte.

\subsection*{CSV-Tabelle (Auszug)}

\begin{center}
\csvreader[
    tabular=|c|c|,
    table head=\hline Zeta-Nullstelle & Beta-Wert \\ \hline,
    late after line=\\\hline,
    head to column names,
    filter test=\ifnumless{\thecsvinputline}{12}
]{beta_skala_v5.csv}{1=\zeta,2=\beta}
{\zeta & \beta}
\end{center}

\section*{Geometrische Interpretation}

Jeder dieser Beta-Werte definiert einen Punkt $H(\beta)$ auf dem Einheitskreis, mit:
\[
H(\beta) = \cos(\pi \beta) + i\sin(\pi \beta)
\]
Der Betrag ist stets 1, die Phase ist $\theta = \pi \beta$.

\end{document}