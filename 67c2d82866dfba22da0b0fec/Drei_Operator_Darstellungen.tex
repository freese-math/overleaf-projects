\documentclass[a4paper,12pt]{article}
\usepackage{amsmath, amssymb, amsfonts}
\usepackage{hyperref}
\usepackage{geometry}
\geometry{a4paper, margin=1in}
\usepackage{graphicx}

\title{Optimierte Parameter und Operator-Darstellung der Freese-Funktion}
\author{Berechnung basierend auf 2.001.052 Nullstellen}
\date{\today}

\begin{document}

\maketitle

\section{Einführung}
Die folgenden Ergebnisse stammen aus der Analyse von 2.001.052 Nullstellen der Riemannschen Zeta-Funktion. Die berechneten Parameter der Freese-Funktion (FF) basieren auf nichtlinearem Curve-Fitting sowie Fourier- und fraktaler Analyse.

\section{Optimierte Parameter der Freese-Funktion}

Die Freese-Funktion wird in der Form
\begin{equation}
    \text{FF}(N) = A \cdot N^{\beta} + C \log(N) + D \sin(N w + \varphi)
\end{equation}
dargestellt, mit den folgenden optimierten Parametern:

\begin{align*}
    A &= 1.8522100261940773 \\
    \beta &= 0.918041466589908 \\
    C &= 249.43750912457116 \\
    D &= 0.01227642588339449 \\
    w &= 0.08000615337071397 \\
    \varphi &= -9007.575132260432
\end{align*}

\section{Fourier-Analyse und fraktale Struktur}
Die Fourier-Analyse der Nullstellenverteilung zeigt zwei signifikante Frequenzen:

\begin{equation}
    f_{\text{signifikant}} = \{4.99737388 \times 10^{-7}, 1.49921216 \times 10^{-6} \}
\end{equation}

Die berechnete fraktale Dimension nach der Box-Counting-Methode beträgt:

\begin{equation}
    D_f = 0.9986841799827781
\end{equation}

\section{Topologische Analyse der Nullstellenverteilung}
Die Betti-Zahlen der Nullstellenverteilung, berechnet mittels persistent homology, ergeben:

\begin{equation}
    \beta_0 = 148, \quad \beta_1 = 0
\end{equation}

Dies deutet auf eine komplexe, aber nicht einfach zusammenhängende Struktur der Nullstellenverteilung hin.

\section{Operator-Darstellung der Freese-Funktion}
Zur weiteren mathematischen Analyse werden die Ableitungen der Freese-Funktion gebildet:

\subsection{Erste Ableitung}
\begin{equation}
    \frac{d}{dN} \text{FF}(N) = A \beta N^{\beta-1} + \frac{C}{N} + D w \cos(N w + \varphi)
\end{equation}

\subsection{Zweite Ableitung}
\begin{equation}
    \frac{d^2}{dN^2} \text{FF}(N) = A \beta (\beta - 1) N^{\beta-2} - \frac{C}{N^2} - D w^2 \sin(N w + \varphi)
\end{equation}

\section{Speicherung der Ergebnisse}
Die berechneten Werte wurden gespeichert unter:

\begin{verbatim}
/content/drive/MyDrive/freese_function_results.csv
\end{verbatim}

\section{Fazit}
Diese Ergebnisse zeigen eine hochpräzise Anpassung der Freese-Funktion an die Nullstellenverteilung der Riemannschen Zeta-Funktion. Die Kombination aus analytischer, spektraler und fraktaler Analyse legt nahe, dass die Freese-Funktion eine fundamentale Struktur in der Verteilung der Nullstellen beschreibt. Weitere Arbeiten sollen die theoretischen Grundlagen und den physikalischen Kontext vertiefen.

\end{document}